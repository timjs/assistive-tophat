% !TEX root=../main.tex

\section{Examples}
\label{sec:examples}

\subsection{Dining Computer Scientists Problem}

The dining philosophers problem is a classic concurrency problem in computer science.
A number of philosophers sit at a round table with a meal in front of them.
In between the plates lies one fork.
In order to eat their meal, each philosopher has to acquire both forks.
This, of course, means that all philosophers cannot eat at the same time, since there are not enough forks.
Deadlock can occur when all philosophers pick up the fork to their right (or left).
Everybody has one fork, an therefor cannot start his meal, but is also not allowed to put his fork back on the table.

We look at dining computer scientists instead.
As the philosophers, the computer scientists have a choice to take either the left or the right fork, provided that the fork is available.
However, events of picking up a fork are performed sequentially.
That is, when one computer scientist decides to pick up his right fork,
we will handle that event first,
after which we will handle the choices from the other scientists.
So, the order of the events is explicitly determined by the scientists.
% This is regardless of choices interfering or not.
It is therefore possible that Now, his right neighbour cannot pick up the fork to his left,
this choice has been disabled.

\cref{lst:dining} lists an implementation in \TOPHAT for this problem, with three computer scientists.
The forks are represented by references containing Booleans.
The value $\True$ indicates that the fork is available,
$\False$ indicates that the fork is being used.
Each computer scientist takes as arguments the two forks that he or she can reach.
After that, they can press continue if the second fork is also available.
For the sake of simplicity, they return the first fork, rather than setting the second fork to False, and then setting both to True again.

\lstset{emph={}}
\begin{TASK}[
    float=ht,
    numbers=right,
    caption={Dining computer scientists problem with three computer scientists.},
    label=lst:dining]
  let fork0 = ref True in
  let fork1 = ref True in
  let fork2 = ref True in
  let scientist = \ name. \ leftFork. \ rightFork.
    if !leftFork
      then leftFork := False >>? \ <<>>.
        if !rightFork then leftFork := True else fail
      else fail
    <?>
    if !rightFork
      then rightFork := False >>? \ <<>>.
        if !leftFork then rightFork := True else fail
       else fail in
  (scientist "Alan Turing" fork0 fork1 <&>
  scientist "Grace Hopper" fork1 fork2 <&>
  scientist "Ada Lovelace" fork2 fork0) >>= \ _ .
    edit "Done"
\end{TASK}


\subsection{Tax subsidy request}

The example program listed in this section is taken from our previous work on symbolic execution for \TOPHAT~\cite{Steenvoorden2019}.
It models a simplified tax subsidy application process for citizens who have installed solar panels.

A subsidy is only given under the following conditions.
\begin{itemize}
\item The roofing company has confirmed that they installed solar panels for the citizen.
\item The tax officer has approved the request.
\item The tax officer can only approve the request if the roofing company has confirmed, and the request is filed within one year of the invoice date.
\item The amount of the granted subsidy is maximal 600 EUR.
\end{itemize}

The listing below gives the \TOPHAT code for this example.

\lstset{emph={invoiceDate,today,confirmed,invoiceAmount,approved}}
\begin{TASK}[
    float=p,
    numbers=right,
    caption={Subsidy request and approval workflow at the Dutch tax office.},
    label=lst:tax]
  let getCurrentDate = enter Date in
  let provideDocuments = enter Amount <&> enter Date in
  let companyConfirm = edit True <?> edit False in
  let officerApprove = \ invoiceDate. \ today. \ confirmed.
    edit False <?> if (today - invoiceDate < 365 /\ confirmed) |\label{lst:tax:officer-approve-def}| then edit True else fail in
  getCurrentDate >>= \ today.|\label{lst:tax:citizen-info}|
  provideDocuments <&> companyConfirm >>= \ <<<<invoiceAmount, invoiceDate>>, confirmed>>. |\label{lst:tax:documents-and-company-confirm}|
  officerApprove invoiceDate today confirmed >>= \ approved.|\label{lst:tax:officer-approve}|
  let subsidyAmount = if approved then min 600 (invoiceAmount / 10) else 0 in
    edit <<subsidyAmount, approved, confirmed, invoiceDate, today>>|\label{lst:tax:result}|
\end{TASK}

In previous work, we have shown that this code indeed adheres to the requirements listed above.

Instead of assisting the developer, by proving the program correct, we would now like to support the end user that is requesting a subsidy.

The end user wants the outcome of this program to be a subsidy amount larger than zero.

Running the simulation function will result in the following set:

\begin{table}[p]
  \centering
  \begin{tabular}{LLL}
    \toprule
    \text{Symbolic value} & \text{Symbolic input} & \text{Path condition} \\
    \midrule
    \tuple{\min\ 600\ (s_{\id{a}}/10),  \True, \True, s_{\id{i}}, s_{\id{t}}} & [s_{\id{t}},\First \First s_{\id{a}}, \First \Second s_{\id{i}}, \Second \First, \Second] & s_{\id{t}}-s_{\id{i}}<365 \\
    \tuple{\min\ 600\ (s_{\id{a}}/10),  \True, \True, s_{\id{i}}, s_{\id{t}}} & [s_{\id{t}},\First \Second s_{\id{i}}, \First \First s_{\id{a}}, \Second \First, \Second] & s_{\id{t}}-s_{\id{i}}<365 \\
    \tuple{\min\ 600\ (s_{\id{a}}/10),  \True, \True, s_{\id{i}}, s_{\id{t}}} & [s_{\id{t}},\Second \First, \First \First s_{\id{a}}, \First \Second s_{\id{i}}, \Second] & s_{\id{t}}-s_{\id{i}}<365 \\
    \tuple{\min\ 600\ (s_{\id{a}}/10),  \True, \True, s_{\id{i}}, s_{\id{t}}} & [s_{\id{t}},\Second \First, \First \Second s_{\id{i}}, \First \First s_{\id{a}}, \Second] & s_{\id{t}}-s_{\id{i}}<365 \\
    \tuple{\min\ 600\ (s_{\id{a}}/10),  \True, \True, s_{\id{i}}, s_{\id{t}}} & [s_{\id{t}},\First \Second s_{\id{i}}, \Second \First, \First \First s_{\id{a}}, \Second] & s_{\id{t}}-s_{\id{i}}<365 \\
    \tuple{\min\ 600\ (s_{\id{a}}/10),  \True, \True, s_{\id{i}}, s_{\id{t}}} & [s_{\id{t}},\First \First s_{\id{a}}, \Second \First, \First \Second s_{\id{i}}, \Second] & s_{\id{t}}-s_{\id{i}}<365 \\
    \tuple{                        0,  \False, \True, s_{\id{i}}, s_{\id{t}}} & [s_{\id{t}},\First \First s_{\id{a}}, \First \Second s_{\id{i}}, \Second \First, \First]  & \True \\
    \tuple{                        0,  \False, \True, s_{\id{i}}, s_{\id{t}}} & [s_{\id{t}},\First \Second s_{\id{i}}, \First \First s_{\id{a}}, \Second \First, \First]  & \True \\
    \tuple{                        0,  \False, \True, s_{\id{i}}, s_{\id{t}}} & [s_{\id{t}},\Second \First, \First \First s_{\id{a}}, \First \Second s_{\id{i}}, \First]  & \True \\
    \tuple{                        0,  \False, \True, s_{\id{i}}, s_{\id{t}}} & [s_{\id{t}},\Second \First, \First \Second s_{\id{i}}, \First \First s_{\id{a}}, \First]  & \True \\
    \tuple{                        0,  \False, \True, s_{\id{i}}, s_{\id{t}}} & [s_{\id{t}},\First \Second s_{\id{i}}, \Second \First, \First \First s_{\id{a}}, \First]  & \True \\
    \tuple{                        0,  \False, \True, s_{\id{i}}, s_{\id{t}}} & [s_{\id{t}},\First \First s_{\id{a}}, \Second \First, \First \Second s_{\id{i}}, \First]  & \True \\
    \tuple{                        0,  \False,\False, s_{\id{i}}, s_{\id{t}}} & [s_{\id{t}},\First \First s_{\id{a}}, \First \Second s_{\id{i}}, \Second, \First]  & \True \\
    \tuple{                        0,  \False,\False, s_{\id{i}}, s_{\id{t}}} & [s_{\id{t}},\First \Second s_{\id{i}}, \First \First s_{\id{a}}, \Second, \First]  & \True \\
    \tuple{                        0,  \False,\False, s_{\id{i}}, s_{\id{t}}} & [s_{\id{t}},\Second \Second,\First \First s_{\id{a}}, \First \Second s_{\id{i}}, \First]  & \True \\
    \tuple{                        0,  \False,\False, s_{\id{i}}, s_{\id{t}}} & [s_{\id{t}},\Second, \First \Second s_{\id{i}}, \First \First s_{\id{a}}, \First]  & \True \\
    \tuple{                        0,  \False,\False, s_{\id{i}}, s_{\id{t}}} & [s_{\id{t}},\First \Second s_{\id{i}}, \Second, \First \First s_{\id{a}}, \First]  & \True \\
    \tuple{                        0,  \False,\False, s_{\id{i}}, s_{\id{t}}} & [s_{\id{t}},\First \First s_{\id{a}}, \Second, \First \Second s_{\id{i}}, \First]  & \True \\
    \bottomrule
  \end{tabular}
  \begin{tabular}{LLL}
    \toprule
    \text{Symbolic value} & \text{Symbolic input} & \text{Path condition} \\
    \midrule
    \tuple{\min\ 600\ (s_{\id{a}}/10),  \True, \True, s_{\id{i}}, s_{\id{t}}} & [s_{\id{t}},\First \First s_{\id{a}}, \First \Second s_{\id{i}}, \Second \First, \Second] & s_{\id{t}}-s_{\id{i}}<365 \\
    \tuple{\min\ 600\ (s_{\id{a}}/10),  \True, \True, s_{\id{i}}, s_{\id{t}}} & [s_{\id{t}},\First \Second s_{\id{i}}, \First \First s_{\id{a}}, \Second \First, \Second] & s_{\id{t}}-s_{\id{i}}<365 \\
    \tuple{\min\ 600\ (s_{\id{a}}/10),  \True, \True, s_{\id{i}}, s_{\id{t}}} & [s_{\id{t}},\Second \First, \First \First s_{\id{a}}, \First \Second s_{\id{i}}, \Second] & s_{\id{t}}-s_{\id{i}}<365 \\
    \tuple{\min\ 600\ (s_{\id{a}}/10),  \True, \True, s_{\id{i}}, s_{\id{t}}} & [s_{\id{t}},\Second \First, \First \Second s_{\id{i}}, \First \First s_{\id{a}}, \Second] & s_{\id{t}}-s_{\id{i}}<365 \\
    \tuple{\min\ 600\ (s_{\id{a}}/10),  \True, \True, s_{\id{i}}, s_{\id{t}}} & [s_{\id{t}},\First \Second s_{\id{i}}, \Second \First, \First \First s_{\id{a}}, \Second] & s_{\id{t}}-s_{\id{i}}<365 \\
    \tuple{\min\ 600\ (s_{\id{a}}/10),  \True, \True, s_{\id{i}}, s_{\id{t}}} & [s_{\id{t}},\First \First s_{\id{a}}, \Second \First, \First \Second s_{\id{i}}, \Second] & s_{\id{t}}-s_{\id{i}}<365 \\
    \bottomrule
  \end{tabular}
  \caption{}
  \label{}
\end{table}

As a goal we set that the subsidy amount should be larger than zero. This leaves us with the following cases.

From the end states we can conclude that invoiceAmount should be larger than zero in order for the result to be larger than zero. This results in an additional constraint.

In the end, we can use the above to give hints to the user.

First, a date must be entered. Then the user can choose to ender an amount larger than zero, a date that is within 365 from now, or a "RL".
etc.
