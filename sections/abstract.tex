% !TEX root=../main.tex

% Context
Software modelling business workflows are omnipresent today's society.
They coordinate collaboration in hospitals, companies, and military institutions.

% Inquiry
These systems obfuscate the influence of current user actions on the desired end result.
In order to make the right decision, they need to oversee the full process and all information in the system.
Both of which are usually buried in the system.

% Approach
We have developed a way to automatically generate next step hints for task oriented programs.
By leveraging symbolic execution, we can calculate these hints without modification of the original program.


% Knowledge
To our knowledge this is the first time that symbolic execution is used to automatically generate next step hints for end users.
We implemented this approach in Haskell to generate next step hints for TopHat programs, a formal system for task oriented programming.

% Grounding
Besides proving the generated hints to be sound and complete, we also demonstrate that the symbolic execution semantics we employ to be correct for sequential input.

% Importance
Next step hints reduce the chance of human error, while still allowing end users to intervene if required.
The overall performance is raised, since the quality of decisions will improve.

% Context: What is the broad context of the work? What is the importance of the general research area?
% Inquiry: What problem or question does the paper address? How has this problem or question been addressed by others (if at all)?
% Approach: What was done that unveiled new knowledge?
% Knowledge: What new facts were uncovered? If the research was not results oriented, what new capabilities are enabled by the work?
% Grounding: What argument, feasibility proof, artifacts, or results and evaluation support this work?
% Importance: Why does this work matter?

\keywords{Task-oriented programming \and Next step hint generation \and Symbolic execution.}
