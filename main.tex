\RequirePackage{amsmath} % fixes vec warning
\documentclass[runningheads]{llncs}

\usepackage[square,sort,comma,numbers]{natbib}

% !TEX root=../main.tex


%% Basics %%%%%%%%%%%%%%%%%%%%%%%%%%%%%%%%%%%%%%%%%%%%%%%%%%%%%%%%%%%%%%%%%%%%%%

%% Fixes %%

\usepackage{underscore}


%% Fonts %%

\usepackage[utf8]{inputenc}
% \usepackage[T1]{fontenc}
%%NOTE: T1 doesn't have the `Th` and `Qu` ligatures :-(
\usepackage[OT1]{fontenc}

\usepackage{stmaryrd}
\usepackage{mathtools}
\usepackage{eurosym}
\usepackage{ dsfont }

%\usepackage{amsthm} %%NOTE: here because defines \openbox which will also be defined by newtxmath...
\usepackage{xcolor}


% \usepackage{tgpagella}
% \usepackage{lucidabr}

\usepackage{libertine}
\usepackage[varqu]{zi4}
\usepackage[libertine]{newtxmath}


%% Programming %%

\usepackage{xargs}
\usepackage{ifthen}


%% Layout %%

% \usepackage{microtype}
\usepackage{xspace}


%% Additions %%%%%%%%%%%%%%%%%%%%%%%%%%%%%%%%%%%%%%%%%%%%%%%%%%%%%%%%%%%%%%%%%%%

%% Textual %%

% \usepackage{titlesec}
\usepackage[inline]{enumitem}
\usepackage{quoting}

%% Maths %%

\usepackage{amsmath}


%% Graphics %%

\usepackage{graphicx}
\usepackage{xcolor}
\usepackage{dblfloatfix}
\usepackage[export]{adjustbox}
% \usepackage[xcolor]{mdframed}
\usepackage{tikz}
\usetikzlibrary{trees,arrows, automata, decorations.markings,decorations.pathmorphing}

%% Tabulations %%

\usepackage{booktabs}
\usepackage{array}


%% Listings %%

\usepackage[final]{listings}


%% References & Bibliography %%

\usepackage[capitalize]{cleveref}
\usepackage{natbib}
% \usepackage[natbibapa,nodoi]{apacite}

% !TEX root=../main.tex


%% Fixes %%

\frenchspacing

\newlength{\hugeskipamount}
\setlength{\hugeskipamount}  {1.2500\baselineskip plus 0.3750\baselineskip minus 0.3750\baselineskip}
\setlength{\bigskipamount}   {0.7500\baselineskip plus 0.2500\baselineskip minus 0.2500\baselineskip}
\setlength{\medskipamount}   {0.3750\baselineskip plus 0.1250\baselineskip minus 0.1250\baselineskip}
\setlength{\smallskipamount} {0.1875\baselineskip plus 0.0625\baselineskip minus 0.0625\baselineskip}

% \widowpenalty=150
% \clubpenalty=150


%% Section spacing %%
%%NOTE: requires 'titlesec'

% \titlespacing*{\section}{0pt}{\hugeskipamount}{\bigskipamount}
% \titlespacing*{\subsection}{0pt}{\bigskipamount}{\medskipamount}
% \titlespacing*{\paragraph}{0pt}{\medskipamount}{1em}


%% Math spacing %%

\setlength{\abovedisplayskip}{\smallskipamount}
\setlength{\belowdisplayskip}{\smallskipamount}

% \setlength{\topsep}{\smallskipamount}



%% Float spacing %%

\setlength{\abovecaptionskip}{\medskipamount}
\setlength{\floatsep}        {\medskipamount}
\setlength{\textfloatsep}    {\bigskipamount}
\setlength{\intextsep}       {\bigskipamount}
\setlength{\dblfloatsep}     {\medskipamount}
\setlength{\dbltextfloatsep} {\bigskipamount}


%% Description spacing %%

\setlist{noitemsep}
\setlist[description]{leftmargin=\parindent}


%% List spacing %%
%% NOTE: requires `paralist`

% \setlength{\pltopsep}   {\medskipamount}
% \setlength{\plpartopsep}{\parskip}
% \setlength{\plitemsep}  {\parskip}
% \setlength{\plparsep}   {\parskip}


%% Listings spacing %%

% \lstset
%   {aboveskip=\smallskipamount
%   ,belowskip=\smallskipamount
%   }


%% Tabular strech %%
%% NOTE: requires `array`

% \renewmacro{arraystretch}
%   {1.1}


%% Quoting %%
%%NOTE: requires 'quoting'

\quotingsetup
  {font=itshape
  ,leftmargin=\parindent
  ,listvskip}


%% Boxes %%

% \mdfsetup
%   {hidealllines=true
%   ,backgroundcolor=lightgray
%   }

% !TEX root=../main.tex


\input macros/auxiliaries


%% Fixes %%%%%%%%%%%%%%%%%%%%%%%%%%%%%%%%%%%%%%%%%%%%%%%%%%%%%%%%%%%%%%%%%%%%%%%

\let\texttilde\textasciitilde


%% Text %%%%%%%%%%%%%%%%%%%%%%%%%%%%%%%%%%%%%%%%%%%%%%%%%%%%%%%%%%%%%%%%%%%%%%%%

\newmacro{separate}
  {\medskip\noindent}

\providemacro{marginnote}
  {\marginpar}
\providemacro{smallcaps}
  {\textsc}
\providemacro{marginwidth}
  {\marginparwidth}

\newmacro{alert}[1]
  {\textbf{#1}}
\newmacro{divert}[1]
  {\textcolor{gray}{#1}}
\newmacro{enquote}[1]
  {``#1''}
\newmacro{fixme}[1]
  {\colorbox{yellow}{#1}\marginnote{\colorbox{yellow}{$\star$}}}
\newmacro{todo}[1]
  {\textcolor{red}{$\star$}\marginnote{\textcolor{red}{#1}}}
  % {}
\newmacro{type}[1]
  {\texttt{#1}}

\newmacro{add}[1]
  {\textcolor{green}{#1}}
\newmacro{remove}[1]
  {\textcolor{red}{#1}}
\newmacro{change}[1]
  {\textcolor{orange}{#1}}
\newmacro{adjust}[2]
  {\remove{#1} \add{#2}}

\newenvironment{fadeout}
  {\color{gray}}
  {}
\newenvironment{emphasize}
  {\begin{quote}\itshape}
  {\end{quote}}
\newenvironment{margintext}[1]
  {\begin{marginfigure}
     \subsection*{#1}}
  {\end{marginfigure}}


%% Lists %%
%% NOTE: requires `paralist`

% %% Use compact lists by default
% \renewenvironment{itemize}
%   {\begin{compactitem}}
%   {\end{compactitem}}
% \renewenvironment{enumerate}
%   {\begin{compactenum}}
%   {\end{compactenum}}
% \renewenvironment{description}
%   {\begin{compactdesc}}
%   {\end{compactdesc}}
% %% Define starred versions as in-paragraph-lists
% \newenvironment{itemize*}
%   {\begin{inparaitem}}
%   {\end{inparaitem}}
% \newenvironment{enumerate*}[1][1=(i)]
%   {\begin{inparaenum}[#1]}
%   {\end{inparaenum}}
% \newenvironment{description*}
%   {\begin{inparadesc}}
%   {\end{inparadesc}}


%% Quotations %%
%% NOTE: requires `quoting`

\let\quote\quoting
\let\endquote\endquoting
\renewenvironment{quotation}
  {\ClassError{Please use the `quote` environment instead of `quotation`}}


%% Column types %%
%% NOTE: requires `array`

\newcolumntype{L}{>{$}l<{$}}
\newcolumntype{C}{>{$}c<{$}}
\newcolumntype{R}{>{$}r<{$}}
\newcolumntype{T}{>{\ttfamily}l}
\newcolumntype{S}{>{\sffamily}l}


%% References %%
%%NOTE: requires `cleveref`

\let\refer\cref
\let\Refer\Cref


%% Citations %%
%% NOTE: requires `natbib`

\let\cite\citep
\let\Cite\Citep
\let\textcite\citet
\let\Textcite\Citet


%% Blocks and Boxes %%

\newenvironment{block}
  {\begin{center}}
  {\end{center}}
% \newenvironment{box}
%   {\begin{mdframed}}
%   {\end{mdframed}}


%% Logos %%

%% \newlogo[.name.]{.text.}
\newmacro{newlogo}[2][1]
  {\ifthenelse{\isempty{#1}}
     {\newlogoaux{#2}{\smallcaps{\lowercase{#2}}}}
     {\newlogoaux{#1}{#2}}}
\newmacro{newlogoaux}[2]
  {\newmacro{#1}{#2\xspace}}


%% Languages %%%%%%%%%%%%%%%%%%%%%%%%%%%%%%%%%%%%%%%%%%%%%%%%%%%%%%%%%%%%%%%%%%%

%%NOTE: `\mathrel` gives a single space width between keywords but removes it after another relational operator.
%%      `\mathop`  gives just a small skip, but doesn't has above bug.
\newmacro{newoperator}[1]
  {\newmathcommand{#1}[op]}
\newmacro{newkeyword}[2][1]
  %%FIXME: this is to complicated: {\newoperator{\ifthenelse{\isempty{#1}}{#2}{#1}}{\text{\sffamily\bfseries #2}}}
  {\ifthenelse{\isempty{#1}}
    {\newoperator{#2}{\text{\normalfont\sffamily\bfseries #2}}}
    {\newoperator{#1}{\text{\normalfont\sffamily\bfseries #2}}}}
\newmacro{newvalue}[2][1]
  {\ifthenelse{\isempty{#1}}
    {\newoperator{#2}{\text{\normalfont\sffamily #2}}}
    {\newoperator{#1}{\text{\normalfont\sffamily #2}}}}
\newmacro{newtype}[2][1]
  {\ifthenelse{\isempty{#1}}
    {\newoperator{#2}{\text{\normalfont\sffamily\scshape #2}}}
    {\newoperator{#1}{\text{\normalfont\sffamily\scshape #2}}}}


%% Math %%%%%%%%%%%%%%%%%%%%%%%%%%%%%%%%%%%%%%%%%%%%%%%%%%%%%%%%%%%%%%%%%%%%%%%%

%% Boxes %%

\newmacro{obox}[2]
  {\makebox[0pt][l]{\ensuremath{#2}}\phantom{\ensuremath{#1}}}

\newmacro{highlight}[1]
  {\colorbox{lightgray}{\ensuremath{#1}}}

%% Spacing %%

\newmacro{Quad}
  {\hspace{1.5em}}
\newmacro{Break}
  {\\[\smallskipamount]}



%% Fractions %%

\newmacro{upon}
  {\genfrac{}{}{0pt}{0}}



%% Symbols %%

%% NOTE: change this to \emptyset when using a font that includes a nice standard emptyset
\let\nothing\varnothing



%% Braces %%

\let\<\langle
\let\>\rangle

\newmathcommand{llbrace}[open] {\{\!|}
\newmathcommand{rrbrace}[close]{|\!\}}

\newmacro{set}[1]
  {\ensuremath{\{#1\}}}
\newmacro{tuple}[1]
  {\ensuremath{\<#1\>}}


%% Operators %%

\let\lt<
\let\gt>
\let\To\Rightarrow

\newmathcommand{pp}[bin]
  {+\!\!+}
\newmathcommand{Mid}
  {\;\mid\;}


%% Shortcuts %%

\newmacro{powerset}[1]
  % {2^{#1}}
  {\mathcal{P}(#1)}

\newmathcommand{n}{\underline{n}}

\newmathcommand{NN}  [bb]{N}
\newmathcommand{ZZ}  [bb]{Z}
\newmathcommand{EE}  [bb]{E}
\newmathcommand{OO}  [bb]{O}
\newmathcommand{QQ}  [bb]{A}
\newmathcommand{RR}  [bb]{R}
\newmathcommand{CC}  [bb]{C}
\newmathcommand{HH}  [bb]{H}

\newmathcommand{LL}  [bb]{L}
\newmathcommand{UU}  [bb]{U}
\newmathcommand{BB}  [bb]{B}
\renewmathcommand{SS}[bb]{S}

\let\to\rightarrow
% \let\implies\Rightarrow
% \let\implies\Longrightarrow
\let\implies\supset
\let\infers\vdash


%% Hints and local definitions %%

\newmacro{hint}[1]
  {\quad\text{\{ #1 \}}}

\newmathcommand{when}[op]
  {\mathbf{when}}
\newmathcommand{where}[op]
  {\mathbf{where}}
\renewmathcommand{and}[op]
  {\mathbf{and}}
\newmathcommand{otherwise}[op]
  {\mathbf{otherwise}}
\newmathcommand{impossible}[op]
  {\mathrm{impossible}}


%% Environments %%

\let\group\begingroup

\newenvironment*{marginequation}[1][1=0pt]
  {\begin{marginfigure}[#1]\equation}
  {\endequation\end{marginfigure}}

\newenvironment*{marginequation*}[1][1=0pt]
  {\begin{marginfigure}[#1]\equation\nonumber}
  {\endequation\end{marginfigure}}


\newenvironment*{function}
  {\begin{tabular}{@{}L@{\ \ }C@{\ \ }L@{}}}
  {\end{tabular}}
\newmacro{signature}[1]
  {\multicolumn{3}{@{}L@{}}{#1}}
\newmacro{inset}[1]
  {\multicolumn{3}{L}{\quad #1}}


\newenvironment*{grammar}
  %%NOTE: the `@{}` suppreses `\tabcolsep` before the first column
  {\begin{block}\begin{tabular}{@{}rRCLl}}
  {\end{tabular}\end{block}}
\newenvironment*{grammar*}
  %%NOTE: the `@{}` suppreses `\tabcolsep` before the first column
  {\begin{block}\begin{tabular}{@{}RLl}}
  {\end{tabular}\end{block}}



%% Theorems %%

% \newtheoremstyle{plain}%
%   {\medskipamount}% space above
%   {\medskipamount}% space below
%   {\itshape}% body font
%   {0pt}% indent amount
%   {\bfseries}% head font
%   {.}% punctuation after head
%   {.5em}% spacing after head
%   {\thmname{#1}\thmnumber{ #2}\thmnote{ {\normalfont(#3)}}}% head spec
% \newtheoremstyle{definition}%
%   {\medskipamount}% space above
%   {\medskipamount}% space below
%   {\normalfont}% body font
%   {0pt}% indent amount
%   {\bfseries}% head font
%   {.}% punctuation after head
%   {.5em}% spacing after head
%   {\thmname{#1}\thmnumber{ #2}\thmnote{ {\normalfont(#3)}}}% head spec
%
%
%\theoremstyle{acmplain}

%\newtheorem{theorem}{Theorem}[section]
%\newtheorem{conjecture}[theorem]{Conjecture}
%\newtheorem{proposition}[theorem]{Proposition}
%\newtheorem{lemma}[theorem]{Lemma}
%\newtheorem{corollary}[theorem]{Corollary}


%\theoremstyle{acmdefinition}

%\newtheorem{example}[theorem]{Example}
%\newtheorem{definition}[theorem]{Definition}



%% Inference rules %%

\newmacro{placerule}[4][1,4]
  {\ensuremath{
    \upon
      {\text{\smallcaps{#1}}\hfill}
      {\dfrac{#2}{#3}\ #4}
  }}

\newmacro{newrule}[4][4]
  {\newmacro{#1}{\placerule[#1]{#2}{#3}[#4]}}
\newmacro{userule}
  {\usemacro}
\newmacro{refrule}[1]
  {\ifthenelse{\isundefined{#1}}
    {\GenericError{}{Rule `#1` is not defined}{}{}}
    {\textsc{#1}}}
% \newmacro{refrule}
%   {\textsc}

% !TEX root=../main.tex


%% Styles %%%%%%%%%%%%%%%%%%%%%%%%%%%%%%%%%%%%%%%%%%%%%%%%%%%%%%%%%%%%%%%%%%%%%%

\lstdefinestyle{common}
  {escapechar=|
  ,numbersep=-9pt % to make numbers appear inside the column; otherwise they are in the margin
  ,aboveskip=0pt
  ,belowskip=0pt
  }

\lstdefinestyle{natural}
  {style=common
  ,columns=fullflexible
  ,gobble=2
  ,breaklines=true
  ,breakatwhitespace=true
  ,literate=
    %{.}{{$\cdot$}}1
    %{.}{{\ }}1
    {<<}{{$\<$}}1
    {>>}{{$\>$}}1
    {->}{{$\to$\ }}2
    % {--}{{--}}1
    %{_}{{\ }}1
    %{\ "}{{\ \textquotedblleft}}2
    %{"\ }{{\textquotedblright\ }}2
  ,basicstyle={\sffamily}
  ,keywordstyle=[1]{\bfseries}
  ,keywordstyle=[2]{\scshape}
  ,keywordstyle=[3]{}
  %,commentstyle={\itshape}
  %,identifierstyle={\itshape}
  ,emphstyle={\itshape}
  %,stringstyle={\rmfamily}
  ,showstringspaces=false
  ,texcl=true
  ,mathescape=true
  %,escapechar=\$
  %,escapeinside={\{\}}
  ,xleftmargin=1\parindent
  }

\lstdefinestyle{flexible}
  {columns=flexible
  ,gobble=2
  ,fontadjust=true
  ,basicstyle={\ttfamily\small}
  ,commentstyle={\itshape}
  ,keywordstyle={\bfseries}
  %,identifierstyle={\itshape}
  %,stringstyle={\ttfamily}
  ,emphstyle={\itshape}
  ,showstringspaces=false
  ,texcl=true
  ,mathescape=true
  %,escapechar=\$
  %,escapeinside={\{\}}
  ,xleftmargin=1\parindent
  }

\lstdefinestyle{literate}
  {style=natural
  ,literate=
    {\\}{{$\lambda$}}1
    {\\\$}{{\$}}1 %NOTE: otherwise eaten by `\`, NOTE: prevents \$ to be parsed as math escape
    {\\/}{{$\vee$}}1
    {/\\}{{$\wedge$}}1
    {A.}{{$\forall$}}1
    {E.}{{$\exist$}}1
    {->}{{$\rightarrow$ }}1
    {<-}{{$\leftarrow$}}1
    {==}{{$\equiv$\ }}1
    {/=}{{$\nequiv$\ }}1
    {<=}{{$\leq$}}1
    {>=}{{$\geq$}}1
    {>>=}{{>>=}}3 %NOTE: otherwise eaten by `>=`
    {\{|}{{$\{\!|\!$}}1
    {|\}}{{$\!|\!\}$}}1
    {\{|*|\}}{{$\{\!|\!\!\star\!\!|\!\}$}}3
  }


%% Definitions %%%%%%%%%%%%%%%%%%%%%%%%%%%%%%%%%%%%%%%%%%%%%%%%%%%%%%%%%%%%%%%%%

%% Tasks %%

\lstdefinelanguage{tasks}
  {sensitive=true
  ,morekeywords=[1]{let,in,if,then,else,case,of,ref,assert,type}
  ,morekeywords=[2]{Bool,Int,String,Unit,List, Ref,Task, Passenger,Seat,Booking, Snack}
  ,moreemph={a,b,c,d,e,f,g,h,i,j,k,l,m,n,o,p,q,r,s,t,u,v,w,x,y,z as,bs,cs,ds,es,fs,gs,hs,is,js,ks,ls,ms,ns,os,ps,qs,rs,ss,ts,us,vs,ws,xs,ys,zs}
  ,morestring=[b]"
  ,morecomment=[l]--
  ,morecomment=[n]{\{-}{-\}}
  }[keywords,strings,comments]
\lstdefinestyle{tasks}
  {style=natural
  ,literate=
    {\\}{{$\lambda$}}1
    {<<}{{$\<$}}1
    {>>}{{$\>$ }}1
    {->}{{$\to$ }}1
    {==}{{$\equiv$ }}1
    {/=}{{$\nequiv$ }}1
    {<=}{{$\leq$ }}1
    {>=}{{$\geq$ }}1
    {*}{{$\times$ }}1
    {`elem`}{{$\in$ }}1
    {\\/}{{$\vee$ }}1
    {/\\}{{$\wedge$ }}1
    {>>=}{{$\Then$ }}1
    {>>?}{{$\Next$ }}1
    {<&>}{{$\And$ }}1
    {<|>}{{$\Or$ }}1
    {<?>}{{$\Xor$ }}1
    {++}{{$\pp$ }}1
    {edit}{{$\Edit$}}1
    {enter}{{$\Enter$}}1
    {update}{{$\Update$}}1
    {fail}{{$\Fail$ }}1
  }

\lstnewenvironment{TASK}[1][]
  {\lstset{language=tasks,style=tasks,#1}}
  {}
\newmacro{TS}[1][1]
  {\lstinline[language=tasks,style=tasks,#1]}
\newmacro{includeTASK}[2][]
  {\lstinputlisting[language=tasks,style=tasks,#1]{#2}}


%% Flows %%

\lstdefinelanguage{flows}
  {sensitive=true
  ,morekeywords=[1]{module,where,define,using,as,yielding,share,holding,with,do,for,fork,then,when,next,done,on,and,or,not,readonly,writeonly,readwrite}
  ,morekeywords=[2]{Bool,Int,String,Shared,List, Date,Document,Photo, Citizen,Company,Declaration}
  ,morekeywords=[3]{True,False,Just,Nothing,List}
  ,morestring=[b]"
  ,morecomment=[l]--
  ,morecomment=[n]{\{-}{-\}}
  }[keywords,strings,comments]

% \lstMakeShortInline[language=flows,style=natural] | % |
\lstnewenvironment{FLOW}[1][]
  {\lstset{language=flows,style=natural,#1}}
  {}
\newmacro{FL}[1][1]
  {\lstinline[language=flows,style=natural,#1]}
\newmacro{includeFLOW}[2][]
  {\lstinputlisting[language=flows,style=natural,#1]{#2}}


%% Clean %%

\lstdefinelanguage{clean}
  {sensitive=true
  %,alsoletter={ABCDEFGHIJKLMNOPQRSTUVWXYZabcdefghijklmnopqrstuvwxyz_`}
  %,alsoletter={~!@\#$\%^\&*-+=?<>:|\\} %$
  ,morekeywords={from,definition,implementation,import,module,system,code,inline,if,case,of,let,let!,in,where,with,class,instance,generic,derive,dynamic,infix,infixl,infixr}
  ,morestring=[b]"
  ,morestring=[b]'
  ,morecomment=[l]//
  ,morecomment=[n]{/*}{*/}
  }[keywords,strings,comments]

\lstnewenvironment{CLEAN}[1][]
  {\lstset{language=clean,style=flexible,#1}}
  {}
\newmacro{CL}[1][1]
  {\lstinline[language=clean,style=flexible,#1]}
\newmacro{includeCLEAN}[2][1]
  {\lstinputlisting[language=clean,style=flexible,#1]{#2}}

% !TEX root=../main.tex

\newlogo{QED}

\newlogo[ITASKS]{iTasks}
\newlogo[MTASKS]{mTasks}
\newlogo{TOP}

\newlogo{BPMN}
\newlogo{BPEL}
\newlogo{UML}
\newlogo{WFN}
\newlogo{YAWL}
\newlogo{IFTTT}

\newlogo{CCS}
\newlogo{CSP}
\newlogo[PICALC]{$\pi$-calculus}

\newlogo[HASKELL]{Haskell}
\newlogo{GHC}
\newlogo[ESTEREL]{Esterel}
\newlogo[HIPHOP]{HopHop}
\newlogo{FRP}

\newlogo{HTML}
\newlogo{XML}
\newlogo{IOT}

\newlogo{DSL}
\newlogo{EDSL}
\newlogo{GUI}
\newlogo{SAT}
\newlogo{SMT}
\newlogo[ZTHREE]{\smallcaps{z3}}
\newlogo[SMTLIB]{\smallcaps{smt-lib}}

\newlogo{STW}
\newlogo{NWO}

\newlogo{IO}
\newlogo{SML}
\newlogo{ML}
\newlogo{UI}
\newlogo{ID}


% !TEX root=../main.tex



\let\phi\varphi

\newmacro{Today}{\text{13 Feb 2020}}
\newmacro{OneYear}{\text{365 days}}


%% Host language %%%%%%%%%%%%%%%%%%%%%%%%%%%%%%%%%%%%%%%%%%%%%%%%%%%%%%%%%%%%%%%


\newkeyword[IF]  {if}
\newkeyword[THEN]{then}
\newkeyword[ELSE]{else}

\newkeyword[Let]{let}
\newkeyword[In]{in}

\newkeyword[Ref] {ref}


\newmacro{If}[3]
  {\IF #1 \THEN #2 \ELSE #3}



%% Values %%


\newmathcommand{unit}{\<\>}


\newvalue{True}
\newvalue{False}
\newvalue[Not]{not}


\newmacro{str}[1]
  {\text{``#1''}}

\newvalue[Map]{map}
\newvalue[Fst]{fst}
\newvalue[Snd]{snd}
\newvalue[Head]{head}
\newvalue[Tail]{tail}
\newvalue[Uniq]{uniq}
\newvalue[Len]{len}



%% Types %%


\newtype{Unit}
\newtype{Bool}
\newtype{Nat}
\newtype{Int}
\newtype{String}
\newtype[Reference]{Ref}
\newtype{Task}
\newtype{Maybe}
\newtype{List}

\newtype{Euro}



%% Object language %%%%%%%%%%%%%%%%%%%%%%%%%%%%%%%%%%%%%%%%%%%%%%%%%%%%%%%%%%%%%


\let\And\relax
\newoperator{Then}  {\blacktriangleright}
\newoperator{Next}  {\vartriangleright}
\newoperator{And}   {\Join}
\newoperator{Or}    {\blacklozenge}
\newoperator{Xor}   {\lozenge}
\newoperator{Edit}  {\square}
\newoperator{View}  {\overline{\square}}
\newoperator{Enter} {\boxtimes}
\newoperator{Update}{\blacksquare}
\newoperator{Watch} {\overline{\blacksquare}}
\newoperator{Fail}  {\lightning}
\newoperator{At}    {@}

\newoperator{AndOr} {\DEPRECATED}



%% Events %%


\newvalue[Left]   {L}
\newvalue[Right]  {R}


\newvalue[Empty]   {E}
\newvalue[Continue]{C}
\newvalue[Pick]    {P}


\newvalue[First]  {F}
\newvalue[Second] {S}
\newvalue[Here]   {H}



%% Semantic functions %%%%%%%%%%%%%%%%%%%%%%%%%%%%%%%%%%%%%%%%%%%%%%%%%%%%%%%%%%


\newmathcommand{eval}[rel]
  {\;\downarrow\;}
\newmathcommand{stride}[rel]
  % {\;\rightarrow\!\shortmid\;}
  % {\;\rightsquigarrow\;}
  {\;\mapsto\;}
\newmathcommand{normalise}[rel]
  {\;\Downarrow\;}
\newmacro{handle}[1]
  {\mathrel{\;\xrightarrow{#1}\;}}
\newmacro{interact}[1]
  {\mathrel{\;\xRightarrow{#1}\;}}
\newmacro{execute}[1]
  {\mathrel{\;\xRightarrow{#1}\!\!^*\;}}

\newmathcommand{Leadsto}[rel]
  {\raisebox{-0.25ex}{$\leadsto$}\mathllap{\raisebox{+0.25ex}{$\leadsto$}}}

\newmathcommand{simeval}[rel]
  {\;\rotatebox[origin=c]{-90}{$\leadsto$}\;}
\newmathcommand{simstride}[rel]
  {\;\mapstochar\kern+0.1em\leadsto\;}
\newmathcommand{simnormalise}[rel]
  {\;\rotatebox[origin=c]{-90}{$\Leadsto$}\;}
\newmathcommand{simhandle}[rel]
  {\;\leadsto\;}
\newmathcommand{siminteract}[rel]
  {\;\Leadsto\;}
\newmathcommand{simulate}[rel]
  {\;\Leadsto^*\;}
  % {\;\raisebox{-0.4ex}{$\leadsto$}\mathllap{\raisebox{+0.4ex}{$\leadsto$}}\kern-0.4ex\circ
  % \approx\!\!>\!\!\circ\;}
  % {\;\bullet\!\!\!\Rightarrow\;\wr\!\!\!\Rightarrow\;}
  % {\;\Longmapsto\;}
  % {\;\wr\!\!\!\Rightarrow\;}

\newmathcommand{Consistent}[rel]
  {\leftrightarrows}

\newmathcommand{Simulate}[it]
  {simulate}
\newmathcommand{Evaluate}[it]
  {evaluate}
\newmathcommand{Again}[it]
  {again}

\newmathcommand{Value}[cal]
  {V}
\newmathcommand{Inputs}[cal]
  {I}
\newmathcommand{Interface}[cal]
  {U}
\newmathcommand{Failing}[cal]
  {F}
\newmathcommand{Watching}[cal]
  {W}
\newmathcommand{Dirty}
  {\Delta}
\newmathcommand{UserInterface}[cal]
  {U}
\newmathcommand{Sat}[cal]
  {S}
\newmathcommand{Hints}[cal]
  {H}


%% Proofs %%%%%%%%%%%%%%%%%%%%%%%%%%%%%%%%%%%%%%%%%%%%%%%%%%%%%%%%%%%%%%%%%%%%%%


% \newcommand{\case}[2]{
%     \noindent\textbf{Case} #1\\
%     \vspace{5mm}
%     \indent\begin{minipage}{\dimexpr\textwidth-3cm}
%     #2
%   \end{minipage}\\\\}

\newcommand{\Case}[2]{
  \bigskip
  \noindent\textbf{Case} #1
  \nopagebreak[4]
  \smallskip
  \par
  \begingroup
    \leftskip\parindent
    \noindent
    #2
    \par
  \endgroup
}

% \newcommand{\case}[2]{
%   \noindent
%   \begin{tabular*}{\textwidth}{lp{0.8\textwidth}}
%     \textbf{Case} & #1 \\
%     \addlinespace
%     & #2
%   \end{tabular*}
%   \medskip
% }



%% Depricated %%%%%%%%%%%%%%%%%%%%%%%%%%%%%%%%%%%%%%%%%%%%%%%%%%%%%%%%%%%%%%%%%%

% !TEX root=main.tex


%% Typing %%%%%%%%%%%%%%%%%%%%%%%%%%%%%%%%%%%%%%%%%%%%%%%%%%%%%%%%%%%%%%%%%%%%%%

\newrule{T-Sym}
  {s:\tau \in \Gamma}
  {\Gamma,\Sigma \infers s:\tau}


%% Evaluation %%%%%%%%%%%%%%%%%%%%%%%%%%%%%%%%%%%%%%%%%%%%%%%%%%%%%%%%%%%%%%%%%%


\newmacro{RelationSE}
  {\tilde{e},\tilde{\sigma} \tilde{\eval} \overline{\tilde{v},\tilde{\sigma}',\phi}}


\newrule{SE-Value}
  {}
  {\tilde{v},\tilde{\sigma}\tilde{\eval} \tilde{v},{\tilde{\sigma},\True}}


\newrule{SE-App}
  {\tilde{e}_1,\tilde{\sigma}\tilde{\eval} \overline{\lambda x:\tau.\tilde{e}_1',\tilde{\sigma}',{\phi_1}} \Quad
   \tilde{e}_2,\tilde{\sigma}'\tilde{\eval} \overline{\tilde{v}_2,\tilde{\sigma}'',{\phi_2}} \Quad
   \tilde{e}_1'[x\mapsto \tilde{v}_2],\tilde{\sigma}''\tilde{\eval} \overline{\tilde{v}_1,\tilde{\sigma}''',{\phi_3}}}
  {\tilde{e}_1 \tilde{e}_2,\tilde{\sigma} \tilde{\eval} \overline{\tilde{v}_1,\tilde{\sigma}''',{\phi_1\land\phi_2\land\phi_3}}}

\newrule{SE-If}
  {\tilde{e}_1,\tilde{\sigma}\tilde{\eval} \overline{\tilde{v}_1,\tilde{\sigma}',{\phi_1}} \Quad
   {\tilde{e}_2,\tilde{\sigma}'\tilde{\eval} \overline{\tilde{v}_2,\tilde{\sigma}'',\phi_2}} \Quad
   {\tilde{e}_3,\tilde{\sigma}'\tilde{\eval} \overline{\tilde{v}_3,\tilde{\sigma}''',\phi_3}}}
  {\If{\tilde{e}_1}{\tilde{e}_2}{\tilde{e}_3},\tilde{\sigma}\tilde{\eval} {\overline{\tilde{v}_2,\tilde{\sigma}'',\phi_1 \land \phi_2\land \tilde{v}_1} \cup \overline{\tilde{v}_3,\tilde{\sigma}''',\phi_1 \land \phi_3 \land \lnot \tilde{v}_1}}}

\newrule{SE-Pair}
  {\tilde{e}_1,\tilde{\sigma}\tilde{\eval} \overline{\tilde{v}_1,\tilde{\sigma}',{\phi_1}} \Quad
   \tilde{e}_2,\tilde{\sigma}'\tilde{\eval} \overline{\tilde{v}_2,\tilde{\sigma}'',{\phi_2}}}
  {\tuple{\tilde{e}_1,\tilde{e}_2},\tilde{\sigma}\tilde{\eval}\overline{\tuple{\tilde{v}_1,\tilde{v}_2},\tilde{\sigma}'',{\phi_1\land\phi_2}}}

\newrule{SE-First}
  {\tilde{e}_1,\tilde{\sigma}\tilde{\eval}\overline{\tilde{v}_1,\tilde{\sigma}',{\phi}}}
  {\Fst\tuple{\tilde{e}_1,\tilde{e}_2},\tilde{\sigma}\tilde{\eval}\overline{\tilde{v}_1,\tilde{\sigma}',{\phi}} }

\newrule{SE-Second}
  {\tilde{e}_2,\tilde{\sigma}\tilde{\eval}\overline{\tilde{v}_2,\tilde{\sigma}',{\phi}}}
  {\Snd\tuple{\tilde{e}_1,\tilde{e}_2},\tilde{\sigma}\tilde{\eval}\overline{\tilde{v}_2,\tilde{\sigma}',\phi} }


%%%%%%%

\newrule{SE-Cons}
  {\tilde{e}_1,\tilde{\sigma} \tilde{\eval} \tilde{v}_1,\tilde{\sigma}',\phi_1\Quad
   \tilde{e}_2,\tilde{\sigma}' \tilde{\eval} \tilde{v}_2,\tilde{\sigma}'',\phi_2}
  {\tilde{e}_1 :: \tilde{e}_2,\tilde{\sigma} \tilde{\eval} \tilde{v}_1:: \tilde{v}_2,\tilde{\sigma}'',\phi_1\land\phi_2}

\newrule{SE-Head}
  {\tilde{e},\tilde{\sigma} \tilde{\eval} \tilde{v}_1::\tilde{v}_2,\tilde{\sigma}',{\phi}}
  {\Head \tilde{e},\tilde{\sigma} \tilde{\eval} \tilde{v}_1,\tilde{\sigma}',{\phi}}

\newrule{SE-Tail}
{\tilde{e},\tilde{\sigma} \tilde{\eval} \tilde{v}_1::\tilde{v}_2,\tilde{\sigma}',{\phi}}
{\Tail \tilde{e},\tilde{\sigma} \tilde{\eval} \tilde{v}_2,\tilde{\sigma}',{\phi}}


%%%%%
\newrule{SE-Ref}
  {\tilde{e},\tilde{\sigma}\tilde{\eval} \overline{\tilde{v},\tilde{\sigma}',\phi} \Quad
   l\not\in Dom(\sigma')}
  {\Ref \tilde{e},\tilde{\sigma}\tilde{\eval} \overline{l,\tilde{\sigma}'[l\mapsto \tilde{v}],\phi}}

\newrule{SE-Deref}
  {\tilde{e},\tilde{\sigma}\tilde{\eval} \overline{l,\tilde{\sigma}',{\phi}}}
  {!\tilde{e},\tilde{\sigma}\tilde{\eval} \overline{\tilde{\sigma}'(l),\tilde{\sigma}',{\phi}}}

\newrule{SE-Assign}
  {\tilde{e}_1,\tilde{\sigma}\tilde{\eval} \overline{l,\tilde{\sigma}',\phi_1} \Quad
   \tilde{e}_2,\tilde{\sigma}'\tilde{\eval} \overline{\tilde{v}_2,\tilde{\sigma}'',\phi_2}}
  {\tilde{e}_1:=\tilde{e}_2,\tilde{\sigma}\tilde{\eval} \overline{\unit,\tilde{\sigma}''[l\mapsto \tilde{v}_2],\phi_1\wedge\phi_2}}

\newrule{SE-Edit}
  {\tilde{e},\tilde{\sigma} \tilde{\eval} \overline{\tilde{v},\tilde{\sigma}',\phi}}
  {\Edit \tilde{e} , \tilde{\sigma}\tilde{\eval} \overline{\Edit \tilde{v},\tilde{\sigma}',\phi}}

\newrule{SE-Enter}
  {}
  {\Enter \tau,\tilde{\sigma} \tilde{\eval} \Enter \tau,\tilde{\sigma},\True}

\newrule{SE-Update}
  {\tilde{e},\tilde{\sigma}\tilde{\eval} \overline{l,\tilde{\sigma}',\phi}}
  {\Update \tilde{e} ,\tilde{\sigma}\tilde{\eval} \overline{\Update l,\tilde{\sigma}',\phi}}


\newrule{SE-Fail}
  {}
  {\Fail,\tilde{\sigma} \tilde{\eval} \Fail,\tilde{\sigma},{\True}}


\newrule{SE-Then}
  {\tilde{e}_1 ,\tilde{\sigma}\tilde{\eval} \overline{\tilde{t}_1,\tilde{\sigma}',{\phi}}}
  {\tilde{e}_1 \Then \tilde{e}_2,\tilde{\sigma} \tilde{\eval} \overline{\tilde{t}_1 \Then \tilde{e}_2,\tilde{\sigma}',{\phi}}}

\newrule{SE-Next}
  {\tilde{e}_1 ,\tilde{\sigma}\tilde{\eval} \overline{\tilde{t}_1,\tilde{\sigma}',{\phi}}}
  {\tilde{e}_1 \Next \tilde{e}_2 ,\tilde{\sigma}\tilde{\eval} \overline{\tilde{t}_1 \Next \tilde{e}_2,\tilde{\sigma}',{\phi}}}


\newrule{SE-And}
  {\tilde{e}_1 ,\tilde{\sigma}\tilde{\eval} \overline{\tilde{t}_1 ,\tilde{\sigma}',\phi_1} \Quad
   \tilde{e}_2 ,\tilde{\sigma}'\tilde{\eval} \overline{\tilde{t}_2,\tilde{\sigma}'',\phi_2}}
  {\tilde{e}_1 \And \tilde{e}_2 ,\tilde{\sigma}\tilde{\eval} \overline{\tilde{t}_1 \And \tilde{t}_2,\tilde{\sigma}'',\phi_1\land\phi_2}}


\newrule{SE-Or}
  {\tilde{e}_1 ,\tilde{\sigma}\tilde{\eval} \overline{\tilde{t}_1 ,\tilde{\sigma}',\phi_1} \Quad
   \tilde{e}_2 ,\tilde{\sigma}'\tilde{\eval} \overline{\tilde{t}_2,\tilde{\sigma}'',\phi_2}}
  {\tilde{e}_1 \Or \tilde{e}_2 ,\tilde{\sigma}\tilde{\eval} \overline{\tilde{t}_1 \Or \tilde{t}_2,\tilde{\sigma}'',\phi_1\land\phi_2}}

\newrule{SE-Xor}
  {}
  {\tilde{e}_1 \Xor \tilde{e}_2 ,\tilde{\sigma}\tilde{\eval} \tilde{e}_1 \Xor \tilde{e}_2,\tilde{\sigma},{\True}}

%% Normalisation %%%%%%%%%%%%%%%%%%%%%%%%%%%%%%%%%%%%%%%%%%%%%%%%%%%%%%%%%%%%%%%


\newmacro{RelationSS}
  {\tilde{t},\tilde{\sigma}\tilde{\stride} \overline{\tilde{t}',\tilde{\sigma}',\phi}}


\newrule{SS-Edit}
  { }
  {\Edit \tilde{v},\tilde{\sigma} \tilde{\stride} \Edit \tilde{v},\tilde{\sigma},\True}

\newrule{SS-Fill}
  { }
  {\Enter \tau,\tilde{\sigma} \tilde{\stride} \Enter \tau,\tilde{\sigma},\True}

\newrule{SS-Update}
  { }
  {\Update l,\tilde{\sigma} \tilde{\stride} \Update l,\tilde{\sigma},\True}


\newrule{SS-Fail}
  { }
  {\Fail,\tilde{\sigma} \tilde{\stride} \Fail,\tilde{\sigma},\True}


\newrule{SS-ThenStay}
  {\tilde{t}_1,\tilde{\sigma} \tilde{\stride} \overline{\tilde{t}_1',\tilde{\sigma}',\phi}}
  {\tilde{t}_1 \Then \tilde{e}_2,\tilde{\sigma} \tilde{\stride} \overline{\tilde{t}_1' \Then \tilde{e}_2,\tilde{\sigma}',\phi}}
  [\Value(\tilde{t}_1',\tilde{\sigma}') = \bot]

\newrule{SS-ThenFail}
  {\tilde{t}_1,\tilde{\sigma} \tilde{\stride} \overline{\tilde{t}_1',\tilde{\sigma}',\phi} \Quad
   \tilde{e}_2\ \tilde{v}_1,\tilde{\sigma}' \tilde{\eval} \overline{\tilde{t}_2,\tilde{\sigma}'',\_}}
  {\tilde{t}_1 \Then \tilde{e}_2,\tilde{\sigma} \tilde{\stride} \overline{\tilde{t}_1' \Then \tilde{e}_2,\tilde{\sigma}',\phi}}
  [\Value(\tilde{t}_1',\tilde{\sigma}') = \tilde{v}_1 \land \Failing(\tilde{t}_2,\tilde{\sigma}'')]

\newrule{SS-ThenCont}
  {\tilde{t}_1,\tilde{\sigma} \tilde{\stride} \overline{\tilde{t}_1',\tilde{\sigma}',\phi_1} \Quad
  \tilde{e}_2\ \tilde{v}_1,\tilde{\sigma}' \tilde{\eval} \overline{\tilde{t}_2 ,\tilde{\sigma}'',\phi_2}}
   % t_2,\sigma'' \stride t_2',\sigma'''}
  {\tilde{t}_1 \Then \tilde{e}_2,\tilde{\sigma} \tilde{\stride} \overline{t_2,\sigma'',{\phi_1\land\phi_2}}}
  [\Value(\tilde{t}_1',\tilde{\sigma}') = \tilde{v}_1 \land \lnot\Failing(\tilde{t}_2,\tilde{\sigma}'')]

\newrule{SS-Next}
  {\tilde{t}_1,\tilde{\sigma} \tilde{\stride} \overline{\tilde{t}_1',\tilde{\sigma}',\phi}}
  {\tilde{t}_1 \Next \tilde{e}_2,\tilde{\sigma} \tilde{\stride} \overline{\tilde{t}_1' \Next \tilde{e}_2,\tilde{\sigma}',\phi}}


\newrule{SS-And}
  {\tilde{t}_1,\tilde{\sigma}  \tilde{\stride} \overline{\tilde{t}_1',\tilde{\sigma}',\phi_1 } \Quad
   \tilde{t}_2,\tilde{\sigma}' \tilde{\stride} \overline{\tilde{t}_2',\tilde{\sigma}'',\phi_2}}
  {\tilde{t}_1 \And \tilde{t}_2,\tilde{\sigma} \tilde{\stride} \overline{\tilde{t}_1' \And \tilde{t}_2',\tilde{\sigma}'',\phi_1\land\phi_2}}


\newrule{SS-OrLeft}
 {\tilde{t}_1,\tilde{\sigma}  \tilde{\stride} \overline{\tilde{t}_1',\tilde{\sigma}',\phi}}
 {\tilde{t}_1 \Or \tilde{t}_2,\tilde{\sigma} \tilde{\stride} \overline{\tilde{t}_1',\tilde{\sigma}',\phi}}
  [\Value(\tilde{t}_1',\tilde{\sigma}') = \tilde{v}_1]

\newrule{SS-OrRight}
  {\tilde{t}_1,\tilde{\sigma}  \tilde{\stride} \overline{\tilde{t}_1',\tilde{\sigma}',\phi_1}  \Quad
   \tilde{t}_2,\tilde{\sigma}' \tilde{\stride} \overline{\tilde{t}_2',\tilde{\sigma}'',\phi_2}}
  {\tilde{t}_1 \Or \tilde{t}_2,\tilde{\sigma} \tilde{\stride} \overline{\tilde{t}_2',\tilde{\sigma}'',\phi_1\land\phi_2}}
  [\Value(\tilde{t}_1',\tilde{\sigma}') = \bot \land \Value(\tilde{t}_2',\tilde{\sigma}'') = \tilde{v}_2]

\newrule{SS-OrNone}
  {\tilde{t}_1,\tilde{\sigma}  \tilde{\stride} \overline{\tilde{t}_1',\tilde{\sigma}' ,\phi_1} \Quad
   \tilde{t}_2,\tilde{\sigma}' \tilde{\stride} \overline{\tilde{t}_2',\tilde{\sigma}'',\phi_2}}
  {\tilde{t}_1 \Or \tilde{t}_2,\tilde{\sigma} \tilde{\stride} \overline{\tilde{t}_1' \Or \tilde{t}_2',\tilde{\sigma}'',\phi_1\land\phi_2}}
  [\Value(\tilde{t}_1',\tilde{\sigma}') = \bot \land \Value(\tilde{t}_2',\tilde{\sigma}'') = \bot]


\newrule{SS-Xor}
  {\ }
  {\tilde{e}_1 \Xor \tilde{e}_2,\tilde{\sigma} \tilde{\stride} \tilde{e}_1 \Xor \tilde{e}_2,\tilde{\sigma},\True}

\newrule{SS-Eval}
    {\tilde{e},\tilde{\sigma} \tilde{\eval} \overline{\tilde{e}',\tilde{\sigma}',\phi_1}  \Quad
     \tilde{e}',\tilde{\sigma}' \tilde{\stride} \overline{\tilde{e}'',\tilde{\sigma}'',\phi_2}}
    {\tilde{e},\tilde{\sigma} \tilde{\stride} \overline{\tilde{e}'',\tilde{\sigma}'',\phi_1\land\phi_2}}
    [\tilde{e} \neq \tilde{e}']

%% Normalisation %%


\newmacro{RelationSN}
  {\tilde{e},\tilde{\sigma} \tilde{\normalise} \overline{\tilde{t},\tilde{\sigma}',\phi}}


\newrule{SN-Done}
    {\tilde{e},\tilde{\sigma} \tilde{\eval} \overline{\tilde{t},\tilde{\sigma}',\phi_1}  \Quad
     \tilde{t},\tilde{\sigma}' \tilde{\stride} \overline{\tilde{t}',\tilde{\sigma}'',\phi_2}}
    {\tilde{e},\tilde{\sigma} \tilde{\normalise} \overline{\tilde{t},\tilde{\sigma}',\phi_1}}
    [\tilde{\sigma}'=\tilde{\sigma}'' \land \tilde{t}=\tilde{t}']

\newrule{SN-Repeat}
    {\tilde{e},\tilde{\sigma} \tilde{\eval} \overline{\tilde{t},\tilde{\sigma}',\phi_1}  \Quad
     \tilde{t},\tilde{\sigma}' \tilde{\stride} \overline{\tilde{t}',\tilde{\sigma}'',\phi_2 } \Quad
     \tilde{t}',\tilde{\sigma}'' \tilde{\normalise} \overline{\tilde{t}'',\tilde{\sigma}''',\phi_3}}
    {\tilde{e},\tilde{\sigma} \tilde{\normalise} \overline{\tilde{t}'',\tilde{\sigma}''',\phi_1 \land \phi_2 \land \phi_3}}
    [\tilde{\sigma}'\neq \tilde{\sigma}''\vee \tilde{t}\neq \tilde{t}']



%% Handling %%


\newmacro{RelationSH}
  {\tilde{t},\tilde{\sigma} \handle{} \overline{\tilde{t}',\tilde{\sigma}',\tilde{\imath},\phi}}


\newrule{SH-Change}
  { \text{fresh }s}
  {\Edit \tilde{v},\tilde{\sigma} \handle{} \Edit s,\tilde{\sigma},s,\True}
  [\tilde{v},s:\tau]

% \newrule{SH-Empty}
%   { }
%   {\Edit v,\sigma \handle{\Empty} \Enter \tau,{\sigma,\True}}
%   [v : \tau]

\newrule{SH-Fill}
  { \text{fresh }\tilde{s}}
  {\Enter \tau,\tilde{\sigma} \handle{} \Edit s,\tilde{\sigma},s,\True}
  [s:\tau]

\newrule{SH-Update}
  { \text{fresh }s}
  {\Update l,\tilde{\sigma} \handle{} \Update l,\tilde{\sigma}[l \mapsto s],s,\True}
  [\sigma(l),s:\tau]

\newrule{SH-PassThen}
  {\tilde{t}_1,\tilde{\sigma} \handle{} \overline{\tilde{t}_1',\tilde{\sigma}',\tilde{\imath},\phi}}
  {\tilde{t}_1 \Then \tilde{e}_2,\tilde{\sigma} \handle{} \overline{\tilde{t}_1' \Then \tilde{e}_2,\tilde{\sigma}',\tilde{\imath},\phi}}

\newrule{SH-PassNext}
  {\tilde{t}_1,\tilde{\sigma} \handle{} \overline{\tilde{t}_1',\tilde{\sigma}',\tilde{\imath},\phi}}
  {\tilde{t}_1 \Next \tilde{e}_2,\tilde{\sigma} \handle{} \overline{\tilde{t}_1' \Next \tilde{e}_2,\tilde{\sigma}',\tilde{\imath},\phi}}
  [\Value{(\tilde{t}_1,\tilde{\sigma})} = \bot]

\newrule{SH-PassNextFail}
  {\tilde{t}_1,\tilde{\sigma} \handle{} \overline{\tilde{t}_1',\tilde{\sigma}_1',\tilde{\imath},\phi} \Quad
  \tilde{e}_2\ \tilde{v}_1,\tilde{\sigma} \tilde{\normalise} \overline{\tilde{t}_2,\tilde{\sigma}_2',{\vphantom{i}\_}}}
  {\tilde{t}_1 \Next \tilde{e}_2,\tilde{\sigma} \handle{} \overline{\tilde{t}_1' \Next \tilde{e}_2,\tilde{\sigma}_1',\tilde{\imath},\phi}}
  [\Value{(\tilde{t}_1,\tilde{\sigma})} = \tilde{v}_1 \land \Failing{(\tilde{t}_2,\tilde{\sigma}_2')}]

\newrule{SH-Next}
  {\tilde{t}_1,\tilde{\sigma} \handle{} \overline{\tilde{t}_1',\tilde{\sigma}_1',\tilde{\imath},\phi_1} \Quad
  \tilde{e}_2\ \tilde{v}_1,\tilde{\sigma} \tilde{\normalise} \overline{\tilde{t}_2,\tilde{\sigma}_2',\phi_2}}
  {\tilde{t}_1 \Next \tilde{e}_2,\tilde{\sigma} \handle{} {\overline{\tilde{t}_1' \Next \tilde{e}_2,\tilde{\sigma}_1',\tilde{\imath},\phi_1}\cup\overline{\tilde{t}_2,\tilde{\sigma}_2',\Continue,\phi_2}}}
  [\Value{(\tilde{t}_1,\tilde{\sigma})} = \tilde{v}_1 \land \neg\Failing{(\tilde{t}_2,\tilde{\sigma}')}]


\newrule{SH-And}
  {\tilde{t}_1,\tilde{\sigma} \handle{} \overline{\tilde{t}_1',\tilde{\sigma}_1',\tilde{\imath}_1,\phi_1} \Quad
   \tilde{t}_2,\tilde{\sigma} \handle{} \overline{\tilde{t}_2',\tilde{\sigma}_2',\tilde{\imath}_2,\phi_2}}
  {\tilde{t}_1 \And \tilde{t}_2,\tilde{\sigma} \handle{} {\overline{\tilde{t}_1' \And \tilde{t}_2,\tilde{\sigma}_1',\First \tilde{\imath}_1,\phi_1}\cup \overline{\tilde{t}_1 \And \tilde{t}_2',\tilde{\sigma}_2'',\Second \tilde{\imath}_2,\phi_2}}}

\newrule{SH-Or}
  {\tilde{t}_1,\tilde{\sigma} \handle{} \overline{\tilde{t}_1',\tilde{\sigma}_1',\tilde{\imath}_1,\phi_1}\Quad
  \tilde{t}_2,\tilde{\sigma} \handle{} \overline{\tilde{t}_2',\tilde{\sigma}_2',\tilde{\imath}_2,\phi_2}}
  {\tilde{t}_1 \Or \tilde{t}_2,\tilde{\sigma} \handle{} {\overline{\tilde{t}_1' \Or \tilde{t}_2,\tilde{\sigma}_1',\First \tilde{\imath}_1,\phi_1}\cup\overline{\tilde{t}_1 \Or \tilde{t}_2',\tilde{\sigma}_2',\Second \tilde{\imath}_2,\phi_2}}}


\newrule{SH-PickLeft}
  {\tilde{e}_1,\tilde{\sigma}\tilde{\normalise} \overline{\tilde{t}_1,\tilde{\sigma}_1,\phi_1} \Quad
   \tilde{e}_2,\tilde{\sigma} \tilde{\normalise} \overline{\tilde{t}_2,\tilde{\sigma}_2,\phi_2}}
  {\tilde{e}_1 \Xor \tilde{e}_2,\tilde{\sigma} \handle{} \tilde{t}_1,\tilde{\sigma}_1,\Left,\phi_1}
  [\neg\Failing(\tilde{t}_1,\tilde{\sigma}_1) \land \Failing(\tilde{t}_2,\tilde{\sigma}_2)]

\newrule{SH-PickRight}
  {\tilde{e}_1,\tilde{\sigma} \tilde{\normalise} \overline{\tilde{t}_1,\tilde{\sigma}_1,\phi_1} \Quad
   \tilde{e}_2,\tilde{\sigma} \tilde{\normalise} \overline{\tilde{t}_2,\tilde{\sigma}_2,\phi_2}}
  {\tilde{e}_1 \Xor \tilde{e}_2,\tilde{\sigma} \handle{} \tilde{t}_2,\tilde{\sigma}_2,\Right,\phi_2}
  [\Failing(\tilde{t}_1,\tilde{\sigma}_1) \land \neg\Failing(\tilde{t}_2,\tilde{\sigma}_2)]

\newrule{SH-Pick}
  {\tilde{e}_1,\tilde{\sigma} \normalise \overline{\tilde{t}_1,\tilde{\sigma}_1,\phi_1} \Quad
   \tilde{e}_2,\tilde{\sigma} \normalise \overline{\tilde{t}_2,\tilde{\sigma}_2,\phi_2}}
  {\tilde{e}_1 \Xor \tilde{e}_2,\tilde{\sigma} \handle{} {\overline{\tilde{t}_1,\tilde{\sigma}_1,\Left,\phi_1}\cup\overline{\tilde{t}_2,\tilde{\sigma}_2,\Right,\phi_2}}}
  [\neg\Failing(\tilde{t}_1,\tilde{\sigma}_1) \land \neg\Failing(\tilde{t}_2,\tilde{\sigma}_2)]


%% Driving %%


\newmacro{RelationSI}
  {\tilde{t},\tilde{\sigma} \drive{} \overline{\tilde{t}',\tilde{\sigma}',\tilde{\imath},\phi}}


\newrule{SI-Handle}
  {\tilde{t},\tilde{\sigma} \handle{} \overline{\tilde{t}',\tilde{\sigma}',{\tilde{\imath},\phi_1}} \Quad
   \tilde{t}',\tilde{\sigma}' \tilde{\normalise} \overline{\tilde{t}'',\tilde{\sigma}'',\phi_2}}
  {\tilde{t},\tilde{\sigma} \drive{} \overline{\tilde{t}'',\tilde{\sigma}'',\tilde{\imath},\phi_1 \land \phi_2}}

%% Firsts %%

\newrule{R-Firsts}
  {t,\sigma\drive{}^*\overline{\tilde{v},\tilde{\imath}:\tilde{is},\Phi}}
  {\Firsts(t,\sigma,g) = \overline{\tilde{\imath},\Phi\land g \tilde{v}}}
  [\Sat(\Phi\land g\tilde{v})]

% !TEX root=main.tex


%% Typing %%%%%%%%%%%%%%%%%%%%%%%%%%%%%%%%%%%%%%%%%%%%%%%%%%%%%%%%%%%%%%%%%%%%%%


\newmacro{RelationT}
  {\Gamma,\Sigma \infers e : \tau}


\newrule{T-ConstBool}
  {c\in B}
  {\Gamma,\Sigma\infers c : \Bool}

\newrule{T-ConstInt}
  {c\in I}
  {\Gamma,\Sigma\infers c : \Int}

\newrule{T-ConstString}
  {c\in S}
  {\Gamma,\Sigma\infers c : \String}


\newrule{T-Unit}
  { }
  {\Gamma,\Sigma\infers \unit : \Unit}


\newrule{T-Var}
  {x:\tau\in\Gamma}
  {\Gamma,\Sigma\infers x:\tau}


\newrule{T-Abs}
  {\Gamma[x:\tau_1] ,\Sigma \infers e:\tau_2}
  {\Gamma,\Sigma \infers \lambda x : \tau_1 . e :\tau_1 \to \tau_2}

\newrule{T-App}
  {\Gamma,\Sigma \infers e_1:\tau_1\to\tau_2 \Quad
   \Gamma,\Sigma \infers e_2:\tau_1}
  {\Gamma,\Sigma \infers e_1 e_2 :\tau_2}


\newrule{T-If}
  {\Gamma,\Sigma \infers e_1:\Bool \Quad
   \Gamma,\Sigma \infers e_2:\tau \Quad
   \Gamma,\Sigma \infers e_3:\tau}
  {\Gamma,\Sigma \infers \If{e_1}{e_2}{e_3}:\tau}


\newrule{T-Pair}
    {\Gamma,\Sigma \infers e_1 : \tau_1  \Quad
     \Gamma,\Sigma \infers e_2 : \tau_2}
    {\Gamma,\Sigma \infers \tuple{e_1, e_2} :\tau_1 \times \tau_2}

\newrule{T-First}
  {\Gamma,\Sigma\infers e_1:\tau}
  {\Gamma,\Sigma\infers \Fst \tuple{e_1,e_2}:\tau}

  \newrule{T-Second}
    {\Gamma,\Sigma\infers e_2:\tau}
    {\Gamma,\Sigma\infers \Snd \tuple{e_1,e_2}:\tau}

%%%%%
\newrule{T-ListEmpty}
  { }
  {\Gamma,\Sigma\infers [\ ]_\beta : \List\beta}

\newrule{T-ListCons}
  {\Gamma,\Sigma\infers e_1:\beta \Quad
   \Gamma,\Sigma\infers e_2:\List\beta}
  {\Gamma,\Sigma\infers e_1 :: e_2 : \List \beta}

\newrule{T-ListHead}
  {\Gamma,\Sigma\infers e:\List\beta}
  {\Gamma,\Sigma\infers \Head e:\beta}

\newrule{T-ListTail}
    {\Gamma,\Sigma\infers e:\List\beta}
    {\Gamma,\Sigma\infers \Tail e:\List\beta}

%%%%%


\newrule{T-Ref}
  {\Gamma,\Sigma \infers e:\beta}
  {\Gamma,\Sigma \infers \Ref e :\Reference \beta}

\newrule{T-Deref}
  {\Gamma,\Sigma \infers e:\Reference \beta}
  {\Gamma,\Sigma\infers\ !e:\beta}

\newrule{T-Assign}
  {\Gamma,\Sigma\infers e_1:\Reference \beta \Quad
   \Gamma,\Sigma\infers e_2:\beta}
  {\Gamma,\Sigma\infers e_1 := e_2:\Unit}

\newrule{T-Loc}
  {\Sigma(l) = \beta}
  {\Gamma,\Sigma\infers l:\Reference \beta}


\newrule{T-Edit}
  {\Gamma,\Sigma \infers e : \tau}
  {\Gamma,\Sigma \infers \Edit e : \Task \tau}

\newrule{T-Enter}
  {}
  {\Gamma,\Sigma \infers \Enter \tau : \Task \tau}

\newrule{T-Update}
  {\Gamma,\Sigma \infers e : \Reference \beta}
  {\Gamma,\Sigma \infers \Update e : \Task \beta}


\newrule{T-Fail}
  {}
  {\Gamma,\Sigma \infers \Fail : \Task \tau}


\newrule{T-Then}
  {\upon{\Gamma,\Sigma \infers e_1 : \Task \tau_1}
   {\Gamma,\Sigma \infers e_2 : \tau_1 \to \Task \tau_2}}
  {\Gamma,\Sigma \infers e_1 \Then e_2 : \Task \tau_2}


\newrule{T-Next}
  {\upon{\Gamma,\Sigma \infers e_1 : \Task \tau_1}
   {\Gamma,\Sigma \infers e_2 : \tau_1 \to \Task \tau_2}}
  {\Gamma,\Sigma \infers e_1 \Next e_2 : \Task \tau_2}


\newrule{T-And}
  {\Gamma,\Sigma \infers e_1 : \Task \tau_1 \Quad
   \Gamma,\Sigma \infers e_2 : \Task \tau_2}
  {\Gamma,\Sigma \infers e_1 \And e_2 : \Task\,(\tau_1 \times \tau_2)}


\newrule{T-Or}
  {\upon{\Gamma,\Sigma \infers e_1 : \Task \tau}
   {\Gamma,\Sigma \infers e_2 : \Task \tau}}
  {\Gamma,\Sigma \infers e_1 \Or e_2 : \Task \tau}


\newrule{T-Xor}
  {\upon{\Gamma,\Sigma \infers e_1 : \Task \tau}
   {\Gamma,\Sigma \infers e_2 : \Task \tau}}
  {\Gamma,\Sigma \infers e_1 \Xor e_2 : \Task \tau}


%% Evaluation %%%%%%%%%%%%%%%%%%%%%%%%%%%%%%%%%%%%%%%%%%%%%%%%%%%%%%%%%%%%%%%%%%

\newmacro{RelationE}
  {e,\sigma \eval v,\sigma'}


\newrule{E-Value}
  {}
  {v,{\sigma}{\eval} v,{\sigma}}


\newrule{E-App}
  {e_1               ,\sigma   \eval \lambda x:\tau.e_1',\sigma'\Quad
   e_2               ,\sigma'  \eval v_2                ,\sigma''\Quad
   e_1'[x\mapsto v_2],\sigma'' \eval v_1                ,\sigma'''}
  {e_1 e_2           ,\sigma   \eval v_1                ,\sigma'''}


\newrule{E-IfTrue}
  {e_1,{\sigma}{\eval} \True,{\sigma}'\Quad
   e_2,{\sigma}'{\eval} {v_2},{\sigma}''}
  {\If{e_1}{e_2}{e_3},{\sigma}{\eval} {v_2},{\sigma}''}

\newrule{E-IfFalse}
  {e_1,{\sigma}{\eval} {v_1},{\sigma}' \Quad
   e_3,{\sigma}'{\eval} {v_3},{\sigma}''}
  {\If{e_1}{e_2}{e_3},{\sigma}{\eval} {v_3},{\sigma}''}


\newrule{E-Pair}
  {e_1,{\sigma}{\eval} {v_1},{\sigma}' \Quad
   e_2,{\sigma}'{\eval} {v_2},{\sigma}''}
  {\tuple{e_1,e_2},{\sigma}{\eval}\tuple{{v_1},{v_2}},{\sigma}''}

\newrule{E-First}
  {e_1,{\sigma}{\eval}{v_1},{\sigma}'}
  {\Fst\tuple{e_1,e_2},{\sigma}{\eval}{v_1},{\sigma}'}

\newrule{E-Second}
  {e_2,{\sigma}{\eval}{v_2},{\sigma}'}
  {\Snd\tuple{e_1,e_2},{\sigma}{\eval} {v_2},{\sigma}' }

%%%%%%%%%

\newrule{E-Cons}
  {e_1,{\sigma}{\eval}{v_1},{\sigma}'\Quad
   e_2,{\sigma}'{\eval}{v_2},{\sigma}''}
  {e_1 :: e_2,{\sigma}{\eval}{v_1}::{v_2},{\sigma}''}

\newrule{E-Head}
  {e,{\sigma}{\eval} {v_1}::{v_2},{\sigma}'}
  {\Head e,{\sigma}{\eval}{v_1},{\sigma}'}

\newrule{E-Tail}
{e,{\sigma}{\eval} {v_1}::{v_2},{\sigma}'}
{\Tail e,{\sigma}{\eval}{v_2},{\sigma}'}

%%%%%%


\newrule{E-Ref}
  {e,{\sigma}{\eval} {v},{\sigma}' \Quad
   l\not\in Dom({\sigma}')}
  {\Ref e,{\sigma}{\eval} l,{\sigma}'[l\mapsto {v}]}

\newrule{E-Deref}
  {e,{\sigma}{\eval} l,{\sigma}'}
  {!e,{\sigma}{\eval} {\sigma}'(l),{\sigma}'}

\newrule{E-Assign}
  {e_1,{\sigma}{\eval} l,{\sigma}' \Quad
   e_2,{\sigma}'{\eval} {v_2},{\sigma}''}
  {e_1:=e_2,{\sigma}{\eval} \unit,{\sigma}''[l\mapsto {v_2}]}

\newrule{E-Edit}
  {e,{\sigma} {\eval} {v},{\sigma}'}
  {\Edit e , {\sigma}{\eval} \Edit {v},{\sigma}'}

\newrule{E-Enter}
  {}
  {\Enter \tau,{\sigma} {\eval} \Enter \tau,{\sigma}}

\newrule{E-Update}
  {e,{\sigma}{\eval} l,{\sigma}'}
  {\Update e ,{\sigma}{\eval} \Update l,{\sigma}'}


\newrule{E-Fail}
  {}
  {\Fail,{\sigma} {\eval} \Fail,{\sigma}}


\newrule{E-Then}
  {e_1 ,{\sigma}{\eval} {t_1},{\sigma}'}
  {e_1 \Then e_2,{\sigma}{\eval}{t_1} \Then e_2,{\sigma}'}

\newrule{E-Next}
  {e_1 ,{\sigma}{\eval} {t_1},{\sigma}'}
  {e_1 \Next e_2 ,{\sigma}{\eval} {t_1} \Next e_2,{\sigma}'}


\newrule{E-And}
  {e_1 ,{\sigma}{\eval}{ t_1 },{\sigma}'\Quad
   e_2 ,{\sigma}'{\eval} {t_2},{\sigma}''}
  {e_1 \And e_2 ,{\sigma}{\eval}{ t_1} \And {t_2},{\sigma}''}


\newrule{E-Or}
  {e_1 ,{\sigma}{\eval}{ t_1} ,{\sigma}'\Quad
   e_2 ,{\sigma}'{\eval} {t_2},{\sigma}''}
  {e_1 \Or e_2 ,{\sigma}{\eval} {t_1} \Or {t_2},{\sigma}''}

\newrule{E-Xor}
  {}
  {e_1 \Xor e_2 ,{\sigma}{\eval} e_1 \Xor e_2,{\sigma}}


%% Normalisation %%%%%%%%%%%%%%%%%%%%%%%%%%%%%%%%%%%%%%%%%%%%%%%%%%%%%%%%%%%%%%%

\newmacro{RelationS}
  {t,{\sigma} {\stride} {t'},{\sigma}'}


\newrule{S-Edit}
  { }
  {\Edit v,{\sigma} {\stride} \Edit v,{\sigma}}

\newrule{S-Fill}
  { }
  {\Enter \tau,{\sigma} {\stride} \Enter \tau,{\sigma}}

\newrule{S-Update}
  { }
  {\Update l,{\sigma} {\stride} \Update l,{\sigma}}


\newrule{S-Fail}
  { }
  {\Fail,{\sigma} {\stride} \Fail,{\sigma}}


\newrule{S-ThenStay}
  {t_1,{\sigma} {\stride} {t_1}',{\sigma}'}
  {t_1 \Then e_2,{\sigma} {\stride} {t_1}' \Then e_2,{\sigma}'}
  [\Value({t_1}',{\sigma}') = \bot]

\newrule{S-ThenFail}
  {t_1,{\sigma} {\stride} {t_1}',{\sigma}' \Quad
   e_2\ {v_1},{\sigma}' {\eval} {t_2},{\sigma}''}
  {t_1 \Then e_2,{\sigma} {\stride} {t_1}' \Then e_2,{\sigma}'}
  [\Value({t_1}',{\sigma}') = {v_1} \land \Failing({t_2},{\sigma}'')]

\newrule{S-ThenCont}
  {t_1,{\sigma} {\stride} {t_1}',{\sigma}'  \Quad
   e_2\ {v_1},{\sigma}' {\eval} {t_2 },{\sigma}''}
  {t_1 \Then e_2,{\sigma} {\stride} {t_2},{\sigma}''}
  [\Value({t_1}',{\sigma}') = {v_1} \land \lnot\Failing({t_2},{\sigma}'')]

\newrule{S-Next}
  {t_1,{\sigma} {\stride} {t_1}',{\sigma}'}
  {t_1 \Next e_2,{\sigma} {\stride} {t_1}' \Next e_2,{\sigma}'}


\newrule{S-And}
  {t_1,{\sigma}  {\stride} {t_1}',{\sigma}'  \Quad
   t_2,{\sigma}' {\stride} {t_2}',{\sigma}''}
  {t_1 \And t_2,{\sigma} {\stride} {t_1}' \And {t_2}',{\sigma}''}


\newrule{S-OrLeft}
  {t_1,{\sigma}  {\stride} {t_1}',{\sigma}'}
  {t_1 \Or t_2,{\sigma} {\stride} {t_1}',{\sigma}'}
  [\Value({t_1}',{\sigma}') = {v_1}]

\newrule{S-OrRight}
  {t_1,{\sigma}  {\stride} {t_1}',{\sigma}'  \Quad
   t_2,{\sigma}' {\stride} {t_2}',{\sigma}''}
  {t_1 \Or t_2,{\sigma} {\stride} {t_2}',{\sigma}''}
  [\Value({t_1}',{\sigma}') = \bot \land \Value({t_2}',{\sigma}'') = {v_2}]

\newrule{S-OrNone}
  {t_1,{\sigma}  {\stride }{t_1}',{\sigma}'  \Quad
   t_2,{\sigma' }{\stride} {t_2}',{\sigma}''}
  {t_1 \Or t_2,{\sigma} {\stride} {t_1}' \Or {t_2}',{\sigma}''}
  [\Value({t_1}',{\sigma}') = \bot \land \Value({t_2}',{\sigma}'') = \bot]


\newrule{S-Xor}
  { }
  {e_1 \Xor e_2,{\sigma} {\stride} e_1 \Xor e_2,{\sigma}}

\newrule{S-Eval}
    {e,{\sigma} {\eval} {e}',{\sigma}'  \Quad
     e',{\sigma}' {\stride} {e}'',{\sigma}''}
    {e,{\sigma} {\stride} {e}'',{\sigma}''}
    [e \neq {e}']


%% Normalisation %%

\newmacro{RelationN}
  {e,{\sigma} {\normalise} {t},{\sigma}'}


\newrule{N-Done}
    {e,{\sigma} {\eval} {t},{\sigma}' \Quad
     {t},{\sigma}' {\stride} {t}',{\sigma}''}
    {e,{\sigma} {\normalise} {t},{\sigma}'}
    [{\sigma}'={\sigma}'' \land {t}={t}']

\newrule{N-Repeat}
    {e,{\sigma} {\eval} {t},{\sigma}'  \Quad
     {t},{\sigma}' {\stride} {t}',{\sigma}''  \Quad
     {t}',{\sigma}'' {\normalise} {t}'',{\sigma}'''}
    {e,{\sigma} {\normalise} {t}'',{\sigma}'''}
    [{\sigma}'\neq {\sigma}''\vee {t}\neq {t}']



%% Handling %%

\newmacro{RelationH}
  {t,{\sigma} \handle{i} {t'},{\sigma}'}


\newrule{H-Change}
  { }
  {\Edit v,{\sigma} \xrightarrow[]{v}' \Edit v',{\sigma}}
  [v,v':\tau]

\newrule{H-Empty}
  { }
  {\Edit v,\sigma \handle{\Empty} \Enter \tau,\sigma,\True}
  [v : \tau]

\newrule{H-Fill}
  { }
  {\Enter \tau,{\sigma} \xrightarrow[]{v} \Edit v,{\sigma}}
  [v:\tau]

\newrule{H-Update}
  { }
  {\Update l,{\sigma} \xrightarrow[]{v} \Update l,{\sigma}[l \mapsto v]}
  [\sigma(l),v:\tau]

\newrule{H-PassThen}
  {t_1,\sigma \xrightarrow[]{i} {t_1'},\sigma'}
  {t_1 \Then e_2,\sigma \xrightarrow[]{i} {t_1'} \Then e_2,\sigma'}

\newrule{H-PassNext}
  {t_1,\sigma \xrightarrow[]{i} {t_1'},\sigma'}
  {t_1 \Next e_2,\sigma \xrightarrow[]{i} {t_1'} \Next e_2,\sigma'}

\newrule{H-Next}
  {e_2\ {v_1},\sigma {\normalise} {t_2},{\sigma}'}
  {t_1 \Next e_2,\sigma \xrightarrow[]{\Continue} {t_2},{\sigma}'}
  [\Value{(t_1,\sigma)} = {v_1} \land \neg\Failing{({t_2},{\sigma}')}]


\newrule{H-FirstAnd}
  {t_1,\sigma \xrightarrow[]{i} {t_1}',{\sigma}'}
  {t_1 \And t_2,\sigma \xrightarrow[]{\First i} {t_1}' \And t_2,{\sigma}'}

\newrule{H-SecondAnd}
  {t_2,\sigma \xrightarrow[]{i} {t_2}',{\sigma}'}
  {t_1 \And t_2,\sigma \xrightarrow[]{\Second i} t_1 \And {t_2}',{\sigma}'}


\newrule{H-FirstOr}
  {t_1,\sigma \xrightarrow[]{i} {t_1}',{\sigma}'}
  {t_1 \Or t_2,\sigma \xrightarrow[]{\First i} {t_1}' \Or t_2,{\sigma}'}

\newrule{H-SecondOr}
  {t_2,\sigma \xrightarrow[]{i} {t_2}',{\sigma}' }
  {t_1 \Or t_2,\sigma \xrightarrow[]{\Second i} t_1 \Or {t_2}',{\sigma}'}


\newrule{H-PickLeft}
  {e_1,\sigma \normalise {t_1},{\sigma}'}
  {e_1 \Xor e_2,\sigma \xrightarrow[]{\Left} {t_1},{\sigma}'}
  [\neg\Failing({t_1},{\sigma}')]

\newrule{H-PickRight}
  {e_2,\sigma {\normalise} {t_2},{\sigma}'}
  {e_1 \Xor e_2,\sigma \xrightarrow[]{\Right} {t_2},{\sigma}'}
  [\neg\Failing({t_2},{\sigma}')]



%% Driving %%

\newmacro{RelationI}
  {{t},{\sigma} \interact{i} {t}',{\sigma}'}



\newrule{I-Handle}
  {t,\sigma \xrightarrow[]{i} {t}',{\sigma}' \Quad
   {t}',{\sigma}' {\normalise} {t}'',{\sigma}''}
  {t,\sigma \xRightarrow[]{i} {t}'',{\sigma}''}

% !TEX root=../main.tex


%% Language %%%%%%%%%%%%%%%%%%%%%%%%%%%%%%%%%%%%%%%%%%%%%%%%%%%%%%%%%%%%%%%%%%%%

\newmacro{G-Language}{
  \begin{grammar}
    Expressions
      & e    &::= & \lambda x:\tau.\ e   & – abstraction \\
      &      &\mid& e_1\ e_2             & – application \\
      &      &\mid& x                    & – variable \\
      &      &\mid& s                    & – symbol \\
      &      &\mid& c                    & – constant \\
    \addlinespace
      &      &\mid& u\ e_1               & – unary operation \\
      &      &\mid& e_1\ o\ e_2          & – binary operation \\
      &      &\mid& \If{e_1}{e_2}{e_3}   & – conditional \\
    \addlinespace
      &      &\mid& \unit                & – unit \\
      &      &\mid& \tuple{e_1, e_2}     & – pair \\
      &      &\mid& \Fst e               & – first projection \\
      &      &\mid& \Snd e               & – second projection \\
    \addlinespace
      &      &\mid& \Ref e               & – reference \\
      &      &\mid& !e                   & – dereference \\
      &      &\mid& e_1 := e_2           & – assignment \\
      % &      &\mid& e_1; e_2             & – sequence \\
      &      &\mid& l                    & – location \\
    \addlinespace
      &      &\mid& p                    & – pretask \\
    \addlinespace
    Constants
      & c    &::= & B                    & – boolean \\
      &      &\mid& I                    & – integer \\
      &      &\mid& S                    & – string \\
    \addlinespace
    Unary operations
      & u    &::= & \lnot                & – logical \\
      &      &\mid& -                    & – numerical \\
      &      &\mid& \Len                 & – sequential \\
    \addlinespace
    Binary operations
      & o    &::= & \land \mid \lor                                     & – logical \\
      &      &\mid& < \mid \le \mid \equiv \mid \nequiv \mid \ge \mid > & – equational \\
      &      &\mid& + \mid - \mid \times \mid /                         & – numerical \\
      &      &\mid& \pp                                                 & – sequential \\
  \end{grammar}
}

\newmacro{G-Language-Compact}{
  \begin{grammar}
    \noalign{Expressions}
    \addlinespace
    & e &::= & \lambda x:\tau.\ e  \Mid  e_1\ e_2                  & – abstraction, application \\
    &       & \mid  & x  \Mid  c \Mid \unit                        & – variable, constant, unit \\
    &       & \mid  & u\ e_1 \Mid e_1\ o\ e_2                      & – unary, binary operation \\
    &       & \mid  & \If{e_1}{e_2}{e_3}                           & – conditional \\
    &       & \mid  & \tuple{e_1, e_2}  \Mid  \Fst e  \Mid  \Snd e & – pair, projections \\
    &       & \mid  & [\ ]_\beta \Mid e_1 :: e_2                   & – nil, cons \\
    &       & \mid  & \Head\ e \Mid \Tail\ e                       & – first element, list tail \\
    &       & \mid  & \Ref e  \Mid  !e  \Mid  e_1 := e_2  \Mid  l  & – references, location \\
    &       & \mid  & p \Mid \highlight{s}                         & – pretask, symbol \\
    \\
    \noalign{Constants}
    \addlinespace
    & c& ::= &  B  \Mid  I  \Mid  S                                & – boolean, integer, string \\
    \\
    \noalign{Unary Operations}
    \addlinespace
    & u &::= &  \lnot \Mid - \Mid \Len \Mid \Uniq                  & – not, negate, length, unique \\
    \\
    \noalign{Binary Operations}
    \addlinespace
    & o &::= & < \Mid \le \Mid \equiv \Mid \nequiv \Mid \ge \Mid > & – equational \\
    &       & \mid  & + \Mid - \Mid \times \Mid /                  & – numerical \\
    &       & \mid  & \pp  \Mid \in                                & – append, elementhood \\
  \end{grammar}
}

\newmacro{G-Language-Flat}{
  \begin{grammar}
    \noalign{Expressions}
    \addlinespace
    & e &::= & \lambda x:\tau.\ e  \Mid  e_1\ e_2 \Mid x \Mid c  \Mid \unit               & – abstraction, application, variable, constant, unit \\
    &       & \mid  & u\ e_1 \Mid e_1\ o\ e_2 \Mid \If{e_1}{e_2}{e_3}                    & – unary, binary operation, conditional \\
    &       & \mid  & \tuple{e_1, e_2}  \Mid  \Fst e  \Mid  \Snd e \Mid [\ ]_\beta \Mid e_1 :: e_2 & – pair, projections,  nil, cons\\
    &       & \mid  & \Head\ e \Mid \Tail\ e, p                      & – first element, list tail, pretask \\
    &       & \mid  & \Ref e  \Mid  !e  \Mid  e_1 := e_2  \Mid  l  & – reference, dereference, assignment, location
  \end{grammar}
  \begin{grammar}
    Constants & c& ::= &  B  \Mid  I  \Mid  S                                & – boolean, integer, string \\
    Unary Operations & u &::= &  \lnot \Mid - \Mid \Len \Mid \Uniq                  & – not, negate, length, unique \\
    Binary Operations  & o &::= & < \Mid \le \Mid \equiv \Mid \nequiv \Mid \ge \Mid > & – equational \\
    &       & \mid  & + \Mid - \Mid \times \Mid /                  & – numerical \\
    &       & \mid  & \pp  \Mid \in                                & – append, elementhood
  \end{grammar}
}

\newmacro{G-SExpressions-Compact}{
  \begin{grammar}
    \noalign{Expressions}
    \addlinespace
    & \tilde{e} &::= & \lambda x:\tau.\ \tilde{e}  \Mid  \tilde{e}_1\ \tilde{e}_2                  & – abstraction, application \\
    &       & \mid  & x  \Mid  c \Mid \unit                        & – variable, constant, unit \\
    &       & \mid  & u\ \tilde{e}_1 \Mid \tilde{e}_1\ o\ \tilde{e}_2                      & – unary, binary operation \\
    &       & \mid  & \If{\tilde{e}_1}{\tilde{e}_2}{\tilde{e}_3}                           & – conditional \\
    &       & \mid  & \tuple{\tilde{e}_1, \tilde{e}_2}  \Mid  \Fst \tilde{e}  \Mid  \Snd \tilde{e} & – pair, projections \\
    &       & \mid  & [\ ]_\beta \Mid \tilde{e}_1 :: \tilde{e}_2                   & – nil, cons \\
    &       & \mid  & \Head\ \tilde{e} \Mid \Tail\ \tilde{e}                       & – first element, list tail \\
    &       & \mid  & \Ref \tilde{e}  \Mid  !\tilde{e}  \Mid  \tilde{e}_1 := \tilde{e}_2  \Mid  l  & – references, location \\
    &       & \mid  & \tilde{p} \Mid s                         & – pretask, symbol \\
  \end{grammar}
}

\newmacro{G-Pretasks}{
  \begin{grammar}
    Pretasks
      & p    &::= & \Edit e              & – valued editor \\
      % &      &\mid& \View e              & – valued read-only editor \\
      &      &\mid& \Enter \tau          & – unvalued editor \\
      &      &\mid& \Update e            & – shared editor \\
      % &      &\mid& \Watch e             & – shared read-only editor \\
    \addlinespace
      &      &\mid& e_1 \Then e_2        & – step \\
      &      &\mid& e_1 \Next e_2        & – user step \\
    \addlinespace
      &      &\mid& e_1 \And e_2         & – composition \\
    \addlinespace
      &      &\mid& e_1 \Or e_2          & – choice \\
      &      &\mid& e_1 \Xor e_2         & – user choice \\
    \addlinespace
      &      &\mid& \Fail                & – fail task \\
  \end{grammar}
}

\newmacro{G-Pretasks-Compact}{
  \begin{grammar}
    \noalign{Pretasks}
    \addlinespace
      & p    &::= & \Edit e \Mid \Enter \tau \Mid \Update e            & – editors: valued, unvalued, shared \\
      &      &\mid& e_1 \Then e_2 \Mid e_1 \Next e_2                   & – steps: internal, external \\
      &      &\mid& \Fail \Mid e_1 \And e_2                            & – fail, composition \\
      &      &\mid& e_1 \Or e_2 \Mid e_1 \Xor e_2                      & – choice: internal, external\\
  \end{grammar}
}

\newmacro{G-Pretasks-Flat}{
  \begin{grammar}
    \noalign{Pretasks}
    \addlinespace
      & p    &::= & \Edit e \Mid \Enter \tau \Mid \Update e  \Mid e_1 \Then e_2 \Mid e_1 \Next e_2             & – editors: valued, empty, shared, steps: internal, external\\
      &      &\mid& e_1 \Or e_2 \Mid e_1 \Xor e_2    \Mid  e_1 \And e_2 \Mid \Fail                 & – choice: internal, external, composition, fail\\
  \end{grammar}
}


\newmacro{G-Types}{
  \begin{grammar}
    Types
      & \tau &::= & \tau_1 \to \tau_2    & – function type \\
      &      &\mid& \Reference \tau      & – reference type \\
      &      &\mid& \Task \tau           & – task type \\
      &      &\mid& \beta                & – basic type \\
      % &      &\mid& \alpha               & – universal type \\
    Basic types
      &\beta &::= & \tau_1 \times \tau_2 & – product type \\
      &      &\mid& \text{List} \beta    & - list type\\
      &      &\mid& \Unit                & – unit type \\
      &      &\mid& \Bool                & – boolean type \\
      &      &\mid& \Int                 & – integer type \\
      &      &\mid& \String              & – string type \\
  \end{grammar}
}

\newmacro{G-Types-Compact}{
  \begin{grammar}
    \noalign{Types}
    \addlinespace
      & \tau  &  ::= & \tau_1 \to \tau_2 \Mid \beta                      & – function, basic \\
      &       & \mid & \Reference \tau \Mid \Task \tau                   & – reference, task \\
      \\
    \noalign{Basic types}
    \addlinespace
      & \beta &  ::= & \tau_1 \times \tau_2 \Mid \List \beta \Mid \Unit  & – product, list, unit \\
      &       & \mid & \Bool \Mid \Int \Mid \String                      & – boolean, integer, string \\
  \end{grammar}
}

\newmacro{G-Tasks-Flat}{
  \begin{grammar}
    \noalign{Tasks}
    \addlinespace
      & t    &::= & \Edit v \Mid \Enter \tau \Mid \Update l  \Mid t_1 \Then e_2 \Mid t_1 \Next e_2             & – editors: valued, empty, shared, steps: internal, external\\
      &      &\mid& t_1 \Or t_2 \Mid e_1 \Xor e_2    \Mid  t_1 \And t_2 \Mid \Fail                 & – choice: internal, external, composition, fail\\
  \end{grammar}
}



\newmacro{G-Types-Flat}{
  \begin{grammar}
    Types & \tau  &  ::= & \tau_1 \to \tau_2 \Mid \beta  \Mid \Reference \tau \Mid \Task \tau  & – function, basic, reference, task \\
    Basic types & \beta & ::=  & \tau_1 \times \tau_2 \Mid \List \beta \Mid \Unit & – product, list, unit\\
                &       & \mid & \Bool \Mid \Int \Mid \String & – boolean, integer, string
  \end{grammar}
}


\newmacro{G-Values}{
  \begin{grammar}
    Values
      & v    &::= & \lambda x:\tau.\ e   & – abstraction \\
      &      &\mid& c                    & – constant \\
      &      &\mid& l                    & – location \\
    \addlinespace
      &      &\mid& s                    & – symbol \\
      &      &\mid& u\ v                 & – symbolic unary operation \\
      &      &\mid& v_1\ o\ v_2          & – symbolic binary operation \\
    \addlinespace
      &      &\mid& \tuple{v_1, v_2}     & – pair value \\
      &      &\mid& \unit                & – unit \\
    \addlinespace
      &      &\mid& t                    & – task \\
  \end{grammar}
}

\newmacro{G-Values-Compact}{
  \begin{grammar}
    \noalign{Values}
    \addlinespace
      & v &  ::= & \lambda x:\tau.\ e \Mid \tuple{v_1, v_2} \Mid \unit & – abstraction, pair, unit \\
      &   & \mid & [\ ]_\beta \Mid v_1 :: v_2 \Mid c                   & - nil, cons, constant \\
      &   & \mid & l \Mid t                                            & – location, task \\
      &   & \mid & u\ v \Mid v_1\ o\ v_2                               & – unary/binary operation \\
  \end{grammar}
}

\newmacro{G-Values-Flat}{
  \begin{grammar}
    \noalign{Values}
    \addlinespace
      & v &  ::= & \lambda x:\tau.\ e \Mid \tuple{v_1, v_2} \Mid \unit \Mid [\ ]_\beta \Mid v_1 :: v_2 & – abstraction, pair, unit, nil, cons \\
      &   & \mid & c \Mid l \Mid t \Mid u\ v \Mid v_1\ o\ v_2                 & – constant, location, task,unary/binary operation
  \end{grammar}
}

\newmacro{G-SValues-Compact}{
  \begin{grammar}
    \noalign{Values}
    \addlinespace
      & \tilde{v} &  ::= & \lambda x:\tau.\ \tilde{e} \Mid \tuple{\tilde{v}_1, \tilde{v}_2} \Mid \unit & – abstraction, pair, unit \\
      &   & \mid & [\ ]_\beta \Mid \tilde{v}_1 :: \tilde{v}_2 \Mid c                           & - nil, cons, constant \\
      &   & \mid & l \Mid \tilde{t} \Mid s                                                     & – location, task, symbol \\
      &   & \mid & u\ \tilde{v} \Mid \tilde{v}_1\ o\ \tilde{v}_2                               & – unary/binary operation \\
  \end{grammar}
}


\newmacro{G-Tasks}{
  \begin{grammar}
    Tasks
      & t    &::= & \Edit v              & – valued editor \\
      % &      &\mid& \View v              & – valued read-only editor \\
      &      &\mid& \Enter \tau          & – unvalued editor \\
      &      &\mid& \Update l            & – shared editor \\
      % &      &\mid& \Watch l             & – shared read-only editor \\
    \addlinespace
      &      &\mid& t_1 \Then e_2        & – step \\
      &      &\mid& t_1 \Next e_2        & – user step \\
    \addlinespace
      &      &\mid& t_1 \And t_2         & – composition \\
    \addlinespace
      &      &\mid& t_1 \Or t_2          & – choice \\
      &      &\mid& e_1 \Xor e_2         & – user choice \\
    \addlinespace
      &      &\mid& \Fail                & – fail task \\
  \end{grammar}
}

\newmacro{G-Tasks-Compact}{
  \begin{grammar}
    \noalign{Tasks}
    \addlinespace
      & t &  ::= & \Edit v \Mid \Enter \tau \Mid \Update l           & – editors \\
      &   & \mid & t_1 \Then e_2 \Mid t_1 \Next e_2                  & – steps \\
      &   & \mid & \Fail \Mid t_1 \And t_2                           & – fail, combination \\
      &   & \mid & t_1 \Or t_2 \Mid e_1 \Xor e_2                     & – choices \\
  \end{grammar}
}

\newmacro{G-STasks-Compact}{
  \begin{grammar}
    \noalign{Tasks}
    \addlinespace
      & \tilde{t} &  ::= & \Edit \tilde{v} \Mid \Enter \tau \Mid \Update l                   & – editors \\
      &           & \mid & \tilde{t}_1 \Then \tilde{e}_2 \Mid \tilde{t}_1 \Next \tilde{e}_2  & – steps \\
      &           & \mid & \Fail \Mid \tilde{t}_1 \And \tilde{t}_2                           & – fail, combination \\
      &           & \mid & \tilde{t}_1 \Or \tilde{t}_2 \Mid \tilde{e}_1 \Xor \tilde{e}_2     & – choices \\
  \end{grammar}
}

\newmacro{G-Inputs}{
  \begin{grammar}
    Symbolic inputs
      & i    & ::=& $s$                  & – symbolic action \\
      &      &\mid& \First i             & – pass to first \\
      &      &\mid& \Second i            & – pass to second
  \end{grammar}
}

\newmacro{G-Inputs-Compact}{
  \begin{grammar}
    \noalign{Symbolic inputs}
    \addlinespace
      & i    & ::=& a \Mid \First i \Mid \Second i  & – symbolic action, to first, to second \\
      \\
    \noalign{Symbolic actions}
    \addlinespace
      & a  & ::=& \highlight{s}                     & – symbol \\
      &    &\mid& \Continue \Mid \Left \Mid \Right  & – continue, go left, go right \\
  \end{grammar}
}

\newmacro{G-CInputs}{
  \begin{grammar}
    Concrete inputs
      & j    & ::=& \bar{a}              & – concrete action \\
      &      &\mid& \First j             & – pass to first \\
      &      &\mid& \Second j            & – pass to second \\
    Concrete actions
      & \bar{a} & ::=& c                    & – constant \\
      &         &\mid& \Continue            & – continue with next task \\
      &         &\mid& \Left                & – go left \\
      &         &\mid& \Right               & – go right \\
  \end{grammar}
}

\newmacro{G-CInputs-Compact}{
  \begin{grammar}
    \noalign{Concrete inputs}
    \addlinespace
      & j    & ::=& a \Mid \First j \Mid \Second j             & – action, to first, to second \\
      \\
    \noalign{Concrete actions}
    \addlinespace
      & a  & ::=& c                    & – constant \\
      &      &\mid& \Continue  \Mid \Left \Mid \Right          & – continue, go left, go right \\
  \end{grammar}
}


\newmacro{G-Predicates}{
  \begin{grammar}
    Predicates
      & \phi &::= & c                    & – constant \\
      &      &\mid& s                    & – symbol \\
      &      &\mid& \Continue            & – continue\\
      &      &\mid& \Left                & – go left \\
      &      &\mid& \Right               & – go right \\
      &      &\mid& u\ \phi              & – symbolic unary operation \\
      &      &\mid& \phi_1\ o\ \phi_2    & – symbolic binary operation \\
  \end{grammar}
}

\newmacro{G-Predicates-Compact}{
  \begin{grammar}
    \noalign{Path conditions}
    \addlinespace
      & \highlight{\phi} &::= & c \Mid s              & – constant, symbol\\
      % &      &\mid& \Continue  \Mid \Left \Mid \Right          & – continue, go left, go right\\
      &      &\mid& \unit \Mid \tuple{\phi_1, \phi_2} & – unit, pairs \\
      &      &\mid& [\ ]_\beta \Mid \phi_1 :: \phi2   & – nil, cons \\
      &      &\mid& u\ \phi \Mid \phi_1\ o\ \phi_2    & – symbolic unary/binary operation \\
  \end{grammar}
}

% !TEX root=../main.tex


%% Language %%%%%%%%%%%%%%%%%%%%%%%%%%%%%%%%%%%%%%%%%%%%%%%%%%%%%%%%%%%%%%%%%%%%

\newmacro{O-Value}{
  \begin{function}
    \signature{\Value : \mathrm{Tasks} \times \mathrm{States} \rightharpoonup \mathrm{Values}} \\
    \Value(\Edit v, \sigma)                &=& v \\
    \Value(\Enter \tau, \sigma)            &=& \bot \\
    \Value(\Update l, \sigma)              &=& \sigma(l) \\
    \Value(\Fail, \sigma)                  &=& \bot \\
    \Value(t_1 \Then e_2, \sigma)          &=& \bot \\
    \Value(t_1 \Next e_2, \sigma)          &=& \bot \\
    \Value(t_1 \And t_2, \sigma)           & & \\
      \inset{= \left\{
          \begin{array}{ll}
            \tuple{v_1, v_2}  & \ \when\ \Value(t_1, \sigma) = \
            v_1 \land \Value(t_2, \sigma) = v_2 \\
            \bot                          & \ \otherwise
          \end{array}
        \right.} \\
    \Value(t_1 \Or t_2, \sigma)            & & \\
      \inset{= \left\{
        \begin{array}{ll}
          v_1                           & \ \when\ \Value(t_1, \sigma) = v_1 \\
          v_2                           & \ \when\ \Value(t_1, \sigma) = \bot \land \Value(t_2, \sigma) = v_2 \\
          \obox{\tuple{v_1, v_2}}{\bot} & \ \otherwise
        \end{array}
        \right.} \\
      \Value(t_1 \Xor t_2, \sigma)           &=& \bot
    \end{function}
  }

  % \newmacro{O-Failing}{
%   \begin{function}
%     \signature{\Failing : \mathrm{Tasks} \times \mathrm{States} \to \mathrm{Booleans}} \\
%     \Failing(\Edit v,\sigma)       &=& \False \\
%     \Failing(\Enter \tau,\sigma)   &=& \False \\
%     \Failing(\Update l,\sigma)     &=& \False \\
%     \Failing(\Fail,\sigma)         &=& \True \\
%     \Failing(t_1 \Then e_2,\sigma) &=& \Failing(t_1,\sigma) \\
%     \Failing(t_1 \Next e_2,\sigma) &=& \Failing(t_1,\sigma) \\
%     \Failing(t_1 \And t_2,\sigma)  &=& \Failing(t_1,\sigma) \land \Failing(t_2,\sigma) \\
%     \Failing(t_1 \Or t_2,\sigma)   &=& \Failing(t_1,\sigma) \land \Failing(t_2,\sigma) \\
%     \Failing(e_1 \Xor e_2,\sigma)  &=& \highlight{\bigwedge \Big( \set{\Failing(t_1,\sigma_1') \mid e_1, \sigma \normalise \overline{t_1,\sigma_1'}}\ \cup} \\
%                           & & \highlight{\phantom{\bigwedge \Big(}\set{\Failing(t_2,\sigma_2') \mid e_2, \sigma \normalise \overline{t_2,\sigma_2'}} \Big)} \\
%     % \Failing(e_1 \Xor e_2,\sigma)  &=& \Failing(t_1,s_1') \land \Failing(t_2,s_2')\\
%     % &&\quad \where\ e_1,\sigma \normalise t_1,s_1' \mathbf{\ and\ } e_2,\sigma \normalise t_2,s_2'
%   \end{function}
% }

\newmacro{O-Failing}{
  \begin{function}
    \signature{\Failing : \mathrm{Tasks} \times \mathrm{States} \to \mathrm{Booleans}} \\
    \Failing(\Edit v, \sigma)       &=& \False \\
    \Failing(\Enter \tau, \sigma)   &=& \False \\
    \Failing(\Update l, \sigma)     &=& \False \\
    \Failing(\Fail, \sigma)         &=& \True \\
    \Failing(t_1 \Then e_2, \sigma) &=& \Failing(t_1, \sigma) \\
    \Failing(t_1 \Next e_2, \sigma) &=& \Failing(t_1, \sigma) \\
    \Failing(t_1 \And t_2, \sigma)  &=& \Failing(t_1, \sigma) \land \Failing(t_2, \sigma) \\
    \Failing(t_1 \Or t_2, \sigma)   &=& \Failing(t_1, \sigma) \land \Failing(t_2, \sigma) \\
    \Failing(e_1 \Xor e_2, \sigma)  &=& \Failing(t_1, \sigma_1') \land \Failing(t_2, \sigma_2') \\
      % \inset{\where\ e_1, \sigma \normalise t_1, \sigma_1' \mathbf{\ and\ } e_2, \sigma \normalise t_2, \sigma_2'} \\
                                    & & \where\ e_1, \sigma \normalise t_1, \sigma_1' \\
                                    & & \mathbf{\ and\ } e_2, \sigma \normalise t_2, \sigma_2'
    % \Failing(u \At t, \sigma)       &=& \Failing(t, \sigma)
  \end{function}
}

\newmacro{O-Inputs}{
  \begin{function}
    \signature{\Inputs : \mathrm{Tasks} \times \mathrm{States} \to \powerset{\mathrm{Inputs}}} \\
    \Inputs(\Edit v,\sigma)             &=& \set{v':\tau}                       \quad\where\ \Edit v : \Task\tau \\
    \Inputs(\Enter \tau,\sigma)         &=& \set{v':\tau} \\
    \Inputs(\Update l,\sigma)           &=& \obox{\set{v':\tau, \Empty}}{\set{v':\tau}} \quad\where\ \Update l : \Task\tau \\
    \Inputs(\Fail,\sigma)               &=& \nothing \\
    \Inputs(t_1 \Then e_2,\sigma)       &=& \Inputs(t_1,\sigma) \\
    \Inputs(t_1 \Next e_2,\sigma)       &=& \Inputs(t_1,\sigma) \cup \set{\Continue \mid \Value(t_1, \sigma) = v_1 \land \\
                                         && e_2\ v_1, \sigma \normalise t_2, \sigma',\phi \land \lnot\Failing(t_2, \sigma')} \\
    \Inputs(t_1 \And t_2,\sigma)        &=& \set{\First\ i \mid i \in \Inputs(t_1,\sigma)} \cup \set{\Second\ i \mid i \in \Inputs(t_2,\sigma)} \\
    \Inputs(t_1 \Or t_2,\sigma)         &=& \set{\First\ i \mid i \in \Inputs(t_1,\sigma)} \cup \set{\Second\ i \mid i \in \Inputs(t_2,\sigma)} \\
    \Inputs(e_1 \Xor e_2,\sigma)        &=& \set{\Left \mid e_1, \sigma \normalise t_1, \sigma',\phi \land \lnot\Failing(t_1, s')} \cup\\
                                         && \set{\Right \mid e_2, \sigma \normalise t_2, \sigma',\phi \land \lnot\Failing(t_2, s')}
  \end{function}
}

\newmacro{O-Hints}{
  \begin{function}
    \signature{\Hints : \id{Tasks} \times \id{States} \times (\id{Values}
      \to \id{Booleans}) \to \powerset{\id{Inputs} \times \id{Predicates}}} \\
    \Hints(t, \sigma, g)
      &=& \set{ \tuple{i, \Phi \land g(v)} \mid t, \sigma \simulate \more{(\tilde{v}, \tilde{\imath} :: I, \Phi)},\; \Sat(\Phi \land g(v))} \\
  \end{function}
}



% If you use the hyperref package, please uncomment the following line
% to display URLs in blue roman font according to Springer's eBook style:
% \renewcommand\UrlFont{\color{blue}\rmfamily}

\begin{document}

%\title{Automatically generated next step hints for end users, using symbolic execution for task oriented programming}
%\title{Leveraging symbolic execution to generate next step hints for task oriented programming}
\title{Generating next step hints for task oriented programs using symbolic execution}

%\titlerunning{Abbreviated paper title}
% If the paper title is too long for the running head, you can set
% an abbreviated paper title here

\author{
  Nico Naus\inst{1}\and
  Tim Steenvoorden\inst{2}
}

\authorrunning{Naus and Steenvoorden}
% First names are abbreviated in the running head.
% If there are more than two authors, 'et al.' is used.

\institute{
  Open University, The Netherlands \email{nico.naus@ou.nl} \and
  Radboud University, Nijmegen, The Netherlands \email{tim@cs.ru.nl}
}

\maketitle              % typeset the header of the contribution

\begin{abstract}
  % !TEX root=../main.tex

% Context
Software modelling business workflows are omnipresent today's society.
They coordinate collaboration in hospitals, companies, and military institutions.

% Inquiry
These systems obfuscate the influence of current user actions on the desired end result.
In order to make the right decision, they need to oversee the full process and all information in the system.
Both of which are usually buried in the system.

% Approach
We have developed a way to automatically generate next step hints for task oriented programs.
By leveraging symbolic execution, we can calculate these hints without modification of the original program.


% Knowledge
To our knowledge this is the first time that symbolic execution is used to automatically generate next step hints for end users.
We implemented this approach in Haskell to generate next step hints for TopHat programs, a formal system for task oriented programming.

% Grounding
Besides proving the generated hints to be sound and complete, we also demonstrate that the symbolic execution semantics we employ to be correct for sequential input.

% Importance
Next step hints reduce the chance of human error, while still allowing end users to intervene if required.
The overall performance is raised, since the quality of decisions will improve.

% Context: What is the broad context of the work? What is the importance of the general research area?
% Inquiry: What problem or question does the paper address? How has this problem or question been addressed by others (if at all)?
% Approach: What was done that unveiled new knowledge?
% Knowledge: What new facts were uncovered? If the research was not results oriented, what new capabilities are enabled by the work?
% Grounding: What argument, feasibility proof, artifacts, or results and evaluation support this work?
% Importance: Why does this work matter?

\keywords{Task-oriented programming \and Next step hint generation \and Symbolic execution.}

\end{abstract}

% !TEX root=../main.tex

\section{Introduction}
\label{sec:intro}

%There are many workflow systems
Software that supports people working together is used in most workplaces nowadays.
Its aim is to automate business workflows, in order to simplify processes, to improve service, or to contain cost.
In settings like hospitals, first responders and military operations, these systems could even prevent the loss of lives.

%they are a problem because

Automation and digitalisation of workflows and business processes comes at a cost.
Using such applications requires workers to undergo training, and often relies on their experience and expertise\todo{Fix problem description.}.

%our approach overcomes this because
To overcome these drawbacks, we propose to integrate a next step hint assistive system into the workflow software.
By combining previous research on symbolic execution for Task-Oriented Programming~\cite{Naus2019} and end-user feedback systems for rule based problems~\cite{DBLP:conf/sfp/NausJ16},
we develop a next step hint end-user feedback system for the Task-Oriented Programming language \TOPHAT~\cite{DBLP:conf/ppdp/SteenvoordenNK19}.
Our solution, which we call Assistive \TOPHAT, generates next step hints from existing code, and does not require extra work by the programmer.

\todo{Specify a bit more the scope of the paper.}
\todo{Say something about the uniqueness of this work, as we do in the related work section.}

\subsection{Contributions}

This paper makes the following contributions.

\begin{itemize}
  \item We describe an end user next step feedback system for \TOPHAT.
  \item We prove soundness and completeness of next step hints generated by this system.
  \item We present an implementation of the user feedback system in Haskell.
  \item We prove the symbolic execution semantics of \TOPHAT to be sound and complete.
\end{itemize}


\subsection{Structure}

\cref{sec:tophat} first introduces the Task-Oriented Programming (\TOP) paradigm and the Task-Oriented Programming language \TOPHAT.
\cref{sec:examples} lists three example programs to illustrate how \TOPHAT works and to show what we like to achieve with Assistive \TOPHAT.
In \cref{sec:symbolic} we introduce briefly the symbolic execution semantics for \TOPHAT.
\cref{sec:assistive} follows with a description of assistive \TOPHAT
and describes the implementation of the system in Haskell.
In \cref{sec:properties} soundness and completeness of the assistive system are shown.
% An implementation of the system in Haskell is described in \cref{sec:implementation}.
\cref{sec:relatedwork} gives an overview of related work, and finally \cref{sec:conclusion} concludes.

% !TEX root=../main.tex

\section{The TopHat language}
\label{sec:tophat}

Task-Oriented Programming (\TOP) is a paradigm first introduced by Plasmeijer et al.~\cite{DBLP:conf/ppdp/PlasmeijerLMAK12}.
It was created to improve the development and quality of software that coordinates collaboration between users.
\TOP provides programmers with a high level of programming abstraction,
while being expressive enough to describe real world collaborations.
It does so by using features from higher-order functional programming languages,
combined with the notion of \emph{tasks}.
Tasks model units of work, which can be performed by a human or by a computer.
From a task specification, a \TOP implementation generates a distributive multi-user (web) application to support this collaboration.

Tasks have a couple of properties, listed below.
\begin{itemize}
  \item
    Tasks model \emph{collaboration}.\\
    Programmers describe what work needs to be done in what way.
  \item
    Tasks are \emph{interactive}.\\
    Users can enter or update information into the system by using \emph{editors}.
  \item
    Tasks can be \emph{observed}.\\
    Therefore, other users or the system itself can make decisions based on the progress of the task.
  \item
    Tasks are \emph{modular}.\\
    They can be combined into bigger tasks by using \emph{combinators}.
    Basic combinators are chosen in such a way, that they represent a way of collaboration between users.
    New combinators can be created by making use of basic combinators and the higher order facilities of the host language.
  \item
    Tasks \emph{share information}.\\
    Information is passed along control flow, or via references to shared data.
  \item
    Tasks are \emph{typed}.\\
    This is not just to ensure safety at runtime,
    but also to automatically derive common program elements.
    \TOP systems automatically generate user interfaces and manage persistent storage of information.
\end{itemize}

Currently, there are three systems implementing the ideas of \TOP.
The reference implementation is the \ITASKS framework~\cite{DBLP:conf/ppdp/PlasmeijerLMAK12},
which is an embedded domain specific language in the non-strict functional programming language Clean~\cite{plasmeijer2002clean}.
\MTASKS~\cite{DBLP:conf/cgo/KoopmanLP18} is a \TOP implementation specifically designed for embedded systems.
A formalisation of \TOP, called \TOPHAT (TopHat), has been created by Steenvoorden, Naus, and Klinik~\cite{DBLP:conf/ppdp/SteenvoordenNK19}.
This work builds further on \TOPHAT and its symbolic counterpart \STOPHAT~\cite{Naus2019}.

\TOPHAT implements \TOP by embedding a task language in the simply typed lambda calculus with references, conditionals, and pairs.
\STOPHAT extends this with built in operators, lists, and most importantly symbols.
References are used to model the shared data component of \TOP.
Below, the different components of \TOPHAT are explained in detail.
The complete syntax and semantics can be found in previous work~\cite{DBLP:conf/ppdp/SteenvoordenNK19}.
In the next subsections we describe the basic constructs of the \TOPHAT language.
\cref{sec:symbolic} details \STOPHAT.

% Since the idea was introduced, a lot of research as been done on \TOP.
% However, the basic principle remain the same.
% \TOP implementations like iTasks~\cite{????} also leverage aspects of the host language in order to allow for a higher level of abstraction,
% such as generic programming and higher order tasks. \todo{is dit wel of niet top?}
%
%
% The next section describes the formal \TOP implementation \TOPHAT.

\subsection{Editors}

Editors form the entry points for interaction and communication with the outside world.
They are the most basic tasks and can be seen as an abstraction over widgets in a \GUI library or on webpage forms.
Users can change the value held by an editor, in the same way they can manipulate widgets in a \GUI.

When a \TOP implementation generates an application from a task specification, it derives user interfaces for the editors.
The appearance of an editor is influenced by its type.
For example, editors of type strings can be represented by simple input fields, dates by calendars, and locations by pins on a map.

There are three different editors in \TOPHAT.
\begin{description}
  \item[$\Edit v$] Valued editor.\\
    This editor holds a value $v$ of a certain type.
    The user can replace the value by new values of the same type.
  \item[$\Enter \tau$] Unvalued editor.\\
    This editor holds no value, and can receive a value of type $\tau$.
    When that happens, it turns into a valued editor.
  \item[$\Update l$] Shared editor.\\
    This editor refers to a store location $l$.
    Its observable value is the value stored at that location.
    When it receives a new value, this value will be stored at location $l$.
\end{description}


\subsection{Combinators}

Editors can be combined into larger tasks using combinators.
The order in which editors and tasks are executed is specified with combinators. Tasks can be performed in sequence, in parallel or a choice can be made between tasks.


The following combinators are available in \TOPHAT.
Here, $t$ stands for tasks and $e$ for arbitrary expressions.
\begin{description}
  \item[$t \Then e$] Step.\\
    Users can work on task $t$.
    As soon as $t$ has an observable value, that value is passed on to the right hand side $e$.
    The expression $e$ is a function, taking the value as an argument, resulting in a new task.
  \item[$t \Next e$] User Step.\\
    Users can work on task $t$.
    When $t$ has an observable value, the step becomes enabled.
    Then, users can send a continue event to the combinator.
    When that happens, the value of $t$ is passed to the right hand side, with which it continues,
    in the same way as normal steps do.
  \item[$t_1 \And t_2$] Pair.\\
    Users can work on tasks $t_1$ and $t_2$ in at the same time.
  \item[$t_1 \Or t_2$] Choice.\\
    The system chooses between $t_1$ or $t_2$,
    based on which task first has an observable value.
    If both tasks have a value, the system chooses the left one.
  \item[$e_1 \Xor e_2$] User choice.\\
    A user has to make a choice between either the left or the right hand side.
    The user continues to work on the chosen task.
\end{description}

In addition to editors and combinators, \TOPHAT also contains the fail task ($\Fail$).
Programmers can use this task to indicate that a task is not reachable or viable.
When the right hand side of a step combinator evaluates to $\Fail$, the step will not proceed to that task.


\subsection{Observations}

Several observations can be made on tasks.
These observations are used by the system to determine the progress of combinators,
to draw the user interface, and they will also be used by Assistive \TOPHAT to provide next step hints.

Using the value function $\Value$, the current value of a task can be determined.
The value function is a partial function, since not all tasks have a value.
For example empty editors and steps do not have a value.
For the parallel and internal choice combinators, their value depends on the value of the subtasks.
Parallel only as a value if both tasks have an observable value.
Internal choice has a value if either of the two tasks has an observable value.

One can also observe whether or not a task is failing, by means of the failing function $\Failing$.
A task is considered to be failing if a user cannot interact with it.
For example, the valued editor is not failing, since the user can update it with a new value.
The task $\Fail$ is failing, as is a parallel combination of failing tasks $\Fail \And \Fail$, since no input can be send to those tasks.
Both observation definitions can be found in \cref{fig:observations}

\begin{figure}[h]
  \begin{minipage}{\textwidth}
    \centering \small
    \usemacro{O-Value}  \usemacro{O-Failing}
  \end{minipage}
  \caption{
    Observations on task $t$.
    $\Value$ gets the value of $t$, $\Failing$ observes if it is unsafe to step to $t$.
    Note that $\Value$ is a partial function.
  }
  \label{fig:observations}
\end{figure}

The step combinator makes use of both functions in order to determine if it can step.
First, it uses $\Value$ to see if the left hand side produces a value.
If that is the case, it uses the $\Failing$ function to see if stepping to the right hand side is successful.


\subsection{Input}

Input events drive evaluation of tasks.
Because tasks are typed, input is typed as well.
Editors only accept input of the correct type.
Examples are replacing a value in an editor,
or sending a continue event to a user step.
When the system receives a valid event, it gives this event to the current task, which reduces to a new task.
Everything in between two events is evaluated atomically with respect to inputs.
This means that tasks are normalised up to the point they await new user interactions.
% In this way the system communicates with the environment.

Input events are synchronous, which means the order of execution is completely determined by the order of the events.
In particular, the order of input events determine the progression of parallel branches.


\subsection{Semantics}
\todo{Fix arrow of striding in figure.}
\begin{figure}[h]
  \centering
  \includegraphics[width=0.8\columnwidth,page=5]{figures/drawings-crop.pdf}
  \caption{
    Semantic functions defined in this report and their relation.
  }
  \label{fig:semantic-functions}
\end{figure}

% !TEX root=../main.tex

\section{A worked example}
\label{sec:examples}

\subsection{Tax subsidy request}

% \citet{conf/sfp/StutterheimAP17} worked with the Dutch tax office to develop a demonstrator for a fictional but realistic law about solar panel subsidies.
% In this section we study a simplified version of this, translated to \TOPHAT, to illustrate how symbolic execution can be used to prove that the program implements the law.
%
% This example proves that a citizen will get subsidy only under the following conditions.
\begin{itemize}
\item The roofing company has confirmed that they installed solar panels for the citizen.
\item The tax officer has approved the request.
\item The tax officer can only approve the request if the roofing company has confirmed, and the request is filed within one year of the invoice date.
\item The amount of the granted subsidy is maximal 600 EUR.
\end{itemize}

\lstset{emph={invoiceDate,today,confirmed,invoiceAmount,approved}}
\begin{TASK}[float=ht
            ,numbers=right
            ,caption=Subsidy request and approval workflow at the Dutch tax office.
            ,label=lst:tax
            ]
  let provideCitizenInformation = enter Date in

    let provideDocuments = enter Amount <&> enter Date in

      let companyConfirm = edit True <?> edit False in

        let officerApprove = \ invoiceDate. \ today. \ confirmed.
          edit False <?> if (today - invoiceDate < 365 /\ confirmed) |\label{lst:tax:officer-approve-def}| then edit True else fail in

            provideCitizenInformation >>= \ today.|\label{lst:tax:citizen-info}|
            provideDocuments <&> companyConfirm >>= |\label{lst:tax:documents-and-company-confirm}|
              \ <<<<invoiceAmount, invoiceDate>>, confirmed>>.
              officerApprove invoiceDate today confirmed >>= \ approved.|\label{lst:tax:officer-approve}|

          let subsidyAmount = if approved then min 600 (invoiceAmount / 10) else 0 in
          
            edit <<subsidyAmount, approved, confirmed, invoiceDate, today>>|\label{lst:tax:result}|
\end{TASK}

% \begin{figure}[ht]
%   \includegraphics[width=\columnwidth]{figures/tax-enter}
%   \caption{
%     Graphical user interface for the task in \cref{lst:tax}.
%     In parallel, the citizen is asked to enter the invoice amount and the invoice date of the installed solar panels,
%     and the roofing company is asked to deny or confirm they actually installed the solar panels.
%   }
%   \label{fig:tax}
% \end{figure}
%
% \Cref{lst:tax} shows the program.
% To enhance readability of the example,
% we omit type annotations and make use of pattern matching on tuples.
% The program works as follows.
% First, the citizen has to enter their personal information (\cref{lst:tax:citizen-info}).
% In the original demonstrator this included the citizen service number, name, and home address.
% Here, we simplified the example so that the citizen only has to enter the invoice date.
% A date is specified using an integer representing the number of days since 1 January 2000.
%
% In the next step (\cref{lst:tax:documents-and-company-confirm}), in parallel the citizen has to provide the invoice documents of the installed solar panels, while the roofing company has to confirm that they have actually installed solar panels at the citizen's address.
% Once the invoice and the confirmation are there, the tax officer has to approve the request (\cref{lst:tax:officer-approve}).
% The officer can always decline the request, but they can only approve it if the roofing company has confirmed and the application date is within one year of the invoice date (\cref{lst:tax:officer-approve-def}).
% The result of the program is the amount of the subsidy, together with all information needed to prove the required properties (\cref{lst:tax:result}).
% The graphical user interface belonging to two steps in this process are shown in \cref{fig:tax}.
%
% The result of the overall task is a tuple with the subsidy amount, the officer's approval, the roofing company's confirmation, the invoice amount, the invoice date, and today's date.
% Returning all this information allows the following predicate to be stated, which verifies the correctness of the implementation.
% The predicate has 5 free variables, which correspond to the returned values.
% \setcounter{equation}{0}
% \begin{align}
% \psi(s,a,c,i,t)
%    & =      s \geq 0 \implies c \label{for:tax:psi-confirmed}
% \\ & \wedge s \geq 0 \implies a \label{for:tax:psi-approved}
% \\ & \wedge a \implies c \wedge t - i \leq 356 \label{for:tax:psi-approve-conditions}
% \\ & \wedge s \leq 600 \label{for:tax:psi-max-subsidy}
% \\ & \wedge \lnot a \implies s \equiv 0 \label{for:tax:psi-unapproved}
% \end{align}
% The predicate $\psi$ states that (\ref{for:tax:psi-confirmed}) if subsidy $s$ has been payed, the roofing company must have confirmed $c$, (\ref{for:tax:psi-approved}) if subsidy has been payed, the officer must have approved $a$, (\ref{for:tax:psi-approve-conditions}) the officer can approve only if the roofing company has confirmed and today's date $t$ is within 356 days of the invoice date $i$, and (\ref{for:tax:psi-max-subsidy}) the subsidy is maximal 600 EUR.
% Finally, (\ref{for:tax:psi-unapproved}) if the officer has not approved, the subsidy must be 0.



Running the simulation function will result in the following set:

\begin{align*}
  ((min 600 (amount/10)&,  \True, \True, i, t),[t,\Left \Left amount,     \Left \Right i, \Right \Left, \Right],t-i<356)\\
  ((min 600 (amount/10)&,  \True, \True, i, t),[t,    \Left \Right i, \Left \Left amount, \Right \Left, \Right],t-i<356)\\
  ((min 600 (amount/10)&,  \True, \True, i, t),[t,      \Right \Left, \Left \Left amount, \Left \Right i, \Right],t-i<356)\\
  ((min 600 (amount/10)&,  \True, \True, i, t),[t,      \Right \Left, \Left \Right i, \Left \Left amount, \Right],t-i<356)\\
  ((min 600 (amount/10)&,  \True, \True, i, t),[t,    \Left \Right i, \Right \Left, \Left \Left amount, \Right],t-i<356)\\
  ((min 600 (amount/10)&,  \True, \True, i, t),[t,\Left \Left amount, \Right \Left, \Left \Right i, \Right],t-i<356)\\
  ((                  0&, \False, \True, i, t),[t,\Left \Left amount, \Left \Right i, \Right \Left, \Left],\True)\\
  ((                  0&, \False, \True, i, t),[t,    \Left \Right i, \Left \Left amount, \Right \Left, \Left],\True)\\
  ((                  0&, \False, \True, i, t),[t,      \Right \Left, \Left \Left amount, \Left \Right i, \Left],\True)\\
  ((                  0&, \False, \True, i, t),[t,      \Right \Left, \Left \Right i, \Left \Left amount, \Left],\True)\\
  ((                  0&, \False, \True, i, t),[t,    \Left \Right i, \Right \Left, \Left \Left amount, \Left],\True)\\
  ((                  0&, \False, \True, i, t),[t,\Left \Left amount, \Right \Left, \Left \Right i, \Left],\True)\\
  ((                  0&, \False,\False, i, t),[t,\Left \Left amount, \Left \Right i, \Right \Right, \Left],\True)\\
  ((                  0&, \False,\False, i, t),[t,\Left \Right i, \Left \Left amount, \Right \Right, \Left],\True)\\
  ((                  0&, \False,\False, i, t),[t,\Right \Right,\Left \Left amount, \Left \Right i, \Left],\True)\\
  ((                  0&, \False,\False, i, t),[t,\Right \Right, \Left \Right i, \Left \Left amount, \Left],\True)\\
  ((                  0&, \False,\False, i, t),[t, \Left \Right i, \Right \Right, \Left \Left amount, \Left],\True)\\
  ((                  0&, \False,\False, i, t),[t, \Left \Left amount, \Right \Right, \Left \Right i, \Left],\True)\\
\end{align*}


As a goal we set that the subsidy amount should be larger than zero. This leaves us with the following cases.


\begin{align*}
  ((min 600 (amount/10)&,  \True, \True, i, t),[t,\Left \Left amount,     \Left \Right i, \Right \Left, \Right],t-i<356)\\
  ((min 600 (amount/10)&,  \True, \True, i, t),[t,    \Left \Right i, \Left \Left amount, \Right \Left, \Right],t-i<356)\\
  ((min 600 (amount/10)&,  \True, \True, i, t),[t,      \Right \Left, \Left \Left amount, \Left \Right i, \Right],t-i<356)\\
  ((min 600 (amount/10)&,  \True, \True, i, t),[t,      \Right \Left, \Left \Right i, \Left \Left amount, \Right],t-i<356)\\
  ((min 600 (amount/10)&,  \True, \True, i, t),[t,    \Left \Right i, \Right \Left, \Left \Left amount, \Right],t-i<356)\\
  ((min 600 (amount/10)&,  \True, \True, i, t),[t,\Left \Left amount, \Right \Left, \Left \Right i, \Right],t-i<356)\\
\end{align*}

From the end states we can conclude that invoiceAmount should be larger than zero in order for the result to be larger than zero. This results in an additional constraint.

In the end, we can use the above to give hints to the user.

First, a date must be entered. Then the user can choose to ender an amount larger than zero, a date that is within 356 from now, or a "RL".
etc.

% !TEX root=../main.tex

\section{Generating next step hints}
\label{sec:assistive}

This section introduces our \ASTOPHAT system.
%What do we wanna do?
The aim of \ASTOPHAT is to automatically provide next step hints.
When users follow these hints, they can be sure that they will reach the goal they described beforehand.
Users can, however, still decide to deviate from the given hints.

During the execution of \TOPHAT programs, users are presented with input fields, choices and continue buttons.
The way in which tasks progress and the resulting task value depend on these inputs.
At any point during execution, we would like to present users with all possible inputs that leads users to the goal they have selected.
These inputs are either concrete actions, like continue, pick the left task, pick the right task;
or a restricted set of values to be entered into an editor.
This set is restricted, since concrete values potentially influence the flow of the program.
To give a concrete example, the user should enter an integer, but this integer must be larger than zero to reach the end goal.

To come to these concrete actions and restricted values, we make use of symbolic execution.
In the next two sections, we briefly describe how symbolic execution for \TOPHAT works
and recap its symbolic semantics presented in earlier work~\cite{Naus2019}.
Thereafter, we show how to turn symbolic execution results into next step hints.
In \cref{sub:assistive-tax,sub:assistive-dining}, we study what these automatically generated hints look like for the examples from \cref{sec:examples}.

%% Our examples work in implementation
All examples have been tested in our implementation.
%% We have an implementation
We added \ASTOPHAT to our existing implementation of \STOPHAT, which is written in Haskell.\footnote{https://github.com/timjs/symbolic-tophat-haskell}
It uses the \ZTHREE \SMT solver under the hood.
%% It works like this
By defining the formal hints function directly on top of the symbolic execution semantics, % in \cref{fig:hints},
we can leverage the already existing symbolic execution for \STOPHAT in the practical implementation.


\subsection{Symbolic execution}
\label{sub:symbolic-execution}

A symbolic execution semantics~\cite{King1975,Boyer1975} aims to execute a program without knowing its input.
Instead, symbols are fed into the program.
During evaluation, the influence of values is recorded in the path condition.
The resulting symbolic value together with the path conditions can be used to prove properties of the program.

\begin{TASK}[
    float=ht,
    % numbers=right,
    caption={Ordering of tuple elements.},
    captionpos=b,
    label=lst:ordering]
  enter Int <&> enter Int >>= \<<x,y>>. if x > y then edit <<y, x>> else edit <<x, y>>
\end{TASK}

Consider the tiny example in \cref{lst:ordering}.
This program asks for two integer values.
After the user has entered this information, the function to the right of the step combinator makes sure the result will be an editor containing a pair,
where the second element is larger then the first.
When we run this program symbolically, we have to create fresh symbols to be entered in either of the two editors, say $s_0$ and $s_1$.
After entering both symbolic values and then normalising the task, there are two possible outcomes, namely
\begin{itemize}
  \item $\tuple{s_1,s_0}$, provided that the path condition $\phi_1 = s_0 > s_1$ holds; or
  \item $\tuple{s_0,s_1}$, with path condition $\phi_2 = \lnot (s_0 > s_1)$.
\end{itemize}

Now, the property that we want to prove for this program is that no matter what the input is, the second element should always be larger than the first.
We write this property as $\psi(\<a, b\>)= a \leq b$.
Looking at the two symbolic runs, we first need to verify that the symbolic runs are indeed viable.
This is done by checking that both $\phi_1$ and $\phi_2$ are satisfiable, written $\Sat(\phi_1)$ and $\Sat(\phi_2)$.
Symbolic runs with a path condition that is not satisfiable are discarded.
Finally, we check that both path conditions conform to the goal property $\psi$, which is the case.
Therefore, we can conclude that the property holds.
When applying this technique to larger programs, it is a powerful tool to show that a program behaves as expected.


\subsection{Symbolic semantics}
\label{sub:symbolic-semantics}

To support symbolic execution in \TOPHAT, we extend our host language with symbols.
In addition, we also need to modify the semantics described in \cref{sub:semantics}, to accommodate symbolic execution.
The observation functions from \cref{sub:observations} are extended in a similar way.
These new semantic relations operate on expressions which may contain symbols.
Instead of stepping to one result, they lead to a set of possible symbolic results, accompanied with a path condition $\phi$.

\begin{table}[ht]
  \caption{Overview of meta variables and semantic relations for concrete and symbolic evaluations.}
  \label{tab:semantic-relations}
  \centering
  \begin{tabular}{l@{\Quad}L@{\Quad}L}
    \toprule
                  & \text{Concrete} & \text{Symbolic} \\
    \midrule
    Expressions   & e               & \tilde{e} \\
    Tasks         & t               & \tilde{t} \\
    States        & \sigma          & \tilde{\sigma} \\
    Inputs        & i               & \simi \\
    \midrule
    Evaluation    & \RelationE      & \RelationSE \\
    Normalisation & \RelationN      & \RelationSN \\
    Striding      & \RelationS      & \RelationSS \\
    Handling      & \RelationH      & \RelationSH \\
    Interacting   & \RelationI      & \RelationSI \\
    \bottomrule
  \end{tabular}
\end{table}

We denote entities containing symbols with an additional tilde,
and symbolic semantic relations with squiggly arrows instead of straight ones.
So $\tilde{t}$, $\tilde{\sigma}$, and $\simi$ are respectively tasks, states, and inputs containing symbols.
\Cref{tab:semantic-relations} gives an overview of the entities in the concrete world,
and their symbolic counterparts.
%
Concrete expressions are a subset of symbolic expressions.
Therefore, symbolic semantic relations can be applied on concrete expressions,
as well as symbolic expressions.

The symbolic interaction semantics ($\siminteract$) results in a set of symbolic runs, each of them just containing one symbolic input.
In other words, the symbolic interaction semantics just looks ahead one symbolic interaction.
To be able to reason about an end state after multiple symbolic interactions,
we introduce the notion of \emph{simulation}.
Informally, simulation performs multiple symbolic interactions after each other,
until the rewritten task has an observable value.
I.e. if $n$ is the number of interactions needed to be done,
$\Value(t_i', \sigma_i')$ has a result for $i = n$ but is undefined for all $i < n$.
Apart from this restriction, we want to permit only viable executions.
This is enforced by validating the satisfiability ($\Sat$) of the conjunction of all sequential path conditions.
More formally, simulating a task for multiple user inputs is defined as follows.

\begin{definition}[Simulation ($\simulate$)]
  \label{def:simulation}
  Let $t$ and $\sigma$ be a concrete task and concrete state.
  We define the simulation relation
  \begin{equation*}
    t,\sigma \simulate \overline{\tilde{v},\tilde{I},\Phi}
  \end{equation*}
  to be the set of results after performing symbolic interaction $n$ times:
  \begin{equation*}
      t,\sigma
        \siminteract \tilde{t}_1,\tilde{\sigma}_1,\simi_1,\phi_1
        \siminteract \cdots
        \siminteract \tilde{t}_n,\tilde{\sigma}_n,\simi_n,\phi_n
  \end{equation*}
  where:
  \begin{itemize}
    \item the $n$th task has a value: $\Value(\tilde{t}_n,\tilde{\sigma}_n) = \tilde{v}$;
    \item all tasks before do not have a value: $\Value(\tilde{t}_{i<n},\tilde{\sigma}_{i<n}) = \bot$;
    \item $\tilde{I} = \simi_1 \cdots \simi_n$ is the concatenation of all symbolic inputs generated along the way;
    \item $\Phi = \phi_1 \land \cdots \land \phi_n$, is the conjunction of all path conditions encountered.
  \end{itemize}
  Furthermore we require that:
  \begin{itemize}
    \item the resulting predicate is satisfiable: $\Sat(\Phi)$.
  \end{itemize}
\end{definition}

The simulation definition used in this paper differs from the one in previous work~\cite{Naus2019}.
Previously, infinite symbolic executions were filtered out by allowing two steps look-ahead in case of idempotent executions.
The definition above only allows finite executions by definition.

\subsection{Next step hints observation}
\label{sub:hints}

%% How do we do this?
As we have seen in \cref{def:simulation}, a symbolic task $\tilde{t}$ is considered done as soon as it has an observable value $\tilde{v}$.
In order to calculate next step hints, one needs to formulate a goal over this resulting value.
Only then, we can calculate next step hints for end users.

\begin{figure}
  \centering
  \usemacro{O-Hints}
  \caption{Definition of next step hint function.}
  \label{fig:hints}
\end{figure}

Hints are calculated by means of the $\Hints$ function listed in \cref{fig:hints}.
As input, it receives a concrete task $t$ and concrete state $\sigma$ together with a goal predicate $g$.
The hints observation simulates $t$ starting in $\sigma$.
This results in a set of symbolic values $\tilde{v}$, together with a list of symbolic inputs $\tilde{i}\cdot\tilde{I}$ and a condition $\Phi$ to reach this path.
We only want to use the symbolic executions that satisfy the goal $g$ when applied to $\tilde{v}$.
Since $\tilde{v}$ could contain symbols, it might be the case that $g(\tilde{v})$ is symbolic and would clash with the path condition $\Phi$.
Therefore, we require that the conjunction of the path condition with the goal is satisfiable ($\Sat(\Phi\land g(\tilde{v}))$).
From the executions that fulfill this requirement, we return the first symbolic input $\simi$ from the complete list of inputs $\simi\cdot\tilde{I}$,
together with the full condition that must hold ($\Phi\land g(\tilde{v})$).
The resulting set contains pairs of symbolic inputs guarded by this condition.

To get a better understanding how $\Hints$ works,
we study it more concretely in the next subsections.
\Cref{sub:assistive-tax} demonstrates on the basis of the tax example listed in \cref{sec:tax}, how the results of the symbolic execution are used to construct automatic next step hints.
\Cref{sub:assistive-dining} shows how hints can be generated during the execution of the example \TOPHAT program listed in \cref{sec:dining}.


\subsection{Tax subsidy request}
\label{sub:assistive-tax}

%recall
Recall the Tax example program in \TOPHAT from \cref{sec:tax},
which models the application for a solar panel tax refund.
The user enters the invoice date and invoice amount, the installation company confirms, and finally the tax officer either approves or denies the request.

\begin{table}[ht]
  \caption{The results of simulating the program from \cref{lst:tax}.}
  \label{table:tax}
  \centering
  \scriptsize
  \begin{tabular}{L@{\Quad}L@{\Quad}L@{\Quad}}
    \toprule
    \text{Symbolic value ($\tilde{v}$)} & \text{Symbolic input ($\tilde{I}$)} & \text{Path condition ($\Phi$)} \\
    \midrule
    \tuple{\min(600, s_{\id{a}}/10),  \True, \True, s_{\id{i}}, \Today} & \First \First s_{\id{a}} \cdot \First \Second s_{\id{i}} \cdot \Second \Left \cdot \Second & (\Today-s_{\id{i}}) < \OneYear \\
    \tuple{\min(600, s_{\id{a}}/10),  \True, \True, s_{\id{i}}, \Today} & \First \Second s_{\id{i}} \cdot \First \First s_{\id{a}} \cdot \Second \Left \cdot \Second & (\Today-s_{\id{i}}) < \OneYear \\
    \tuple{\min(600, s_{\id{a}}/10),  \True, \True, s_{\id{i}}, \Today} & \Second \Left \cdot \First \First s_{\id{a}} \cdot \First \Second s_{\id{i}} \cdot \Second & (\Today-s_{\id{i}}) < \OneYear \\
    \tuple{\min(600, s_{\id{a}}/10),  \True, \True, s_{\id{i}}, \Today} & \Second \Left \cdot \First \Second s_{\id{i}} \cdot \First \First s_{\id{a}} \cdot \Second & (\Today-s_{\id{i}}) < \OneYear \\
    \tuple{\min(600, s_{\id{a}}/10),  \True, \True, s_{\id{i}}, \Today} & \First \Second s_{\id{i}} \cdot \Second \Left \cdot \First \First s_{\id{a}} \cdot \Second & (\Today-s_{\id{i}}) < \OneYear \\
    \tuple{\min(600, s_{\id{a}}/10),  \True, \True, s_{\id{i}}, \Today} & \First \First s_{\id{a}} \cdot \Second \Left \cdot \First \Second s_{\id{i}} \cdot \Second & (\Today-s_{\id{i}}) < \OneYear \\
    \tuple{                       0,  \False, \True, s_{\id{i}}, \Today} & \First \First s_{\id{a}} \cdot \First \Second s_{\id{i}} \cdot \Second \Left \cdot \First  & \True \\
    \tuple{                       0,  \False, \True, s_{\id{i}}, \Today} & \First \Second s_{\id{i}} \cdot \First \First s_{\id{a}} \cdot \Second \Left \cdot \First  & \True \\
    \tuple{                       0,  \False, \True, s_{\id{i}}, \Today} & \Second \Left \cdot \First \First s_{\id{a}} \cdot \First \Second s_{\id{i}} \cdot \First  & \True \\
    \tuple{                       0,  \False, \True, s_{\id{i}}, \Today} & \Second \Left \cdot \First \Second s_{\id{i}} \cdot \First \First s_{\id{a}} \cdot \First  & \True \\
    \tuple{                       0,  \False, \True, s_{\id{i}}, \Today} & \First \Second s_{\id{i}} \cdot \Second \Left \cdot \First \First s_{\id{a}} \cdot \First  & \True \\
    \tuple{                       0,  \False, \True, s_{\id{i}}, \Today} & \First \First s_{\id{a}} \cdot \Second \Left \cdot \First \Second s_{\id{i}} \cdot \First  & \True \\
    \tuple{                       0,  \False,\False, s_{\id{i}}, \Today} & \First \First s_{\id{a}} \cdot \First \Second s_{\id{i}} \cdot \Second \cdot \First  & \True \\
    \tuple{                       0,  \False,\False, s_{\id{i}}, \Today} & \First \Second s_{\id{i}} \cdot \First \First s_{\id{a}} \cdot \Second \cdot \First  & \True \\
    \tuple{                       0,  \False,\False, s_{\id{i}}, \Today} & \Second \Second \cdot\First \First s_{\id{a}} \cdot \First \Second s_{\id{i}} \cdot \First  & \True \\
    \tuple{                       0,  \False,\False, s_{\id{i}}, \Today} & \Second \cdot \First \Second s_{\id{i}} \cdot \First \First s_{\id{a}} \cdot \First  & \True \\
    \tuple{                       0,  \False,\False, s_{\id{i}}, \Today} & \First \Second s_{\id{i}} \cdot \Second \cdot \First \First s_{\id{a}} \cdot \First  & \True \\
    \tuple{                       0,  \False,\False, s_{\id{i}}, \Today} & \First \First s_{\id{a}} \cdot \Second \cdot \First \Second s_{\id{i}} \cdot \First  & \True \\
    \bottomrule
  \end{tabular}
\end{table}

In this section, we will demonstrate what symbolic execution looks like for this example, and how we generate next step hints from the symbolic execution results.
First, we call the simulate function $\simulate$ on the program, with an empty state.
The resulting set of symbolic executions is listed in \cref{table:tax}.
Each line represents one symbolic execution.
In the first column, the resulting symbolic value $\tilde{v}$ is listed.
The second column lists the symbolic input $\tilde{I}$ that was produced to arrive at that value, followed by the path condition $\Phi$ in the third column.
The symbolic values that are produced are $s_i$ for the invoice date and $s_a$ for the invoice amount.

The definition of $\Hints$ describes how these results should be used in order to calculate next step hints.
First of all, we need a goal $g$ to select the symbolic runs that we are interested in.
The most straight forward goal would be that we want to end up in a situation where we get a subsidy amount larger than zero.
This goal can be formulated as $g(\tuple{v,\_,\_,\_,\_}) = v > 0$.

The first six symbolic runs listed in \cref{table:tax} fulfill this goal condition.
From those runs, we then take the first symbolic input, together with the path condition conjugated with the goal.
After removing duplicates and redundant information, the result of $\Hints$ is as follows.
\begin{block}
  \begin{tabular}{LCL}
    \tuple{\First \First s_{\id{a}}  &, & \min(600, s_{\id{a}}/10) > 0}  \\
    \tuple{\First \Second s_{\id{i}} &, & (\Today-s_{\id{i}}) < \OneYear} \\
    \tuple{\Second \Left             &, & \True}
  \end{tabular}
\end{block}
This means that, at this stage, users have three possible options.\footnote{
  Note that the first branch, entering an amount, is denoted by $\First \First$;
  the second branch, entering the invoice date, is denoted by $\First \Second$;
  and the third branch, making a left/right choice, is denoted by $\Second$.
}
\begin{enumerate}
  \item The applicant may enter an amount $s_\id{a}$ for which $\min(600, s_{\id{a}}/10) > 0$ should hold.
  \item The applicant may enter an invoice date $s_\id{i}$ for which $(\Today-s_{\id{i}}) < \OneYear$ should hold.
  \item The company should take the left choice ($\Left$) to confirm they installed the solar panels.
\end{enumerate}


\subsection{Dining Computer Scientists}
\label{sub:assistive-dining}

%% Recall
Recall the example program Dining Computer Scientists from \cref{sec:dining}.
Three computer scientist sit at a table and have to coordinate how to their meals.
%% What hints do we want to give?
We want to calculate all possible next steps that lead to the goal.
The goal in this example is for all computer scientists to finish their meal.
%% Write the goal
In terms of the resulting task value, this means that we want to reach the value "Full bellies".
Witten as a predicate, we get $g(v) = v \equiv \text{"Full bellies"}$.

%% What does the result of H look like?
Let us assume that both Grace Hopper and Ada Lovelace have already picked up the forks to their left ($\lbl{fork1}$ and $\lbl{fork2}$ respectively).
We then find ourselves in the situation shown in \cref{fig:dining-middle}.

%what does the result of H look like?
Let us assume that both Grace Hopper and Ada Lovelace have already picked up the forks to their left (fork1 and fork2 respectively).
We then find ourselves in the following situation.

\begin{figure}
\begin{minipage}[r]{0.55\textwidth}
  \begin{align*}
  t =\ &\text{scientist}\ "\text{Alan Turing}"\ \text{fork0}\ \text{fork1} \And\\
      &\unit \Next \lambda \unit .\\
      &\Quad \If{!\text{fork2}}{\text{fork1} := \True}{\Fail} \And\\
      &\unit \Next \lambda \unit .\\
      &\Quad \If{!\text{fork0}}{\text{fork2} := \True}{\Fail} \Then\ \lambda \_.\\
      &\Quad \Edit "\text{Full bellies}"\\
  \sigma =\ &\{\text{fork0}\mapsto \True, \text{fork1}\mapsto \False,\text{fork2}\mapsto \False\}
  \end{align*}

\end{minipage}
\begin{minipage}[r]{0.05\textwidth}
  \
\end{minipage}
\begin{minipage}[r]{0.3\textwidth}
\tikzset{every picture/.style={line width=0.75pt}} %set default line width to 0.75pt

  \begin{tikzpicture}[x=0.75pt,y=0.75pt,yscale=-0.5,xscale=0.5]

%table
  \draw   (229,147.5) .. controls (229,90.89) and (274.89,45) .. (331.5,45) .. controls (388.11,45) and (434,90.89) .. (434,147.5) .. controls (434,204.11) and (388.11,250) .. (331.5,250) .. controls (274.89,250) and (229,204.11) .. (229,147.5) -- cycle ;

%plates
  \draw   (305,84) .. controls (305,70.19) and (316.19,59) .. (330,59) .. controls (343.81,59) and (355,70.19) .. (355,84) .. controls (355,97.81) and (343.81,109) .. (330,109) .. controls (316.19,109) and (305,97.81) .. (305,84) -- cycle ;
  \draw   (360,180) .. controls (360,166.19) and (371.19,155) .. (385,155) .. controls (398.81,155) and (410,166.19) .. (410,180) .. controls (410,193.81) and (398.81,205) .. (385,205) .. controls (371.19,205) and (360,193.81) .. (360,180) -- cycle ;
  \draw   (253,183) .. controls (253,169.19) and (264.19,158) .. (278,158) .. controls (291.81,158) and (303,169.19) .. (303,183) .. controls (303,196.81) and (291.81,208) .. (278,208) .. controls (264.19,208) and (253,196.81) .. (253,183) -- cycle ;

%forks
\draw    (375.33,247.64) -- (408.37,279.64) ;
\draw    (370.11,253.03) .. controls (385.17,266.22) and (393.52,257.6) .. (379.85,242.97) ;

\draw    (294.3,133.27) -- (256.15,107.56) ;
\draw    (298.49,127.05) .. controls (281.34,116.7) and (274.64,126.65) .. (290.66,138.66) ;

\draw    (366.16,44.59) -- (373.86,-0.76) ;
\draw    (373.55,45.85) .. controls (375.91,25.96) and (364.08,23.95) .. (359.75,43.5) ;

%names
  \draw (252,242) node  [align=left] {Alan};
  \draw (330,24) node  [align=left] {Grace};
  \draw (416,243) node  [align=left] {Ada};

  \end{tikzpicture}
\end{minipage}
\caption{Task, state and visual representation of dining computer scientists after two moves.}
\label{fig:dining-middle}
\end{figure}

Calling $\Hints(t,\sigma,g)$ will result in just one hint, namely
\begin{equation*}
  \tuple{\Second\Second\Continue,\True}
\end{equation*}
This means that the only step towards goal $g$ is for the third scientist,\footnote{
  The third branch is denoted by $\Second\Second$.
  The action $\Continue$ means pushing the continue button.
}
which is Ada Lovelace, to pick up the right fork.
Although it is also possible for Alan Turing to pick up the fork to his left,
this step is not a valid hint and performing this action will result in deadlock.

% !TEX root=../main.tex

\section{Properties}
\label{sec:properties}

In this section, we want to validate our approach by proving correctness.
For the hints function, which forms the heart of Assistive \TOPHAT, we want to prove that its results are both sound and complete.
Since the hints function relies on \STOPHAT, and more specifically, the simulate function,
we first prove correctness of simulate.


\subsection{Correctness of simulate}

The symbolic execution semantics is correct when all symbolic runs relate to a concrete run,
and the other way around, when all concrete runs are contained in the set of all symbolic executions.
These properties are defined as soundness and completeness respectively.

The simulate function can be considered a big-step symbolic execution semantics.
In order to prove certain properties with respect to the concrete semantics,
we need a concrete big-step concrete execution semantics.
Therefore the evaluate function is defined.

\begin{definition}[Evaluate]
  $t,\sigma\interact{I}^* v$ where

  \begin{minipage}[c]{0.4\textwidth}
    \begin{align*}
      t,\sigma             & \interact{i_1}  t_1,\sigma_1\\
      t_1,\sigma_1         & \interact{i_2}  t_2,\sigma_2\\
                           & \hspace{3mm}\vdots    \\
      t_{n-1},\sigma_{n-1} & \interact{i_n}  t_n,\sigma_n
    \end{align*}
\end{minipage}
\begin{minipage}[c]{0.1\textwidth}
  \Quad
\end{minipage}
\begin{minipage}[c]{0.3\textwidth}
  \begin{align*}
  \text{With }&\Value(t_n,\sigma_n)=v\\
  &\Value(t_{i<n},\sigma_{i<n})=\bot\\
  &I=i_1,\cdots,i_n
\end{align*}
\end{minipage}
\label{def:evaluate}
\end{definition}

Using the evaluate definition, we can state soundness and completeness for $\Simulate$ are listed below.

\begin{lemma}[Soundness of simulate]
  \label{lem:soundsimulate}
  For all tasks $t$ and states $\sigma$
  such that $t,\sigma\simulate\overline{\tilde{v},\tilde{I},\Phi}$
  then for all results $(\tilde{v},\tilde{I}=[\tilde{\imath}_0,\cdots,\tilde{\imath}_n],\Phi)$
  there exists a concrete input $I$ with the same length as the symbolic input $\tilde{I}$
  such that $t,\sigma\interact{I}^*v$
  with $[s_i\mapsto c_i]\tilde{v}=v$ and $[s_i\mapsto c_i]\Phi$
  where $s_i\in\tilde{i_i}$ and $c_i\in i_i$.
\end{lemma}

\begin{lemma}[Completeness of simulate]
  \label{lem:completesimulate}
  For all tasks $t$, states $\sigma$ and lists of input $I$
  such that $t,\sigma\interact{I}^*v$
  there exists a symbolic value $\tilde{v}$ and a symbolic input $\tilde{I}$ with the same length as $I$,
  such that $(\tilde{v},\tilde{I},\Phi)\in t,\sigma\simulate$,
  with $\tilde{i_i}\sim i_i$, $[s_i\mapsto c_i]\tilde{v}=v$ and $[s_i\mapsto c_i]\Phi$,
  where $s_i\in\tilde{i_i}$ and $c_i\in i_i$.
\end{lemma}

Where $\tilde{\imath}\sim i$ is defined as follows.

\begin{definition}[Input simulation]
  A symbolic input $\tilde{\imath}$ simulates a concrete input $i$ denoted as $\tilde{\imath}\sim i$ in the following cases.\\
  $s\sim a$, where $s$ is a symbol and $a$ a concrete action.\\
  $\tilde{\imath}\sim i\implies \First \tilde{\imath} \sim \First i$\\
  $\tilde{\imath}\sim i\implies \Second \tilde{\imath} \sim \Second i$
\end{definition}

And $s\in \tilde{\imath}$ and $c\in i$ are defined as follows.\\
\\
\noindent
\begin{minipage}[c]{0.5\textwidth}
  \begin{definition}[Value from input]\\
    $c\in \First i = c\in i $\\
    $c\in \Second i = c\in i $\\
    $c\in a = a $
  \end{definition}
\end{minipage}
\begin{minipage}[c]{0.5\textwidth}
  \begin{definition}[Symbol from input]\\
    $s\in \First \tilde{\imath} = s\in \tilde{\imath} $\\
    $s\in \Second \tilde{\imath} = s\in \tilde{\imath} $\\
    $s\in a = a $
  \end{definition}
\end{minipage}\\
\\

Our strategy to prove these two lemma's is outlined in \cref{fig:proofstructure}.

\begin{figure}[t]
  \tikzstyle{drive} = [decoration={markings,mark=at position
     1 with {\arrow[semithick]{angle 60}}},
     double distance=1.4pt, shorten >= 2.3pt,
     preaction = {decorate},
     postaction = {draw,line width=1pt, white,shorten >= 2.3pt}]
  \tikzstyle{sdrive} = [->,decorate, decoration={coil,aspect=0,amplitude=.5mm},
     double distance=1.4pt, shorten >= 2.3pt,]

\begin{tikzpicture}[
            > = stealth, % arrow head style
            shorten > = 1pt, % don't touch arrow head to node
            auto,
            node distance = 3cm, % distance between nodes
            semithick, % line style
        ]


        % j = 0
        \node (0)  {$t,\sigma$};
        \node (l0) [right of=0,xshift=1.4cm] {$t,\sigma\Consistent_{[\ ]} t,\sigma,\True$};

        % j = 1
        \node (c1) [below left of=0] {$t_1,\sigma_1$};
        \node (s1) [below right of=0] {$\tilde{t}_1,\tilde{\sigma}_1,\tilde{i}_1,\phi_1$};
        \node (l1) [right of=s1,text width=4cm] {\begin{tabular}{l}
        $t_1,\sigma_1\Consistent_{[s_1\mapsto c_1]}\tilde{t_1},\tilde{\sigma_1},\phi_1$\\
                                  $\Sat(\phi_1)$\\
                                  $\Value(t_1,\sigma_1)=\bot$\end{tabular}};

        %
        \node (cc) [below of=c1,yshift=1.5cm] {\vdots};
        \node (ss) [below of=s1,yshift=1.5cm] {\vdots};

        % j = k
        \node (ck) [below of=cc,yshift=1.5cm] {$t_k,\sigma_k$};
        \node (sk) [below of=ss,yshift=1.5cm] {$\tilde{t}_k,\tilde{\sigma}_k,\tilde{i}_k\phi_k$};
        \node (lk) [right of=sk,text width=4cm] {\begin{tabular}{l}
        $t_k,\sigma_k\Consistent_{[s_1\mapsto c_1,\cdots,s_k\mapsto c_k]}\tilde{t_k},\tilde{\sigma_k},\phi_1\land\cdots\land\phi_k$\\
        $\Sat(\phi_1\land\cdots\land\phi_k)$\\
        $\Value(t_k,\sigma_k)=\bot$\end{tabular}};

        %
        \node (ccc) [below of=ck,yshift=1.5cm] {\vdots};
        \node (sss) [below of=sk,yshift=1.5cm] {\vdots};

        \node (cn) [below of=ccc,yshift=1.5cm] {$t_n,\sigma_n$};
        \node (sn) [below of=sss,yshift=1.5cm] {$\tilde{t}_n,\tilde{\sigma}_n,\tilde{i}_n,\phi_n$};
        \node (l1) [right of=sn,text width=4cm] {\begin{tabular}{l}
        $t_n,\sigma_n\Consistent_{[s_1\mapsto c_1,\cdots,s_n\mapsto c_n]}\tilde{t_n},\tilde{\sigma_n},\phi_1\land\cdots\land\phi_n$\\
        $\Sat(\phi_1\land\cdots\land\phi_n)$\\
        $\Value(t_n,\sigma_n)=v$\Quad $\Value(\tilde{t}_n,\tilde{\sigma}_n)=\tilde{v}$\\
        $I=[i_1,\cdots,i_n]$\Quad $\tilde{I}=[\tilde{i}_1,\cdots,\tilde{i}_n]$\end{tabular}};

        \draw[drive] (0) to (c1);
        \node (0label) [below left of=0,yshift=1.3cm,xshift=0.8cm] {$i_1$};
        \draw[sdrive] (0) -- (s1);

        \draw[drive] (c1) to (cc);
        \draw[sdrive] (s1) -- (ss);

        \draw[drive] (cc) to (ck);
        \node (cclabel) [below left of=cc,yshift=1.5cm,xshift=1.8cm] {$i_k$};
        \draw[sdrive] (ss) -- (sk);

        \draw[drive] (ck) to (ccc);
        \draw[sdrive] (sk) -- (sss);

        \draw[drive] (ccc) to (cn);
        \node (cnlabel) [below left of=ccc,yshift=1.5cm,xshift=1.8cm] {$i_n$};
        \draw[sdrive] (sss) -- (sn);


        \path[dashed,->] ([yshift=.25em]c1.east) edge node {\cref{lem:completedriving}} ([yshift=.25em]s1.west);
        \path[dashed,->] ([yshift=-.25em]s1.west) edge node {\cref{lem:sounddriving}} ([yshift=-.25em]c1.east);

        \path[dashed,->] ([yshift=.25em]ck.east) edge node {\cref{lem:completedriving}} ([yshift=.25em]sk.west);
        \path[dashed,->] ([yshift=-.25em]sk.west) edge node {\cref{lem:sounddriving}} ([yshift=-.25em]ck.east);

        \path[dashed,->] ([yshift=.25em]cn.east) edge node {\cref{lem:completesimulate}} ([yshift=.25em]sn.west);
        \path[dashed,->] ([yshift=-.25em]sn.west) edge node {\cref{lem:soundsimulate}} ([yshift=-.25em]cn.east);


    \end{tikzpicture}
    \caption{Proof structure}
      \label{fig:proofstructure}
\end{figure}

At the top, we start out with any task $t$ and state $\sigma$.
The left side of the diagram is an overview of the evaluate function.
Inputs $i_1$ until $i_n$ are sequentially applied, until the task has an observable value.

On the right side, symbolic execution is performed.
One step of the driving semantics is taken, which results in a symbolic task, state, input
and a path condition.
Provided that the path condition holds, driving is executed sequentially until the symbolic task has an observable symbolic value.

Proving soundness and completeness of simulate now comes down to relating the left and right side of the diagram.
From symbolic to concrete, right to left, is soundness, as stated in \cref{lem:soundsimulate}.
From concrete to symbolic, left to right, is completeness, as stated in \cref{lem:soundsimulate}.

Since simulate and evaluate rely on the (symbolic) handling semantics,
we prove soundness and completeness of those semantics first.
Looking at \cref{fig:proofstructure}, there are two different settings in which the (symbolic) handling semantics are applied.
At the top, both symbolic and concrete execution start out with the same task and state.
But further down, the task and state differ for both semantics.
The task and state are related to each other however.
Symbolic execution introduces symbols, the concrete semantics handles concrete values.
This relation is expressed by the consistence relation listed in \cref{def:consistence}.

\begin{definition}[Consistence relation $\Consistent$]
  \label{def:consistence}
A concrete task $t$ and concrete state $\sigma$
are considered to be consistent with a symbolic task $\tilde{t}$, symbolic state $\tilde{\sigma}$ and path condition $\Phi$
under a certain mapping $M=[s_1\mapsto c_1,\cdots,s_n,\mapsto c_n]$, denoted as $t,\sigma \Consistent_M \tilde{t},\tilde{\sigma},\Phi$
if and only if $M\tilde{t}=t$, $M\tilde{\sigma}=\sigma$ and $M\Phi$
\end{definition}

Now \cref{lem:sounddriving} and \cref{lem:completedriving} express soundness and completeness of driving respectively.

\begin{lemma}[Soundness of driving]
  \label{lem:sounddriving}
  For all concrete tasks $t$, concrete states $\sigma$, symbolic tasks $\tilde{t}$, symbolic states $\tilde{\sigma}$ path conditions $\Phi$ and mappings $M$,
  we have that $t,\sigma\Consistent_M\tilde{t},\tilde{\sigma},\Phi$ implies
  that for all pairs $(\tilde{t}',\tilde{\sigma}',\tilde{\imath},\phi)$ in $\tilde{t},\tilde{\sigma}\interact{}\overline{\tilde{t}',\tilde{\sigma}',\tilde{\imath},\phi}$,
  $\Sat(\Phi\land\phi)$ implies that there exists an input $i$ such that $\tilde{\imath}\sim i$,  $t,\sigma\interact{i}t',\sigma'$ and $t',\sigma' \Consistent_{M.[s\mapsto c]} \tilde{t}',\tilde{\sigma}',\Phi\land\phi$ where where $s\in\tilde{\imath}$ and $c\in i$.
\end{lemma}

\begin{lemma}[Completeness of driving]
  \label{lem:completedriving}
  For all concrete tasks $t$, concrete states $\sigma$, symbolic tasks $\tilde{t}$, symbolic states $\tilde{\sigma}$ path conditions $\Phi$ and mappings $M$,
  we have that $t,\sigma\Consistent_{M}\tilde{t},\tilde{\sigma},\Phi$ implies
  that for all inputs $i$ such that $t,\sigma\interact{i}t',\sigma'$,
  there exists a symbolic input $\tilde{\imath}$, $\tilde{\imath}\sim i$ such that
  $\tilde{t},\tilde{\sigma}\interact{}\overline{\tilde{t}',\tilde{\sigma}',\tilde{\imath},\phi}$, $\Sat(\Phi\land\phi)$ and $t',\sigma'\Consistent_{M.[s\mapsto c]}\tilde{t}',\tilde{\sigma}',\Phi\land\phi$ where where $s\in\tilde{\imath}$ and $c\in i$.
\end{lemma}

In other words, if a symbolic and concrete task and state are related, they will still be related after (symbolic) handling.

The top case, where both the symbolic and concrete semantics start out with the same task and state, can be seen as a special case of the consistence relation.
Obviously a task and state are consistent with themselves, using the empty mapping and the path condition $\True$.

The full proof of all four lemma's is listed in the appendix.


\subsection{Correctness of hints}

Now that soundness and completeness of simulate has been proven, we can prove that our hints function is correct.
We consider our function to be correct when it produces correct hints.
Intuitively, for a set of next step hints to be correct, they should adhere to the following requirements.

\begin{itemize}
  \item valid steps a user can actually take.
  \item bring the user closer to the desired goal.
  \item contain ALL steps that bring the user closer to their goal.
\end{itemize}

We separate these requirements into two lemma's, namely soundness and completeness.

\begin{theorem}[Soundness of hints]
  \label{thm:soundhint}

For all tasks $t$, states $\sigma$ and goals $g$,
for every next step hint $(\tilde{\imath},\Phi)$ in $\Hints(t,\sigma,g)$,
there exists a sequence of inputs $I$ and input $i$ such that $\tilde{\imath}\sim i$,
$\Sat([s\mapsto c]\Phi)$, $t,\sigma\interact{i} t',\sigma'\interact{I}^* v$ and $gv$.
\end{theorem}

\begin{theorem}[Completeness of hints]
  \label{thm:completehint}
  For all tasks $t$, states $\sigma$, lists of input $(i:is)$ and goals $g$,
  if $t,\sigma,\interact{i:is}^*v$ and $g v$, then there exists a symbolic input $\tilde{\imath}$ and path condition $\Phi$
  such that $(\tilde{\imath},\Phi)\in\Hints(t,\sigma)$ with $\tilde{\imath}\sim i$ and $\Sat([s\mapsto c]\Phi)$ with $c\in i$ and $s\in i$.
\end{theorem}


The proofs of these two threorems are quite straight forward.

\begin{proof}[\cref{thm:soundhint}]
  \cref{thm:soundhint} follows from the definition of $\Hints$ and \cref{lem:soundsimulate} as follows.

  The definition of $\Hints$ gives us that for every pair $(\tilde{\imath},\Phi)$ produced by $\Hints$,
  there exists a pair $(\tilde{v},\tilde{\imath}:\tilde{is},\Phi)$ with $\Sat(\Phi\land g \tilde{v})$.
  Then by \cref{lem:soundsimulate} we have that there exists a sequence of concrete inputs $I$ such that
  $t,\sigma\interact{I}^*v$ and $g v$.
\end{proof}


\begin{proof}[\cref{thm:completehint}]
  In order to prove that $i$ is contained in $\Hints(t,\sigma)$, we need to show that $(\tilde{v},\tilde{\imath}:\tilde{is},\Phi)\in t,\sigma\interact{}^*$, with $\tilde{\imath}\sim i$ and $\Sat([s_0\mapsto c_0,\cdots,s_n\mapsto c_n]\land g\tilde{v})$, where $[c_0,\cdots,c_n]\in i:is$ and $[s_0,\cdots,s_n]\in \tilde{\imath}:\tilde{is}$.

  By \cref{lem:completesimulate}, we directly obtain that this indeed exists. Therefore we know that $\tilde{\imath}$ and $\Phi$ exist.
\end{proof}

% % !TEX root=../main.tex

\section{Implementation}
\label{sec:implementation}

% !TEX root=../main.tex

\section{Related work}
\label{sec:relatedwork}

In previous work, we have attempted to provide end-users with next step hints by viewing workflows as rule based problems~\cite{DBLP:conf/sfp/NausJ16}.
By abstracting over workflows, reasoning about them becomes simpler.
A standard search algorithm can be run to find a path to the desired goal state.
Two drawbacks of this approach however are that only very general hints can be given, that range over multiple steps, and that a programmer needs to augment existing workflows with extra information in order to convert it to a rule-based problem.

Stutterheim et al.~\cite{DBLP:conf/sfp/StutterheimPA14} have developed Tonic, a task visualiser for iTasks with limited path prediction capabilities.
The main goal is not to provide hints to end-users, but the system is able to handle the complete task language, and visualise the effects of user input on the progression of tasks.

In order to overcome the problems of our own previous research and the limited use of Tonic for end-user hints, we have combined symbolic execution, together with workflow modelling and next-step hint generation.
To our knowledge, this is the first work describing the combination of these techniques in this way.
The different components coming together in this paper have been studied extensively.
The following sections give an overview of the work done in those areas.

\subsection{Symbolic execution}


Symbolic execution \cite{King1975,Boyer1975} is typically being applied to imperative programming languages,
but in recent years it has been used for functional programming languages as well.
Ongoing work by Hallahan et al.~\cite{HallahanXP2017} and Xue~\cite{Xue2019} aims to implement a symbolic execution engine for Haskell.
Giantios et al.~\cite{GiantsiosPS2017} use symbolic execution for a mix of concrete and symbolic testing of programs written in a subset of Core Erlang.
Their goal is to find executions that lead to a runtime error, either due to an assertion violation or an unhandled exception.
Chang et al.~\cite{ChangKT2018} present a symbolic execution engine for a typed lambda calculus with mutable state where only some language constructs recognise symbolic values.
They claim that their approach is easier to implement than full symbolic execution and simplifies the burden on the solver, while still considering all execution paths.

\subsection{Workflow modelling}

Workflow modelling have been studied extensively from different viewpoints.
Since many software exists that automates workflows, it is a research topic that potentially has a huge impact on society.

\emph{Workflow patterns} are regarded as special design patterns in software engineering.
Similar to the combinators in \TOP, they describe recurring patterns in workflow systems.
Van der Aalst et al.~\cite{journals/dpd/AalstHKB03} identify common patterns, and examine their availability in industry workflow frameworks.

\emph{Workflow Nets} allow for the modelling an analysis of business processes~\cite{DBLP:journals/jcsc/Aalst98}.
Worflow Nets are a subclass of Petri nets, and are therefore graphical in nature.
Research on Workflow Nets includes verification of models~\cite{DBLP:conf/apn/Aalst97} and complexity analysis~\cite{DBLP:journals/infsof/LassenA09}, just to name a few.


\emph{iTasks}~\cite{DBLP:conf/ppdp/PlasmeijerLMAK12} is an implementation of \TOP in the programming language Clean.
It differs from the above mentioned modelling techniques, since it is not graphical in nature.
iTasks supports higher order workflows, and leverages techniques from functional and generic programming.


\subsection{Automatic hint generation in intelligent tutoring systems}

The intelligent tutoring systems (ITS) research community is very large.
Work that is most relevant to our own is the research into automatic hint generation.
More traditional ITS rely heavily on experts to write dedicated hints for every specific case of an exercise.
Automatic hint generation attempts to overcome this burden by calculating a hint rather than having every case specified.

Heeren et al.~\cite{DBLP:journals/scp/HeerenJ14} develop a framework for so called domain reasoners that allow for automatic hint generation.
Feedback is calculated automatically from a high-level description of an exercise class.
Their approach is applicable to domains like logic, mathematics and linear algebra.
Paquette et al.~\cite{DBLP:conf/its/PaquetteLBM12} present a different automatic next-step hint ITS, that is used to provide hints to students in a programming exercise.

Based on the work mentioned above by Heeren et al., an ITS for Haskell exercises has been developed by Gerdes et al.~\cite{DBLP:journals/aiedu/GerdesHJB17}.
It tuns out that programming exercises is a popular area for Automatic hint generation.
Keuning et al.~\cite{DBLP:journals/jeric/KeuningJH19} have written an excellent literature study of this research area.

% !TEX root=../main.tex

\section{Conclusion}
\label{sec:conclusion}


%% Acknowledgements
\section*{Acknowledgements}
% !TEX root=../main.tex

This research is supported by the Dutch Technology Foundation STW, which is part
of the Netherlands Organisation for Scientific Research (NWO), and which is
partly funded by the Ministry of Economic Affairs.


%% Bibliography
% BibTeX users should specify bibliography style 'splncs04'.
% References will then be sorted and formatted in the correct style.
\bibliographystyle{splncs04}
\bibliography{bibliography}

%% Appendix
\newpage
\appendix

% !TEX root=../main.tex

\section{Complete syntax}

\begin{figure}[h]
  \small
  \usemacro{G-Language-Compact}
  \caption{Language grammar} \label{fig:language-grammar}
\end{figure}

\begin{figure}[h]
  \small
  \usemacro{G-Pretasks-Compact}
  \caption{Task grammar} \label{fig:task-grammar}
\end{figure}


\begin{figure}[h]
  \small
  \usemacro{G-Types-Compact}
  \caption{Type grammar} \label{fig:type-grammar}
\end{figure}

\begin{figure}[h]
  \small
  \usemacro{G-Values-Compact}
  \caption{Value grammar} \label{fig:value-grammar}
\end{figure}

% !TEX root=../main.tex

\section{\TOPHAT semantics}

\subsection{Typing rules}

\begin{gather*}
  \boxed{\RelationT} \Break
  \userule{T-ConstBool} \Quad
  \userule{T-ConstInt} \Quad
  \userule{T-ConstString} \Quad
  \userule{T-Unit}\Break
  \userule{T-Var} \Quad
  \userule{T-Loc} \Quad
  \userule{T-Pair} \Break
  \userule{T-First} \Quad
  \userule{T-Second} \Quad
  \userule{T-ListEmpty} \Break
  \userule{T-ListCons} \Quad
  \userule{T-ListHead} \Quad
  \userule{T-ListTail}\Break
  \userule{T-Abs} \Quad
  \userule{T-App} \Break
  \userule{T-If} \Quad
  \userule{T-Ref} \Break
  \userule{T-Deref} \Quad
  \userule{T-Assign}\Break
  \userule{T-Edit}\Quad
  \userule{T-Enter}\Quad
  \userule{T-Update}\Break
  \userule{T-Fail}\Quad
  \userule{T-Then}\Quad
  \userule{T-Next}\Break
  \userule{T-And}\Quad
  \userule{T-Or}\Quad
  \userule{T-Xor}
\end{gather*}

\subsection{Evaluation rules}

\begin{gather*}
  \boxed{\RelationE} \Break
  \userule{E-App} \Break
  \userule{E-IfTrue} \Quad
  \userule{E-Ref} \Break
  \userule{E-IfFalse} \Quad
  \userule{E-Deref} \Quad
  \userule{E-Value} \Break
  \userule{E-Assign} \Quad
  \userule{E-Pair} \Break
  \userule{E-First} \Quad
  \userule{E-Second}\Quad
  \userule{E-Cons}\Break
  \userule{E-Head}\Quad
  \userule{E-Tail}\Quad
  \userule{E-Edit} \Break
  \userule{E-Update} \Quad
  \userule{E-Then} \Quad
  \userule{E-Next} \Break
  \userule{E-And} \Quad
  \userule{E-Or}
\end{gather*}

\subsection{Striding rules}

\begin{gather*}
  \boxed{\RelationS} \Break
  \userule{S-ThenStay} \Break
  \userule{S-ThenFail} \Break
  \userule{S-ThenCont}\Break
  \userule{S-OrLeft} \Break
  \userule{S-OrRight} \Break
  \userule{S-OrNone}\Break
  \userule{S-Edit} \Quad \userule{S-Fill} \Quad \userule{S-Update} \Break
  \userule{S-Fail} \Quad \userule{S-Xor}\Quad
  \userule{S-Next} \Break
  \userule{S-And}
\end{gather*}

\subsection{Normalisation rules}

\begin{gather*}
  \boxed{\RelationN} \Break
  \userule{N-Done} \Break
  \userule{N-Repeat}
\end{gather*}

\subsection{Handling rules}

\begin{gather*}
  \boxed{\RelationH} \Break
  \userule{H-Change} \Quad
  \userule{H-Fill} \Break
  \userule{H-Update}\Quad
  \userule{H-Next} \Break
  \userule{H-PickLeft} \Quad
  \userule{H-PickRight}\Break
  \userule{H-PassThen} \Quad
  \userule{H-PassNext} \Quad
  \userule{H-FirstAnd} \Break \userule{H-SecondAnd} \Quad
  \userule{H-FirstOr}  \Quad \userule{H-SecondOr}
\end{gather*}


\subsection{Driving rules}

\begin{gather*}
  \boxed{\RelationI} \Break
  \userule{I-Handle}
\end{gather*}

% !TEX root=../main.tex

\section{Complete symbolic semantics}

\subsection{Symbolic evaluation rules}
\label{sec:symbolic-evaluation-rules}

\begin{gather*}
  % \small
  \boxed{\RelationSE} \Break
  \userule{SE-Value}\Quad
  \userule{SE-Pair} \Break
  \userule{SE-First}\Quad
  \userule{SE-Second}\Break
  \userule{SE-Cons}\Quad
  \userule{SE-Head}\Break
  \userule{SE-Tail}\Break
  \userule{SE-App} \Break
  \userule{SE-If} \Break
  \userule{SE-Ref} \Quad
  \userule{SE-Deref} \Break
  \userule{SE-Assign} \Quad
  \userule{SE-Edit} \Break
  \userule{SE-Enter}\Quad
  \userule{SE-Update}\Break
  \userule{SE-Then}\Quad
  \userule{SE-Next}\Break
  \userule{SE-And}\Quad
  \userule{SE-Or} \Break
  \userule{SE-Xor}\Quad
  \userule{SE-Fail}
\end{gather*}

\subsection{Symbolic striding rules}

\begin{gather*}
  % \small
  \boxed{\RelationSS} \Break
  \userule{SS-ThenStay} \Break
  \userule{SS-ThenFail} \Break
  \userule{SS-ThenCont} \Break
  \userule{SS-OrLeft} \Break
  \userule{SS-OrRight} \Break
  \userule{SS-OrNone} \Break
  \userule{SS-Edit} \Quad
  \userule{SS-Fill} \Quad
  \userule{SS-Update} \Break
  \userule{SS-Fail} \Quad
  \userule{SS-Xor} \Break
  \userule{SS-Next} \Quad
  \userule{SS-And}
\end{gather*}

\subsection{Symbolic normalisation rules}

 \begin{gather*}
   % \small
   \boxed{\RelationSN} \Break
   \userule{SN-Done} \Break
   \userule{SN-Repeat}
 \end{gather*}


\subsection{Symbolic handling rules}
\label{sec:symbolic-handling-rules}

 \begin{gather*}
   % \small
   \boxed{\RelationSH} \Break
   \userule{SH-Change} \Quad
   \userule{SH-Fill} \Break
   \userule{SH-Update}\Quad
   \userule{SH-PassNext} \Break
   \userule{SH-PassNextFail}\Break
   \userule{SH-Next} \Break
   \userule{SH-PassThen} \Break
   \userule{SH-Pick} \Break
   \userule{SH-PickLeft} \Break
   \userule{SH-PickRight}\Break
   \userule{SH-And}\Break
   \userule{SH-Or}
\end{gather*}

\subsection{Symbolic driving rules}

\begin{gather*}
  % \small
  \boxed{\RelationSI}\Break
  \userule{SI-Handle}
\end{gather*}

% !TEX root=../main.tex

\section{Soundness proofs}
\label{sec:soundnessproofs}


\begin{proof}[Soundness of simulate]
  The structure of this proof is outlined in \cref{fig:proofstructure}.

  We have $t$ and $\sigma$ such that $t,\sigma\interact{}^*\overline{\tilde{v},\tilde{I},\Phi}$.
  By definition of simulation $(\interact{}^*)$, we know that for each tuple $(\tilde{v},\tilde{I},\Phi)$,
  the following sequence of symbolic drive steps has occurred.
  \begin{align*}
      t,\sigma\interact{}&\tilde{t}_1,\tilde{\sigma}_1,\tilde{\imath}_1,\phi_1&\\
                      &\tilde{t}_1,\tilde{\sigma}_1\interact{}&\tilde{t}_2,\tilde{\sigma}_2,\tilde{\imath}_2,\phi_2\\
                      &                                    &\tilde{t}_2,\tilde{\sigma}_2\interact{}&\cdots&\\
                      &                                    &                                    &\cdots&
                      \interact{}\tilde{t}_n,\tilde{\sigma}_n,\tilde{\imath}_n,\phi_n
  \end{align*}

  with $\Value(\tilde{t}_n,\tilde{\sigma}_n)=\tilde{v}$ and $\Sat(\phi_1\land\cdots\land\phi_n)$.

  We need to show that there exits an $I$ such that $t,\sigma\interact{I}^*v$, which is defined similarly as follows.

  $t,\sigma\interact{i_1}t_1,\sigma_1\interact{i_2}t_2,\sigma_2\interact{i_3}\cdots \interact{i_n}t_n,\sigma_n$ with $\Value(t_n,\sigma_n)$.

  By \cref{lem:sounddriving}, we know that $t,\sigma\interact{i_1}t_1,\sigma_1$ exists, since $t,\sigma\Consistent_{\emptyset}t,\sigma,\True$.
  This also gives us that $\tilde{i_1}\sim i_1$, and $t_1,\sigma_1\Consistent_{[s_1\mapsto c_1]}\tilde{t}_1,\tilde{\sigma}_1,\phi_1$ with $s_1\in\tilde{\imath}_1$ and $c_1\in i_1$.

  By repeatedly applying \cref{lem:sounddriving}, until we arrive at $\tilde{t}_n,\sigma{t}_n$,
  we can show that there indeed exists an $I$ such that $t,\sigma\interact{I}^*v$ with $[s_1\mapsto c_1,\cdots,s_n\mapsto c_n]\tilde{v}=v$
  and $[s_1\mapsto c_1,\cdots,s_n\mapsto c_n]\Phi$, namely $I=[i_1,\cdots,i_n]$.

\end{proof}






\begin{proof}[Soundness of driving]
  The symbolic driving semantics consists of only one rule, \refrule{SI-Handle}.
  Given that $t,\sigma\Consistent_M\tilde{t},\tilde{\sigma},\Phi$ and $\tilde{t},\tilde{\sigma}\interact{}\overline{\tilde{t}',\tilde{\sigma}',\tilde{\imath},\phi_1}$,
  \cref{lem:soundhandle} gives us that for each pair $(\tilde{t}',\tilde{\sigma}',\tilde{\imath},\phi_1)$
  there exists an input $i$ such that $\tilde{\imath}\sim i$, $t,\sigma\handle{i}t',\sigma'$
  and $t',\sigma'\Consistent_{M.[s\mapsto c]}\tilde{t'},\tilde{\sigma'},\Phi\land\phi_1$.

  Then, by \cref{lem:soundnorm}, given that $\tilde{t'},\tilde{\sigma'}\normalise\overline{\tilde{t''},\tilde{\sigma''}',\phi_2}$,
  we obtain that for each pair $(\tilde{t''},\tilde{\sigma''}',\phi_2)$, we have that $\Sat(\Phi\land\phi_1\land\phi_2)$ implies
  that $t',\sigma'\hat{\normalise}t'',\sigma''$ with $t'',\sigma''\Consistent_{M.[s\mapsto c]}\tilde{t''},\tilde{\sigma''},\Phi\land\phi_1\land\phi_2$.
\end{proof}


\begin{lemma}[Soundness of handling]
  \label{lem:soundhandle}

  For all concrete tasks $t$, concrete states $\sigma$, symbolic tasks $\tilde{t}$, symbolic states $\tilde{\sigma}$ path conditions $\Phi$ and mappings $M$,
  we have that $t,\sigma\Consistent_{M}\tilde{t},\tilde{\sigma},\Phi$ implies
  that for all symbolic inputs $\tilde{\imath}$ such that $\tilde{t},\tilde{\sigma}\handle{}\overline{\tilde{t}',\tilde{\sigma}',\tilde{\imath},\phi}$ and
  for all pairs $(\tilde{t}',\tilde{\sigma}',\tilde{\imath},\phi)$,
  $\Sat(\Phi\land\phi)$ implies that there exists an input $i$ such that $\tilde{\imath}\sim i$,  $t,\sigma\handle{i}t',\sigma'$ and $t',\sigma'\Consistent_{M.[s\mapsto c]}\tilde{t}',\tilde{\sigma}',\Phi\land\phi$ where where $s\in\tilde{\imath}$ and $c\in i$.

\end{lemma}



\begin{lemma}[Soundness of normalisation]
  \label{lem:soundnorm}
  For all concrete expressions $e$, concrete states $\sigma$, symbolic expressions $\tilde{e}$, symbolic states $\tilde{\sigma}$ path conditions $\Phi$ and mappings $M$,
  we have that $e,\sigma\Consistent_{M}\tilde{e},\tilde{\sigma},\Phi$ implies
  that if $\tilde{e},\tilde{\sigma}\simnormalise\overline{\tilde{t},\tilde{\sigma}',\phi}$,
  then for all pairs $(\tilde{t},\tilde{\sigma}',\phi)$ it holds that $\Sat(\Phi\land\phi)$ implies
  that $e,\sigma\normalise t,\sigma'$ with $t,\sigma'\Consistent_{M}\tilde{t},\tilde{\sigma'},\Phi\land\phi$.
\end{lemma}

\begin{lemma}[Soundness of striding]
  \label{lem:soundstride}
  for all concrete tasks $t$, concrete states $\sigma$, symbolic tasks $\tilde{t}$, symbolic states $\tilde{\sigma}$ path conditions $\Phi$ and mappings $M$,
  we have that $t,\sigma\Consistent_{M}\tilde{t},\tilde{\sigma},\Phi$ implies
  that if $\tilde{t},\tilde{\sigma}\stride\overline{\tilde{t}',\tilde{\sigma}',\phi}$,
  then for all pairs $(\tilde{t}',\tilde{\sigma}',\phi)$ it holds that $\Sat(\Phi\land\phi)$ implies
  that $t,\sigma\hat{\stride}t',\sigma'$ with $t',\sigma'\Consistent_{M}\tilde{t'},\tilde{\sigma'},\Phi\land\phi$.
\end{lemma}

\begin{lemma}[Soundness of evaluation]
  \label{lem:soundeval}
  For all concrete expressions $e$, concrete states $\sigma$, symbolic expressions $\tilde{e}$, symbolic states $\tilde{\sigma}$ path conditions $\Phi$ and mappings $M$,
  we have that $e,\sigma\Consistent_{M}\tilde{e},\tilde{\sigma},\Phi$ implies
  that if $\tilde{e},\tilde{\sigma}\eval\overline{\tilde{v},\tilde{\sigma}',\phi}$,
  then for all pairs $(\tilde{v},\tilde{\sigma}',\phi)$ it holds that $\Sat(\Phi\land\phi)$ implies
  that $e,\sigma\hat{\eval}v,\sigma'$ with $v,\sigma'\Consistent_{M}\tilde{v},\tilde{\sigma'},\Phi\land\phi$.
\end{lemma}


\begin{proof}[Soundness of handle]

  We prove \cref{lem:soundhandle} by induction over $\tilde{t}$.

  \Case{$\tilde{t}=\Enter \tau$}
 {One rule applies, namely \userule{SH-Fill}\\
 Since we have $t,\sigma\Consistent_M\Enter \tau,\tilde{\sigma},\Phi$, we know that $t$ must be $\Enter \tau$ too, $\tilde{t}$ contains no symbols.
 There exists only one symbolic execution, namely $\Enter \tau,\tilde{\sigma}\handle{}\Edit s,\tilde{\sigma},s,\True$.
 We need to show that there exists an $i$ such that $s\sim i$ and $\Edit v,\sigma\handle{i}t',\sigma'$.

 Any concrete value $c$ of type $\tau$ will do. Now we have to show that we end up with $\Edit c,\sigma\Consistent_{M.[s\mapsto c]}\Edit s,\tilde{\sigma},\Phi\land\True$, which holds trivially.
 }

  \Case{$\tilde{t}=\Edit \tilde{v}$}
  {One rule applies, namely \userule{SH-Change}\\
  Since we have $t,\sigma\Consistent_M\Edit \tilde{v},\tilde{\sigma},\Phi$, we know that either $\tilde{v}$ is a concrete value, or $M$ contains a mapping such that $M\tilde{v}$ becomes a concrete value $c$. We know therefore that $t$ must be $\Edit c$.

  There exists only one symbolic execution, namely $\Edit \tilde{v},\tilde{\sigma}\handle{}\Edit s,\tilde{\sigma},s,\True$.
  We need to show that there exists an $i$ such that $s\sim i$ and $\Edit c,\sigma\handle{i}t',\sigma'$.

  Any concrete value $c'$ of the same type as $c$ will do. Now we have to show that we end up with $\Edit c',\sigma\Consistent_{M.[s\mapsto c']}\Edit s,\tilde{\sigma},\Phi\land\True$, which holds trivially.
  }


\Case{$\tilde{t}=\Update l$}
{One rule applies, namely \userule{SH-Update}\\

Since we have $t,\sigma\Consistent_M\Update l,\tilde{\sigma},\Phi$, we know that $t$ must be $\Update l$ too, $\tilde{t}$ contains no symbols.
There exists only one symbolic execution, namely $\Update l,\tilde{\sigma}\handle{}\Update l,\tilde{\sigma}[l\mapsto s],s,\True$.
We need to show that there exists an $i$ such that $s\sim i$ and $\Update l,\sigma\handle{i}t',\sigma'$.

Any concrete value $c$ of the same type as $l$ will do. Now we have to show that we end up with $\Update l,\sigma[l\mapsto c]\Consistent_{M.[s\mapsto c]}\Update l,\tilde{\sigma}[l\mapsto s],\Phi\land\True$, which holds trivially.
}


\Case{$\tilde{t}=\tilde{t}_1\Next \tilde{e}_2$}
 {
Since we have $t,\sigma\Consistent_M\tilde{t}_1\Next \tilde{e}_2,\tilde{\sigma},\Phi$, we know that $M \tilde{t}_1\Next \tilde{e}_2=t$, which comes down to $t_1\Next e_2$ for some concrete $t_1$ and $e_2$.

 In this case, three rules apply.\\

 \Case{\userule{SH-Next}}

{
In this case, we have two sets of symbolic executions.

For all tuples $(\tilde{t}_1'\Next \tilde{e}_2,\tilde{\sigma}_1',\tilde{\imath},\phi_1)$, we know by application of the induction hypothesis that
there exits an $i$ such that $\tilde{\imath}\sim i$, $t_1,\sigma\handle{i}t_1',\sigma'$ and
$t_1',\sigma'\Consistent_{M.[s\mapsto c]}\tilde{t}_1',\tilde{\sigma}',\Phi\land\phi_1$ where $c\in i$ and $s\in \tilde{\imath}$.
Therefore we also have $t_1'\Next e_2,\sigma_1'\Consistent_{M.[s\mapsto c]}\tilde{t}_1'\Next \tilde{e}_2,\tilde{\sigma}_1',\Phi\land\phi_1$.

For all tuples $(\tilde{t}_2,\tilde{\sigma}_2',\Continue,\phi_2)$, we first have by \cref{lem:valpres} that
$v_1,\sigma\Consistent_M\tilde{v}_1,\tilde{\sigma},\Phi$.
Now, before we can apply \cref{lem:soundnorm}, we need to establish that
$e_2\ v_1,\sigma\Consistent_M\tilde{e}_2\ \tilde{v}_1,\tilde{\sigma},\Phi$ holds.
This means that we have to show that $M \tilde{e}_2\ \tilde{v}_1 = e_2\ v_1$.
Since application of the mapping is distributive, it suffices to show that $M\tilde{v}_1=v_1$, which is given,
and $M\tilde{e}_2=e_2$, which follows from the premise as well.

At this point, by application of \cref{lem:soundnorm}, we obtain that $e_2\ v_1,\sigma\normalise t_2,\sigma_2'$
and $t_1,\sigma_2'\Consistent_M\tilde{t}_2,\tilde{\sigma}_2',\Phi\land\phi_2$
}
%
\Case{\userule{SH-PassNext}}
{
For all tuples $(\tilde{t}_1'\Next \tilde{e}_2,\tilde{\sigma}_1',\tilde{\imath},\phi_1)$, we know by application of the induction hypothesis that
there exits an $i$ such that $\tilde{\imath}\sim i$, $t_1,\sigma\handle{i}t_1',\sigma'$ and
$t_1',\sigma'\Consistent_{M.[s\mapsto c]}\tilde{t}_1',\tilde{\sigma}',\Phi\land\phi_1$ where $c\in i$ and $s\in \tilde{\imath}$.
Therefore we also have $t_1'\Next e_2,\sigma_1'\Consistent_{M.[s\mapsto c]}\tilde{t}_1'\Next \tilde{e}_2,\tilde{\sigma}_1',\Phi\land\phi_1$.
}

\Case{\userule{SH-PassNextFail}}
{
For all tuples $(\tilde{t}_1'\Next \tilde{e}_2,\tilde{\sigma}_1',\tilde{\imath},\phi_1)$, we know by application of the induction hypothesis that
there exits an $i$ such that $\tilde{\imath}\sim i$, $t_1,\sigma\handle{i}t_1',\sigma'$ and
$t_1',\sigma'\Consistent_{M.[s\mapsto c]}\tilde{t}_1',\tilde{\sigma}',\Phi\land\phi_1$ where $c\in i$ and $s\in \tilde{\imath}$.
Therefore we also have $t_1'\Next e_2,\sigma_1'\Consistent_{M.[s\mapsto c]}\tilde{t}_1'\Next \tilde{e}_2,\tilde{\sigma}_1',\Phi\land\phi_1$.
}
}

\Case{$\tilde{t}=\tilde{t}_1\Then \tilde{e}_2$}
{One rule applies, namely \userule{SH-PassThen}\\
For all tuples $(\tilde{t}_1'\Then\tilde{e}_2,\tilde{\sigma}',\tilde{\imath},\phi)$, we know by application of the induction hypothesis that
there exists an $i$ such that $\tilde{\imath}\sim i$, $t_1,\sigma\handle{1}t_1',\sigma'$ and
$t_1',\sigma'\Consistent_{M.[s\mapsto c]}\tilde{t}_1',\tilde{\sigma}',\Phi\land\phi_1$ where $c\in i$ and $s\in \tilde{\imath}$.
Therefore we also have $t_1'\Then e_2,\sigma_1'\Consistent_{M.[s\mapsto c]}\tilde{t}_1'\Then \tilde{e}_2,\tilde{\sigma}_1',\Phi\land\phi_1,M.[s\mapsto c])$.
}

\Case{$\tilde{t}=\tilde{e}_1\Xor \tilde{e}_2$}
 {
In this case, three rules apply.\\
   \Case{\userule{SH-Pick}}
   {
   In this case, we have two sets of symbolic executions.

   For all tuples $(\tilde{t}_1,\tilde{\sigma}_1,\Left,\phi_1)$,
   we obtain from \cref{lem:soundnorm} that $e_1,\sigma\normalise t_1,\sigma_1$ with
   $t_1,\sigma_1\Consistent_M\tilde{t}_1,\tilde{\sigma}_1,\Phi\land\phi_1$.

   For all tuples $(\tilde{t}_2,\tilde{\sigma}_2,\Right,\phi_2)$,
   we obtain from \cref{lem:soundnorm} that $e_2,\sigma\normalise t_2,\sigma_2$ with
   $t_2,\sigma_2\Consistent_M\tilde{t}_2,\tilde{\sigma}_2,\Phi\land\phi_2$.
   \todo{is it problematic that $i$ is not captured in the relation here?}
   }
%
 \Case{\userule{SH-PickLeft}}
  {
  For all tuples $(\tilde{t}_1,\tilde{\sigma}_1,\Left,\phi_1)$,
  we obtain from \cref{lem:soundnorm} that $e_1,\sigma\normalise t_1,\sigma_1$ with
  $t_1,\sigma_1\Consistent_M\tilde{t}_1,\tilde{\sigma}_1,\Phi\land\phi_1$.
%
}
  \Case{\userule{SH-PickRight}}
  {
  For all tuples $(\tilde{t}_2,\tilde{\sigma}_2,\Right,\phi_2)$,
  we obtain from \cref{lem:soundnorm} that $e_2,\sigma\normalise t_2,\sigma_2$ with
  $t_2,\sigma_2\Consistent_M\tilde{t}_2,\tilde{\sigma}_2,\Phi\land\phi_2$.
  }
 }
%
\Case{$\tilde{t}=\tilde{t}_1\And \tilde{t}_2$}
{
In this case, one rule applies. \userule{SH-And}

In this case, we have two sets of symbolic executions.

  For all tuples $(\tilde{t}_1'\And\tilde{t}_2,\tilde{\sigma}_1',\First \tilde{\imath}_1,\phi_1)$,
  we know by application of the induction hypothesis that there exists an $i$ such that
  $\tilde{\imath}_1\sim i$, $t_1,\sigma\handle{i}t_1',\sigma_1'$ and
  $t_1',\sigma_1'\Consistent_{M.[s\mapsto c]}\tilde{t}_1',\tilde{\sigma}_1',\Phi\land\phi_1$.
  Then by \refrule{H-FirstAnd}, we know that also $t_1\And t_2,\sigma\handle{\First i}t_1'\And t_2,\sigma_1'$.
  It follows trivially that $t_1'\And t_2,\sigma_1'\Consistent_{M.[s\mapsto c]}\tilde{t}_1'\And \tilde{t}_2,\tilde{\sigma}_1',\Phi\land\phi_1$.

  For all tuples $(\tilde{t}_1\And\tilde{t}_2',\tilde{\sigma}_2',\Second \tilde{\imath}_2,\phi_2)$,
  we know by application of the induction hypothesis that there exists an $i$ such that
  $\tilde{\imath}_2\sim i$, $t_2,\sigma\handle{i}t_2',\sigma_2'$ and
  $t_2',\sigma_2'\Consistent_{M.[s\mapsto c]}\tilde{t}_2',\tilde{\sigma}_2',\Phi\land\phi_2$.
  Then by \refrule{H-SecondAnd}, we know that also $t_1\And t_2,\sigma\handle{\Second i}t_1\And t_2',\sigma_2'$.
  It follows trivially that $t_1\And t_2',\sigma_2'\Consistent_{M.[s\mapsto c]}\tilde{t}_1\And \tilde{t}_2',\tilde{\sigma}_2',\Phi\land\phi_2$.
}
%
 \Case{$\tilde{t}=\tilde{e}_1\Or \tilde{e}_2$}
{One rule applies, namely \userule{SH-Or}\\
\todo{I forgot this case}
%
% In the case that $M\phi_1$, we need to demonstrate that \userule{H-FirstOr} with $\hat{\sigma}=M\sigma$ and $M\First i =\First j$,
% $M t_1'\Or t_2\equiv \hat{t_1'}\And t_2$ and $M\sigma'\equiv \hat{\sigma'}$.
%
% By the induction hypothesis we obtain the following.\\
% $\forall M_1 . M_1 \phi_1 \implies t_1,M_1\sigma \xrightarrow[]{M_1 i} \hat{t_1'},\hat{\sigma'}\land M_1 t_1'\equiv\hat{t_1'}\land M_1\sigma' \equiv \hat{\sigma'}$.
%
% Since $M$ satisfies $\phi$, we have $t_1,M\sigma\xrightarrow[]{M i} \hat{t_1'},\hat{\sigma'}$ and $M\sigma'\equiv\hat{\sigma'}$,
% which we needed to show, as well as $M t_1'\Or t_2\equiv \hat{t_1'}\And t_2$, which follows from $M t_1' \equiv \hat{t_1'}$.
%
% In the case that $M\phi_2$, we need to demonstrate that \userule{H-SecondOr} with $\hat{\sigma}=M\sigma$ and $M\Second i = \Second j$,
% $M t_1\Or t_2'\equiv t_1\And \hat{t_2}$ and $M\sigma'\equiv \hat{\sigma'}$.
%
% By the induction hypothesis we obtain the following.\\
% $\forall M_1 . M_1 \phi_1 \implies t_2,M_1\sigma \xrightarrow[]{M_1 i} \hat{t_2'},\hat{\sigma'}\land M_1 t_2'\equiv\hat{t_2'}\land M_1\sigma' \equiv \hat{\sigma'}$
%
% Since $M$ satisfies $\phi$, we have $t_2,M\sigma\xrightarrow[]{M i} \hat{t_2'},\hat{\sigma'}$ and $M\sigma'\equiv\hat{\sigma'}$,
% which we needed to show, as well as $M t_1\Or t_2'\equiv t_1\And \hat{t_2'}$, which follows from $M t_2' \equiv \hat{t_2'}$.
}

\end{proof}


\begin{lemma}[$\Value$ preserves consistence]
  \label{lem:valpres}
  For all concrete tasks $t$, concrete states $\sigma$, symbolic tasks $\tilde{t}$, symbolic states $\tilde{\sigma}$, path conditions $\Phi$ and mappings $M=[s_1\mapsto c_1\cdots s_n\mapsto c_n]$,
  if $t,\sigma\Consistent_M\tilde{t},\tilde{\sigma},\Phi$ and $\Value(t,\sigma)=v$ and $\Value(\tilde{t},\tilde{\sigma})$,
  then also $v,\sigma\Consistent_M\tilde{v},\tilde{\sigma},\Phi$
\end{lemma}

\begin{proof}[$\Value$ preserves consistence]

  \Case{$\tilde{t}=\Edit s$}
    {
    If we have $t,\sigma\Consistent_M\Edit s,\tilde{\sigma},\Phi$, then we know that $t$ must be $\Edit c$ for some concrete value of the same type as $s$.

    Then by definition of $\Value$, we have $\Value(\Edit c,\sigma)=c$ and $\Value(\Edit s,\tilde{\sigma})=s$.
    Since we have $M(\Edit s)=\Edit c$ from the premise, we know that $M s = c$, since mapping propagates.
    Therefore $c,\sigma\Consistent_M s,\tilde{\sigma},\Phi$.
    }

  \Case{$\tilde{t}=\Enter \tau$}
    {
    If we have $t,\sigma\Consistent_M\Enter \tau,\tilde{\sigma},\Phi$, then we know that $t$ is also $\Enter\tau$.

    By definition of $\Value$, $\Value(\Enter \tau,\sigma)=\bot$ and $\Value(\Enter\tau,\tilde{\sigma})=\bot$,
    so this case holds trivially.
    }

  \Case{$\tilde{t}=\Update l$}
    {
    If we have $t,\sigma\Consistent_M\Update l,\tilde{\sigma},\Phi$, then we know that $t$ is also $\Update l$.

    By definition of $\Value$, $\Value(\Update l,\sigma)=\sigma(l)$ and $\Value(\Update l,\tilde{\sigma})=\tilde{\sigma}(l)$.

    We now need to show that $M(\tilde{\sigma}(l))=\sigma(l)$. From the premise we know that $M\tilde{\sigma}=\sigma$, from which this immediately follows.
    }

  \Case{$\tilde{t}=\Fail$}
    {
    If we have $t,\sigma\Consistent_M\Fail,\tilde{\sigma},\Phi$, then we know that $t$ is also $\Fail$.

    By definition of $\Value$, $\Value(\Fail,\sigma)=\bot$ and $\Value(\Fail,\tilde{\sigma})=\bot$, so we know that this case holds trivially.
    }

  \Case{$\tilde{t}=\tilde{t}_1\Then \tilde{e}_2$}
    {
    If we have $t,\sigma\Consistent_M\tilde{t}_1\Then \tilde{e}_2,\tilde{\sigma},\Phi$, then we know that $t$ is $t_1\Then e_2$.

    By definition of $\Value$, $\Value(t_1\Then e_2,\sigma)=\bot$ and $\Value(\tilde{t}_1\Then \tilde{e}_2,\tilde{\sigma})=\bot$, so we know that this case holds trivially.

    }


  \Case{$\tilde{t}=\tilde{t}_1\Next \tilde{e}_2$}
    {
    If we have $t,\sigma\Consistent_M\tilde{t}_1\Next \tilde{e}_2,\tilde{\sigma},\Phi$, then we know that $t$ is $t_1\Next e_2$.

    By definition of $\Value$, $\Value(t_1\Next e_2,\sigma)=\sigma(l)$ and $\Value(\tilde{t}_1\Next \tilde{e}_2,\tilde{\sigma})=\bot$, so we know that this case holds trivially.

    }

  \Case{$\tilde{t}=\tilde{t}_1\And \tilde{t}_2$}
    {
    If we have $t,\sigma\Consistent_M\tilde{t}_1\And \tilde{t}_2,\tilde{\sigma},\Phi$, then we know that $t$ is also $t_1\And t_2$.

    By definition of $\Value$, we can find ourselves in one of two cases.

    If $\Value(\tilde{t}_1,\sigma)=\tilde{v}_1$ and $\Value(\tilde{t}_2,\sigma)=\tilde{v}_2$,
    then $\Value(t_1\And t_2,\sigma)=\tuple{v_1,v_2}$ and $\Value(\tilde{t}_1\And \tilde{t}_2,\tilde{\sigma})=\tuple{\tilde{v}_1,\tilde{v}_2}$.
    This case follows from the induction hypothesis.

    Otherwise, if either one of the two branches returns $\bot$, we have that
    $\Value(t_1\And t_2,\sigma)=\bot$ and $\Value(\tilde{t}_1\And \tilde{t}_2,\tilde{\sigma})=\bot$,
    so we know that this case holds trivially

    }

  \Case{$\tilde{t}=\tilde{t}_1\Or \tilde{t}_2$}
    {
    If we have $t,\sigma\Consistent_M\tilde{t}_1\Or \tilde{t}_2,\tilde{\sigma},\Phi$, then we know that $t$ is also $t_1\Or t_2$.

    By definition of $\Value$, we find ourselves in one of three cases.

    If $\Value(\tilde{t}_1,\tilde{\sigma})=\tilde{v}_1$, then $\Value(\tilde{t}_1\Or\tilde{t}_2,\tilde{\sigma})=\tilde{v}_1$
    and $\Value(t_1\Or t_2,\sigma)=v_1$. This case follows from the induction hypothesis.

    Otherwise, if $\Value(\tilde{t}_2,\tilde{\sigma})=\tilde{v}_2$, then $\Value(\tilde{t}_1\Or\tilde{t}_2,\tilde{\sigma})=\tilde{v}_2$
    and $\Value(t_1\Or t_2,\sigma)=v_2$. This case follows from the induction hypothesis.

    Otherwise, if either one of the two branches returns $\bot$, we have that
    $\Value(t_1\Or t_2,\sigma)=\bot$ and $\Value(\tilde{t}_1\Or \tilde{t}_2,\tilde{\sigma})=\bot$,
    so we know that this case holds trivially.

    }

  \Case{$\tilde{t}=\tilde{t}_1\Xor \tilde{t}_2$}
    {
    If we have $t,\sigma\Consistent_M\tilde{t}_1\Xor \tilde{t}_2,\tilde{\sigma},\Phi$, then we know that $t$ is $t_1\Xor t_2$.

    By definition of $\Value$, $\Value(t_1\Xor t_2,\sigma)=\bot$ and $\Value(\tilde{t}_1\Xor \tilde{t}_2,\tilde{\sigma})=\bot$, so we know that this case holds trivially.

    }
\end{proof}

\begin{proof}[Soundness of normalise]
  We prove Lemma~\ref{lem:soundnorm} by induction over $\tilde{e}$.

  From the premise, we can assume that $e,\sigma\Consistent_M\tilde{e},\tilde{\sigma},\Phi$.
  Now, given that $\tilde{e},\sigma{e}\simnormalise \overline{\tilde{t},\tilde{\sigma}',\phi}$,
  we need to demonstrate that for all pairs $(\tilde{t},\tilde{\sigma}',\phi)$,
  $\Sat(\Phi\land\phi)$ implies that $e,\sigma\normalise t,\sigma'$ with $t,\sigma'\Consistent_M\tilde{t},\tilde{\sigma}',\Phi\land\phi$.

The base case is when the SN-Done rule applies.\\
\userule{SN-Done}\\

In this case, we obtain from \cref{lem:soundeval} that
$e,\sigma\normalise t,\sigma'$ with $t,\sigma'\Consistent_M\tilde{t},\tilde{\sigma}',\Phi\land\phi$,
which is exactly what we needed to show.

The only induction step is when\\
\userule{SN-Repeat} applies.

In this case, we obtain from \cref{lem:soundeval} that
$e,\sigma\normalise t,\sigma'$ with $t,\sigma'\Consistent_M\tilde{t},\tilde{\sigma}',\Phi\land\phi_1$,
which is exactly what we needed to show.
Furthermore, by \cref{lem:soundstride} we obtain that
$t,\sigma'\stride t',\sigma''$ with $t',\sigma''\Consistent_M\tilde{t}',\tilde{\sigma}'',\Phi\land\phi_1\land\phi_2$.
Then finally, by application of the induction hypothesis, we obtain what we needed to prove.
$t',\sigma''\normalise t'',\sigma'''$ with $t'',\sigma'''\Consistent_M\tilde{t}'',\tilde{\sigma}''',\Phi\land\phi_1\land\phi_2\land\phi_3$.

\end{proof}

\begin{proof}[Soundness of stride]

  Provided that $t,\sigma\Consistent_M\tilde{t},\tilde{\sigma},\Phi$ and $\tilde{t},\tilde{\sigma}\simstride \overline{\tilde{t}',\tilde{\sigma}',\phi}$,
  we want to show that for all pairs $(\tilde{t}',\tilde{\sigma}',\phi)$,
  we have $\Sat(\Phi\land\phi)$ implies that $t,\sigma\stride t',\sigma'$
  We prove Lemma~\ref{lem:soundstride} by induction over $t$.



  \Case{$\tilde{t}=\Edit \tilde{v}$}
  { One rule applies, namely \userule{SS-Edit}\\
    Given that $t,\sigma\Consistent_M\Edit\tilde{v},\tilde{\sigma},\Phi$ and $\Edit\tilde{v},\tilde{\sigma}\simstride\Edit\tilde{v},\tilde{\sigma},\True$,
    we know that $t=\Edit M \tilde{v}$, and we have $\Edit M \tilde{v},\sigma\stride\Edit M\tilde{v},\sigma$ by \refrule{S-Edit} and
    $\Edit M \tilde{v},\sigma\Consistent_M\Edit\tilde{v},\tilde{\sigma},\Phi$, since none of the tasks and states were altered.
   }
  %
   \Case{$t=\Enter \tau$}
  {One rule applies, namely \userule{SS-Fill}\\
  Given that $t,\sigma\Consistent_M\Enter \tau,\tilde{\sigma},\Phi$ and $\Enter \tau,\tilde{\sigma}\simstride\Enter \tau,\tilde{\sigma},\True$,
  we know that $t=\Enter \tau$, and we have $\Enter \tau,\sigma\stride\Enter \tau,\sigma$ by \refrule{S-Fill} and
  $\Enter \tau,\sigma\Consistent_M\Edit\tilde{v},\tilde{\sigma},\Phi$, since none of the tasks and states were altered.
  }

  \Case{$t=\Update l$}
   {One rule applies, namely \userule{SS-Update}\\
   Given that $t,\sigma\Consistent_M\Update l,\tilde{\sigma},\Phi$ and $\Update l,\tilde{\sigma}\simstride\Update l,\tilde{\sigma},\True$,
   we know that $t=\Update l$, and we have $\Update l,\sigma\stride\Update l,\sigma$ by \refrule{S-Update} and
   $\Update l,\sigma\Consistent_M\Update l,\tilde{\sigma},\Phi$, since none of the tasks and states were altered.
  }

  \Case{$t=\Fail$}
   {One rule applies, namely \userule{SS-Fail}\\
   Given that $t,\sigma\Consistent_M\Fail,\tilde{\sigma},\Phi$ and $\Fail,\tilde{\sigma}\simstride\Fail,\tilde{\sigma},\True$,
   we know that $t=\Fail$, and we have $\Fail,\sigma\stride\Fail,\sigma$ by \refrule{S-Fail} and
   $\Fail,\sigma\Consistent_M\Fail,\tilde{\sigma},\Phi$, since none of the tasks and states were altered.
   }
  %
  \Case{$t=\tilde{e}_1\Xor \tilde{e}_2$}
   {One rule applies, namely \userule{SS-Xor}\\
   Given that $t,\sigma\Consistent_M\tilde{e}_1\Xor \tilde{e}_2,\tilde{\sigma},\Phi$ and $\tilde{e}_1\Xor \tilde{e}_2,\tilde{\sigma}\simstride\tilde{e}_1\Xor \tilde{e}_2,\tilde{\sigma},\True$,
   we know that $t=M\tilde{e}_1\Xor M\tilde{e}_2$, and we have $M\tilde{e}_1\Xor M\tilde{e}_2,\sigma\stride M\tilde{e}_1\Xor M\tilde{e}_2,\sigma$ by \refrule{S-Xor} and
   $M\tilde{e}_1\Xor M\tilde{e}_2,\sigma\Consistent_M\tilde{e}_1\Xor \tilde{e}_2,\tilde{\sigma},\Phi$, since none of the tasks and states were altered.
   }




\Case{$\tilde{t}=\tilde{t}_1\Then \tilde{e}_2$}
 {
Three rules apply.\\
 \Case{\userule{SS-ThenStay}}
   {
     Provided that $t,\sigma\Consistent_M\tilde{t}_1\Then \tilde{e}_2,\tilde{\sigma},\Phi$
     and $\tilde{t}_1\Then \tilde{e}_2,\tilde{\sigma}\simstride\tilde{t}'_1\Then \tilde{e}_2,\tilde{\sigma}',\phi_1$,
     we obtain from the induction hypothesis that $t_1,\sigma\stride t_1',\sigma'$ and $t_1',\sigma'\Consistent_M\tilde{t}_1',\tilde{\sigma}',\Phi$.
     From this, we can directly conclude that $t_1\Then e_2,\sigma\stride t_1'\Then e_2,\sigma'$ and $t_1'\Then e_2,\sigma'\Consistent_M\tilde{t}_1'\Then\tilde{e}_2,\tilde{\sigma}',\Phi$.
  }
 \Case{\userule{SS-ThenFail}}
   { Provided that $t,\sigma\Consistent_M\tilde{t}_1\Then \tilde{e}_2,\tilde{\sigma},\Phi$
   and $\tilde{t}_1\Then \tilde{e}_2,\tilde{\sigma}\simstride\tilde{t}'_1\Then \tilde{e}_2,\tilde{\sigma}',\phi_1$,
   we obtain from the induction hypothesis that $t_1,\sigma\stride t_1',\sigma'$ and $t_1',\sigma'\Consistent_M\tilde{t}_1',\tilde{\sigma}',\Phi$.
   From this, we can directly conclude that $t_1\Then e_2,\sigma\stride t_1'\Then e_2,\sigma'$ and $t_1'\Then e_2,\sigma'\Consistent_M\tilde{t}_1'\Then\tilde{e}_2,\tilde{\sigma}',\Phi$.
   }
 \Case{\userule{SS-ThenCont}}
   {Provided that $t,\sigma\Consistent_M\tilde{t}_1\Then \tilde{e}_2,\tilde{\sigma},\Phi$
   and $\tilde{t}_1\Then \tilde{e}_2,\tilde{\sigma}\simstride\tilde{t}_2,\tilde{\sigma}',\phi_1\land\phi_2$
   with $\tilde{t}_1,\tilde{\sigma}\simstride\tilde{t}_1',\tilde{\sigma}',\phi_1$ and $\Value(\tilde{t}_1',\tilde{\sigma}')=\tilde{v}_1$,
   we obtain from the induction hypothesis that $t_1,\sigma\stride t_1',\sigma'$ and $t_1',\sigma'\Consistent_M\tilde{t}_1',\tilde{\sigma}',\Phi$.
   Then from the consistence relation, we can conclude that $\Value(t_1',\sigma')=\Value(M t_1',M \sigma')=M\tilde{v}_1$.

   At this point, we have $e_2 M\tilde{v}_1,\sigma'\Consistent_M\tilde{e}_2 \tilde{v}_1,\tilde{\sigma}',\Phi\land\phi_1$ and $\tilde{e}_2 \tilde{v}_1,\tilde{\sigma}'\simeval\tilde{t}_2,\tilde{sigma}'',\phi_2$.
   This allows us to apply \cref{lem:soundeval} to obtain $e_2 (M\tilde{v}_1),\sigma'\eval t_2,\sigma''$ and $t_2,\sigma''\Consistent_M\tilde{t}_2,\tilde{\sigma}'',\Phi\land\phi_1\land\phi_2$.

   From this, we can directly conclude that $t_1\Then e_2,\sigma\stride t_2,\sigma''$ and $t_2,\sigma''\Consistent_M\tilde{t}_2,\tilde{\sigma}'',\Phi\land\phi_1\land\phi_2$.

   }
 }

\Case{$\tilde{t}=\tilde{t}_1\Or \tilde{t}_2$}
 {
One of three rules applies.\\
\Case{\userule{SS-OrLeft}}
   {Provided that $t,\sigma\Consistent_M\tilde{t}_1\Or \tilde{t}_2,\tilde{\sigma},\Phi$
   and $\tilde{t}_1\Or \tilde{t}_2,\tilde{\sigma}\simstride\tilde{t}'_1,\tilde{\sigma}',\phi$,
   we obtain from the induction hypothesis that $t_1,\sigma\stride t_1',\sigma'$ and $t_1',\sigma'\Consistent_M\tilde{t}_1',\tilde{\sigma}',\Phi\land\phi$.
   From this, we can directly conclude that $t_1\Or t_2,\sigma\stride t_1',\sigma'$.
  }
 \Case{\userule{SS-OrRight}}
   { Provided that $t,\sigma\Consistent_M\tilde{t}_1\Or \tilde{t}_2,\tilde{\sigma},\Phi$
   and $\tilde{t}_1\Or \tilde{t}_2,\tilde{\sigma}\simstride\tilde{t}'_2,\tilde{\sigma}'',\phi_1\land\phi_2$,
   we obtain from the induction hypothesis that $t_1,\sigma\stride t_1',\sigma'$ and $t_1',\sigma'\Consistent_M\tilde{t}_1',\tilde{\sigma}',\Phi\land\phi_1$.
   Then by a second application of the induction hypothesis, we obtain that $t_2,\sigma'\stride t_2',\sigma''$ and $t_2',\sigma''\Consistent_M\tilde{t}_2',\tilde{\sigma}'',\Phi\land\phi_1\land\phi_2$.
   This leads us to conclude $t_1\Or t_2,\sigma\stride t_2',\sigma''$.
   }
\Case{\userule{SS-OrNone}}
  {  Provided that $t,\sigma\Consistent_M\tilde{t}_1\Or \tilde{t}_2,\tilde{\sigma},\Phi$
  and $\tilde{t}_1\Or \tilde{t}_2,\tilde{\sigma}\simstride\tilde{t}'_2,\tilde{\sigma}'',\phi_1\land\phi_2$,
  we obtain from the induction hypothesis that $t_1,\sigma\stride t_1',\sigma'$ and $t_1',\sigma'\Consistent_M\tilde{t}_1',\tilde{\sigma}',\Phi\land\phi_1$.
  Then by a second application of the induction hypothesis, we obtain that $t_2,\sigma'\stride t_2',\sigma''$ and $t_2',\sigma''\Consistent_M\tilde{t}_2',\tilde{\sigma}'',\Phi\land\phi_1\land\phi_2$.
  This leads us to conclude $t_1\Or t_2,\sigma\stride t_1'\Or t_2',\sigma''$ and $t_1'\Or t_2',\sigma''\Consistent_M\tilde{t}_1'\Or\tilde{t}_2',\tilde{\sigma}'',\Phi\land\phi_1\land\phi_2$.
   }
 }

 \Case{$\tilde{t}=\tilde{t}_1\Next \tilde{e}_2$}
 {One rule applies, namely \userule{SS-Next}\\
 Provided that $t,\sigma\Consistent_M\tilde{t}_1\Next \tilde{e}_2,\tilde{\sigma},\Phi$
 and $\tilde{t}_1\Next \tilde{e}_2,\tilde{\sigma}\simstride\tilde{t}'_1\Then \tilde{e}_2,\tilde{\sigma}',\phi$,
 we obtain from the induction hypothesis that $t_1,\sigma\stride t_1',\sigma'$ and $t_1',\sigma'\Consistent_M\tilde{t}_1',\tilde{\sigma}',\Phi\land\phi$.
 From this, we can directly conclude that $t_1\Next e_2,\sigma\stride t_1'\Next e_2,\sigma'$ and $t_1'\Next e_2,\sigma'\Consistent_M\tilde{t}_1'\Next\tilde{e}_2,\tilde{\sigma}',\Phi\land\phi$.
}

\Case{$\tilde{t}=\tilde{t}_1\And \tilde{t}_2$}
{One rule applies, namely \userule{SS-And}\\
Provided that $t,\sigma\Consistent_M\tilde{t}_1\And \tilde{t}_2,\tilde{\sigma},\Phi$
and $\tilde{t}_1\And \tilde{t}_2,\tilde{\sigma}\simstride\tilde{t}'_1\And\tilde{t}'_2,\tilde{\sigma}'',\phi_1\land\phi_2$,
we obtain from the induction hypothesis that $t_1,\sigma\stride t_1',\sigma'$ and $t_1',\sigma'\Consistent_M\tilde{t}_1',\tilde{\sigma}',\Phi\land\phi_1$.
Then by a second application of the induction hypothesis, we obtain that $t_2,\sigma'\stride t_2',\sigma''$ and $t_2',\sigma''\Consistent_M\tilde{t}_2',\tilde{\sigma}'',\Phi\land\phi_1\land\phi_2$.
This leads us to conclude $t_1\And t_2,\sigma\stride t_1'\And t_2',\sigma''$ and $t_1'\And t_2',\sigma''\Consistent_M\tilde{t}_1'\And\tilde{t}_2',\tilde{\sigma}'',\Phi\land\phi_1\land\phi_2$.

}

\end{proof}

\begin{proof}[Soundness of evaluate]

  We prove Lemma~\ref{lem:soundeval} by induction over $\tilde{e}$.

  \Case{$\tilde{e}=\tilde{v}$}
    {One rule applies, namely \userule{SE-Value}\\
    We assume $e,\sigma\Consistent_M\tilde{v},\tilde{\sigma},\Phi$ and $\tilde{v},\tilde{\sigma}\simeval\tilde{v},\tilde{\sigma},\True$.
    By \refrule{E-Value} we have $v,\sigma\eval v,\sigma$, so this case holds trivially.
    }

  \Case{$\tilde{e}=\tuple{\tilde{e}_1,\tilde{e}_2}$}
    {One rule applies, namely \userule{SE-Pair}\\
    Provided that $e,\sigma\Consistent_m\tuple{\tilde{e}_1,\tilde{e}_2},\tilde{\sigma},\Phi$ and $\tuple{\tilde{e}_1,\tilde{e}_2},\tilde{\sigma}\simeval\tuple{\tilde{v}_1,\tilde{v}_2},\tilde{\sigma}'',\phi_1\land\phi_2$,
    we obtain from the induction hypothesis that $e_1,\sigma\eval v_1,\sigma'$ with $v_1,\sigma'\Consistent_m\tilde{v}_1,\tilde{\sigma}',\Phi\land\phi_1$.
    Then by a second application of the induction hypothesis, we obtain that $e_2,\sigma'\eval v_2,\sigma''$ with $v_2,\sigma''\Consistent_m\tilde{v}_2,\tilde{\sigma}'',\Phi\land\phi_1\land\phi_2$.
    From this, we can conclude that $\tuple{e_1,e_2},\sigma\eval\tuple{v_1,v_2},\sigma''$ with $\tuple{v_1,v_2},\sigma''\Consistent_m\tuple{\tilde{v}_1,\tilde{v}_2},\tilde{\sigma}'',\Phi\land\phi_1\land\phi_2$.
    }

  \Case{$\tilde{e}=\Fst \tuple{\tilde{e}_1,\tilde{e}_2}$}
  {
    One rule applies, namely \userule{SE-First}\\
    Provided that $e,\sigma\Consistent_m\Fst\tuple{\tilde{e}_1,\tilde{e}_2},\tilde{\sigma},\Phi$ and $\Fst\tuple{\tilde{e}_1,\tilde{e}_2},\tilde{\sigma}\simeval\tilde{v}_1,\tilde{\sigma}'',\phi$,
    we obtain from the induction hypothesis that $e_1,\sigma\eval v_1,\sigma'$ with $v_1,\sigma'\Consistent_m\tilde{v}_1,\tilde{\sigma}',\Phi\land\phi$.
    From this, we can conclude that $\Fst\tuple{e_1,e_2},\sigma\eval v_1,\sigma'$.
    }

  \Case{$e=\Snd \tuple{\tilde{e}_1,\tilde{e}_2}$}
  { One rule applies, namely \userule{SE-Second}\\
  Provided that $e,\sigma\Consistent_m\Snd\tuple{\tilde{e}_1,\tilde{e}_2},\tilde{\sigma},\Phi$ and $\Snd\tuple{\tilde{e}_1,\tilde{e}_2},\tilde{\sigma}\simeval\tilde{v}_2,\tilde{\sigma}'',\phi$,
  we obtain from the induction hypothesis that $e_1,\sigma\eval v_2,\sigma'$ with $v_2,\sigma'\Consistent_m\tilde{v}_1,\tilde{\sigma}',\Phi\land\phi$.
  From this, we can conclude that $\Snd\tuple{e_1,e_2},\sigma\eval v_2,\sigma'$.
    }

  \Case{$\tilde{e}=\tilde{e}_1::\tilde{e}_2$}
    {One rule applies, namely \userule{SE-Cons}\\
    Provided that $e,\sigma\Consistent_m\tilde{e}_1::\tilde{e}_2,\tilde{\sigma},\Phi$ and $\tilde{e}_1::\tilde{e}_2,\tilde{\sigma}\simeval\tilde{v}_1::\tilde{v}_2,\tilde{\sigma}'',\phi_1\land\phi_2$,
    we obtain from the induction hypothesis that $e_1,\sigma\eval v_1,\sigma'$ with $v_1,\sigma'\Consistent_m\tilde{v}_1,\tilde{\sigma}',\Phi\land\phi_1$.
    Then by a second application of the induction hypothesis, we obtain that $e_2,\sigma'\eval v_2,\sigma''$ with $v_2,\sigma''\Consistent_m\tilde{v}_2,\tilde{\sigma}'',\Phi\land\phi_1\land\phi_2$.
    From this, we can conclude that $e_1 :: e_2,\sigma\eval v_1 :: v_2,\sigma''$ with $v_1 :: v_2,\sigma''\Consistent_m\tilde{v}_1 :: \tilde{v}_2,\tilde{\sigma}'',\Phi\land\phi_1\land\phi_2$.
   }

  \Case{$\tilde{e}=\Head \tilde{e}$}
    {One rule applies, namely \userule{SE-Head}\\
    Provided that $e,\sigma\Consistent_m\Head \tilde{e},\tilde{\sigma},\Phi$ and $\Head \tilde{e},\tilde{\sigma}\simeval\tilde{v}_1,\tilde{\sigma}',\phi$,
    we obtain from the induction hypothesis that $e,\sigma\eval v_1::v_2,\sigma'$ with $v_1::v_2,\sigma'\Consistent_m\tilde{v}_1::\tilde{v}_2,\tilde{\sigma}',\Phi\land\phi$.
    From this, we can conclude that $\Head e,\sigma\eval v_1,\sigma'$.
    }

  \Case{$\tilde{e}=\Tail \tilde{e}$}
    {One rule applies, namely \userule{SE-Tail}\\
    Provided that $e,\sigma\Consistent_m\Tail \tilde{e},\tilde{\sigma},\Phi$ and $\Tail \tilde{e},\tilde{\sigma}\simeval\tilde{v}_2,\tilde{\sigma}',\phi$,
    we obtain from the induction hypothesis that $e,\sigma\eval v_1::v_2,\sigma'$ with $v_1::v_2,\sigma'\Consistent_m\tilde{v}_1::\tilde{v}_2,\tilde{\sigma}',\Phi\land\phi$.
    From this, we can conclude that $\Tail e,\sigma\eval v_2,\sigma'$.
      }

  \Case{$\tilde{e}=\tilde{e}_1 \tilde{e}_2$}
    {One rule applies, namely\\ \userule{SE-App}\\

    Provided that $e,\sigma\Consistent_m\tilde{e}_1 \tilde{e}_2,\tilde{\sigma},\Phi$ and $\tilde{e}_1 \tilde{e}_2,\tilde{\sigma}\simeval\tilde{v}_1,\tilde{\sigma}''',\phi_1\land\phi_2\land\phi_3$,
    we obtain from the induction hypothesis that $e_1,\sigma\eval \lambda x:\tau.{e_1}',\sigma'$ with $\lambda x:\tau.{e_1}',\sigma'\Consistent_m\lambda x:\tau.\tilde{e}_1',\tilde{\sigma}',\Phi\land\phi_1$.
    Then by a second application of the induction hypothesis, we obtain that $e_2,\sigma'\eval v_2,\sigma''$ with $v_2,\sigma''\Consistent_m\tilde{v}_2,\tilde{\sigma}'',\Phi\land\phi_1\land\phi_2$.
    A third and final application of the induction hypothesis gives us that $e_1'[x\mapsto v_2],\sigma'' \eval v_1,\sigma'''$ with
    $v_1,\sigma'''\Consistent_m\tilde{v}_1,\tilde{\sigma}''',\Phi\land\phi_1\land\phi_2\land\phi_3$.
    From this, we can conclude that $e_1 e_2,\sigma\eval v_1,\sigma'''$.
    }

  \Case{$\tilde{e}=\If{\tilde{e}_1}{\tilde{e}_2}{\tilde{e}_3}$}
     {One rule applies, namely\\ \userule{SE-If}\\
     \todo{this case requires more than simply IH application}
     % In case that $M\phi_1\land M\phi_2 \land Mv_1$,
     % we need to demonstrate that
     % \userule{E-IfTrue} with $\hat{\sigma}=M\sigma$,
     % $M v_2 = \hat{v_2}$ and $M\sigma''=\hat{\sigma''}$.
     %
     % From the induction hypothesis, we obtain the following.\\
     % $\forall M_1 . M_1\phi_1 \implies e_1,M_1\sigma\hat{\eval}\hat{v_1},\hat{\sigma'}
     % \land M_1v_1 \equiv \hat{v_1} \land M_1\sigma'\equiv\hat{\sigma'}$
     % and\\
     % $\forall M_2 . M_2\phi_2 \implies e_2,M_2\sigma'\hat{\eval}\hat{v_2},\hat{\sigma''}
     % \land M_2 v_2 \equiv \hat{v_2} \land M_2\sigma'' \equiv\hat{\sigma''}$.
     %
     % Since $M$ satisfies $\phi_1$, and $M v_1=\True$, we know from the application of the induction hypothesis above, that $\hat{v_1}=\True$.\\
     % Furthermore, $M$ satisfies $\phi_2$, so we directly obtain that $Mv_2=\hat{v_2}$ and $M\sigma''=\hat{\sigma''}$.
     %
     % In case that $M\phi_1\land M\phi_3 \land M\neg v_1$,
     % we need to demonstrate that
     % \userule{E-IfFalse} with $\hat{\sigma}=M\sigma$,
     % $M v_3 = \hat{v_3}$ and $M\sigma''=\hat{\sigma''}$.\\
     %
     % From the induction hypothesis, we obtain the following.\\
     % $\forall M_1 . M_1\phi_1 \implies e_1,M_1\sigma\hat{\eval}\hat{v_1},\hat{\sigma'}
     % \land M_1v_1 \equiv \hat{v_1} \land M_1\sigma'\equiv\hat{\sigma'}$
     % and\\
     % $\forall M_3 . M_3\phi_3 \implies e_3,M_3\sigma'\hat{\eval}\hat{v_3},\hat{\sigma''}
     % \land M_3 v_3 \equiv \hat{v_3} \land M_3\sigma'' \equiv\hat{\sigma''}$.
     %
     % Since $M$ satisfies $\phi_1$, and $M v_1=\False$, we know from the application of the induction hypothesis above, that $\hat{v_1}=\False$.\\
     % Furthermore, $M$ satisfies $\phi_3$, so we directly obtain that $Mv_3=\hat{v_3}$ and $M\sigma''=\hat{\sigma''}$.
    }

  \Case{$\tilde{e}=\Ref \tilde{e}$}
    {One rule applies, namely \userule{SE-Ref}\\
    Provided that $e,\sigma\Consistent_m\Ref \tilde{e},\tilde{\sigma},\Phi$ and $\Ref \tilde{e},\tilde{\sigma}\simeval l,\tilde{\sigma}'[l\mapsto\tilde{v}],\phi$,
    we obtain from the induction hypothesis that $e,\sigma\eval v_1,\sigma'$ with $v_1,\sigma'\Consistent_m\tilde{v}_1,\tilde{\sigma}',\Phi\land\phi$.
    From this, we can conclude that $\Ref e,\sigma\eval l,\sigma'[l\mapsto v]$ with $l,\sigma'[l\mapsto v]\Consistent_m l,\tilde{\sigma}'[l\mapsto \tilde{v}],\Phi\land\phi$.
    }

  \Case{$\tilde{e}=!\tilde{e}$}
    {One rule applies, namely \userule{SE-Deref}\\
    Provided that $e,\sigma\Consistent_m !\tilde{e},\tilde{\sigma},\Phi$ and $!\tilde{e},\tilde{\sigma}\simeval\tilde{\sigma}'(l),\tilde{\sigma}',\phi$,
    we obtain from the induction hypothesis that $e,\sigma\eval l,\sigma'$ with $l,\sigma'\Consistent_m l,\tilde{\sigma}',\Phi\land\phi$.
    From this, we can conclude that $!e,\sigma\eval \sigma'(l),\sigma'$ with $\sigma'(l),\sigma'\Consistent_m \tilde{\sigma}'(l),\tilde{\sigma}',\Phi\land\phi$.
  }

  \Case{$\tilde{e}=\tilde{e}_1:=\tilde{e}_2$}
    {
    One rule applies, namely \userule{SE-Assign}\\
    Provided that $e,\sigma\Consistent_m\tilde{e}_1:=\tilde{e}_2,\tilde{\sigma},\Phi$ and $\tilde{e}_1:=\tilde{e}_2,\tilde{\sigma}\simeval\unit,\tilde{\sigma}''[l\mapsto \tilde{v}_2],\phi_1\land\phi_2$,
    we obtain from the induction hypothesis that $e_1,\sigma\eval l,\sigma'$ with $l,\sigma'\Consistent_m l,\tilde{\sigma}',\Phi\land\phi_1$.
    Then by a second application of the induction hypothesis, we obtain that $e_2,\sigma'\eval v_2,\sigma''$ with $v_2,\sigma''\Consistent_m\tilde{v}_2,\tilde{\sigma}'',\Phi\land\phi_1\land\phi_2$.
    From this, we can conclude that $e_1 := e_2,\sigma\eval \unit,\sigma''[l\mapsto v_2]$ with $\unit,\sigma''[l\mapsto v_2]\Consistent_m\unit,\tilde{\sigma}''[l\mapsto\tilde{v}_2],\Phi\land\phi_1\land\phi_2$.
    }

  \Case{$\tilde{e}=\Edit \tilde{e}$}
    {One rule applies, namely \userule{SE-Edit}\\
    Provided that $e,\sigma\Consistent_m \Edit\tilde{e},\tilde{\sigma},\Phi$ and $\Edit\tilde{e},\tilde{\sigma}\simeval\Edit\tilde{v},\tilde{\sigma}',\phi$,
    we obtain from the induction hypothesis that $e,\sigma\eval v,\sigma'$ with $v,\sigma'\Consistent_m \tilde{v},\tilde{\sigma}',\Phi\land\phi$.
    From this, we can conclude that $\Edit e,\sigma\eval \Edit v,\sigma'$ with $\Edit v,\sigma'\Consistent_m \Edit\tilde{v},\tilde{\sigma}',\Phi\land\phi$.

    }

  \Case{$\tilde{e}=\Enter \tau$}
    {
    One rule applies, namely \userule{SE-Enter}\\
    We assume $e,\sigma\Consistent_M\Enter\tau,\tilde{\sigma},\Phi$ and $\Enter\tau,\tilde{\sigma}\simeval\Enter\tau,\tilde{\sigma},\True$.
    By \refrule{E-Enter} we have $\Enter\tau,\sigma\eval \Enter\tau,\sigma$, so this case holds trivially.
    }

  \Case{$\tilde{e}=\Update \tilde{e}$}
    {One rule applies, namely \userule{SE-Update}\\
    Provided that $e,\sigma\Consistent_m \Update\tilde{e},\tilde{\sigma},\Phi$ and $\Update\tilde{e},\tilde{\sigma}\simeval\Update l,\tilde{\sigma}',\phi$,
    we obtain from the induction hypothesis that $e,\sigma\eval l,\sigma'$ with $l,\sigma'\Consistent_m l,\tilde{\sigma}',\Phi\land\phi$.
    From this, we can conclude that $\Update e,\sigma\eval \Update l ,\sigma'$ with $\Update l,\sigma'\Consistent_m \Update l,\tilde{\sigma}',\Phi\land\phi$.

    }

  \Case{$\tilde{e}=\tilde{e}_1\Then \tilde{e}_2$}
    {One rule applies, namely \userule{SE-Then}\\
    Provided that $e,\sigma\Consistent_m \tilde{e}_1\Then \tilde{e}_2,\tilde{\sigma},\Phi$ and $\tilde{e}_1\Then \tilde{e}_2,\tilde{\sigma}\simeval\tilde{t}_1\Then \tilde{e}_2,\tilde{\sigma}',\phi$,
    we obtain from the induction hypothesis that $e_1,\sigma\eval t_1,\sigma'$ with $t_1,\sigma'\Consistent_m \tilde{t}_1,\tilde{\sigma}',\Phi\land\phi$.
    From this, we can conclude that $e_1\Then e_2,\sigma\eval t_1\Then e_2,\sigma'$ with $t_1\Then e_2,\sigma'\Consistent_m \tilde{t}_1\Then e_2,\tilde{\sigma}',\Phi\land\phi$.

    }

  \Case{$\tilde{e}=\tilde{e}_1\Next \tilde{e}_2$}
    {One rule applies, namely \userule{SE-Next}\\
    Provided that $e,\sigma\Consistent_m \tilde{e}_1\Next \tilde{e}_2,\tilde{\sigma},\Phi$ and $\tilde{e}_1\Next \tilde{e}_2,\tilde{\sigma}\simeval\tilde{t}_1\Next \tilde{e}_2,\tilde{\sigma}',\phi$,
    we obtain from the induction hypothesis that $e_1,\sigma\eval t_1,\sigma'$ with $t_1,\sigma'\Consistent_m \tilde{t}_1,\tilde{\sigma}',\Phi\land\phi$.
    From this, we can conclude that $e_1\Next e_2,\sigma\eval t_1\Next e_2,\sigma'$ with $t_1\Next e_2,\sigma'\Consistent_m \tilde{t}_1\Next e_2,\tilde{\sigma}',\Phi\land\phi$.

    }

  \Case{$\tilde{e}=\tilde{e}_1\Or \tilde{e}_2$}
    {One rule applies, namely \userule{SE-Or}\\
    Provided that $e,\sigma\Consistent_m\tilde{e}_1 \Or \tilde{e}_2,\tilde{\sigma},\Phi$ and $\tilde{e}_1 \Or \tilde{e}_2,\tilde{\sigma}\simeval\tilde{v}_1\Or \tilde{v}_2,\tilde{\sigma}'',\phi_1\land\phi_2$,
    we obtain from the induction hypothesis that $e_1,\sigma\eval v_1,\sigma'$ with $v_1,\sigma'\Consistent_m\tilde{v}_1,\tilde{\sigma}',\Phi\land\phi_1$.
    Then by a second application of the induction hypothesis, we obtain that $e_2,\sigma'\eval v_2,\sigma''$ with $v_2,\sigma''\Consistent_m\tilde{v}_2,\tilde{\sigma}'',\Phi\land\phi_1\land\phi_2$.
    From this, we can conclude that $e_1 \Or e_2,\sigma\eval v_1 \Or v_2,\sigma''$ with $v_1 \Or v_2,\sigma''\Consistent_m\tilde{v}_1 \Or \tilde{v}_2,\tilde{\sigma}'',\Phi\land\phi_1\land\phi_2$.

    }

  \Case{$\tilde{e}=\tilde{e}_1\Xor \tilde{e}_2$}
    {  One rule applies, namely \userule{SE-Xor}\\
    We assume $e,\sigma\Consistent_M\tilde{e}_1\Xor \tilde{e}_2,\tilde{\sigma},\Phi$ and $\tilde{e}_1\Xor \tilde{e}_2,\tilde{\sigma}\simeval\tilde{e}_1\Xor \tilde{e}_2,\tilde{\sigma},\True$.
    By \refrule{E-Xor} we have $e_1\Xor e_2,\sigma\eval e_1\Xor e_2,\sigma$, so this case holds trivially.

    }

  \Case{$\tilde{e}=\Fail$}
    {  One rule applies, namely \userule{SE-Fail}\\
    We assume $e,\sigma\Consistent_M\Fail,\tilde{\sigma},\Phi$ and $\Fail,\tilde{\sigma}\simeval\Fail,\tilde{\sigma},\True$.
    By \refrule{E-Fail} we have $\Fail,\sigma\eval \Fail,\sigma$, so this case holds trivially.

    }

\end{proof}

% !TEX root=../main.tex

\section{Completeness proofs}
\label{sec:completenessproofs}

\begin{proof}[Completeness of simulate]
  The structure of this proof is outlined in \cref{fig:proofstructure}.

  We have $t$ and $\sigma$ such that $t,\sigma\interact{I}^*v$.
  By definition of $\interact{I}^*$, we have the following.

  $t,\sigma\interact{i_1}t_1,\sigma_1\interact{i_2}\cdots \interact{i_n}t_n,\sigma_n$ with $\Value(t_n,\sigma_n)$ and $I=[i_1,\cdots,i_n]$.

  We need to show that we have $(\tilde{v},\tilde{I},\Phi)\in t,\sigma\interact{}^*)$,
  which is defined as follows.

  \begin{align*}
      t,\sigma\interact{}&\tilde{t}_1,\tilde{\sigma}_1,\tilde{\imath}_1,\phi_1&\\
                      &\tilde{t}_1,\tilde{\sigma}_1\interact{}&\tilde{t}_2,\tilde{\sigma}_2,\tilde{\imath}_2,\phi_2\\
                      &                                    &\tilde{t}_2,\tilde{\sigma}_2\interact{}&\cdots&\\
                      &                                    &                                    &\cdots&
                      \interact{}\tilde{t}_n,\tilde{\sigma}_n,\tilde{\imath}_n,\phi_n
  \end{align*}

  with $\Value(\tilde{t}_n,\tilde{\sigma}_n)=\tilde{v}$ and $\Sat(\phi_1\land\cdots\land\phi_n)$.

  By \cref{lem:completedriving}, we know that $t,\sigma\interact{}\tilde{t}_1,\tilde{\sigma}_1,\tilde{\imath}_1,\phi_1$ exists,
  since $t,\sigma,t\Consistent_{\emptyset}\sigma,\True$.
  This also gives us that $\tilde{\imath}_1\sim i_1$ and $t_1,\sigma_1\Consistent_{[s_1\mapsto c_1]}\tilde{t}_1,\tilde{\sigma}_1,\phi_1$ with $s_1\in\tilde{\imath}_1$ and $c_1\in i_1$.

  By repeated application of \cref{lem:completedriving}, untill we arrive at $t_n,\sigma_n$,
  we can show that there exists a $\tilde{I}$ such that $t,\sigma\interact{}^*\tilde{t}_n,\tilde{\sigma}_n,\tilde{I},\Phi$,
  namely $[\tilde{\imath}_1,\cdots,\tilde{\imath}_n]$.

\end{proof}

\begin{lemma}[Completeness of handling]
  \label{lem:completeHandle}
  For all concrete tasks $t$, concrete states $\sigma$, concrete inputs $i$, symbolic tasks $\tilde{t}$, symbolic states $\tilde{\sigma}$ path conditions $\Phi$ and mappings $M$,
  we have that $t,\sigma\Consistent_{M}\tilde{t},\tilde{\sigma},\Phi$ and
  $t,\sigma\handle{i}t',\sigma'$ together with
  $\tilde{t},\tilde{\sigma}\handle{}\overline{\tilde{t}',\tilde{\sigma}',\tilde{\imath},\phi}$,
  and for all pairs $(\tilde{t}',\tilde{\sigma}',\tilde{\imath},\phi)$ we have that $\Sat(\Phi\land\phi)$ and $\imath\sim i$ implies $t',\sigma'\Consistent_{M.[s\mapsto c]}\tilde{t}',\tilde{\sigma}',\Phi\land\phi$ where where $s\in\tilde{\imath}$ and $c\in i$.
\end{lemma}

\begin{lemma}[Completeness of normalisation]
  \label{lem:completeNormalise}
  For all concrete expressions $e$, concrete states $\sigma$, symbolic expressions $\tilde{e}$, symbolic states $\tilde{\sigma}$ path conditions $\Phi$ and mappings $M$,
  we have that $e,\sigma\Consistent_{M}\tilde{e},\tilde{\sigma},\Phi$
  and $e,\sigma\normalise t,\sigma'$,
  then $\tilde{e},\tilde{\sigma}\normalise\overline{\tilde{t},\tilde{\sigma}',\phi}$,
  and for all pairs $(\tilde{t},\tilde{\sigma}',\phi)$ we have that $\Sat(\Phi\land\phi)$ implies $t,\sigma'\Consistent_{M}\tilde{t},\tilde{\sigma}',\Phi\land\phi$.
\end{lemma}

\begin{lemma}[Completeness of striding]
  \label{lem:completeStride}
  For all concrete tasks $t$, concrete states $\sigma$, symbolic tasks $\tilde{t}$, symbolic states $\tilde{\sigma}$ path conditions $\Phi$ and mappings $M$,
  we have that $t,\sigma\Consistent_{M}\tilde{t},\tilde{\sigma},\Phi$
  and $t,\sigma\stride t',\sigma'$,
  then $\tilde{t},\tilde{\sigma}\stride\overline{\tilde{t'},\tilde{\sigma}',\phi}$,
  and for all pairs $(\tilde{t'},\tilde{\sigma}',\phi)$ we have that $\Sat(\Phi\land\phi)$ implies $t',\sigma'\Consistent_{M}\tilde{t'},\tilde{\sigma}',\Phi\land\phi$.
\end{lemma}

\begin{lemma}[Completeness of evaluate]
  \label{lem:completeEval}
  For all concrete expressions $e$, concrete states $\sigma$, symbolic expressions $\tilde{e}$, symbolic states $\tilde{\sigma}$ path conditions $\Phi$ and mappings $M$,
  we have that $e,\sigma\Consistent_{M}\tilde{e},\tilde{\sigma},\Phi$
  and $e,\sigma\eval v,\sigma'$,
  then $\tilde{e},\tilde{\sigma}\eval\overline{\tilde{v},\tilde{\sigma}',\phi}$,
  and for all pairs $(\tilde{v},\tilde{\sigma}',\phi)$ we have that $\Sat(\Phi\land\phi)$ implies $v,\sigma'\Consistent_{M}\tilde{v},\tilde{\sigma}',\Phi\land\phi$.
\end{lemma}

\begin{proof}[Completeness of handle]
  We prove Lemma~\ref{lem:completeHandle} by induction over $t$.\\

    \Case{$t=\Edit v$}
    {
    Provided that $\Edit v,\sigma\Consistent_{M}\tilde{t},\tilde{\sigma},\Phi$ and \userule{H-Change},
    then $\Edit\tilde{v},\tilde{\sigma}\handle{}\Edit s,\tilde{\sigma},s,\True$.
    $\Sat(\Phi\land\True)=\Sat(\Phi)$, which follows from the premise.
    Furthermore we have $s\sim v'$ by definition.
    Then finally $\Edit v',\sigma\Consistent{M[s\mapsto v']}\Edit s,\tilde{\sigma},\Phi$ since $M[s\mapsto v'] s = v'$.


    }

    \Case{$t=\Enter \tau$}
    {
    Provided that $\Enter \tau,\sigma\Consistent_{M}\tilde{t},\tilde{\sigma},\Phi$ and \userule{H-Fill},
    then $\Enter\tau,\tilde{\sigma}\handle{}\Edit s,\tilde{\sigma},s,\True$.
    $\Sat(\Phi\land\True)=\Sat(\Phi)$, which follows from the premise.
    Furthermore we have $s\sim v$ by definition.
    Then finally $\Edit v,\sigma\Consistent{M[s\mapsto v]}\Edit s,\tilde{\sigma},\Phi$ since $M[s\mapsto v] s = v$.
    }

    \Case{$t=\Update l$}
    {
    Provided that $\Update l,\sigma\Consistent_{M}\tilde{t},\tilde{\sigma},\Phi$ and \userule{H-Update},
    then $\Update l,\tilde{\sigma}\handle{}\Update l,\tilde{\sigma}[l\mapsto s],s,\True$.
    $\Sat(\Phi\land\True)=\Sat(\Phi)$, which follows from the premise.
    Furthermore we have $s\sim v$ by definition.
    Then finally $\Update l,\sigma[l\mapsto v]\Consistent{M[s\mapsto v]}\Update l,\tilde{\sigma}[l\mapsto s],\Phi$ since $M[s\mapsto v] s = v$.
     }

    \Case{$t=t_1\Next e_2$}
    {Two rules apply in this case\\
      \Case{\userule{H-Next}}
      {
      Provided that $t_1\Next e_2,\sigma\Consistent_{M}\tilde{t},\tilde{\sigma},\Phi$ and \userule{H-Next},
      then \userule{SH-Next}.
      The simulation step results in two sets, from which only the second adheres to the requirement that the symbolic input should simulate the concrete input.
      For this set, $\overline{\tilde{t}_2,\tilde{\sigma}_2',\Continue,\phi_2}$, we have $\Sat(\Phi\land\phi_2)$ implies
      $t_2,\sigma_2'\Consistent_M\tilde{t}_2,\tilde{\sigma}_2',\Phi\land\phi_2$,
      Which follows directly from \cref{lem:completeNormalise}.
      }
      \Case{\userule{H-PassNext}}
      {
      Provided that $t_1\Next e_2,\sigma\Consistent_{M}\tilde{t},\tilde{\sigma},\Phi$ and \userule{H-PassNext}.
      There are three symbolic rules that apply, namely \userule{SH-PassNext}, \userule{SH-PassNextFail} and\\
      \userule{SH-Next}.
      We are only interested in the runs that produce a symbolic input that simulates the concrete input $i$.
      Whichever rule applies, we deal with the same premise because of this restriction.
      This allows us to apply the induction hypothesis and obtain that
      $\Sat(\Phi\land\phi_1)\implies t_1',\sigma'\Consistent_{M.[s\mapsto c]}\tilde{t}_1',\tilde{\sigma}',\Phi\land\phi_1$.
      From this, we can directly conclude that $t_1'\Next e_2,\sigma'\Consistent_{M.[s\mapsto c]}\tilde{t}_1'\Next \tilde{e}_2,\tilde{\sigma}',\Phi\land\phi_1$.
      }
    }


    \Case{$t=t_1\Then e_2$}
    {
    Provided that $t_1\Then e_2,\sigma\Consistent_{M}\tilde{t},\tilde{\sigma},\Phi$ and \userule{H-PassThen},
    then \userule{SH-PassThen}.
    By application of the induction hypothesis, we obtain $\Sat(\Phi\land\phi)$ implies $t_1',\sigma'\Consistent_{M}\tilde{t}_1',\tilde{\sigma}',\Phi\land\phi$
    from which we can conclude that $t_1'\Then e_2,\sigma'\Consistent_{M}\tilde{t}_1'\Then \tilde{e}_2,\tilde{\sigma}',\Phi\land\phi$.
    }
    \Case{$t=e_1\Xor e_2$}
    {
    Two rules apply in this case.\\
    \Case{\userule{H-PickLeft}}
      {
        % Take $i=s$. $s\sim \Left$ holds by definition.\\
        % Lemma~\ref{lem:completeNormalise} gives us the following.\\
        % There exists a symbolic execution $e_1,\sigma\normalise t_1,\sigma_1,\phi$.\\
        % There exists a symbolic execution $e_2,\sigma_1\normalise t_2,\sigma_2,\phi$.\\
        %
        % We can now conclude that a symbolic execution exists.
        % Either by the \refrule{SH-PickLeft} rule, in case $\Failing(t_2,\sigma_2)$, or by the \refrule{SH-Pick} rule in case $\neg\Failing(t_2,\sigma_2)$.
      }
    \Case{\userule{H-PickRight}}
      {
      % Take $i=s$. $s\sim \Left$ holds by definition.\\
      % Lemma~\ref{lem:completeNormalise} gives us the following.\\
      % There exists a symbolic execution $e_1,\sigma\normalise t_1,\sigma_1,\phi$.\\
      % There exists a symbolic execution $e_2,\sigma_1\normalise t_2,\sigma_2,\phi$.\\
      %
      % We can now conclude that a symbolic execution exists.
      % Either by the \refrule{SH-PickRight} rule, in case $\Failing(t_1,\sigma_1)$, or by the \refrule{SH-Pick} rule in case $\neg\Failing(t_1,\sigma_1)$.
      }
    }
    \Case{$t=t_1\Or t_2$}
      {
      Two rules applies in this case.\\
      \Case{\userule{H-FirstOr}}
      {
      % Take $i=\First i$.
      %
      % By application of the induction hypothesis, we obtain the following.\\
      % For all $t_1,\sigma,j$ such that $t_1,\sigma\xrightarrow[]{j}t_1',\sigma'$ there exists an $i\sim j$ such that $t_1'',\sigma''\handle{}t_1''',\sigma''',i,\phi$.\\
      %
      % From this, we can conclude that $\First i\sim \First j$.
      % From \refrule{SH-Or}, and the conclusion of the induction hypothesis,
      % we can conclude that there exists an $i$ such that $t_1\Or t_2,\sigma\handle{}t_1'\Or t_2,\sigma',i,\phi$.
      }
      \Case{\userule{H-SecondOr}}
      {
      % Take $i=\Second i$.
      %
      % By application of the induction hypothesis, we obtain the following.\\
      % For all $t_2,\sigma,j$ such that $t_2,\sigma\xrightarrow[]{j}t_2',\sigma'$ there exists an $i\sim j$ such that $t_2'',\sigma''\handle{}t_2''',\sigma''',i,\phi$.\\
      %
      % From this, we can conclude that $\Second i\sim \Second j$.
      % From \refrule{SH-Or}, and the conclusion of the induction hypothesis,
      % we can conclude that there exists an $i$ such that $t_1\Or t_2,\sigma\handle{}t_1\Or t_2',\sigma',i,\phi$.
      }
      }
    \Case{$t=t_1\And t_2$}
      {
      Two rules applies in this case.\\
      \Case{\userule{H-FirstAnd}}
      {
      % Take $i=\First i$.
      %
      % By application of the induction hypothesis, we obtain the following.\\
      % For all $t_1,\sigma,j$ such that $t_1,\sigma\xrightarrow[]{j}t_1',\sigma'$ there exists an $i\sim j$ such that $t_1'',\sigma''\handle{}t_1''',\sigma''',i,\phi$.\\
      %
      % From this, we can conclude that $\First i\sim \First j$.
      % From \refrule{SH-And}, and the conclusion of the induction hypothesis,
      % we can conclude that there exists an $i$ such that $t_1\And t_2,\sigma\handle{}t_1'\Or t_2,\sigma',i,\phi$.
      }
      \Case{\userule{H-SecondAnd}}
      {
      % Take $i=\Second i$.
      %
      % By application of the induction hypothesis, we obtain the following.\\
      % For all $t_2,\sigma,j$ such that $t_2,\sigma\xrightarrow[]{j}t_2',\sigma'$ there exists an $i\sim j$ such that $t_2'',\sigma''\handle{}t_2''',\sigma''',i,\phi$.\\
      %
      % From this, we can conclude that $\First i\sim \Second j$.
      % From \refrule{SH-And}, and the conclusion of the induction hypothesis,
      % we can conclude that there exists an $i$ such that $t_1\And t_2,\sigma\handle{}t_1\And t_2',\sigma',i,\phi$.
      }
      }
\end{proof}

\begin{proof}[Completeness of normalise]
  We prove Lemma~\ref{lem:completeNormalise} by induction over $e$.

  % From the premise, we can assume that $e,\sigma\Consistent_M\tilde{e},\tilde{\sigma},\Phi$.
  % Now, given that $\tilde{e},\sigma{e}\simnormalise \overline{\tilde{t},\tilde{\sigma}',\phi}$,
  % we need to demonstrate that for all pairs $(\tilde{t},\tilde{\sigma}',\phi)$,
  % $\Sat(\Phi\land\phi)$ implies that $e,\sigma\normalise t,\sigma'$ with $t,\sigma'\Consistent_M\tilde{t},\tilde{\sigma}',\Phi\land\phi$.

  The base case is when the N-Done rule applies.\\
  \userule{N-Done}\\

  % In this case, we obtain from \cref{lem:soundeval} that
  % $e,\sigma\normalise t,\sigma'$ with $t,\sigma'\Consistent_M\tilde{t},\tilde{\sigma}',\Phi\land\phi$,
  % which is exactly what we needed to show.
  %
  % The only induction step is when\\
  \userule{N-Repeat} applies.

  % In this case, we obtain from \cref{lem:soundeval} that
  % $e,\sigma\normalise t,\sigma'$ with $t,\sigma'\Consistent_M\tilde{t},\tilde{\sigma}',\Phi\land\phi_1$,
  % which is exactly what we needed to show.
  % Furthermore, by \cref{lem:soundstride} we obtain that
  % $t,\sigma'\stride t',\sigma''$ with $t',\sigma''\Consistent_M\tilde{t}',\tilde{\sigma}'',\Phi\land\phi_1\land\phi_2$.
  % Then finally, by application of the induction hypothesis, we obtain what we needed to prove.
  % $t',\sigma''\normalise t'',\sigma'''$ with $t'',\sigma'''\Consistent_M\tilde{t}'',\tilde{\sigma}''',\Phi\land\phi_1\land\phi_2\land\phi_3$.
\end{proof}

\begin{proof}[Completeness of stride]



  % Provided that $t,\sigma\Consistent_M\tilde{t},\tilde{\sigma},\Phi$ and $\tilde{t},\tilde{\sigma}\simstride \overline{\tilde{t}',\tilde{\sigma}',\phi}$,
  % we want to show that for all pairs $(\tilde{t}',\tilde{\sigma}',\phi)$,
  % we have $\Sat(\Phi\land\phi)$ implies that $t,\sigma\stride t',\sigma'$
  % We prove Lemma~\ref{lem:soundstride} by induction over $t$.



  \Case{$t=\Edit v$}
  { One rule applies, namely \userule{S-Edit}\\
    % Given that $t,\sigma\Consistent_M\Edit\tilde{v},\tilde{\sigma},\Phi$ and $\Edit\tilde{v},\tilde{\sigma}\simstride\Edit\tilde{v},\tilde{\sigma},\True$,
    % we know that $t=\Edit M \tilde{v}$, and we have $\Edit M \tilde{v},\sigma\stride\Edit M\tilde{v},\sigma$ by \refrule{S-Edit} and
    % $\Edit M \tilde{v},\sigma\Consistent_M\Edit\tilde{v},\tilde{\sigma},\Phi$, since none of the tasks and states were altered.
   }
  %
   \Case{$t=\Enter \tau$}
  {One rule applies, namely \userule{S-Fill}\\
  % Given that $t,\sigma\Consistent_M\Enter \tau,\tilde{\sigma},\Phi$ and $\Enter \tau,\tilde{\sigma}\simstride\Enter \tau,\tilde{\sigma},\True$,
  % we know that $t=\Enter \tau$, and we have $\Enter \tau,\sigma\stride\Enter \tau,\sigma$ by \refrule{S-Fill} and
  % $\Enter \tau,\sigma\Consistent_M\Edit\tilde{v},\tilde{\sigma},\Phi$, since none of the tasks and states were altered.
  }

  \Case{$t=\Update l$}
   {One rule applies, namely \userule{S-Update}\\
   % Given that $t,\sigma\Consistent_M\Update l,\tilde{\sigma},\Phi$ and $\Update l,\tilde{\sigma}\simstride\Update l,\tilde{\sigma},\True$,
   % we know that $t=\Update l$, and we have $\Update l,\sigma\stride\Update l,\sigma$ by \refrule{S-Update} and
   % $\Update l,\sigma\Consistent_M\Update l,\tilde{\sigma},\Phi$, since none of the tasks and states were altered.
  }

  \Case{$t=\Fail$}
   {One rule applies, namely \userule{S-Fail}\\
   % Given that $t,\sigma\Consistent_M\Fail,\tilde{\sigma},\Phi$ and $\Fail,\tilde{\sigma}\simstride\Fail,\tilde{\sigma},\True$,
   % we know that $t=\Fail$, and we have $\Fail,\sigma\stride\Fail,\sigma$ by \refrule{S-Fail} and
   % $\Fail,\sigma\Consistent_M\Fail,\tilde{\sigma},\Phi$, since none of the tasks and states were altered.
   }
  %
  \Case{$t=e_1\Xor e_2$}
   {One rule applies, namely \userule{S-Xor}\\
   % Given that $t,\sigma\Consistent_M\tilde{e}_1\Xor \tilde{e}_2,\tilde{\sigma},\Phi$ and $\tilde{e}_1\Xor \tilde{e}_2,\tilde{\sigma}\simstride\tilde{e}_1\Xor \tilde{e}_2,\tilde{\sigma},\True$,
   % we know that $t=M\tilde{e}_1\Xor M\tilde{e}_2$, and we have $M\tilde{e}_1\Xor M\tilde{e}_2,\sigma\stride M\tilde{e}_1\Xor M\tilde{e}_2,\sigma$ by \refrule{S-Xor} and
   % $M\tilde{e}_1\Xor M\tilde{e}_2,\sigma\Consistent_M\tilde{e}_1\Xor \tilde{e}_2,\tilde{\sigma},\Phi$, since none of the tasks and states were altered.
   }




  \Case{$t=t_1\Then e_2$}
  {
  Three rules apply.\\
  \Case{\userule{S-ThenStay}}
   {
     % Provided that $t,\sigma\Consistent_M\tilde{t}_1\Then \tilde{e}_2,\tilde{\sigma},\Phi$
     % and $\tilde{t}_1\Then \tilde{e}_2,\tilde{\sigma}\simstride\tilde{t}'_1\Then \tilde{e}_2,\tilde{\sigma}',\phi_1$,
     % we obtain from the induction hypothesis that $t_1,\sigma\stride t_1',\sigma'$ and $t_1',\sigma'\Consistent_M\tilde{t}_1',\tilde{\sigma}',\Phi$.
     % From this, we can directly conclude that $t_1\Then e_2,\sigma\stride t_1'\Then e_2,\sigma'$ and $t_1'\Then e_2,\sigma'\Consistent_M\tilde{t}_1'\Then\tilde{e}_2,\tilde{\sigma}',\Phi$.
  }
  \Case{\userule{S-ThenFail}}
   {
   % Provided that $t,\sigma\Consistent_M\tilde{t}_1\Then \tilde{e}_2,\tilde{\sigma},\Phi$
   % and $\tilde{t}_1\Then \tilde{e}_2,\tilde{\sigma}\simstride\tilde{t}'_1\Then \tilde{e}_2,\tilde{\sigma}',\phi_1$,
   % we obtain from the induction hypothesis that $t_1,\sigma\stride t_1',\sigma'$ and $t_1',\sigma'\Consistent_M\tilde{t}_1',\tilde{\sigma}',\Phi$.
   % From this, we can directly conclude that $t_1\Then e_2,\sigma\stride t_1'\Then e_2,\sigma'$ and $t_1'\Then e_2,\sigma'\Consistent_M\tilde{t}_1'\Then\tilde{e}_2,\tilde{\sigma}',\Phi$.
   }
  \Case{\userule{S-ThenCont}}
   {
   % Provided that $t,\sigma\Consistent_M\tilde{t}_1\Then \tilde{e}_2,\tilde{\sigma},\Phi$
   % and $\tilde{t}_1\Then \tilde{e}_2,\tilde{\sigma}\simstride\tilde{t}_2,\tilde{\sigma}',\phi_1\land\phi_2$
   % with $\tilde{t}_1,\tilde{\sigma}\simstride\tilde{t}_1',\tilde{\sigma}',\phi_1$ and $\Value(\tilde{t}_1',\tilde{\sigma}')=\tilde{v}_1$,
   % we obtain from the induction hypothesis that $t_1,\sigma\stride t_1',\sigma'$ and $t_1',\sigma'\Consistent_M\tilde{t}_1',\tilde{\sigma}',\Phi$.
   % Then from the consistence relation, we can conclude that $\Value(t_1',\sigma')=\Value(M t_1',M \sigma')=M\tilde{v}_1$.
   %
   % At this point, we have $e_2 M\tilde{v}_1,\sigma'\Consistent_M\tilde{e}_2 \tilde{v}_1,\tilde{\sigma}',\Phi\land\phi_1$ and $\tilde{e}_2 \tilde{v}_1,\tilde{\sigma}'\tilde{\eval}\tilde{t}_2,\tilde{sigma}'',\phi_2$.
   % This allows us to apply \cref{lem:soundeval} to obtain $e_2 (M\tilde{v}_1),\sigma'\eval t_2,\sigma''$ and $t_2,\sigma''\Consistent_M\tilde{t}_2,\tilde{\sigma}'',\Phi\land\phi_1\land\phi_2$.
   %
   % From this, we can directly conclude that $t_1\Then e_2,\sigma\stride t_2,\sigma''$ and $t_2,\sigma''\Consistent_M\tilde{t}_2,\tilde{\sigma}'',\Phi\land\phi_1\land\phi_2$.
   }
  }

  \Case{$t=t_1\Or t_2$}
  {
  One of three rules applies.\\
  \Case{\userule{S-OrLeft}}
   {
   % Provided that $t,\sigma\Consistent_M\tilde{t}_1\Or \tilde{t}_2,\tilde{\sigma},\Phi$
   % and $\tilde{t}_1\Or \tilde{t}_2,\tilde{\sigma}\simstride\tilde{t}'_1,\tilde{\sigma}',\phi$,
   % we obtain from the induction hypothesis that $t_1,\sigma\stride t_1',\sigma'$ and $t_1',\sigma'\Consistent_M\tilde{t}_1',\tilde{\sigma}',\Phi\land\phi$.
   % From this, we can directly conclude that $t_1\Or t_2,\sigma\stride t_1',\sigma'$.
  }
  \Case{\userule{S-OrRight}}
   {
   % Provided that $t,\sigma\Consistent_M\tilde{t}_1\Or \tilde{t}_2,\tilde{\sigma},\Phi$
   % and $\tilde{t}_1\Or \tilde{t}_2,\tilde{\sigma}\simstride\tilde{t}'_2,\tilde{\sigma}'',\phi_1\land\phi_2$,
   % we obtain from the induction hypothesis that $t_1,\sigma\stride t_1',\sigma'$ and $t_1',\sigma'\Consistent_M\tilde{t}_1',\tilde{\sigma}',\Phi\land\phi_1$.
   % Then by a second application of the induction hypothesis, we obtain that $t_2,\sigma'\stride t_2',\sigma''$ and $t_2',\sigma''\Consistent_M\tilde{t}_2',\tilde{\sigma}'',\Phi\land\phi_1\land\phi_2$.
   % This leads us to conclude $t_1\Or t_2,\sigma\stride t_2',\sigma''$.
   }
  \Case{\userule{S-OrNone}}
  {
  % Provided that $t,\sigma\Consistent_M\tilde{t}_1\Or \tilde{t}_2,\tilde{\sigma},\Phi$
  % and $\tilde{t}_1\Or \tilde{t}_2,\tilde{\sigma}\simstride\tilde{t}'_2,\tilde{\sigma}'',\phi_1\land\phi_2$,
  % we obtain from the induction hypothesis that $t_1,\sigma\stride t_1',\sigma'$ and $t_1',\sigma'\Consistent_M\tilde{t}_1',\tilde{\sigma}',\Phi\land\phi_1$.
  % Then by a second application of the induction hypothesis, we obtain that $t_2,\sigma'\stride t_2',\sigma''$ and $t_2',\sigma''\Consistent_M\tilde{t}_2',\tilde{\sigma}'',\Phi\land\phi_1\land\phi_2$.
  % This leads us to conclude $t_1\Or t_2,\sigma\stride t_1'\Or t_2',\sigma''$ and $t_1'\Or t_2',\sigma''\Consistent_M\tilde{t}_1'\Or\tilde{t}_2',\tilde{\sigma}'',\Phi\land\phi_1\land\phi_2$.
   }
  }

  \Case{$t=t_1\Next e_2$}
  {
  One rule applies, namely \userule{S-Next}\\
  % Provided that $t,\sigma\Consistent_M\tilde{t}_1\Next \tilde{e}_2,\tilde{\sigma},\Phi$
  % and $\tilde{t}_1\Next \tilde{e}_2,\tilde{\sigma}\simstride\tilde{t}'_1\Then \tilde{e}_2,\tilde{\sigma}',\phi$,
  % we obtain from the induction hypothesis that $t_1,\sigma\stride t_1',\sigma'$ and $t_1',\sigma'\Consistent_M\tilde{t}_1',\tilde{\sigma}',\Phi\land\phi$.
  % From this, we can directly conclude that $t_1\Next e_2,\sigma\stride t_1'\Next e_2,\sigma'$ and $t_1'\Next e_2,\sigma'\Consistent_M\tilde{t}_1'\Next\tilde{e}_2,\tilde{\sigma}',\Phi\land\phi$.
  }

  \Case{$t=t_1\And t_2$}
  {
  One rule applies, namely \userule{S-And}\\
  % Provided that $t,\sigma\Consistent_M\tilde{t}_1\And \tilde{t}_2,\tilde{\sigma},\Phi$
  % and $\tilde{t}_1\And \tilde{t}_2,\tilde{\sigma}\simstride\tilde{t}'_1\And\tilde{t}'_2,\tilde{\sigma}'',\phi_1\land\phi_2$,
  % we obtain from the induction hypothesis that $t_1,\sigma\stride t_1',\sigma'$ and $t_1',\sigma'\Consistent_M\tilde{t}_1',\tilde{\sigma}',\Phi\land\phi_1$.
  % Then by a second application of the induction hypothesis, we obtain that $t_2,\sigma'\stride t_2',\sigma''$ and $t_2',\sigma''\Consistent_M\tilde{t}_2',\tilde{\sigma}'',\Phi\land\phi_1\land\phi_2$.
  % This leads us to conclude $t_1\And t_2,\sigma\stride t_1'\And t_2',\sigma''$ and $t_1'\And t_2',\sigma''\Consistent_M\tilde{t}_1'\And\tilde{t}_2',\tilde{\sigma}'',\Phi\land\phi_1\land\phi_2$.

  }
\end{proof}

\begin{proof}[Completeness of evaluate]
  We prove Lemma~\ref{lem:completeEval} by induction over $e$.

  \Case{$e=v$}
    {One rule applies, namely \userule{E-Value}\\
    Since $v,\sigma\Consistent_M \tilde{e},\tilde{\sigma},\Phi$, we know that $\tilde{e}=\tilde{v}$.
    By \refrule{SE-Value}, we have $\tilde{v},\tilde{\sigma}\tilde{\eval}\tilde{v},\tilde{\sigma},\True$.
    Since the expressions did not change, this case holds trivially.
    }

  \Case{$e=\tuple{e_1,e_2}$}
    {
    Provided that $\tuple{e_1, e_2},\sigma\Consistent_M \tilde{e},\tilde{\sigma},\Phi$ and \userule{E-Pair},
    then by application of the induction hypothesis we obtain $\tilde{e}_1,\tilde{\sigma}\tilde{\eval}\tilde{v}_1,\tilde{\sigma}',\phi_1$
    and $v_1,\sigma'\Consistent_M \tilde{v}_1,\tilde{\sigma}',\Phi\land\phi_1$.
    A second application of the induction hypothesis gives us  $\tilde{e}_2,\tilde{\sigma}'\tilde{\eval}\tilde{v}_2,\tilde{\sigma}'',\phi_2$
    and $v_2,\sigma''\Consistent_M \tilde{v}_2,\tilde{\sigma}'',\Phi\land\phi_2$.
    By \refrule{SE-Pair}, we have $\tuple{\tilde{e}_1, \tilde{e}_2},\tilde{\sigma}\tilde{\eval}\tuple{\tilde{v}_1,\tilde{v}_2},\tilde{\sigma}'',\phi_1\land\phi_2$ and $\tuple{v_1,v_2},\sigma''\Consistent_M \tuple{\tilde{v}_1,\tilde{v}_2},\tilde{\sigma}'',\Phi\land\phi_1\land\phi_2$.
    }

  \Case{$e=\Fst \tuple{e_1,e_2}$}
  {
    Provided that $\Fst \tuple{e_1,e_2},\sigma\Consistent_M \tilde{e},\tilde{\sigma},\Phi$ and \userule{E-First},
    then by application of the induction hypothesis we obtain $\tilde{e}_1,\tilde{\sigma}\tilde{\eval}\tilde{v}_1,\tilde{\sigma}',\phi$
    and $v_1,\sigma'\Consistent_M \tilde{v}_1,\tilde{\sigma}',\Phi\land\phi$.
    By \refrule{SE-First}, we have $\Fst \tuple{\tilde{e}_1,\tilde{e}_2},\tilde{\sigma}\tilde{\eval}\tilde{v}_1,\tilde{\sigma}',\phi$.
    }

  \Case{$e=\Snd \tuple{e_1,e_2}$}
  {
  Provided that $\Snd \tuple{e_1,e_2},\sigma\Consistent_M \tilde{e},\tilde{\sigma},\Phi$ and \userule{E-Second},
  then by application of the induction hypothesis we obtain $\tilde{e}_2,\tilde{\sigma}\tilde{\eval}\tilde{v}_2,\tilde{\sigma}',\phi$
  and $v_2,\sigma'\Consistent_M \tilde{v}_2,\tilde{\sigma}',\Phi\land\phi$.
  By \refrule{SE-Second}, we have $\Snd \tuple{\tilde{e}_1,\tilde{e}_2},\tilde{\sigma}\tilde{\eval}\tilde{v}_2,\tilde{\sigma}',\phi$.
    }

  \Case{$e=e_1::e_2$}
    {
    Provided that $e_1:: e_2,\sigma\Consistent_M \tilde{e},\tilde{\sigma},\Phi$ and \userule{E-Cons},
    then by application of the induction hypothesis we obtain $\tilde{e}_1,\tilde{\sigma}\tilde{\eval}\tilde{v}_1,\tilde{\sigma}',\phi_1$
    and $v_1,\sigma'\Consistent_M \tilde{v}_1,\tilde{\sigma}',\Phi\land\phi_1$.
    A second application of the induction hypothesis gives us  $\tilde{e}_2,\tilde{\sigma}'\tilde{\eval}\tilde{v}_2,\tilde{\sigma}'',\phi_2$
    and $v_2,\sigma''\Consistent_M \tilde{v}_2,\tilde{\sigma}'',\Phi\land\phi_2$.
    By \refrule{SE-Cons}, we have $\tilde{e}_1 :: \tilde{e}_2,\tilde{\sigma}\tilde{\eval}\tilde{v}_1::\tilde{v}_2,\tilde{\sigma}'',\phi_1\land\phi_2$ and $v_1::v_2,\sigma''\Consistent_M \tilde{v}_1::\tilde{v}_2,\tilde{\sigma}'',\Phi\land\phi_1\land\phi_2$.
   }

  \Case{$e=\Head e$}
    {
    Provided that $\Head e,\sigma\Consistent_M \tilde{e},\tilde{\sigma},\Phi$ and \userule{E-Head},
    then by application of the induction hypothesis we obtain $\tilde{e},\tilde{\sigma}\tilde{\eval}\tilde{v}_1 :: \tilde{v}_2,\tilde{\sigma}',\phi$
    and $v_1::v_2,\sigma'\Consistent_M \tilde{v}_1 :: \tilde{v}_2,\tilde{\sigma}',\Phi\land\phi$.
    By \refrule{SE-Head}, we have $\tilde{v}_1 :: \tilde{v}_2,\tilde{\sigma}\tilde{\eval}\tilde{v}_1,\tilde{\sigma}',\phi$.
    }

  \Case{$e=\Tail e$}
    {
    Provided that $\Tail e,\sigma\Consistent_M \tilde{e},\tilde{\sigma},\Phi$ and \userule{E-Tail},
    then by application of the induction hypothesis we obtain $\tilde{e},\tilde{\sigma}\tilde{\eval}\tilde{v}_1 :: \tilde{v}_2,\tilde{\sigma}',\phi$
    and $v_1::v_2,\sigma'\Consistent_M \tilde{v}_1 :: \tilde{v}_2,\tilde{\sigma}',\Phi\land\phi$.
    By \refrule{SE-Tail}, we have $\tilde{v}_1 :: \tilde{v}_2,\tilde{\sigma}\tilde{\eval}\tilde{v}_2,\tilde{\sigma}',\phi$.
      }

  \Case{$e=e_1 e_2$}
    {

    Provided that $e_1 e_2,\sigma\Consistent_M \tilde{e},\tilde{\sigma},\Phi$ and\\
    \userule{E-App},
    then by application of the induction hypothesis we obtain $\tilde{e}_1,\tilde{\sigma}\tilde{\eval}\lambda x:\tau .\tilde{e}_1',\tilde{\sigma}',\phi_1$
    and $\lambda x:\tau .e_1',\sigma'\Consistent_M \lambda x:\tau .\tilde{e}_1',\tilde{\sigma}',\Phi\land\phi_1$.
    A second application of the induction hypothesis gives us  $\tilde{e}_2,\tilde{\sigma}'\tilde{\eval}\tilde{v}_2,\tilde{\sigma}'',\phi_2$
    and $v_2,\sigma''\Consistent_M \tilde{v}_2,\tilde{\sigma}'',\Phi\land\phi_1\land\phi_2$.
    Then finally by a third application of the induction hypothesis, we get  $\tilde{e}_1'[x\mapsto \tilde{v}_2],\tilde{\sigma}''\tilde{\eval}\tilde{v}_1,\tilde{\sigma}''',\phi_3$
    and $v_1,\sigma'''\Consistent_M \tilde{v}_1,\tilde{\sigma}''',\Phi\land\phi_1\land\phi_2\land\phi_3$.
    By \refrule{SE-App}, we have $\tilde{e}_1 \tilde{e}_2,\tilde{\sigma}\tilde{\eval}\tilde{v}_1,\tilde{\sigma}''',\phi_1\land\phi_2\land\phi_2$.
    }

  \Case{$e=\If{e_1}{e_2}{e_3}$}
     {\Case{1}
     {
     Provided that $\If{e_1}{e_2}{e_3},\sigma\Consistent_M \tilde{e},\tilde{\sigma},\Phi$ and\\
     \userule{E-IfTrue},
     then by application of the induction hypothesis we obtain $\tilde{e}_1,\tilde{\sigma}\tilde{\eval}\tilde{v}_1,\tilde{\sigma}',\phi_1$
     and $\True,\sigma'\Consistent_M \tilde{v}_1,\tilde{\sigma}',\Phi\land\phi_1$.
     A second application of the induction hypothesis gives us  $\tilde{e}_2,\tilde{\sigma}'\tilde{\eval}\tilde{v}_2,\tilde{\sigma}'',\phi_2$
     and $v_2,\sigma''\Consistent_M \tilde{v}_2,\tilde{\sigma}'',\Phi\land\phi_1\land\phi_2$.
     By \refrule{SE-If}, we have $\If{\tilde{e}_1}{\tilde{e}_2}{\tilde{e}_3},\tilde{\sigma}\tilde{\eval}\tilde{v}_2,\tilde{\sigma}'',\phi_1\land\phi_2\land \tilde{v}_1$.
     }
      \Case{2}{
      Provided that $\If{e_1}{e_2}{e_3},\sigma\Consistent_M \tilde{e},\tilde{\sigma},\Phi$ and\\
      \userule{E-IfFalse},
      then by application of the induction hypothesis we obtain $\tilde{e}_1,\tilde{\sigma}\tilde{\eval}\tilde{v}_1,\tilde{\sigma}',\phi_1$
      and $\False,\sigma'\Consistent_M \tilde{v}_1,\tilde{\sigma}',\Phi\land\phi_1$.
      A second application of the induction hypothesis gives us  $\tilde{e}_3,\tilde{\sigma}'\tilde{\eval}\tilde{v}_3,\tilde{\sigma}'',\phi_2$
      and $v_3,\sigma''\Consistent_M \tilde{v}_3,\tilde{\sigma}'',\Phi\land\phi_1\land\phi_2$.
      By \refrule{SE-If}, we have $\If{\tilde{e}_1}{\tilde{e}_2}{\tilde{e}_3},\tilde{\sigma}\tilde{\eval}\tilde{v}_3,\tilde{\sigma}'',\phi_1\land\phi_3\land \neg\tilde{v}_1$.
      }

    }

  \Case{$e=\Ref e$}
    {
    Provided that $\Ref e,\sigma\Consistent_M \tilde{e},\tilde{\sigma},\Phi$ and \userule{E-Ref},
    then by application of the induction hypothesis we obtain $\tilde{e},\tilde{\sigma}\tilde{\eval}\tilde{v},\tilde{\sigma}',\phi$
    and $v,\sigma'\Consistent_M \tilde{v},\tilde{\sigma}',\Phi\land\phi$.
    By \refrule{SE-Ref}, we have $\Ref \tilde{e},\tilde{\sigma}\tilde{\eval}l,\tilde{\sigma}'[l\mapsto\tilde{v}],\phi$ and $l,\sigma'[l\mapsto v]\Consistent_M l,\tilde{\sigma}'[f\mapsto \tilde{v}],\Phi\land\phi$.
    }

  \Case{$e=!e$}
    {
    Provided that $!e,\sigma\Consistent_M \tilde{e},\tilde{\sigma},\Phi$ and \userule{E-Deref},
    then by application of the induction hypothesis we obtain $\tilde{e},\tilde{\sigma}\tilde{\eval}l,\tilde{\sigma}',\phi$
    and $l,\sigma'\Consistent_M l,\tilde{\sigma}',\Phi\land\phi$.
    By \refrule{SE-Deref}, we have $!\tilde{e},\tilde{\sigma}\tilde{\eval}\tilde{\sigma}'(l),\tilde{\sigma}',\phi$ and $\sigma'(l),\sigma'\Consistent_M \tilde{\sigma}'(l),\tilde{\sigma}',\Phi\land\phi$.
  }

  \Case{$e=e_1:=e_2$}
    {
    Provided that $e_1:= e_2,\sigma\Consistent_M \tilde{e},\tilde{\sigma},\Phi$ and \userule{E-Assign},
    then by application of the induction hypothesis we obtain $\tilde{e}_1,\tilde{\sigma}\tilde{\eval}l,\tilde{\sigma}',\phi_1$
    and $l,\sigma'\Consistent_M l,\tilde{\sigma}',\Phi\land\phi_1$.
    A second application of the induction hypothesis gives us  $\tilde{e}_2,\tilde{\sigma}'\tilde{\eval}\tilde{v}_2,\tilde{\sigma}'',\phi_2$
    and $v_2,\sigma''\Consistent_M \tilde{v}_2,\tilde{\sigma}'',\Phi\land\phi_2$.
    By \refrule{SE-Assign}, we have $\tilde{e}_1 := \tilde{e}_2,\tilde{\sigma}\tilde{\eval}\unit,\tilde{\sigma}''[l\mapsto\tilde{v}_2],\phi_1\land\phi_2$ and $\Unit,\sigma''[l\mapsto v_2]\Consistent_M \Unit,\tilde{\sigma}''[l\mapsto\tilde{v}_2],\Phi\land\phi_1\land\phi_2$.
    }

  \Case{$e=\Edit e$}
    {
    Provided that $\Edit e,\sigma\Consistent_M \tilde{e},\tilde{\sigma},\Phi$ and \userule{E-Edit},
    then by application of the induction hypothesis we obtain $\tilde{e},\tilde{\sigma}\tilde{\eval}\tilde{v},\tilde{\sigma}',\phi$
    and $v,\sigma'\Consistent_M \tilde{v},\tilde{\sigma}',\Phi\land\phi$.
    By \refrule{SE-Edit}, we have $\Edit\tilde{e},\tilde{\sigma}\tilde{\eval}\Edit\tilde{v},\tilde{\sigma}',\phi$ and $\Edit v,\sigma'\Consistent_M \Edit \tilde{v},\tilde{\sigma}',\Phi\land\phi$.

    }

  \Case{$e=\Enter \tau$}
    {
    One rule applies, namely \userule{E-Enter}\\
    Since $\Enter \tau,\sigma\Consistent_M \tilde{e},\tilde{\sigma},\Phi$, we know that $\tilde{e}=\Enter \tau$.
    By \refrule{SE-Enter}, we have $\Enter \tau,\tilde{\sigma}\tilde{\eval}\Enter \tau,\tilde{\sigma},\True$.
    Since the expressions did not change, this case holds trivially.
    }

  \Case{$e=\Update e$}
    {Provided that $\Update e,\sigma\Consistent_M \tilde{e},\tilde{\sigma},\Phi$ and \userule{E-Update},
    then by application of the induction hypothesis we obtain $\tilde{e},\tilde{\sigma}\tilde{\eval}l,\tilde{\sigma}',\phi$
    and $l,\sigma'\Consistent_M l,\tilde{\sigma}',\Phi\land\phi$.
    By \refrule{SE-Update}, we have $\Update\tilde{e},\tilde{\sigma}\tilde{\eval}\Update l,\tilde{\sigma}',\phi$ and $\Update l,\sigma'\Consistent_M \Update l ,\tilde{\sigma}',\Phi\land\phi$.

    }

  \Case{$e=e_1\Then e_2$}
    {
    Provided that $e_1\Then e_2,\sigma\Consistent_M \tilde{e},\tilde{\sigma},\Phi$ and \userule{E-Then},
    then by application of the induction hypothesis we obtain $\tilde{e}_1,\tilde{\sigma}\tilde{\eval}\tilde{v}_1,\tilde{\sigma}',\phi$
    and $v_1,\sigma'\Consistent_M \tilde{v}_1,\tilde{\sigma}',\Phi\land\phi$.
    By \refrule{SE-Then}, we have $\tilde{e}_1\Then \tilde{e}_2,\tilde{\sigma}\tilde{\eval}\tilde{v}_1\Then\tilde{e}_2,\tilde{\sigma}',\phi$ and $v_1\Then e_2 ,\sigma'\Consistent_M \tilde{v}_1\Then\tilde{e}_2 ,\tilde{\sigma}',\Phi\land\phi$.

    }

  \Case{$e=e_1\Next e_2$}
    {
    Provided that $e_1\Next e_2,\sigma\Consistent_M \tilde{e},\tilde{\sigma},\Phi$ and \userule{E-Next},
    then by application of the induction hypothesis we obtain $\tilde{e}_1,\tilde{\sigma}\tilde{\eval}\tilde{v}_1,\tilde{\sigma}',\phi$
    and $v_1,\sigma'\Consistent_M \tilde{v}_1,\tilde{\sigma}',\Phi\land\phi$.
    By \refrule{SE-Next}, we have $\tilde{e}_1\Next \tilde{e}_2,\tilde{\sigma}\tilde{\eval}\tilde{v}_1\Next\tilde{e}_2,\tilde{\sigma}',\phi$ and $v_1\Then e_2 ,\sigma'\Consistent_M \tilde{v}_1\Next\tilde{e}_2 ,\tilde{\sigma}',\Phi\land\phi$.

    }

  \Case{$e=e_1\Or e_2$}
    {
    Provided that $e_1\Or e_2,\sigma\Consistent_M \tilde{e},\tilde{\sigma},\Phi$ and \userule{E-Or},
    then by application of the induction hypothesis we obtain $\tilde{e}_1,\tilde{\sigma}\tilde{\eval}\tilde{t}_1,\tilde{\sigma}',\phi_1$
    and $t_1,\sigma'\Consistent_M \tilde{t}_1,\tilde{\sigma}',\Phi\land\phi_1$.
    A second application of the induction hypothesis gives us  $\tilde{e}_2,\tilde{\sigma}'\tilde{\eval}\tilde{t}_2,\tilde{\sigma}'',\phi_2$
    and $t_2,\sigma''\Consistent_M \tilde{t}_2,\tilde{\sigma}'',\Phi\land\phi_2$.
    By \refrule{SE-Or}, we have $\tilde{e}_1\Or \tilde{e}_2,\tilde{\sigma}\tilde{\eval}\tilde{t}_1\Or\tilde{t}_2,\tilde{\sigma}'',\phi_1\land\phi_2$ and $t_1\Or t_2 ,\sigma''\Consistent_M \tilde{t}_1\Or\tilde{t}_2 ,\tilde{\sigma}'',\Phi\land\phi_1\land\phi_2$.

    }

  \Case{$e=e_1\Xor e_2$}
    {  One rule applies, namely \userule{E-Xor}\\
    Since $e_1\Xor e_2,\sigma\Consistent_M \tilde{e},\tilde{\sigma},\Phi$, we know that $\tilde{e}=\tilde{e}_1\Xor \tilde{e}_2$.
    By \refrule{SE-Xor}, we have $\tilde{e}_1\Xor \tilde{e}_2,\tilde{\sigma}\tilde{\eval}\tilde{e}_1\Xor \tilde{e}_2,\tilde{\sigma},\True$.
    Since the expressions did not change, this case holds trivially.

    }

  \Case{$e=\Fail$}
    {  One rule applies, namely \userule{E-Fail}\\
    Since $\Fail,\sigma\Consistent_M \tilde{e},\tilde{\sigma},\Phi$, we know that $\tilde{e}=\Fail$.
    By \refrule{SE-Fail}, we have $\Fail,\tilde{\sigma}\tilde{\eval}\Fail,\tilde{\sigma},\True$.
    Since the expressions did not change, this case holds trivially.


    }
\end{proof}


\end{document}
