% !TEX root=../main.tex

\section{Related work}
\label{sec:relatedwork}

This paper is an advancement of the research into supporting end-users of workflow systems.
In previous work, we have attempted to provide end-users with next step hints by viewing workflows as rule based problems~\cite{DBLP:conf/sfp/NausJ16}.
By abstracting over workflows, reasoning about them becomes simpler.
A standard search algorithm can be run to find a path to the desired goal state.
Two drawbacks of this approach however are that only global hints can be given, and that a programmer needs to augment existing workflows with extra information in order to convert it to a rule-based problem.

Stutterheim et al.~\cite{DBLP:conf/sfp/StutterheimPA14} have developed Tonic, a task visualiser for iTasks with limited path prediction capabilities.
The main goal is not to provide hints to end-users, but the system is able to handle the complete task language, and visualise the effects of user input on the progression of tasks.

In order to overcome the problems of our own previous research and the limited use of Tonic for end-user hints, we have combined symbolic execution, together with workflow modelling and next-step hint generation.
To our knowledge, this is the first work describing the combination of these techniques in this way.
The different components coming together in this paper have been studied extensively.
The following sections give an overview of the work done in those areas.

\subsection{Symbolic execution}


Symbolic execution \cite{King1975,Boyer1975} is typically being applied to imperative programming languages, for example Bucur et al.~\cite{BucurKC2014} prototype a symbolic execution engine for interpreted imperative languages.
Cadar et al.~\cite{CadarDE2008} use it to generate test cases for programs that can be compiled to LLVM bytecode.
Jaffar et al.~\cite{JaffarMNS2012} use it for verifying safety properties of C programs.

In recent years it has been used for functional programming languages as well.
To name some examples, there is ongoing work by Hallahan et al.~\cite{HallahanXP2017} and Xue~\cite{Xue2019} to implement a symbolic execution engine for Haskell.
Giantios et al.~\cite{GiantsiosPS2017} use symbolic execution for a mix of concrete and symbolic testing of programs written in a subset of Core Erlang.
Their goal is to find executions that lead to a runtime error, either due to an assertion violation or an unhandled exception.
Chang et al.~\cite{ChangKT2018} present a symbolic execution engine for a typed lambda calculus with mutable state where only some language constructs recognize symbolic values.
They claim that their approach is easier to implement than full symbolic execution and simplifies the burden on the solver, while still considering all execution paths.

The difficulty of symbolic execution for functional languages lies in symbolic higher-order values, that is functions as arguments to other functions.
Hallahan et al solve this with a technique called \emph{defunctionalization}, which requires all source code to be present, so that a symbolic function can only be one of the present lambda expressions or function definitions.
Giantosis et al also require all source code to be present, but they only analyze first-order functions.
They can execute higher-order functions, but only with concrete arguments.
Our method also requires closed well-typed terms, so we never execute a higher-order function in isolation.
Furthermore, we currently do not allow functions and tasks as task values.
Together, this means that symbolic values can never be functions.
\todo{THIS SECTION IS COPIED OVER FROM SYMBOLIC TOPHAT! DO NOT SEND IN LIKE THIS!}

\subsection{Workflow modelling}

Workflows have been studied extensively from different viewpoints.
Since many software exists that automates workflows, it is a research topic that potentially can have a huge impact on society.

\emph{Workflow patterns} are regarded as special design patterns in software engineering.
Similar to the combinators in \TOP, they describe recurring patterns in workflow systems.
Van der Aalst et al.~\cite{journals/dpd/AalstHKB03} identify common patterns, and examine their availability in industry workflow frameworks.

\emph{Workflow Nets} allow for the moddeling an analysis of business processes.
They are graphical in nature, based on petrinets.

\emph{iTasks} is an implementation of \TOP.



\subsection{Intelligent tutoring systems}
