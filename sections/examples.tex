% !TEX root=../main.tex

\section{Examples}
\label{sec:examples}

\subsection{Dining Philosophers Problem}

The dining philosophers problem is a classic concurrency problem in computer science.
A number of philosophers sit at a round table with a meal in front of them.
In between the plates lies one fork.
In order to eat their meal, each philosopher as to acquire both forks.
This of course means that the philosophers cannot all eat at the same time, since there are not enough forks.

\cref{lst:dining} lists an implementation in \TOPHAT for this problem, with three philosophers.
\lstset{emph={}}
\begin{TASK}[float=ht
            ,numbers=right
            ,caption=Dining Philosophers Problem with three philosophers
            ,label=lst:dining
            ]
  let fork0 = Ref True in
  let fork1 = Ref True in
  let fork2 = Ref True in

  let philosopher = \ name. \ leftFork. \ rightFork.
          if !leftFork
             then edit leftFork := False >>? if !rightFork
                                          then edit leftFork := True
                                          else Fail
             else Fail

          <?>

          if !rightFork
             then edit rightFork := False >>? if !leftFork
                                          then edit rightFork := True
                                          else Fail
             else Fail

          in

    philosopher "Alan Turing" fork0 fork1 <&>
    philosopher "Grace Hopper" fork1 fork2 <&>
    philosopher "Ada Lovelace" fork2 fork0

    >>= \ _ . edit "Done"
\end{TASK}

The forks are represented by shares, containing a boolean.
The value True indicates that the fork is available.

Each philosopher takes as arguments the two forks that he or she can reach.
They have a choice to take either the left or the right fork, provided that the fork is available.
After that, they can press continue if the second fork is also available.
For the sake of simplicity, they return the first fork, rather than setting the second fork to False, and then setting both to True again.


\subsection{Tax subsidy request}

The example program listed in this section is taken from our previous work on symbolic execution for \TOPHAT~\cite{Steenvoorden2019}.
It models a simplified tax subsidy application process for citizens who have installed solar panels.

A subsidy is only given under the following conditions.
\begin{itemize}
\item The roofing company has confirmed that they installed solar panels for the citizen.
\item The tax officer has approved the request.
\item The tax officer can only approve the request if the roofing company has confirmed, and the request is filed within one year of the invoice date.
\item The amount of the granted subsidy is maximal 600 EUR.
\end{itemize}

The listing below gives the \TOPHAT code for this example.

\lstset{emph={invoiceDate,today,confirmed,invoiceAmount,approved}}
\begin{TASK}[float=ht
            ,numbers=right
            ,caption=Subsidy request and approval workflow at the Dutch tax office.
            ,label=lst:tax
            ]
  let getCurrentDate = enter Date in

    let provideDocuments = enter Amount <&> enter Date in

      let companyConfirm = edit True <?> edit False in

        let officerApprove = \ invoiceDate. \ today. \ confirmed.
          edit False <?> if (today - invoiceDate < 365 /\ confirmed) |\label{lst:tax:officer-approve-def}| then edit True else fail in

            getCurrentDate >>= \ today.|\label{lst:tax:citizen-info}|
            provideDocuments <&> companyConfirm >>= |\label{lst:tax:documents-and-company-confirm}|
              \ <<<<invoiceAmount, invoiceDate>>, confirmed>>.
              officerApprove invoiceDate today confirmed >>= \ approved.|\label{lst:tax:officer-approve}|

          let subsidyAmount = if approved then min 600 (invoiceAmount / 10) else 0 in

            edit <<subsidyAmount, approved, confirmed, invoiceDate, today>>|\label{lst:tax:result}|
\end{TASK}

In previous work, we have shown that this code indeed adheres to the requirements listed above.

Instead of assisting the developer, by proving the program correct, we would now like to support the end user that is requesting a subsidy.

The end user wants the outcome of this program to be a subsidy amount larger than zero.

Running the simulation function will result in the following set:

\begin{tabular}{l|l|l}
  Symbolic value & Symbolic input & Path condition\\
  \hline
  $(min\ 600\ (s_{amount}/10),  \True, \True, s_i, s_t)$ & $[s_t,\Left \Left s_{amount}, \Left \Right s_i, \Right \Left, \Right]$ & $s_t-s_i<365$\\
  $(min\ 600\ (s_{amount}/10),  \True, \True, s_i, s_t)$ & $[s_t,\Left \Right s_i, \Left \Left s_{amount}, \Right \Left, \Right]$ & $s_t-s_i<365$\\
  $(min\ 600\ (s_{amount}/10),  \True, \True, s_i, s_t)$ & $[s_t,\Right \Left, \Left \Left s_{amount}, \Left \Right s_i, \Right]$ & $s_t-s_i<365$\\
  $(min\ 600\ (s_{amount}/10),  \True, \True, s_i, s_t)$ & $[s_t,\Right \Left, \Left \Right s_i, \Left \Left s_{amount}, \Right]$ & $s_t-s_i<365$\\
  $(min\ 600\ (s_{amount}/10),  \True, \True, s_i, s_t)$ & $[s_t,\Left \Right s_i, \Right \Left, \Left \Left s_{amount}, \Right]$ & $s_t-s_i<365$\\
  $(min\ 600\ (s_{amount}/10),  \True, \True, s_i, s_t)$ & $[s_t,\Left \Left s_{amount}, \Right \Left, \Left \Right s_i, \Right]$ & $s_t-s_i<365$\\
  $(                        0, \False, \True, s_i, s_t)$ & $[s_t,\Left \Left s_{amount}, \Left \Right s_i, \Right \Left, \Left]$  & $\True$\\
  $(                        0, \False, \True, s_i, s_t)$ & $[s_t,\Left \Right s_i, \Left \Left s_{amount}, \Right \Left, \Left]$  & $\True$\\
  $(                        0, \False, \True, s_i, s_t)$ & $[s_t,\Right \Left, \Left \Left s_{amount}, \Left \Right s_i, \Left]$  & $\True$\\
  $(                        0, \False, \True, s_i, s_t)$ & $[s_t,\Right \Left, \Left \Right s_i, \Left \Left s_{amount}, \Left]$  & $\True$\\
  $(                        0, \False, \True, s_i, s_t)$ & $[s_t,\Left \Right s_i, \Right \Left, \Left \Left s_{amount}, \Left]$  & $\True$\\
  $(                        0, \False, \True, s_i, s_t)$ & $[s_t,\Left \Left s_{amount}, \Right \Left, \Left \Right s_i, \Left]$  & $\True$\\
  $(                        0, \False,\False, s_i, s_t)$ & $[s_t,\Left \Left s_{amount}, \Left \Right s_i, \Right\Right, \Left]$  & $\True$\\
  $(                        0, \False,\False, s_i, s_t)$ & $[s_t,\Left \Right s_i, \Left \Left s_{amount}, \Right\Right, \Left]$  & $\True$\\
  $(                        0, \False,\False, s_i, s_t)$ & $[s_t,\Right \Right,\Left \Left s_{amount}, \Left \Right s_i, \Left]$  & $\True$\\
  $(                        0, \False,\False, s_i, s_t)$ & $[s_t,\Right\Right, \Left \Right s_i, \Left \Left s_{amount}, \Left]$  & $\True$\\
  $(                        0, \False,\False, s_i, s_t)$ & $[s_t,\Left \Right s_i, \Right\Right, \Left \Left s_{amount}, \Left]$  & $\True$\\
  $(                        0, \False,\False, s_i, s_t)$ & $[s_t,\Left \Left s_{amount}, \Right\Right, \Left \Right s_i, \Left]$  & $\True$
\end{tabular}


As a goal we set that the subsidy amount should be larger than zero. This leaves us with the following cases.

\begin{tabular}{l|l|l}
  Symbolic value & Symbolic input & Path condition\\
  \hline
  $(min\ 600\ (s_{amount}/10),  \True, \True, s_i, s_t)$ & $[s_t,\Left \Left s_{amount}, \Left \Right s_i, \Right \Left, \Right]$ & $s_t-s_i<365$\\
  $(min\ 600\ (s_{amount}/10),  \True, \True, s_i, s_t)$ & $[s_t,\Left \Right s_i, \Left \Left s_{amount}, \Right \Left, \Right]$ & $s_t-s_i<365$\\
  $(min\ 600\ (s_{amount}/10),  \True, \True, s_i, s_t)$ & $[s_t,\Right \Left, \Left \Left s_{amount}, \Left \Right s_i, \Right]$ & $s_t-s_i<365$\\
  $(min\ 600\ (s_{amount}/10),  \True, \True, s_i, s_t)$ & $[s_t,\Right \Left, \Left \Right s_i, \Left \Left s_{amount}, \Right]$ & $s_t-s_i<365$\\
  $(min\ 600\ (s_{amount}/10),  \True, \True, s_i, s_t)$ & $[s_t,\Left \Right s_i, \Right \Left, \Left \Left s_{amount}, \Right]$ & $s_t-s_i<365$\\
  $(min\ 600\ (s_{amount}/10),  \True, \True, s_i, s_t)$ & $[s_t,\Left \Left s_{amount}, \Right \Left, \Left \Right s_i, \Right]$ & $s_t-s_i<365$
\end{tabular}

From the end states we can conclude that invoiceAmount should be larger than zero in order for the result to be larger than zero. This results in an additional constraint.

In the end, we can use the above to give hints to the user.

First, a date must be entered. Then the user can choose to ender an amount larger than zero, a date that is within 365 from now, or a "RL".
etc.
