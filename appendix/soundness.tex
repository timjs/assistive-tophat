% !TEX root=../main.tex

\section{Soundness proofs}
\label{sec:soundnessproofs}

\begin{proof}[Soundness of handle]

  We prove \cref{lem:soundhandle} by induction over $\tilde{t}$.

  \case{$\tilde{t}=\Edit v$}
  {One rule applies, namely \userule{SH-Change}\\
  Since we have $\mathds{C}(t,\sigma,\Edit v,\tilde{\sigma},\Phi,M)$, we know that $t$ must be $\Edit v$ too, $\tilde{t}$ contains no symbols.\fixme{not true! v could be a symbol!}
  There exists only one symbolic execution, namely $\Edit v,\tilde{\sigma}\handle{}\Edit s,\tilde{\sigma},s,\True$.
  We need to show that there exists an $i$ such that $s\sim i$ and $\Edit v,\sigma\handle{i}t',\sigma'$.

  Any concrete value $c$ of the same type as $v$ will do. Now we have to show that we end up with $\mathds{C}(\Edit c,\sigma,\Edit s,\tilde{\sigma},\Phi\land\True,M.[s\mapsto c])$, which holds trivially.
  }

 \case{$\tilde{t}=\Enter \tau$}
{One rule applies, namely \userule{SH-Fill}\\
Since we have $\mathds{C}(t,\sigma,\Enter \tau,\tilde{\sigma},\Phi,M)$, we know that $t$ must be $\Enter \tau$ too, $\tilde{t}$ contains no symbols.
There exists only one symbolic execution, namely $\Enter \tau,\tilde{\sigma}\handle{}\Edit s,\tilde{\sigma},s,\True$.
We need to show that there exists an $i$ such that $s\sim i$ and $\Edit v,\sigma\handle{i}t',\sigma'$.

Any concrete value $c$ of type $\tau$ will do. Now we have to show that we end up with $\mathds{C}(\Edit c,\sigma,\Edit s,\tilde{\sigma},\Phi\land\True,M.[s\mapsto c])$, which holds trivially.
}

\case{$\tilde{t}=\Update l$}
{One rule applies, namely \userule{SH-Update}\\

Since we have $\mathds{C}(t,\sigma,\Update l,\tilde{\sigma},\Phi,M)$, we know that $t$ must be $\Update l$ too, $\tilde{t}$ contains no symbols.
There exists only one symbolic execution, namely $\Update l,\tilde{\sigma}\handle{}\Update l,\tilde{\sigma}[l\mapsto s],s,\True$.
We need to show that there exists an $i$ such that $s\sim i$ and $\Update l,\sigma\handle{i}t',\sigma'$.

Any concrete value $c$ of the same type as $l$ will do. Now we have to show that we end up with $\mathds{C}(\Update l,\sigma[l\mapsto c],\Update l,\tilde{\sigma}[l\mapsto s],\Phi\land\True,M.[s\mapsto c])$, which holds trivially.




% Provided that $M\True$
% we need to demonstrate that  \userule{H-Update} with $\hat{\sigma}=M\sigma$ and $M s = v$,
% $M \Update l \equiv \Update l$ and $ M\sigma[l\mapsto s]\equiv \hat{\sigma}[l\mapsto v]$.
%
% \userule{H-Update} with $\hat{\sigma}=M\sigma$ follows trivially.
% $M \Update l \equiv \Update l$ follows trivially, since locations cannot contain symbols. $ M\sigma[l\mapsto s]\equiv \hat{\sigma}[l\mapsto v]$ can be concluded from the fact that $\hat{\sigma}=M\sigma$ and $M s = v$.
%
}
%
% \case{$t=t_1\Next e_2$}
% {In this case, two rules apply.\\
%
%
% \case{\userule{SH-Next}}
%
% {In the case $M\phi_1$, we need to demonstrate that \userule{H-PassNext} with $\hat{\sigma}=M\sigma$ and $j= M i$,
% $M t_1' \Next e_2 \equiv \hat{t_1'}\Next e_2$ and $M\sigma'\equiv\hat{\sigma'}$.
%
% By the induction hypothesis we obtain the following.\\
% $\forall M_1 . M_1 \phi_1 \implies t_1,M_1\sigma \xrightarrow[]{M_1 i} \hat{t_1'},\hat{\sigma'}\land M_1 t_1'\equiv\hat{t_1'}\land M_1\sigma' \equiv \hat{\sigma'}$
%
% Since $M$ satisfies $\phi$, we have $t_1,M\sigma \xrightarrow[]{M i} \hat{t_1'},\hat{\sigma'}$, $M\sigma'\equiv\hat{\sigma'}$,
% which we needed to show, as well as $M t_1' \Next e_2 \equiv \hat{t_1'}\Next e_2$ since this can be concluded from $M t_1'\equiv \hat{t_1'}$.
%
% In the case $M\phi_2$, we need to demonstrate that \userule{H-Next} with $\hat{\sigma}=M\sigma$,
% $M t_2 \equiv \hat{t_2}$ and $M\sigma'\equiv\hat{\sigma'}$.
%
% From Lemma~\ref{lem:soundnorm} we obtain that $\forall M_1. M_1 \phi \implies e_2 v_1,M\sigma\hat{\normalise}\hat{t_2},\hat{\sigma'}\land M t_2\equiv\hat{t_2}\land M \sigma'\equiv\hat{\sigma'}$.
%
% This gives us exactly what we needed to prove this case.
%   }
%
% \case{\userule{SH-PassNext}}
% {
% Provided that $M\phi$, we need to demonstrate that \userule{H-PassNext} with $\hat{\sigma}=M\sigma$ and $j= M i$,
% $M t_1' \Next e_2 \equiv \hat{t_1'}\Next e_2$ and $M\sigma'\equiv\hat{\sigma'}$.
%
% By the induction hypothesis we obtain the following.\\
% $\forall M_1 . M_1 \phi_1 \implies t_1,M_1\sigma \xrightarrow[]{M_1 i} \hat{t_1'},\hat{\sigma'}\land M_1 t_1'\equiv\hat{t_1'}\land M_1\sigma' \equiv \hat{\sigma'}$
%
% Since $M$ satisfies $\phi$, we have $t_1,M\sigma \xrightarrow[]{M i} \hat{t_1'},\hat{\sigma'}$, $M\sigma'\equiv\hat{\sigma'}$,
% which we needed to show, as well as $M t_1' \Next e_2 \equiv \hat{t_1'}\Next e_2$ since this can be concluded from $M t_1'\equiv \hat{t_1'}$.
% }
% }
%
% \case{$t=t_1\Then e_2$}
% {One rule applies, namely \userule{SH-PassThen}\\
% Provided that $M\phi$, we need to demonstrate that \userule{H-PassThen} with $\hat{\sigma}=M\sigma$ and $j= M i$,
% $M t_1'\Then e_2\equiv \hat{t_1'}\Then e_2$ and $M\sigma'\equiv\hat{\sigma'}$.
%
% By the induction hypothesis we obtain the following.\\
% $\forall M_1 . M_1 \phi_1 \implies t_1,M_1\sigma \xrightarrow[]{M_1 i} \hat{t_1'},\hat{\sigma'}\land M_1 t_1'\equiv\hat{t_1'}\land M_1\sigma' \equiv \hat{\sigma'}$
%
% Since $M$ satisfies $\phi$, we have $t_1,M\sigma \xrightarrow[]{M i} \hat{t_1'},\hat{\sigma'}$ and $M\sigma'\equiv\hat{\sigma'}$,
% which we needed to show, as well as $M t_1' \Then e_2 \equiv \hat{t_1'}\Then e_2$ since this can be concluded from $M t_1'\equiv \hat{t_1'}$.
%
% }
%
% \case{$t=e_1\Xor e_2$}
% {
% In this case, three rules apply.\\
%   \case{\userule{SH-Pick}}
%   {
%   Either we have that $M(\phi_1\wedge s=\Left)$ or $M(\phi_2\wedge s=\Right)$.
%   In the first case, the proof is identical to the SH-PickLeft rule.
%   In the second cse, the proof is identical to the SH-PickRight rule.
%   }
%
%   \case{\userule{SH-PickLeft}}
%   {Provided that $M(\phi_2\wedge s=\Left)$, we need to demonstrate that \userule{H-PickLeft} with $\hat{\sigma}=M\sigma$,
%   $M t_1\equiv \hat{t_1}$ and $M\sigma'\equiv \hat{\sigma'}$.
%
%   From Lemma~\ref{lem:soundnorm} we obtain that $\forall M_1. M_1 \phi \implies e_1,M\sigma\hat{\normalise}\hat{t_1},\hat{\sigma'}\land M t_1\equiv\hat{t_1}\land M \sigma'\equiv\hat{\sigma'}$.
%
%   Since $M$ satisfies $\phi$, we have $e_1,M\sigma \hat{\normalise}[] \hat{t_1},\hat{\sigma'}$ and $M\sigma'\equiv\hat{\sigma'}$,
%   which we needed to show, as well as $M t_1 \equiv \hat{t_1}$.
%
%   }
%   \case{\userule{SH-PickRight}}
%   {Provided that $M(\phi_2\wedge s=\Right)$ we need to demonstrate that \userule{H-PickRight} with $\hat{\sigma}=M\sigma$,
%   $M t_2\equiv \hat{t_2}$ and $M\sigma'\equiv \hat{\sigma'}$.
%
%   From Lemma~\ref{lem:soundnorm} we obtain that $\forall M_1. M_1 \phi \implies e_2,M\sigma\hat{\normalise}\hat{t_2},\hat{\sigma'}\land M t_2\equiv\hat{t_2}\land M \sigma'\equiv\hat{\sigma'}$.
%
%   Since $M$ satisfies $\phi$, we have $e_2,M\sigma \hat{\normalise}[] \hat{t_2},\hat{\sigma'}$ and $M\sigma'\equiv\hat{\sigma'}$,
%   which we needed to show, as well as $M t_2 \equiv \hat{t_2}$.
%   }
% }
%
% \case{$t=t_1\And t_2$}
% {
% In this case, two rules apply.\\
%   \case{\userule{SH-FirstAnd}}
%   {Provided that $M\phi$, we need to demonstrate that \userule{H-FirstAnd} with $\hat{\sigma}=M\sigma$,
%   $M t_1'\And t_2\equiv \hat{t_1'}\And t_2$ and $M\sigma'\equiv \hat{\sigma'}$.
%
%   By the induction hypothesis we obtain the following.\\
%   $\forall M_1 . M_1 \phi_1 \implies t_1,M_1\sigma \xrightarrow[]{M_1 i} \hat{t_1'},\hat{\sigma'}\land M_1 t_1'\equiv\hat{t_1'}\land M_1\sigma' \equiv \hat{\sigma'}$
%
%   Since $M$ satisfies $\phi$, we have $t_1,M\sigma\xrightarrow[]{M i} \hat{t_1'},\hat{\sigma'}$ and $M\sigma'\equiv\hat{\sigma'}$,
%   which we needed to show, as well as $M t_1'\And t_2\equiv \hat{t_1'}\And t_2$, which follows from $M t_1' \equiv \hat{t_1'}$.
%
%   }
%   \case{\userule{SH-SecondAnd}}
%   {Provided that $M\phi$, we need to demonstrate that \userule{H-SecondAnd} with $\hat{\sigma}=M\sigma$,
%   $M t_1\And t_2'\equiv t_1\And \hat{t_2}$ and $M\sigma'\equiv \hat{\sigma'}$.
%
%   By the induction hypothesis we obtain the following.\\
%   $\forall M_1 . M_1 \phi_1 \implies t_2,M_1\sigma \xrightarrow[]{M_1 i} \hat{t_2'},\hat{\sigma'}\land M_1 t_2'\equiv\hat{t_2'}\land M_1\sigma' \equiv \hat{\sigma'}$
%
%   Since $M$ satisfies $\phi$, we have $t_2,M\sigma\xrightarrow[]{M i} \hat{t_2'},\hat{\sigma'}$ and $M\sigma'\equiv\hat{\sigma'}$,
%   which we needed to show, as well as $M t_1\And t_2'\equiv t_1\And \hat{t_2'}$, which follows from $M t_2' \equiv \hat{t_2'}$.}
% }
%
% \case{$t=e_1\Or e_2$}
% {One rule applies, namely \userule{SH-Or}\\
%
% In the case that $M\phi_1$, we need to demonstrate that \userule{H-FirstOr} with $\hat{\sigma}=M\sigma$ and $M\First i =\First j$,
% $M t_1'\Or t_2\equiv \hat{t_1'}\And t_2$ and $M\sigma'\equiv \hat{\sigma'}$.
%
% By the induction hypothesis we obtain the following.\\
% $\forall M_1 . M_1 \phi_1 \implies t_1,M_1\sigma \xrightarrow[]{M_1 i} \hat{t_1'},\hat{\sigma'}\land M_1 t_1'\equiv\hat{t_1'}\land M_1\sigma' \equiv \hat{\sigma'}$.
%
% Since $M$ satisfies $\phi$, we have $t_1,M\sigma\xrightarrow[]{M i} \hat{t_1'},\hat{\sigma'}$ and $M\sigma'\equiv\hat{\sigma'}$,
% which we needed to show, as well as $M t_1'\Or t_2\equiv \hat{t_1'}\And t_2$, which follows from $M t_1' \equiv \hat{t_1'}$.
%
% In the case that $M\phi_2$, we need to demonstrate that \userule{H-SecondOr} with $\hat{\sigma}=M\sigma$ and $M\Second i = \Second j$,
% $M t_1\Or t_2'\equiv t_1\And \hat{t_2}$ and $M\sigma'\equiv \hat{\sigma'}$.
%
% By the induction hypothesis we obtain the following.\\
% $\forall M_1 . M_1 \phi_1 \implies t_2,M_1\sigma \xrightarrow[]{M_1 i} \hat{t_2'},\hat{\sigma'}\land M_1 t_2'\equiv\hat{t_2'}\land M_1\sigma' \equiv \hat{\sigma'}$
%
% Since $M$ satisfies $\phi$, we have $t_2,M\sigma\xrightarrow[]{M i} \hat{t_2'},\hat{\sigma'}$ and $M\sigma'\equiv\hat{\sigma'}$,
% which we needed to show, as well as $M t_1\Or t_2'\equiv t_1\And \hat{t_2'}$, which follows from $M t_2' \equiv \hat{t_2'}$.}

\end{proof}
