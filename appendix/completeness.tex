% !TEX root=../main.tex

\section{Completeness proofs}
\label{sec:completenessproofs}

\begin{lemma}[Completeness of handling]
  \label{lem:completeHandle}
  For all concrete tasks $t$, concrete states $\sigma$, symbolic tasks $\tilde{t}$, symbolic states $\tilde{\sigma}$ path conditions $\Phi$ and mappings $M$,
  we have that $t,\sigma\Consistent_{M}\tilde{t},\tilde{\sigma},\Phi$ implies
  that for all inputs $i$ such that $t,\sigma\handle{i}t',\sigma'$,
  there exists a symbolic input $\tilde{\imath}$, $\tilde{\imath}\sim i$ such that
  $\tilde{t},\tilde{\sigma}\handle{}\overline{\tilde{t}',\tilde{\sigma}',\tilde{\imath},\phi}$,
  and for all pairs $(\tilde{t}',\tilde{\sigma}',\tilde{\imath},\phi)$ we have that $\Sat(\Phi\land\phi)$ implies $t',\sigma'\Consistent_{M.[s\mapsto c]}\tilde{t}',\tilde{\sigma}',\Phi\land\phi$ where where $s\in\tilde{\imath}$ and $c\in i$.
\end{lemma}

\begin{lemma}[Completeness of normalisation]
  \label{lem:completeNormalise}
  For all concrete expressions $e$, concrete states $\sigma$, symbolic expressions $\tilde{e}$, symbolic states $\tilde{\sigma}$ path conditions $\Phi$ and mappings $M$,
  we have that $e,\sigma\Consistent_{M}\tilde{e},\tilde{\sigma},\Phi$
  and $e,\sigma\normalise t,\sigma'$,
  then $\tilde{e},\tilde{\sigma}\normalise\overline{\tilde{t},\tilde{\sigma}',\phi}$,
  and for all pairs $(\tilde{t},\tilde{\sigma}',\phi)$ we have that $\Sat(\Phi\land\phi)$ implies $t,\sigma'\Consistent_{M}\tilde{t},\tilde{\sigma}',\Phi\land\phi$.
\end{lemma}

\begin{lemma}[Completeness of striding]
  \label{lem:completeStride}
  For all concrete tasks $t$, concrete states $\sigma$, symbolic tasks $\tilde{t}$, symbolic states $\tilde{\sigma}$ path conditions $\Phi$ and mappings $M$,
  we have that $t,\sigma\Consistent_{M}\tilde{t},\tilde{\sigma},\Phi$
  and $t,\sigma\stride t',\sigma'$,
  then $\tilde{t},\tilde{\sigma}\stride\overline{\tilde{t'},\tilde{\sigma}',\phi}$,
  and for all pairs $(\tilde{t'},\tilde{\sigma}',\phi)$ we have that $\Sat(\Phi\land\phi)$ implies $t',\sigma'\Consistent_{M}\tilde{t'},\tilde{\sigma}',\Phi\land\phi$.
\end{lemma}

\begin{lemma}[Completeness of evaluate]
  \label{lem:completeEval}
  For all concrete expressions $e$, concrete states $\sigma$, symbolic expressions $\tilde{e}$, symbolic states $\tilde{\sigma}$ path conditions $\Phi$ and mappings $M$,
  we have that $e,\sigma\Consistent_{M}\tilde{e},\tilde{\sigma},\Phi$
  and $e,\sigma\eval v,\sigma'$,
  then $\tilde{e},\tilde{\sigma}\eval\overline{\tilde{v},\tilde{\sigma}',\phi}$,
  and for all pairs $(\tilde{v},\tilde{\sigma}',\phi)$ we have that $\Sat(\Phi\land\phi)$ implies $v,\sigma'\Consistent_{M}\tilde{v},\tilde{\sigma}',\Phi\land\phi$.
\end{lemma}

\begin{proof}[Completeness of handle]
  We prove Lemma~\ref{lem:completeHandle} by induction over $t$.\\

    \case{$t=\Edit v$}
    {One rule applies in this case, namely \userule{H-Change}\\
    % Take $i=s$ and assume $\sigma''=\sigma$. $s\sim v'$ holds by definition.
    % Then by the SH-Change rule, we know that a symbolic execution exists.

    }

    \case{$t=\Enter \tau$}
    {One rule applies in this case, namely \userule{H-Fill}\\
    % Take $i=s$ and assume $\sigma''=\sigma$. $s\sim v$ holds by definition.
    % Then by the SH-Fill rule,
    % we know that a symbolic execution exists.
    }

    \case{$t=\Update l$}
    {One rule applies in this case, namely \userule{H-Update}\\
    % Take $i=s$ and assume $\sigma''=\sigma$. $s\sim v'$ holds by definition.
    % Then by the SH-Update rule,
    % we know that a symbolic execution exists.
     }

    \case{$t=t_1\Next e_2$}
    {Two rules apply in this case\\
      \case{\userule{H-Next}}
      {
      % Take $i=s$ and assume $\sigma''=\sigma$. $s\sim \Continue$ holds by definition.
      % Then by the SH-Next rule, we know that a symbolic execution exists.
      }
      \case{\userule{H-PassNext}}
      {
      % By application of the induction hypothesis, we obtain the following.\\
      % For all $t_1,\sigma,j$ such that $t_1,\sigma\xrightarrow[j]{}t_1',\sigma'$ there exists an $i\sim j$ such that $t_1'',\sigma''\handle{}t_1''',\sigma''',i,\phi$.\\
      % From this we can conclude that there exists a symbolic execution $t_1\Next e_2,\sigma\handle{} t_1'''\Next e_2,\sigma''',i,\phi$, and that $i\sim j$.
      }
    }


    \case{$t=t_1\Then e_2$}
    {
    One rule applies in this case, namely \userule{H-PassThen}\\
    %
    % By application of the induction hypothesis, we obtain the following.\\
    % For all $t_1,\sigma,j$ such that $t_1,\sigma\xrightarrow[j]{}t_1',\sigma'$ there exists an $i\sim j$ such that $t_1'',\sigma''\handle{}t_1''',\sigma''',i,\phi$.\\
    % From this we can conclude that there exists a symbolic execution $t_1\Then e_2,\sigma\handle{} t_1'''\Then e_2,\sigma''',i,\phi$, and $i\sim j$.
    }
    \case{$t=e_1\Xor e_2$}
    {
    Two rules apply in this case.\\
    \case{\userule{H-PickLeft}}
      {
        % Take $i=s$. $s\sim \Left$ holds by definition.\\
        % Lemma~\ref{lem:completeNormalise} gives us the following.\\
        % There exists a symbolic execution $e_1,\sigma\normalise t_1,\sigma_1,\phi$.\\
        % There exists a symbolic execution $e_2,\sigma_1\normalise t_2,\sigma_2,\phi$.\\
        %
        % We can now conclude that a symbolic execution exists.
        % Either by the \refrule{SH-PickLeft} rule, in case $\Failing(t_2,\sigma_2)$, or by the \refrule{SH-Pick} rule in case $\neg\Failing(t_2,\sigma_2)$.
      }
    \case{\userule{H-PickRight}}
      {
      % Take $i=s$. $s\sim \Left$ holds by definition.\\
      % Lemma~\ref{lem:completeNormalise} gives us the following.\\
      % There exists a symbolic execution $e_1,\sigma\normalise t_1,\sigma_1,\phi$.\\
      % There exists a symbolic execution $e_2,\sigma_1\normalise t_2,\sigma_2,\phi$.\\
      %
      % We can now conclude that a symbolic execution exists.
      % Either by the \refrule{SH-PickRight} rule, in case $\Failing(t_1,\sigma_1)$, or by the \refrule{SH-Pick} rule in case $\neg\Failing(t_1,\sigma_1)$.
      }
    }
    \case{$t=t_1\Or t_2$}
      {
      Two rules applies in this case.\\
      \case{\userule{H-FirstOr}}
      {
      % Take $i=\First i$.
      %
      % By application of the induction hypothesis, we obtain the following.\\
      % For all $t_1,\sigma,j$ such that $t_1,\sigma\xrightarrow[]{j}t_1',\sigma'$ there exists an $i\sim j$ such that $t_1'',\sigma''\handle{}t_1''',\sigma''',i,\phi$.\\
      %
      % From this, we can conclude that $\First i\sim \First j$.
      % From \refrule{SH-Or}, and the conclusion of the induction hypothesis,
      % we can conclude that there exists an $i$ such that $t_1\Or t_2,\sigma\handle{}t_1'\Or t_2,\sigma',i,\phi$.
      }
      \case{\userule{H-SecondOr}}
      {
      % Take $i=\Second i$.
      %
      % By application of the induction hypothesis, we obtain the following.\\
      % For all $t_2,\sigma,j$ such that $t_2,\sigma\xrightarrow[]{j}t_2',\sigma'$ there exists an $i\sim j$ such that $t_2'',\sigma''\handle{}t_2''',\sigma''',i,\phi$.\\
      %
      % From this, we can conclude that $\Second i\sim \Second j$.
      % From \refrule{SH-Or}, and the conclusion of the induction hypothesis,
      % we can conclude that there exists an $i$ such that $t_1\Or t_2,\sigma\handle{}t_1\Or t_2',\sigma',i,\phi$.
      }
      }
    \case{$t=t_1\And t_2$}
      {
      Two rules applies in this case.\\
      \case{\userule{H-FirstAnd}}
      {
      % Take $i=\First i$.
      %
      % By application of the induction hypothesis, we obtain the following.\\
      % For all $t_1,\sigma,j$ such that $t_1,\sigma\xrightarrow[]{j}t_1',\sigma'$ there exists an $i\sim j$ such that $t_1'',\sigma''\handle{}t_1''',\sigma''',i,\phi$.\\
      %
      % From this, we can conclude that $\First i\sim \First j$.
      % From \refrule{SH-And}, and the conclusion of the induction hypothesis,
      % we can conclude that there exists an $i$ such that $t_1\And t_2,\sigma\handle{}t_1'\Or t_2,\sigma',i,\phi$.
      }
      \case{\userule{H-SecondAnd}}
      {
      % Take $i=\Second i$.
      %
      % By application of the induction hypothesis, we obtain the following.\\
      % For all $t_2,\sigma,j$ such that $t_2,\sigma\xrightarrow[]{j}t_2',\sigma'$ there exists an $i\sim j$ such that $t_2'',\sigma''\handle{}t_2''',\sigma''',i,\phi$.\\
      %
      % From this, we can conclude that $\First i\sim \Second j$.
      % From \refrule{SH-And}, and the conclusion of the induction hypothesis,
      % we can conclude that there exists an $i$ such that $t_1\And t_2,\sigma\handle{}t_1\And t_2',\sigma',i,\phi$.
      }
      }
\end{proof}

\begin{proof}[Completeness of normalise]
  We prove Lemma~\ref{lem:completeNormalise} by induction over $e$.

  % From the premise, we can assume that $e,\sigma\Consistent_M\tilde{e},\tilde{\sigma},\Phi$.
  % Now, given that $\tilde{e},\sigma{e}\tilde{\normalise} \overline{\tilde{t},\tilde{\sigma}',\phi}$,
  % we need to demonstrate that for all pairs $(\tilde{t},\tilde{\sigma}',\phi)$,
  % $\Sat(\Phi\land\phi)$ implies that $e,\sigma\normalise t,\sigma'$ with $t,\sigma'\Consistent_M\tilde{t},\tilde{\sigma}',\Phi\land\phi$.

  The base case is when the N-Done rule applies.\\
  \userule{N-Done}\\

  % In this case, we obtain from \cref{lem:soundeval} that
  % $e,\sigma\normalise t,\sigma'$ with $t,\sigma'\Consistent_M\tilde{t},\tilde{\sigma}',\Phi\land\phi$,
  % which is exactly what we needed to show.
  %
  % The only induction step is when\\
  \userule{N-Repeat} applies.

  % In this case, we obtain from \cref{lem:soundeval} that
  % $e,\sigma\normalise t,\sigma'$ with $t,\sigma'\Consistent_M\tilde{t},\tilde{\sigma}',\Phi\land\phi_1$,
  % which is exactly what we needed to show.
  % Furthermore, by \cref{lem:soundstride} we obtain that
  % $t,\sigma'\stride t',\sigma''$ with $t',\sigma''\Consistent_M\tilde{t}',\tilde{\sigma}'',\Phi\land\phi_1\land\phi_2$.
  % Then finally, by application of the induction hypothesis, we obtain what we needed to prove.
  % $t',\sigma''\normalise t'',\sigma'''$ with $t'',\sigma'''\Consistent_M\tilde{t}'',\tilde{\sigma}''',\Phi\land\phi_1\land\phi_2\land\phi_3$.
\end{proof}

\begin{proof}[Completeness of stride]



  % Provided that $t,\sigma\Consistent_M\tilde{t},\tilde{\sigma},\Phi$ and $\tilde{t},\tilde{\sigma}\tilde{\stride} \overline{\tilde{t}',\tilde{\sigma}',\phi}$,
  % we want to show that for all pairs $(\tilde{t}',\tilde{\sigma}',\phi)$,
  % we have $\Sat(\Phi\land\phi)$ implies that $t,\sigma\stride t',\sigma'$
  % We prove Lemma~\ref{lem:soundstride} by induction over $t$.



  \case{$t=\Edit v$}
  { One rule applies, namely \userule{S-Edit}\\
    % Given that $t,\sigma\Consistent_M\Edit\tilde{v},\tilde{\sigma},\Phi$ and $\Edit\tilde{v},\tilde{\sigma}\tilde{\stride}\Edit\tilde{v},\tilde{\sigma},\True$,
    % we know that $t=\Edit M \tilde{v}$, and we have $\Edit M \tilde{v},\sigma\stride\Edit M\tilde{v},\sigma$ by \refrule{S-Edit} and
    % $\Edit M \tilde{v},\sigma\Consistent_M\Edit\tilde{v},\tilde{\sigma},\Phi$, since none of the tasks and states were altered.
   }
  %
   \case{$t=\Enter \tau$}
  {One rule applies, namely \userule{S-Fill}\\
  % Given that $t,\sigma\Consistent_M\Enter \tau,\tilde{\sigma},\Phi$ and $\Enter \tau,\tilde{\sigma}\tilde{\stride}\Enter \tau,\tilde{\sigma},\True$,
  % we know that $t=\Enter \tau$, and we have $\Enter \tau,\sigma\stride\Enter \tau,\sigma$ by \refrule{S-Fill} and
  % $\Enter \tau,\sigma\Consistent_M\Edit\tilde{v},\tilde{\sigma},\Phi$, since none of the tasks and states were altered.
  }

  \case{$t=\Update l$}
   {One rule applies, namely \userule{S-Update}\\
   % Given that $t,\sigma\Consistent_M\Update l,\tilde{\sigma},\Phi$ and $\Update l,\tilde{\sigma}\tilde{\stride}\Update l,\tilde{\sigma},\True$,
   % we know that $t=\Update l$, and we have $\Update l,\sigma\stride\Update l,\sigma$ by \refrule{S-Update} and
   % $\Update l,\sigma\Consistent_M\Update l,\tilde{\sigma},\Phi$, since none of the tasks and states were altered.
  }

  \case{$t=\Fail$}
   {One rule applies, namely \userule{S-Fail}\\
   % Given that $t,\sigma\Consistent_M\Fail,\tilde{\sigma},\Phi$ and $\Fail,\tilde{\sigma}\tilde{\stride}\Fail,\tilde{\sigma},\True$,
   % we know that $t=\Fail$, and we have $\Fail,\sigma\stride\Fail,\sigma$ by \refrule{S-Fail} and
   % $\Fail,\sigma\Consistent_M\Fail,\tilde{\sigma},\Phi$, since none of the tasks and states were altered.
   }
  %
  \case{$t=e_1\Xor e_2$}
   {One rule applies, namely \userule{S-Xor}\\
   % Given that $t,\sigma\Consistent_M\tilde{e}_1\Xor \tilde{e}_2,\tilde{\sigma},\Phi$ and $\tilde{e}_1\Xor \tilde{e}_2,\tilde{\sigma}\tilde{\stride}\tilde{e}_1\Xor \tilde{e}_2,\tilde{\sigma},\True$,
   % we know that $t=M\tilde{e}_1\Xor M\tilde{e}_2$, and we have $M\tilde{e}_1\Xor M\tilde{e}_2,\sigma\stride M\tilde{e}_1\Xor M\tilde{e}_2,\sigma$ by \refrule{S-Xor} and
   % $M\tilde{e}_1\Xor M\tilde{e}_2,\sigma\Consistent_M\tilde{e}_1\Xor \tilde{e}_2,\tilde{\sigma},\Phi$, since none of the tasks and states were altered.
   }




  \case{$t=t_1\Then e_2$}
  {
  Three rules apply.\\
  \case{\userule{S-ThenStay}}
   {
     % Provided that $t,\sigma\Consistent_M\tilde{t}_1\Then \tilde{e}_2,\tilde{\sigma},\Phi$
     % and $\tilde{t}_1\Then \tilde{e}_2,\tilde{\sigma}\tilde{\stride}\tilde{t}'_1\Then \tilde{e}_2,\tilde{\sigma}',\phi_1$,
     % we obtain from the induction hypothesis that $t_1,\sigma\stride t_1',\sigma'$ and $t_1',\sigma'\Consistent_M\tilde{t}_1',\tilde{\sigma}',\Phi$.
     % From this, we can directly conclude that $t_1\Then e_2,\sigma\stride t_1'\Then e_2,\sigma'$ and $t_1'\Then e_2,\sigma'\Consistent_M\tilde{t}_1'\Then\tilde{e}_2,\tilde{\sigma}',\Phi$.
  }
  \case{\userule{S-ThenFail}}
   {
   % Provided that $t,\sigma\Consistent_M\tilde{t}_1\Then \tilde{e}_2,\tilde{\sigma},\Phi$
   % and $\tilde{t}_1\Then \tilde{e}_2,\tilde{\sigma}\tilde{\stride}\tilde{t}'_1\Then \tilde{e}_2,\tilde{\sigma}',\phi_1$,
   % we obtain from the induction hypothesis that $t_1,\sigma\stride t_1',\sigma'$ and $t_1',\sigma'\Consistent_M\tilde{t}_1',\tilde{\sigma}',\Phi$.
   % From this, we can directly conclude that $t_1\Then e_2,\sigma\stride t_1'\Then e_2,\sigma'$ and $t_1'\Then e_2,\sigma'\Consistent_M\tilde{t}_1'\Then\tilde{e}_2,\tilde{\sigma}',\Phi$.
   }
  \case{\userule{S-ThenCont}}
   {
   % Provided that $t,\sigma\Consistent_M\tilde{t}_1\Then \tilde{e}_2,\tilde{\sigma},\Phi$
   % and $\tilde{t}_1\Then \tilde{e}_2,\tilde{\sigma}\tilde{\stride}\tilde{t}_2,\tilde{\sigma}',\phi_1\land\phi_2$
   % with $\tilde{t}_1,\tilde{\sigma}\tilde{\stride}\tilde{t}_1',\tilde{\sigma}',\phi_1$ and $\Value(\tilde{t}_1',\tilde{\sigma}')=\tilde{v}_1$,
   % we obtain from the induction hypothesis that $t_1,\sigma\stride t_1',\sigma'$ and $t_1',\sigma'\Consistent_M\tilde{t}_1',\tilde{\sigma}',\Phi$.
   % Then from the consistence relation, we can conclude that $\Value(t_1',\sigma')=\Value(M t_1',M \sigma')=M\tilde{v}_1$.
   %
   % At this point, we have $e_2 M\tilde{v}_1,\sigma'\Consistent_M\tilde{e}_2 \tilde{v}_1,\tilde{\sigma}',\Phi\land\phi_1$ and $\tilde{e}_2 \tilde{v}_1,\tilde{\sigma}'\tilde{\eval}\tilde{t}_2,\tilde{sigma}'',\phi_2$.
   % This allows us to apply \cref{lem:soundeval} to obtain $e_2 (M\tilde{v}_1),\sigma'\eval t_2,\sigma''$ and $t_2,\sigma''\Consistent_M\tilde{t}_2,\tilde{\sigma}'',\Phi\land\phi_1\land\phi_2$.
   %
   % From this, we can directly conclude that $t_1\Then e_2,\sigma\stride t_2,\sigma''$ and $t_2,\sigma''\Consistent_M\tilde{t}_2,\tilde{\sigma}'',\Phi\land\phi_1\land\phi_2$.
   }
  }

  \case{$t=t_1\Or t_2$}
  {
  One of three rules applies.\\
  \case{\userule{S-OrLeft}}
   {
   % Provided that $t,\sigma\Consistent_M\tilde{t}_1\Or \tilde{t}_2,\tilde{\sigma},\Phi$
   % and $\tilde{t}_1\Or \tilde{t}_2,\tilde{\sigma}\tilde{\stride}\tilde{t}'_1,\tilde{\sigma}',\phi$,
   % we obtain from the induction hypothesis that $t_1,\sigma\stride t_1',\sigma'$ and $t_1',\sigma'\Consistent_M\tilde{t}_1',\tilde{\sigma}',\Phi\land\phi$.
   % From this, we can directly conclude that $t_1\Or t_2,\sigma\stride t_1',\sigma'$.
  }
  \case{\userule{S-OrRight}}
   {
   % Provided that $t,\sigma\Consistent_M\tilde{t}_1\Or \tilde{t}_2,\tilde{\sigma},\Phi$
   % and $\tilde{t}_1\Or \tilde{t}_2,\tilde{\sigma}\tilde{\stride}\tilde{t}'_2,\tilde{\sigma}'',\phi_1\land\phi_2$,
   % we obtain from the induction hypothesis that $t_1,\sigma\stride t_1',\sigma'$ and $t_1',\sigma'\Consistent_M\tilde{t}_1',\tilde{\sigma}',\Phi\land\phi_1$.
   % Then by a second application of the induction hypothesis, we obtain that $t_2,\sigma'\stride t_2',\sigma''$ and $t_2',\sigma''\Consistent_M\tilde{t}_2',\tilde{\sigma}'',\Phi\land\phi_1\land\phi_2$.
   % This leads us to conclude $t_1\Or t_2,\sigma\stride t_2',\sigma''$.
   }
  \case{\userule{S-OrNone}}
  {
  % Provided that $t,\sigma\Consistent_M\tilde{t}_1\Or \tilde{t}_2,\tilde{\sigma},\Phi$
  % and $\tilde{t}_1\Or \tilde{t}_2,\tilde{\sigma}\tilde{\stride}\tilde{t}'_2,\tilde{\sigma}'',\phi_1\land\phi_2$,
  % we obtain from the induction hypothesis that $t_1,\sigma\stride t_1',\sigma'$ and $t_1',\sigma'\Consistent_M\tilde{t}_1',\tilde{\sigma}',\Phi\land\phi_1$.
  % Then by a second application of the induction hypothesis, we obtain that $t_2,\sigma'\stride t_2',\sigma''$ and $t_2',\sigma''\Consistent_M\tilde{t}_2',\tilde{\sigma}'',\Phi\land\phi_1\land\phi_2$.
  % This leads us to conclude $t_1\Or t_2,\sigma\stride t_1'\Or t_2',\sigma''$ and $t_1'\Or t_2',\sigma''\Consistent_M\tilde{t}_1'\Or\tilde{t}_2',\tilde{\sigma}'',\Phi\land\phi_1\land\phi_2$.
   }
  }

  \case{$t=t_1\Next e_2$}
  {
  One rule applies, namely \userule{S-Next}\\
  % Provided that $t,\sigma\Consistent_M\tilde{t}_1\Next \tilde{e}_2,\tilde{\sigma},\Phi$
  % and $\tilde{t}_1\Next \tilde{e}_2,\tilde{\sigma}\tilde{\stride}\tilde{t}'_1\Then \tilde{e}_2,\tilde{\sigma}',\phi$,
  % we obtain from the induction hypothesis that $t_1,\sigma\stride t_1',\sigma'$ and $t_1',\sigma'\Consistent_M\tilde{t}_1',\tilde{\sigma}',\Phi\land\phi$.
  % From this, we can directly conclude that $t_1\Next e_2,\sigma\stride t_1'\Next e_2,\sigma'$ and $t_1'\Next e_2,\sigma'\Consistent_M\tilde{t}_1'\Next\tilde{e}_2,\tilde{\sigma}',\Phi\land\phi$.
  }

  \case{$t=t_1\And t_2$}
  {
  One rule applies, namely \userule{S-And}\\
  % Provided that $t,\sigma\Consistent_M\tilde{t}_1\And \tilde{t}_2,\tilde{\sigma},\Phi$
  % and $\tilde{t}_1\And \tilde{t}_2,\tilde{\sigma}\tilde{\stride}\tilde{t}'_1\And\tilde{t}'_2,\tilde{\sigma}'',\phi_1\land\phi_2$,
  % we obtain from the induction hypothesis that $t_1,\sigma\stride t_1',\sigma'$ and $t_1',\sigma'\Consistent_M\tilde{t}_1',\tilde{\sigma}',\Phi\land\phi_1$.
  % Then by a second application of the induction hypothesis, we obtain that $t_2,\sigma'\stride t_2',\sigma''$ and $t_2',\sigma''\Consistent_M\tilde{t}_2',\tilde{\sigma}'',\Phi\land\phi_1\land\phi_2$.
  % This leads us to conclude $t_1\And t_2,\sigma\stride t_1'\And t_2',\sigma''$ and $t_1'\And t_2',\sigma''\Consistent_M\tilde{t}_1'\And\tilde{t}_2',\tilde{\sigma}'',\Phi\land\phi_1\land\phi_2$.

  }
\end{proof}

\begin{proof}[Completeness of evaluate]
  We prove Lemma~\ref{lem:completeEval} by induction over $e$.

  \case{$e=v$}
    {One rule applies, namely \userule{E-Value}\\
    % We assume $e,\sigma\Consistent_M\tilde{v},\tilde{\sigma},\Phi$ and $\tilde{v},\tilde{\sigma}\tilde{\eval}\tilde{v},\tilde{\sigma},\True$.
    % By \refrule{E-Value} we have $v,\sigma\eval v,\sigma$, so this case holds trivially.
    }

  \case{$e=\tuple{e_1,e_2}$}
    {One rule applies, namely \userule{E-Pair}\\
    % Provided that $e,\sigma\Consistent_m\tuple{\tilde{e}_1,\tilde{e}_2},\tilde{\sigma},\Phi$ and $\tuple{\tilde{e}_1,\tilde{e}_2},\tilde{\sigma}\tilde{\eval}\tuple{\tilde{v}_1,\tilde{v}_2},\tilde{\sigma}'',\phi_1\land\phi_2$,
    % we obtain from the induction hypothesis that $e_1,\sigma\eval v_1,\sigma'$ with $v_1,\sigma'\Consistent_m\tilde{v}_1,\tilde{\sigma}',\Phi\land\phi_1$.
    % Then by a second application of the induction hypothesis, we obtain that $e_2,\sigma'\eval v_2,\sigma''$ with $v_2,\sigma''\Consistent_m\tilde{v}_2,\tilde{\sigma}'',\Phi\land\phi_1\land\phi_2$.
    % From this, we can conclude that $\tuple{e_1,e_2},\sigma\eval\tuple{v_1,v_2},\sigma''$ with $\tuple{v_1,v_2},\sigma''\Consistent_m\tuple{\tilde{v}_1,\tilde{v}_2},\tilde{\sigma}'',\Phi\land\phi_1\land\phi_2$.
    }

  \case{$e=\Fst \tuple{e_1,e_2}$}
  {
    One rule applies, namely \userule{E-First}\\
    % Provided that $e,\sigma\Consistent_m\Fst\tuple{\tilde{e}_1,\tilde{e}_2},\tilde{\sigma},\Phi$ and $\Fst\tuple{\tilde{e}_1,\tilde{e}_2},\tilde{\sigma}\tilde{\eval}\tilde{v}_1,\tilde{\sigma}'',\phi$,
    % we obtain from the induction hypothesis that $e_1,\sigma\eval v_1,\sigma'$ with $v_1,\sigma'\Consistent_m\tilde{v}_1,\tilde{\sigma}',\Phi\land\phi$.
    % From this, we can conclude that $\Fst\tuple{e_1,e_2},\sigma\eval v_1,\sigma'$.
    }

  \case{$e=\Snd \tuple{e_1,e_2}$}
  { One rule applies, namely \userule{E-Second}\\
  % Provided that $e,\sigma\Consistent_m\Snd\tuple{\tilde{e}_1,\tilde{e}_2},\tilde{\sigma},\Phi$ and $\Snd\tuple{\tilde{e}_1,\tilde{e}_2},\tilde{\sigma}\tilde{\eval}\tilde{v}_2,\tilde{\sigma}'',\phi$,
  % we obtain from the induction hypothesis that $e_1,\sigma\eval v_2,\sigma'$ with $v_2,\sigma'\Consistent_m\tilde{v}_1,\tilde{\sigma}',\Phi\land\phi$.
  % From this, we can conclude that $\Snd\tuple{e_1,e_2},\sigma\eval v_2,\sigma'$.
    }

  \case{$e=e_1::e_2$}
    {One rule applies, namely \userule{E-Cons}\\
    % Provided that $e,\sigma\Consistent_m\tilde{e}_1::\tilde{e}_2,\tilde{\sigma},\Phi$ and $\tilde{e}_1::\tilde{e}_2,\tilde{\sigma}\tilde{\eval}\tilde{v}_1::\tilde{v}_2,\tilde{\sigma}'',\phi_1\land\phi_2$,
    % we obtain from the induction hypothesis that $e_1,\sigma\eval v_1,\sigma'$ with $v_1,\sigma'\Consistent_m\tilde{v}_1,\tilde{\sigma}',\Phi\land\phi_1$.
    % Then by a second application of the induction hypothesis, we obtain that $e_2,\sigma'\eval v_2,\sigma''$ with $v_2,\sigma''\Consistent_m\tilde{v}_2,\tilde{\sigma}'',\Phi\land\phi_1\land\phi_2$.
    % From this, we can conclude that $e_1 :: e_2,\sigma\eval v_1 :: v_2,\sigma''$ with $v_1 :: v_2,\sigma''\Consistent_m\tilde{v}_1 :: \tilde{v}_2,\tilde{\sigma}'',\Phi\land\phi_1\land\phi_2$.
   }

  \case{$e=\Head e$}
    {One rule applies, namely \userule{E-Head}\\
    % Provided that $e,\sigma\Consistent_m\Head \tilde{e},\tilde{\sigma},\Phi$ and $\Head \tilde{e},\tilde{\sigma}\tilde{\eval}\tilde{v}_1,\tilde{\sigma}',\phi$,
    % we obtain from the induction hypothesis that $e,\sigma\eval v_1::v_2,\sigma'$ with $v_1::v_2,\sigma'\Consistent_m\tilde{v}_1::\tilde{v}_2,\tilde{\sigma}',\Phi\land\phi$.
    % From this, we can conclude that $\Head e,\sigma\eval v_1,\sigma'$.
    }

  \case{$e=\Tail e$}
    {One rule applies, namely \userule{E-Tail}\\
    % Provided that $e,\sigma\Consistent_m\Tail \tilde{e},\tilde{\sigma},\Phi$ and $\Tail \tilde{e},\tilde{\sigma}\tilde{\eval}\tilde{v}_2,\tilde{\sigma}',\phi$,
    % we obtain from the induction hypothesis that $e,\sigma\eval v_1::v_2,\sigma'$ with $v_1::v_2,\sigma'\Consistent_m\tilde{v}_1::\tilde{v}_2,\tilde{\sigma}',\Phi\land\phi$.
    % From this, we can conclude that $\Tail e,\sigma\eval v_2,\sigma'$.
      }

  \case{$e=e_1 e_2$}
    {One rule applies, namely\\ \userule{E-App}\\

    % Provided that $e,\sigma\Consistent_m\tilde{e}_1 \tilde{e}_2,\tilde{\sigma},\Phi$ and $\tilde{e}_1 \tilde{e}_2,\tilde{\sigma}\tilde{\eval}\tilde{v}_1,\tilde{\sigma}''',\phi_1\land\phi_2\land\phi_3$,
    % we obtain from the induction hypothesis that $e_1,\sigma\eval \lambda x:\tau.{e_1}',\sigma'$ with $\lambda x:\tau.{e_1}',\sigma'\Consistent_m\lambda x:\tau.\tilde{e}_1',\tilde{\sigma}',\Phi\land\phi_1$.
    % Then by a second application of the induction hypothesis, we obtain that $e_2,\sigma'\eval v_2,\sigma''$ with $v_2,\sigma''\Consistent_m\tilde{v}_2,\tilde{\sigma}'',\Phi\land\phi_1\land\phi_2$.
    % A third and final application of the induction hypothesis gives us that $e_1'[x\mapsto v_2],\sigma'' \eval v_1,\sigma'''$ with
    % $v_1,\sigma'''\Consistent_m\tilde{v}_1,\tilde{\sigma}''',\Phi\land\phi_1\land\phi_2\land\phi_3$.
    % From this, we can conclude that $e_1 e_2,\sigma\eval v_1,\sigma'''$.
    }
    \case{$e=\If{e_1}{e_2}{e_3}$}
       {One rule applies, namely\\ \userule{E-IfTrue}\\

      }
  \case{$e=\If{e_1}{e_2}{e_3}$}
     {One rule applies, namely\\ \userule{E-IfFalse}\\

    }

  \case{$e=\Ref e$}
    {One rule applies, namely \userule{E-Ref}\\
    % Provided that $e,\sigma\Consistent_m\Ref \tilde{e},\tilde{\sigma},\Phi$ and $\Ref \tilde{e},\tilde{\sigma}\tilde{\eval}l,\tilde{\sigma}'[l\mapsto\tilde{v}],\phi$,
    % we obtain from the induction hypothesis that $e,\sigma\eval v_1,\sigma'$ with $v_1,\sigma'\Consistent_m\tilde{v}_1,\tilde{\sigma}',\Phi\land\phi$.
    % From this, we can conclude that $\Ref e,\sigma\eval l,\sigma'[l\mapsto v]$ with $l,\sigma'[l\mapsto v]\Consistent_m l,\tilde{\sigma}'[l\mapsto \tilde{v}],\Phi\land\phi$.
    }

  \case{$e=!e$}
    {One rule applies, namely \userule{E-Deref}\\
    % Provided that $e,\sigma\Consistent_m !\tilde{e},\tilde{\sigma},\Phi$ and $!\tilde{e},\tilde{\sigma}\tilde{\eval}\tilde{\sigma}'(l),\tilde{\sigma}',\phi$,
    % we obtain from the induction hypothesis that $e,\sigma\eval l,\sigma'$ with $l,\sigma'\Consistent_m l,\tilde{\sigma}',\Phi\land\phi$.
    % From this, we can conclude that $!e,\sigma\eval \sigma'(l),\sigma'$ with $\sigma'(l),\sigma'\Consistent_m \tilde{\sigma}'(l),\tilde{\sigma}',\Phi\land\phi$.
  }

  \case{$e=e_1:=e_2$}
    {
    One rule applies, namely \userule{E-Assign}\\
    % Provided that $e,\sigma\Consistent_m\tilde{e}_1:=\tilde{e}_2,\tilde{\sigma},\Phi$ and $\tilde{e}_1:=\tilde{e}_2,\tilde{\sigma}\tilde{\eval}\unit,\tilde{\sigma}''[l\mapsto \tilde{v}_2],\phi_1\land\phi_2$,
    % we obtain from the induction hypothesis that $e_1,\sigma\eval l,\sigma'$ with $l,\sigma'\Consistent_m l,\tilde{\sigma}',\Phi\land\phi_1$.
    % Then by a second application of the induction hypothesis, we obtain that $e_2,\sigma'\eval v_2,\sigma''$ with $v_2,\sigma''\Consistent_m\tilde{v}_2,\tilde{\sigma}'',\Phi\land\phi_1\land\phi_2$.
    % From this, we can conclude that $e_1 := e_2,\sigma\eval \unit,\sigma''[l\mapsto v_2]$ with $\unit,\sigma''[l\mapsto v_2]\Consistent_m\unit,\tilde{\sigma}''[l\mapsto\tilde{v}_2],\Phi\land\phi_1\land\phi_2$.
    }

  \case{$e=\Edit e$}
    {One rule applies, namely \userule{E-Edit}\\
    % Provided that $e,\sigma\Consistent_m \Edit\tilde{e},\tilde{\sigma},\Phi$ and $\Edit\tilde{e},\tilde{\sigma}\tilde{\eval}\Edit\tilde{v},\tilde{\sigma}',\phi$,
    % we obtain from the induction hypothesis that $e,\sigma\eval v,\sigma'$ with $v,\sigma'\Consistent_m \tilde{v},\tilde{\sigma}',\Phi\land\phi$.
    % From this, we can conclude that $\Edit e,\sigma\eval \Edit v,\sigma'$ with $\Edit v,\sigma'\Consistent_m \Edit\tilde{v},\tilde{\sigma}',\Phi\land\phi$.

    }

  \case{$e=\Enter \tau$}
    {
    One rule applies, namely \userule{E-Enter}\\
    % We assume $e,\sigma\Consistent_M\Enter\tau,\tilde{\sigma},\Phi$ and $\Enter\tau,\tilde{\sigma}\tilde{\eval}\Enter\tau,\tilde{\sigma},\True$.
    % By \refrule{E-Enter} we have $\Enter\tau,\sigma\eval \Enter\tau,\sigma$, so this case holds trivially.
    }

  \case{$e=\Update e$}
    {One rule applies, namely \userule{E-Update}\\
    % Provided that $e,\sigma\Consistent_m \Update\tilde{e},\tilde{\sigma},\Phi$ and $\Update\tilde{e},\tilde{\sigma}\tilde{\eval}\Update l,\tilde{\sigma}',\phi$,
    % we obtain from the induction hypothesis that $e,\sigma\eval l,\sigma'$ with $l,\sigma'\Consistent_m l,\tilde{\sigma}',\Phi\land\phi$.
    % From this, we can conclude that $\Update e,\sigma\eval \Update l ,\sigma'$ with $\Update l,\sigma'\Consistent_m \Update l,\tilde{\sigma}',\Phi\land\phi$.

    }

  \case{$e=e_1\Then e_2$}
    {One rule applies, namely \userule{E-Then}\\
    % Provided that $e,\sigma\Consistent_m \tilde{e}_1\Then \tilde{e}_2,\tilde{\sigma},\Phi$ and $\tilde{e}_1\Then \tilde{e}_2,\tilde{\sigma}\tilde{\eval}\tilde{t}_1\Then \tilde{e}_2,\tilde{\sigma}',\phi$,
    % we obtain from the induction hypothesis that $e_1,\sigma\eval t_1,\sigma'$ with $t_1,\sigma'\Consistent_m \tilde{t}_1,\tilde{\sigma}',\Phi\land\phi$.
    % From this, we can conclude that $e_1\Then e_2,\sigma\eval t_1\Then e_2,\sigma'$ with $t_1\Then e_2,\sigma'\Consistent_m \tilde{t}_1\Then e_2,\tilde{\sigma}',\Phi\land\phi$.

    }

  \case{$e=e_1\Next e_2$}
    {One rule applies, namely \userule{E-Next}\\
    % Provided that $e,\sigma\Consistent_m \tilde{e}_1\Next \tilde{e}_2,\tilde{\sigma},\Phi$ and $\tilde{e}_1\Next \tilde{e}_2,\tilde{\sigma}\tilde{\eval}\tilde{t}_1\Next \tilde{e}_2,\tilde{\sigma}',\phi$,
    % we obtain from the induction hypothesis that $e_1,\sigma\eval t_1,\sigma'$ with $t_1,\sigma'\Consistent_m \tilde{t}_1,\tilde{\sigma}',\Phi\land\phi$.
    % From this, we can conclude that $e_1\Next e_2,\sigma\eval t_1\Next e_2,\sigma'$ with $t_1\Next e_2,\sigma'\Consistent_m \tilde{t}_1\Next e_2,\tilde{\sigma}',\Phi\land\phi$.

    }

  \case{$e=e_1\Or e_2$}
    {One rule applies, namely \userule{E-Or}\\
    % Provided that $e,\sigma\Consistent_m\tilde{e}_1 \Or \tilde{e}_2,\tilde{\sigma},\Phi$ and $\tilde{e}_1 \Or \tilde{e}_2,\tilde{\sigma}\tilde{\eval}\tilde{v}_1\Or \tilde{v}_2,\tilde{\sigma}'',\phi_1\land\phi_2$,
    % we obtain from the induction hypothesis that $e_1,\sigma\eval v_1,\sigma'$ with $v_1,\sigma'\Consistent_m\tilde{v}_1,\tilde{\sigma}',\Phi\land\phi_1$.
    % Then by a second application of the induction hypothesis, we obtain that $e_2,\sigma'\eval v_2,\sigma''$ with $v_2,\sigma''\Consistent_m\tilde{v}_2,\tilde{\sigma}'',\Phi\land\phi_1\land\phi_2$.
    % From this, we can conclude that $e_1 \Or e_2,\sigma\eval v_1 \Or v_2,\sigma''$ with $v_1 \Or v_2,\sigma''\Consistent_m\tilde{v}_1 \Or \tilde{v}_2,\tilde{\sigma}'',\Phi\land\phi_1\land\phi_2$.

    }

  \case{$e=e_1\Xor e_2$}
    {  One rule applies, namely \userule{E-Xor}\\
    % We assume $e,\sigma\Consistent_M\tilde{e}_1\Xor \tilde{e}_2,\tilde{\sigma},\Phi$ and $\tilde{e}_1\Xor \tilde{e}_2,\tilde{\sigma}\tilde{\eval}\tilde{e}_1\Xor \tilde{e}_2,\tilde{\sigma},\True$.
    % By \refrule{E-Xor} we have $e_1\Xor e_2,\sigma\eval e_1\Xor e_2,\sigma$, so this case holds trivially.

    }

  \case{$e=\Fail$}
    {  One rule applies, namely \userule{E-Fail}\\
    % We assume $e,\sigma\Consistent_M\Fail,\tilde{\sigma},\Phi$ and $\Fail,\tilde{\sigma}\tilde{\eval}\Fail,\tilde{\sigma},\True$.
    % By \refrule{E-Fail} we have $\Fail,\sigma\eval \Fail,\sigma$, so this case holds trivially.

    }
\end{proof}
