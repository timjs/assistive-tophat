% !TEX root=../main.tex

\section{Completeness proofs}
\label{sec:completenessproofs}

\begin{proof}[Completeness of simulate]
  The structure of this proof is outlined in \cref{fig:proofstructure}.

  We have $t$ and $\sigma$ such that $t,\sigma\interact{I}^*v$.
  By definition of $\interact{I}^*$, we have the following.

  $t,\sigma\interact{i_1}t_1,\sigma_1\interact{i_2}\cdots \interact{i_n}t_n,\sigma_n$ with $\Value(t_n,\sigma_n)$ and $I=[i_1,\cdots,i_n]$.

  We need to show that we have $(\tilde{v},\tilde{I},\Phi)\in t,\sigma\interact{}^*)$,
  which is defined as follows.

  \begin{align*}
      t,\sigma\interact{}&\tilde{t}_1,\tilde{\sigma}_1,\tilde{\imath}_1,\phi_1&\\
                      &\tilde{t}_1,\tilde{\sigma}_1\interact{}&\tilde{t}_2,\tilde{\sigma}_2,\tilde{\imath}_2,\phi_2\\
                      &                                    &\tilde{t}_2,\tilde{\sigma}_2\interact{}&\cdots&\\
                      &                                    &                                    &\cdots&
                      \interact{}\tilde{t}_n,\tilde{\sigma}_n,\tilde{\imath}_n,\phi_n
  \end{align*}

  with $\Value(\tilde{t}_n,\tilde{\sigma}_n)=\tilde{v}$ and $\Sat(\phi_1\land\cdots\land\phi_n)$.

  By \cref{lem:completedriving}, we know that $t,\sigma\interact{}\tilde{t}_1,\tilde{\sigma}_1,\tilde{\imath}_1,\phi_1$ exists,
  since $t,\sigma,t\Consistent_{\emptyset}\sigma,\True$.
  This also gives us that $\tilde{\imath}_1\sim i_1$ and $t_1,\sigma_1\Consistent_{[s_1\mapsto c_1]}\tilde{t}_1,\tilde{\sigma}_1,\phi_1$ with $s_1\in\tilde{\imath}_1$ and $c_1\in i_1$.

  By repeated application of \cref{lem:completedriving}, untill we arrive at $t_n,\sigma_n$,
  we can show that there exists a $\tilde{I}$ such that $t,\sigma\interact{}^*\tilde{t}_n,\tilde{\sigma}_n,\tilde{I},\Phi$,
  namely $[\tilde{\imath}_1,\cdots,\tilde{\imath}_n]$.

\end{proof}

\begin{lemma}[Completeness of handling]
  \label{lem:completeHandle}
  For all concrete tasks $t$, concrete states $\sigma$, concrete inputs $i$, symbolic tasks $\tilde{t}$, symbolic states $\tilde{\sigma}$ path conditions $\Phi$ and mappings $M$,
  we have that $t,\sigma\Consistent_{M}\tilde{t},\tilde{\sigma},\Phi$ and
  $t,\sigma\handle{i}t',\sigma'$ together with
  $\tilde{t},\tilde{\sigma}\handle{}\overline{\tilde{t}',\tilde{\sigma}',\tilde{\imath},\phi}$,
  and for all pairs $(\tilde{t}',\tilde{\sigma}',\tilde{\imath},\phi)$ we have that $\Sat(\Phi\land\phi)$ and $\imath\sim i$ implies $t',\sigma'\Consistent_{M.[s\mapsto c]}\tilde{t}',\tilde{\sigma}',\Phi\land\phi$ where where $s\in\tilde{\imath}$ and $c\in i$.
\end{lemma}

\begin{lemma}[Completeness of normalisation]
  \label{lem:completeNormalise}
  For all concrete expressions $e$, concrete states $\sigma$, symbolic expressions $\tilde{e}$, symbolic states $\tilde{\sigma}$ path conditions $\Phi$ and mappings $M$,
  we have that $e,\sigma\Consistent_{M}\tilde{e},\tilde{\sigma},\Phi$
  and $e,\sigma\normalise t,\sigma'$,
  then $\tilde{e},\tilde{\sigma}\normalise\overline{\tilde{t},\tilde{\sigma}',\phi}$,
  and for all pairs $(\tilde{t},\tilde{\sigma}',\phi)$ we have that $\Sat(\Phi\land\phi)$ implies $t,\sigma'\Consistent_{M}\tilde{t},\tilde{\sigma}',\Phi\land\phi$.
\end{lemma}

\begin{lemma}[Completeness of striding]
  \label{lem:completeStride}
  For all concrete tasks $t$, concrete states $\sigma$, symbolic tasks $\tilde{t}$, symbolic states $\tilde{\sigma}$ path conditions $\Phi$ and mappings $M$,
  we have that $t,\sigma\Consistent_{M}\tilde{t},\tilde{\sigma},\Phi$
  and $t,\sigma\stride t',\sigma'$,
  then $\tilde{t},\tilde{\sigma}\stride\overline{\tilde{t'},\tilde{\sigma}',\phi}$,
  and for all pairs $(\tilde{t'},\tilde{\sigma}',\phi)$ we have that $\Sat(\Phi\land\phi)$ implies $t',\sigma'\Consistent_{M}\tilde{t'},\tilde{\sigma}',\Phi\land\phi$.
\end{lemma}

\begin{lemma}[Completeness of evaluate]
  \label{lem:completeEval}
  For all concrete expressions $e$, concrete states $\sigma$, symbolic expressions $\tilde{e}$, symbolic states $\tilde{\sigma}$ path conditions $\Phi$ and mappings $M$,
  we have that $e,\sigma\Consistent_{M}\tilde{e},\tilde{\sigma},\Phi$
  and $e,\sigma\eval v,\sigma'$,
  then $\tilde{e},\tilde{\sigma}\eval\overline{\tilde{v},\tilde{\sigma}',\phi}$,
  and for all pairs $(\tilde{v},\tilde{\sigma}',\phi)$ we have that $\Sat(\Phi\land\phi)$ implies $v,\sigma'\Consistent_{M}\tilde{v},\tilde{\sigma}',\Phi\land\phi$.
\end{lemma}

\begin{proof}[Completeness of handle]
  We prove Lemma~\ref{lem:completeHandle} by induction over $t$.\\

    \Case{$t=\Edit v$}
    {
    Provided that $\Edit v,\sigma\Consistent_{M}\tilde{t},\tilde{\sigma},\Phi$ and \userule{H-Change},
    then $\Edit\tilde{v},\tilde{\sigma}\handle{}\Edit s,\tilde{\sigma},s,\True$.
    $\Sat(\Phi\land\True)=\Sat(\Phi)$, which follows from the premise.
    Furthermore we have $s\sim v'$ by definition.
    Then finally $\Edit v',\sigma\Consistent{M[s\mapsto v']}\Edit s,\tilde{\sigma},\Phi$ since $M[s\mapsto v'] s = v'$.


    }

    \Case{$t=\Enter \tau$}
    {
    Provided that $\Enter \tau,\sigma\Consistent_{M}\tilde{t},\tilde{\sigma},\Phi$ and \userule{H-Fill},
    then $\Enter\tau,\tilde{\sigma}\handle{}\Edit s,\tilde{\sigma},s,\True$.
    $\Sat(\Phi\land\True)=\Sat(\Phi)$, which follows from the premise.
    Furthermore we have $s\sim v$ by definition.
    Then finally $\Edit v,\sigma\Consistent{M[s\mapsto v]}\Edit s,\tilde{\sigma},\Phi$ since $M[s\mapsto v] s = v$.
    }

    \Case{$t=\Update l$}
    {
    Provided that $\Update l,\sigma\Consistent_{M}\tilde{t},\tilde{\sigma},\Phi$ and \userule{H-Update},
    then $\Update l,\tilde{\sigma}\handle{}\Update l,\tilde{\sigma}[l\mapsto s],s,\True$.
    $\Sat(\Phi\land\True)=\Sat(\Phi)$, which follows from the premise.
    Furthermore we have $s\sim v$ by definition.
    Then finally $\Update l,\sigma[l\mapsto v]\Consistent{M[s\mapsto v]}\Update l,\tilde{\sigma}[l\mapsto s],\Phi$ since $M[s\mapsto v] s = v$.
     }

    \Case{$t=t_1\Next e_2$}
    {Two rules apply in this case\\
      \Case{\userule{H-Next}}
      {
      Provided that $t_1\Next e_2,\sigma\Consistent_{M}\tilde{t},\tilde{\sigma},\Phi$ and \userule{H-Next},
      then \userule{SH-Next}.
      The simulation step results in two sets, from which only the second adheres to the requirement that the symbolic input should simulate the concrete input.
      For this set, $\overline{\tilde{t}_2,\tilde{\sigma}_2',\Continue,\phi_2}$, we have $\Sat(\Phi\land\phi_2)$ implies
      $t_2,\sigma_2'\Consistent_M\tilde{t}_2,\tilde{\sigma}_2',\Phi\land\phi_2$,
      Which follows directly from \cref{lem:completeNormalise}.
      }
      \Case{\userule{H-PassNext}}
      {
      Provided that $t_1\Next e_2,\sigma\Consistent_{M}\tilde{t},\tilde{\sigma},\Phi$ and \userule{H-PassNext}.
      There are three symbolic rules that apply, namely \userule{SH-PassNext}, \userule{SH-PassNextFail} and\\
      \userule{SH-Next}.
      We are only interested in the runs that produce a symbolic input that simulates the concrete input $i$.
      Whichever rule applies, we deal with the same premise because of this restriction.
      This allows us to apply the induction hypothesis and obtain that
      $\Sat(\Phi\land\phi_1)\implies t_1',\sigma'\Consistent_{M.[s\mapsto c]}\tilde{t}_1',\tilde{\sigma}',\Phi\land\phi_1$.
      From this, we can directly conclude that $t_1'\Next e_2,\sigma'\Consistent_{M.[s\mapsto c]}\tilde{t}_1'\Next \tilde{e}_2,\tilde{\sigma}',\Phi\land\phi_1$.
      }
    }


    \Case{$t=t_1\Then e_2$}
    {
    Provided that $t_1\Then e_2,\sigma\Consistent_{M}\tilde{t},\tilde{\sigma},\Phi$ and \userule{H-PassThen},
    then \userule{SH-PassThen}.
    By application of the induction hypothesis, we obtain $\Sat(\Phi\land\phi)$ implies $t_1',\sigma'\Consistent_{M}\tilde{t}_1',\tilde{\sigma}',\Phi\land\phi$
    from which we can conclude that $t_1'\Then e_2,\sigma'\Consistent_{M}\tilde{t}_1'\Then \tilde{e}_2,\tilde{\sigma}',\Phi\land\phi$.
    }
    \Case{$t=e_1\Xor e_2$}
    {
    Two rules apply in this case.\\
    \Case{\userule{H-PickLeft}}
      {
        % Take $i=s$. $s\sim \Left$ holds by definition.\\
        % Lemma~\ref{lem:completeNormalise} gives us the following.\\
        % There exists a symbolic execution $e_1,\sigma\normalise t_1,\sigma_1,\phi$.\\
        % There exists a symbolic execution $e_2,\sigma_1\normalise t_2,\sigma_2,\phi$.\\
        %
        % We can now conclude that a symbolic execution exists.
        % Either by the \refrule{SH-PickLeft} rule, in case $\Failing(t_2,\sigma_2)$, or by the \refrule{SH-Pick} rule in case $\neg\Failing(t_2,\sigma_2)$.
      }
    \Case{\userule{H-PickRight}}
      {
      % Take $i=s$. $s\sim \Left$ holds by definition.\\
      % Lemma~\ref{lem:completeNormalise} gives us the following.\\
      % There exists a symbolic execution $e_1,\sigma\normalise t_1,\sigma_1,\phi$.\\
      % There exists a symbolic execution $e_2,\sigma_1\normalise t_2,\sigma_2,\phi$.\\
      %
      % We can now conclude that a symbolic execution exists.
      % Either by the \refrule{SH-PickRight} rule, in case $\Failing(t_1,\sigma_1)$, or by the \refrule{SH-Pick} rule in case $\neg\Failing(t_1,\sigma_1)$.
      }
    }
    \Case{$t=t_1\Or t_2$}
      {
      Two rules applies in this case.\\
      \Case{\userule{H-FirstOr}}
      {
      % Take $i=\First i$.
      %
      % By application of the induction hypothesis, we obtain the following.\\
      % For all $t_1,\sigma,j$ such that $t_1,\sigma\xrightarrow[]{j}t_1',\sigma'$ there exists an $i\sim j$ such that $t_1'',\sigma''\handle{}t_1''',\sigma''',i,\phi$.\\
      %
      % From this, we can conclude that $\First i\sim \First j$.
      % From \refrule{SH-Or}, and the conclusion of the induction hypothesis,
      % we can conclude that there exists an $i$ such that $t_1\Or t_2,\sigma\handle{}t_1'\Or t_2,\sigma',i,\phi$.
      }
      \Case{\userule{H-SecondOr}}
      {
      % Take $i=\Second i$.
      %
      % By application of the induction hypothesis, we obtain the following.\\
      % For all $t_2,\sigma,j$ such that $t_2,\sigma\xrightarrow[]{j}t_2',\sigma'$ there exists an $i\sim j$ such that $t_2'',\sigma''\handle{}t_2''',\sigma''',i,\phi$.\\
      %
      % From this, we can conclude that $\Second i\sim \Second j$.
      % From \refrule{SH-Or}, and the conclusion of the induction hypothesis,
      % we can conclude that there exists an $i$ such that $t_1\Or t_2,\sigma\handle{}t_1\Or t_2',\sigma',i,\phi$.
      }
      }
    \Case{$t=t_1\And t_2$}
      {
      Two rules applies in this case.\\
      \Case{\userule{H-FirstAnd}}
      {
      % Take $i=\First i$.
      %
      % By application of the induction hypothesis, we obtain the following.\\
      % For all $t_1,\sigma,j$ such that $t_1,\sigma\xrightarrow[]{j}t_1',\sigma'$ there exists an $i\sim j$ such that $t_1'',\sigma''\handle{}t_1''',\sigma''',i,\phi$.\\
      %
      % From this, we can conclude that $\First i\sim \First j$.
      % From \refrule{SH-And}, and the conclusion of the induction hypothesis,
      % we can conclude that there exists an $i$ such that $t_1\And t_2,\sigma\handle{}t_1'\Or t_2,\sigma',i,\phi$.
      }
      \Case{\userule{H-SecondAnd}}
      {
      % Take $i=\Second i$.
      %
      % By application of the induction hypothesis, we obtain the following.\\
      % For all $t_2,\sigma,j$ such that $t_2,\sigma\xrightarrow[]{j}t_2',\sigma'$ there exists an $i\sim j$ such that $t_2'',\sigma''\handle{}t_2''',\sigma''',i,\phi$.\\
      %
      % From this, we can conclude that $\First i\sim \Second j$.
      % From \refrule{SH-And}, and the conclusion of the induction hypothesis,
      % we can conclude that there exists an $i$ such that $t_1\And t_2,\sigma\handle{}t_1\And t_2',\sigma',i,\phi$.
      }
      }
\end{proof}

\begin{proof}[Completeness of normalise]
  We prove Lemma~\ref{lem:completeNormalise} by induction over $e$.

  % From the premise, we can assume that $e,\sigma\Consistent_M\tilde{e},\tilde{\sigma},\Phi$.
  % Now, given that $\tilde{e},\sigma{e}\simnormalise \overline{\tilde{t},\tilde{\sigma}',\phi}$,
  % we need to demonstrate that for all pairs $(\tilde{t},\tilde{\sigma}',\phi)$,
  % $\Sat(\Phi\land\phi)$ implies that $e,\sigma\normalise t,\sigma'$ with $t,\sigma'\Consistent_M\tilde{t},\tilde{\sigma}',\Phi\land\phi$.

  The base case is when the N-Done rule applies.\\
  \userule{N-Done}\\

  % In this case, we obtain from \cref{lem:soundeval} that
  % $e,\sigma\normalise t,\sigma'$ with $t,\sigma'\Consistent_M\tilde{t},\tilde{\sigma}',\Phi\land\phi$,
  % which is exactly what we needed to show.
  %
  % The only induction step is when\\
  \userule{N-Repeat} applies.

  % In this case, we obtain from \cref{lem:soundeval} that
  % $e,\sigma\normalise t,\sigma'$ with $t,\sigma'\Consistent_M\tilde{t},\tilde{\sigma}',\Phi\land\phi_1$,
  % which is exactly what we needed to show.
  % Furthermore, by \cref{lem:soundstride} we obtain that
  % $t,\sigma'\stride t',\sigma''$ with $t',\sigma''\Consistent_M\tilde{t}',\tilde{\sigma}'',\Phi\land\phi_1\land\phi_2$.
  % Then finally, by application of the induction hypothesis, we obtain what we needed to prove.
  % $t',\sigma''\normalise t'',\sigma'''$ with $t'',\sigma'''\Consistent_M\tilde{t}'',\tilde{\sigma}''',\Phi\land\phi_1\land\phi_2\land\phi_3$.
\end{proof}

\begin{proof}[Completeness of stride]



  % Provided that $t,\sigma\Consistent_M\tilde{t},\tilde{\sigma},\Phi$ and $\tilde{t},\tilde{\sigma}\simstride \overline{\tilde{t}',\tilde{\sigma}',\phi}$,
  % we want to show that for all pairs $(\tilde{t}',\tilde{\sigma}',\phi)$,
  % we have $\Sat(\Phi\land\phi)$ implies that $t,\sigma\stride t',\sigma'$
  % We prove Lemma~\ref{lem:soundstride} by induction over $t$.



  \Case{$t=\Edit v$}
  { One rule applies, namely \userule{S-Edit}\\
    % Given that $t,\sigma\Consistent_M\Edit\tilde{v},\tilde{\sigma},\Phi$ and $\Edit\tilde{v},\tilde{\sigma}\simstride\Edit\tilde{v},\tilde{\sigma},\True$,
    % we know that $t=\Edit M \tilde{v}$, and we have $\Edit M \tilde{v},\sigma\stride\Edit M\tilde{v},\sigma$ by \refrule{S-Edit} and
    % $\Edit M \tilde{v},\sigma\Consistent_M\Edit\tilde{v},\tilde{\sigma},\Phi$, since none of the tasks and states were altered.
   }
  %
   \Case{$t=\Enter \tau$}
  {One rule applies, namely \userule{S-Fill}\\
  % Given that $t,\sigma\Consistent_M\Enter \tau,\tilde{\sigma},\Phi$ and $\Enter \tau,\tilde{\sigma}\simstride\Enter \tau,\tilde{\sigma},\True$,
  % we know that $t=\Enter \tau$, and we have $\Enter \tau,\sigma\stride\Enter \tau,\sigma$ by \refrule{S-Fill} and
  % $\Enter \tau,\sigma\Consistent_M\Edit\tilde{v},\tilde{\sigma},\Phi$, since none of the tasks and states were altered.
  }

  \Case{$t=\Update l$}
   {One rule applies, namely \userule{S-Update}\\
   % Given that $t,\sigma\Consistent_M\Update l,\tilde{\sigma},\Phi$ and $\Update l,\tilde{\sigma}\simstride\Update l,\tilde{\sigma},\True$,
   % we know that $t=\Update l$, and we have $\Update l,\sigma\stride\Update l,\sigma$ by \refrule{S-Update} and
   % $\Update l,\sigma\Consistent_M\Update l,\tilde{\sigma},\Phi$, since none of the tasks and states were altered.
  }

  \Case{$t=\Fail$}
   {One rule applies, namely \userule{S-Fail}\\
   % Given that $t,\sigma\Consistent_M\Fail,\tilde{\sigma},\Phi$ and $\Fail,\tilde{\sigma}\simstride\Fail,\tilde{\sigma},\True$,
   % we know that $t=\Fail$, and we have $\Fail,\sigma\stride\Fail,\sigma$ by \refrule{S-Fail} and
   % $\Fail,\sigma\Consistent_M\Fail,\tilde{\sigma},\Phi$, since none of the tasks and states were altered.
   }
  %
  \Case{$t=e_1\Xor e_2$}
   {One rule applies, namely \userule{S-Xor}\\
   % Given that $t,\sigma\Consistent_M\tilde{e}_1\Xor \tilde{e}_2,\tilde{\sigma},\Phi$ and $\tilde{e}_1\Xor \tilde{e}_2,\tilde{\sigma}\simstride\tilde{e}_1\Xor \tilde{e}_2,\tilde{\sigma},\True$,
   % we know that $t=M\tilde{e}_1\Xor M\tilde{e}_2$, and we have $M\tilde{e}_1\Xor M\tilde{e}_2,\sigma\stride M\tilde{e}_1\Xor M\tilde{e}_2,\sigma$ by \refrule{S-Xor} and
   % $M\tilde{e}_1\Xor M\tilde{e}_2,\sigma\Consistent_M\tilde{e}_1\Xor \tilde{e}_2,\tilde{\sigma},\Phi$, since none of the tasks and states were altered.
   }




  \Case{$t=t_1\Then e_2$}
  {
  Three rules apply.\\
  \Case{\userule{S-ThenStay}}
   {
     % Provided that $t,\sigma\Consistent_M\tilde{t}_1\Then \tilde{e}_2,\tilde{\sigma},\Phi$
     % and $\tilde{t}_1\Then \tilde{e}_2,\tilde{\sigma}\simstride\tilde{t}'_1\Then \tilde{e}_2,\tilde{\sigma}',\phi_1$,
     % we obtain from the induction hypothesis that $t_1,\sigma\stride t_1',\sigma'$ and $t_1',\sigma'\Consistent_M\tilde{t}_1',\tilde{\sigma}',\Phi$.
     % From this, we can directly conclude that $t_1\Then e_2,\sigma\stride t_1'\Then e_2,\sigma'$ and $t_1'\Then e_2,\sigma'\Consistent_M\tilde{t}_1'\Then\tilde{e}_2,\tilde{\sigma}',\Phi$.
  }
  \Case{\userule{S-ThenFail}}
   {
   % Provided that $t,\sigma\Consistent_M\tilde{t}_1\Then \tilde{e}_2,\tilde{\sigma},\Phi$
   % and $\tilde{t}_1\Then \tilde{e}_2,\tilde{\sigma}\simstride\tilde{t}'_1\Then \tilde{e}_2,\tilde{\sigma}',\phi_1$,
   % we obtain from the induction hypothesis that $t_1,\sigma\stride t_1',\sigma'$ and $t_1',\sigma'\Consistent_M\tilde{t}_1',\tilde{\sigma}',\Phi$.
   % From this, we can directly conclude that $t_1\Then e_2,\sigma\stride t_1'\Then e_2,\sigma'$ and $t_1'\Then e_2,\sigma'\Consistent_M\tilde{t}_1'\Then\tilde{e}_2,\tilde{\sigma}',\Phi$.
   }
  \Case{\userule{S-ThenCont}}
   {
   % Provided that $t,\sigma\Consistent_M\tilde{t}_1\Then \tilde{e}_2,\tilde{\sigma},\Phi$
   % and $\tilde{t}_1\Then \tilde{e}_2,\tilde{\sigma}\simstride\tilde{t}_2,\tilde{\sigma}',\phi_1\land\phi_2$
   % with $\tilde{t}_1,\tilde{\sigma}\simstride\tilde{t}_1',\tilde{\sigma}',\phi_1$ and $\Value(\tilde{t}_1',\tilde{\sigma}')=\tilde{v}_1$,
   % we obtain from the induction hypothesis that $t_1,\sigma\stride t_1',\sigma'$ and $t_1',\sigma'\Consistent_M\tilde{t}_1',\tilde{\sigma}',\Phi$.
   % Then from the consistence relation, we can conclude that $\Value(t_1',\sigma')=\Value(M t_1',M \sigma')=M\tilde{v}_1$.
   %
   % At this point, we have $e_2 M\tilde{v}_1,\sigma'\Consistent_M\tilde{e}_2 \tilde{v}_1,\tilde{\sigma}',\Phi\land\phi_1$ and $\tilde{e}_2 \tilde{v}_1,\tilde{\sigma}'\tilde{\eval}\tilde{t}_2,\tilde{sigma}'',\phi_2$.
   % This allows us to apply \cref{lem:soundeval} to obtain $e_2 (M\tilde{v}_1),\sigma'\eval t_2,\sigma''$ and $t_2,\sigma''\Consistent_M\tilde{t}_2,\tilde{\sigma}'',\Phi\land\phi_1\land\phi_2$.
   %
   % From this, we can directly conclude that $t_1\Then e_2,\sigma\stride t_2,\sigma''$ and $t_2,\sigma''\Consistent_M\tilde{t}_2,\tilde{\sigma}'',\Phi\land\phi_1\land\phi_2$.
   }
  }

  \Case{$t=t_1\Or t_2$}
  {
  One of three rules applies.\\
  \Case{\userule{S-OrLeft}}
   {
   % Provided that $t,\sigma\Consistent_M\tilde{t}_1\Or \tilde{t}_2,\tilde{\sigma},\Phi$
   % and $\tilde{t}_1\Or \tilde{t}_2,\tilde{\sigma}\simstride\tilde{t}'_1,\tilde{\sigma}',\phi$,
   % we obtain from the induction hypothesis that $t_1,\sigma\stride t_1',\sigma'$ and $t_1',\sigma'\Consistent_M\tilde{t}_1',\tilde{\sigma}',\Phi\land\phi$.
   % From this, we can directly conclude that $t_1\Or t_2,\sigma\stride t_1',\sigma'$.
  }
  \Case{\userule{S-OrRight}}
   {
   % Provided that $t,\sigma\Consistent_M\tilde{t}_1\Or \tilde{t}_2,\tilde{\sigma},\Phi$
   % and $\tilde{t}_1\Or \tilde{t}_2,\tilde{\sigma}\simstride\tilde{t}'_2,\tilde{\sigma}'',\phi_1\land\phi_2$,
   % we obtain from the induction hypothesis that $t_1,\sigma\stride t_1',\sigma'$ and $t_1',\sigma'\Consistent_M\tilde{t}_1',\tilde{\sigma}',\Phi\land\phi_1$.
   % Then by a second application of the induction hypothesis, we obtain that $t_2,\sigma'\stride t_2',\sigma''$ and $t_2',\sigma''\Consistent_M\tilde{t}_2',\tilde{\sigma}'',\Phi\land\phi_1\land\phi_2$.
   % This leads us to conclude $t_1\Or t_2,\sigma\stride t_2',\sigma''$.
   }
  \Case{\userule{S-OrNone}}
  {
  % Provided that $t,\sigma\Consistent_M\tilde{t}_1\Or \tilde{t}_2,\tilde{\sigma},\Phi$
  % and $\tilde{t}_1\Or \tilde{t}_2,\tilde{\sigma}\simstride\tilde{t}'_2,\tilde{\sigma}'',\phi_1\land\phi_2$,
  % we obtain from the induction hypothesis that $t_1,\sigma\stride t_1',\sigma'$ and $t_1',\sigma'\Consistent_M\tilde{t}_1',\tilde{\sigma}',\Phi\land\phi_1$.
  % Then by a second application of the induction hypothesis, we obtain that $t_2,\sigma'\stride t_2',\sigma''$ and $t_2',\sigma''\Consistent_M\tilde{t}_2',\tilde{\sigma}'',\Phi\land\phi_1\land\phi_2$.
  % This leads us to conclude $t_1\Or t_2,\sigma\stride t_1'\Or t_2',\sigma''$ and $t_1'\Or t_2',\sigma''\Consistent_M\tilde{t}_1'\Or\tilde{t}_2',\tilde{\sigma}'',\Phi\land\phi_1\land\phi_2$.
   }
  }

  \Case{$t=t_1\Next e_2$}
  {
  One rule applies, namely \userule{S-Next}\\
  % Provided that $t,\sigma\Consistent_M\tilde{t}_1\Next \tilde{e}_2,\tilde{\sigma},\Phi$
  % and $\tilde{t}_1\Next \tilde{e}_2,\tilde{\sigma}\simstride\tilde{t}'_1\Then \tilde{e}_2,\tilde{\sigma}',\phi$,
  % we obtain from the induction hypothesis that $t_1,\sigma\stride t_1',\sigma'$ and $t_1',\sigma'\Consistent_M\tilde{t}_1',\tilde{\sigma}',\Phi\land\phi$.
  % From this, we can directly conclude that $t_1\Next e_2,\sigma\stride t_1'\Next e_2,\sigma'$ and $t_1'\Next e_2,\sigma'\Consistent_M\tilde{t}_1'\Next\tilde{e}_2,\tilde{\sigma}',\Phi\land\phi$.
  }

  \Case{$t=t_1\And t_2$}
  {
  One rule applies, namely \userule{S-And}\\
  % Provided that $t,\sigma\Consistent_M\tilde{t}_1\And \tilde{t}_2,\tilde{\sigma},\Phi$
  % and $\tilde{t}_1\And \tilde{t}_2,\tilde{\sigma}\simstride\tilde{t}'_1\And\tilde{t}'_2,\tilde{\sigma}'',\phi_1\land\phi_2$,
  % we obtain from the induction hypothesis that $t_1,\sigma\stride t_1',\sigma'$ and $t_1',\sigma'\Consistent_M\tilde{t}_1',\tilde{\sigma}',\Phi\land\phi_1$.
  % Then by a second application of the induction hypothesis, we obtain that $t_2,\sigma'\stride t_2',\sigma''$ and $t_2',\sigma''\Consistent_M\tilde{t}_2',\tilde{\sigma}'',\Phi\land\phi_1\land\phi_2$.
  % This leads us to conclude $t_1\And t_2,\sigma\stride t_1'\And t_2',\sigma''$ and $t_1'\And t_2',\sigma''\Consistent_M\tilde{t}_1'\And\tilde{t}_2',\tilde{\sigma}'',\Phi\land\phi_1\land\phi_2$.

  }
\end{proof}

\begin{proof}[Completeness of evaluate]
  We prove Lemma~\ref{lem:completeEval} by induction over $e$.

  \Case{$e=v$}
    {One rule applies, namely \userule{E-Value}\\
    Since $v,\sigma\Consistent_M \tilde{e},\tilde{\sigma},\Phi$, we know that $\tilde{e}=\tilde{v}$.
    By \refrule{SE-Value}, we have $\tilde{v},\tilde{\sigma}\tilde{\eval}\tilde{v},\tilde{\sigma},\True$.
    Since the expressions did not change, this case holds trivially.
    }

  \Case{$e=\tuple{e_1,e_2}$}
    {
    Provided that $\tuple{e_1, e_2},\sigma\Consistent_M \tilde{e},\tilde{\sigma},\Phi$ and \userule{E-Pair},
    then by application of the induction hypothesis we obtain $\tilde{e}_1,\tilde{\sigma}\tilde{\eval}\tilde{v}_1,\tilde{\sigma}',\phi_1$
    and $v_1,\sigma'\Consistent_M \tilde{v}_1,\tilde{\sigma}',\Phi\land\phi_1$.
    A second application of the induction hypothesis gives us  $\tilde{e}_2,\tilde{\sigma}'\tilde{\eval}\tilde{v}_2,\tilde{\sigma}'',\phi_2$
    and $v_2,\sigma''\Consistent_M \tilde{v}_2,\tilde{\sigma}'',\Phi\land\phi_2$.
    By \refrule{SE-Pair}, we have $\tuple{\tilde{e}_1, \tilde{e}_2},\tilde{\sigma}\tilde{\eval}\tuple{\tilde{v}_1,\tilde{v}_2},\tilde{\sigma}'',\phi_1\land\phi_2$ and $\tuple{v_1,v_2},\sigma''\Consistent_M \tuple{\tilde{v}_1,\tilde{v}_2},\tilde{\sigma}'',\Phi\land\phi_1\land\phi_2$.
    }

  \Case{$e=\Fst \tuple{e_1,e_2}$}
  {
    Provided that $\Fst \tuple{e_1,e_2},\sigma\Consistent_M \tilde{e},\tilde{\sigma},\Phi$ and \userule{E-First},
    then by application of the induction hypothesis we obtain $\tilde{e}_1,\tilde{\sigma}\tilde{\eval}\tilde{v}_1,\tilde{\sigma}',\phi$
    and $v_1,\sigma'\Consistent_M \tilde{v}_1,\tilde{\sigma}',\Phi\land\phi$.
    By \refrule{SE-First}, we have $\Fst \tuple{\tilde{e}_1,\tilde{e}_2},\tilde{\sigma}\tilde{\eval}\tilde{v}_1,\tilde{\sigma}',\phi$.
    }

  \Case{$e=\Snd \tuple{e_1,e_2}$}
  {
  Provided that $\Snd \tuple{e_1,e_2},\sigma\Consistent_M \tilde{e},\tilde{\sigma},\Phi$ and \userule{E-Second},
  then by application of the induction hypothesis we obtain $\tilde{e}_2,\tilde{\sigma}\tilde{\eval}\tilde{v}_2,\tilde{\sigma}',\phi$
  and $v_2,\sigma'\Consistent_M \tilde{v}_2,\tilde{\sigma}',\Phi\land\phi$.
  By \refrule{SE-Second}, we have $\Snd \tuple{\tilde{e}_1,\tilde{e}_2},\tilde{\sigma}\tilde{\eval}\tilde{v}_2,\tilde{\sigma}',\phi$.
    }

  \Case{$e=e_1::e_2$}
    {
    Provided that $e_1:: e_2,\sigma\Consistent_M \tilde{e},\tilde{\sigma},\Phi$ and \userule{E-Cons},
    then by application of the induction hypothesis we obtain $\tilde{e}_1,\tilde{\sigma}\tilde{\eval}\tilde{v}_1,\tilde{\sigma}',\phi_1$
    and $v_1,\sigma'\Consistent_M \tilde{v}_1,\tilde{\sigma}',\Phi\land\phi_1$.
    A second application of the induction hypothesis gives us  $\tilde{e}_2,\tilde{\sigma}'\tilde{\eval}\tilde{v}_2,\tilde{\sigma}'',\phi_2$
    and $v_2,\sigma''\Consistent_M \tilde{v}_2,\tilde{\sigma}'',\Phi\land\phi_2$.
    By \refrule{SE-Cons}, we have $\tilde{e}_1 :: \tilde{e}_2,\tilde{\sigma}\tilde{\eval}\tilde{v}_1::\tilde{v}_2,\tilde{\sigma}'',\phi_1\land\phi_2$ and $v_1::v_2,\sigma''\Consistent_M \tilde{v}_1::\tilde{v}_2,\tilde{\sigma}'',\Phi\land\phi_1\land\phi_2$.
   }

  \Case{$e=\Head e$}
    {
    Provided that $\Head e,\sigma\Consistent_M \tilde{e},\tilde{\sigma},\Phi$ and \userule{E-Head},
    then by application of the induction hypothesis we obtain $\tilde{e},\tilde{\sigma}\tilde{\eval}\tilde{v}_1 :: \tilde{v}_2,\tilde{\sigma}',\phi$
    and $v_1::v_2,\sigma'\Consistent_M \tilde{v}_1 :: \tilde{v}_2,\tilde{\sigma}',\Phi\land\phi$.
    By \refrule{SE-Head}, we have $\tilde{v}_1 :: \tilde{v}_2,\tilde{\sigma}\tilde{\eval}\tilde{v}_1,\tilde{\sigma}',\phi$.
    }

  \Case{$e=\Tail e$}
    {
    Provided that $\Tail e,\sigma\Consistent_M \tilde{e},\tilde{\sigma},\Phi$ and \userule{E-Tail},
    then by application of the induction hypothesis we obtain $\tilde{e},\tilde{\sigma}\tilde{\eval}\tilde{v}_1 :: \tilde{v}_2,\tilde{\sigma}',\phi$
    and $v_1::v_2,\sigma'\Consistent_M \tilde{v}_1 :: \tilde{v}_2,\tilde{\sigma}',\Phi\land\phi$.
    By \refrule{SE-Tail}, we have $\tilde{v}_1 :: \tilde{v}_2,\tilde{\sigma}\tilde{\eval}\tilde{v}_2,\tilde{\sigma}',\phi$.
      }

  \Case{$e=e_1 e_2$}
    {

    Provided that $e_1 e_2,\sigma\Consistent_M \tilde{e},\tilde{\sigma},\Phi$ and\\
    \userule{E-App},
    then by application of the induction hypothesis we obtain $\tilde{e}_1,\tilde{\sigma}\tilde{\eval}\lambda x:\tau .\tilde{e}_1',\tilde{\sigma}',\phi_1$
    and $\lambda x:\tau .e_1',\sigma'\Consistent_M \lambda x:\tau .\tilde{e}_1',\tilde{\sigma}',\Phi\land\phi_1$.
    A second application of the induction hypothesis gives us  $\tilde{e}_2,\tilde{\sigma}'\tilde{\eval}\tilde{v}_2,\tilde{\sigma}'',\phi_2$
    and $v_2,\sigma''\Consistent_M \tilde{v}_2,\tilde{\sigma}'',\Phi\land\phi_1\land\phi_2$.
    Then finally by a third application of the induction hypothesis, we get  $\tilde{e}_1'[x\mapsto \tilde{v}_2],\tilde{\sigma}''\tilde{\eval}\tilde{v}_1,\tilde{\sigma}''',\phi_3$
    and $v_1,\sigma'''\Consistent_M \tilde{v}_1,\tilde{\sigma}''',\Phi\land\phi_1\land\phi_2\land\phi_3$.
    By \refrule{SE-App}, we have $\tilde{e}_1 \tilde{e}_2,\tilde{\sigma}\tilde{\eval}\tilde{v}_1,\tilde{\sigma}''',\phi_1\land\phi_2\land\phi_2$.
    }

  \Case{$e=\If{e_1}{e_2}{e_3}$}
     {\Case{1}
     {
     Provided that $\If{e_1}{e_2}{e_3},\sigma\Consistent_M \tilde{e},\tilde{\sigma},\Phi$ and\\
     \userule{E-IfTrue},
     then by application of the induction hypothesis we obtain $\tilde{e}_1,\tilde{\sigma}\tilde{\eval}\tilde{v}_1,\tilde{\sigma}',\phi_1$
     and $\True,\sigma'\Consistent_M \tilde{v}_1,\tilde{\sigma}',\Phi\land\phi_1$.
     A second application of the induction hypothesis gives us  $\tilde{e}_2,\tilde{\sigma}'\tilde{\eval}\tilde{v}_2,\tilde{\sigma}'',\phi_2$
     and $v_2,\sigma''\Consistent_M \tilde{v}_2,\tilde{\sigma}'',\Phi\land\phi_1\land\phi_2$.
     By \refrule{SE-If}, we have $\If{\tilde{e}_1}{\tilde{e}_2}{\tilde{e}_3},\tilde{\sigma}\tilde{\eval}\tilde{v}_2,\tilde{\sigma}'',\phi_1\land\phi_2\land \tilde{v}_1$.
     }
      \Case{2}{
      Provided that $\If{e_1}{e_2}{e_3},\sigma\Consistent_M \tilde{e},\tilde{\sigma},\Phi$ and\\
      \userule{E-IfFalse},
      then by application of the induction hypothesis we obtain $\tilde{e}_1,\tilde{\sigma}\tilde{\eval}\tilde{v}_1,\tilde{\sigma}',\phi_1$
      and $\False,\sigma'\Consistent_M \tilde{v}_1,\tilde{\sigma}',\Phi\land\phi_1$.
      A second application of the induction hypothesis gives us  $\tilde{e}_3,\tilde{\sigma}'\tilde{\eval}\tilde{v}_3,\tilde{\sigma}'',\phi_2$
      and $v_3,\sigma''\Consistent_M \tilde{v}_3,\tilde{\sigma}'',\Phi\land\phi_1\land\phi_2$.
      By \refrule{SE-If}, we have $\If{\tilde{e}_1}{\tilde{e}_2}{\tilde{e}_3},\tilde{\sigma}\tilde{\eval}\tilde{v}_3,\tilde{\sigma}'',\phi_1\land\phi_3\land \neg\tilde{v}_1$.
      }

    }

  \Case{$e=\Ref e$}
    {
    Provided that $\Ref e,\sigma\Consistent_M \tilde{e},\tilde{\sigma},\Phi$ and \userule{E-Ref},
    then by application of the induction hypothesis we obtain $\tilde{e},\tilde{\sigma}\tilde{\eval}\tilde{v},\tilde{\sigma}',\phi$
    and $v,\sigma'\Consistent_M \tilde{v},\tilde{\sigma}',\Phi\land\phi$.
    By \refrule{SE-Ref}, we have $\Ref \tilde{e},\tilde{\sigma}\tilde{\eval}l,\tilde{\sigma}'[l\mapsto\tilde{v}],\phi$ and $l,\sigma'[l\mapsto v]\Consistent_M l,\tilde{\sigma}'[f\mapsto \tilde{v}],\Phi\land\phi$.
    }

  \Case{$e=!e$}
    {
    Provided that $!e,\sigma\Consistent_M \tilde{e},\tilde{\sigma},\Phi$ and \userule{E-Deref},
    then by application of the induction hypothesis we obtain $\tilde{e},\tilde{\sigma}\tilde{\eval}l,\tilde{\sigma}',\phi$
    and $l,\sigma'\Consistent_M l,\tilde{\sigma}',\Phi\land\phi$.
    By \refrule{SE-Deref}, we have $!\tilde{e},\tilde{\sigma}\tilde{\eval}\tilde{\sigma}'(l),\tilde{\sigma}',\phi$ and $\sigma'(l),\sigma'\Consistent_M \tilde{\sigma}'(l),\tilde{\sigma}',\Phi\land\phi$.
  }

  \Case{$e=e_1:=e_2$}
    {
    Provided that $e_1:= e_2,\sigma\Consistent_M \tilde{e},\tilde{\sigma},\Phi$ and \userule{E-Assign},
    then by application of the induction hypothesis we obtain $\tilde{e}_1,\tilde{\sigma}\tilde{\eval}l,\tilde{\sigma}',\phi_1$
    and $l,\sigma'\Consistent_M l,\tilde{\sigma}',\Phi\land\phi_1$.
    A second application of the induction hypothesis gives us  $\tilde{e}_2,\tilde{\sigma}'\tilde{\eval}\tilde{v}_2,\tilde{\sigma}'',\phi_2$
    and $v_2,\sigma''\Consistent_M \tilde{v}_2,\tilde{\sigma}'',\Phi\land\phi_2$.
    By \refrule{SE-Assign}, we have $\tilde{e}_1 := \tilde{e}_2,\tilde{\sigma}\tilde{\eval}\unit,\tilde{\sigma}''[l\mapsto\tilde{v}_2],\phi_1\land\phi_2$ and $\Unit,\sigma''[l\mapsto v_2]\Consistent_M \Unit,\tilde{\sigma}''[l\mapsto\tilde{v}_2],\Phi\land\phi_1\land\phi_2$.
    }

  \Case{$e=\Edit e$}
    {
    Provided that $\Edit e,\sigma\Consistent_M \tilde{e},\tilde{\sigma},\Phi$ and \userule{E-Edit},
    then by application of the induction hypothesis we obtain $\tilde{e},\tilde{\sigma}\tilde{\eval}\tilde{v},\tilde{\sigma}',\phi$
    and $v,\sigma'\Consistent_M \tilde{v},\tilde{\sigma}',\Phi\land\phi$.
    By \refrule{SE-Edit}, we have $\Edit\tilde{e},\tilde{\sigma}\tilde{\eval}\Edit\tilde{v},\tilde{\sigma}',\phi$ and $\Edit v,\sigma'\Consistent_M \Edit \tilde{v},\tilde{\sigma}',\Phi\land\phi$.

    }

  \Case{$e=\Enter \tau$}
    {
    One rule applies, namely \userule{E-Enter}\\
    Since $\Enter \tau,\sigma\Consistent_M \tilde{e},\tilde{\sigma},\Phi$, we know that $\tilde{e}=\Enter \tau$.
    By \refrule{SE-Enter}, we have $\Enter \tau,\tilde{\sigma}\tilde{\eval}\Enter \tau,\tilde{\sigma},\True$.
    Since the expressions did not change, this case holds trivially.
    }

  \Case{$e=\Update e$}
    {Provided that $\Update e,\sigma\Consistent_M \tilde{e},\tilde{\sigma},\Phi$ and \userule{E-Update},
    then by application of the induction hypothesis we obtain $\tilde{e},\tilde{\sigma}\tilde{\eval}l,\tilde{\sigma}',\phi$
    and $l,\sigma'\Consistent_M l,\tilde{\sigma}',\Phi\land\phi$.
    By \refrule{SE-Update}, we have $\Update\tilde{e},\tilde{\sigma}\tilde{\eval}\Update l,\tilde{\sigma}',\phi$ and $\Update l,\sigma'\Consistent_M \Update l ,\tilde{\sigma}',\Phi\land\phi$.

    }

  \Case{$e=e_1\Then e_2$}
    {
    Provided that $e_1\Then e_2,\sigma\Consistent_M \tilde{e},\tilde{\sigma},\Phi$ and \userule{E-Then},
    then by application of the induction hypothesis we obtain $\tilde{e}_1,\tilde{\sigma}\tilde{\eval}\tilde{v}_1,\tilde{\sigma}',\phi$
    and $v_1,\sigma'\Consistent_M \tilde{v}_1,\tilde{\sigma}',\Phi\land\phi$.
    By \refrule{SE-Then}, we have $\tilde{e}_1\Then \tilde{e}_2,\tilde{\sigma}\tilde{\eval}\tilde{v}_1\Then\tilde{e}_2,\tilde{\sigma}',\phi$ and $v_1\Then e_2 ,\sigma'\Consistent_M \tilde{v}_1\Then\tilde{e}_2 ,\tilde{\sigma}',\Phi\land\phi$.

    }

  \Case{$e=e_1\Next e_2$}
    {
    Provided that $e_1\Next e_2,\sigma\Consistent_M \tilde{e},\tilde{\sigma},\Phi$ and \userule{E-Next},
    then by application of the induction hypothesis we obtain $\tilde{e}_1,\tilde{\sigma}\tilde{\eval}\tilde{v}_1,\tilde{\sigma}',\phi$
    and $v_1,\sigma'\Consistent_M \tilde{v}_1,\tilde{\sigma}',\Phi\land\phi$.
    By \refrule{SE-Next}, we have $\tilde{e}_1\Next \tilde{e}_2,\tilde{\sigma}\tilde{\eval}\tilde{v}_1\Next\tilde{e}_2,\tilde{\sigma}',\phi$ and $v_1\Then e_2 ,\sigma'\Consistent_M \tilde{v}_1\Next\tilde{e}_2 ,\tilde{\sigma}',\Phi\land\phi$.

    }

  \Case{$e=e_1\Or e_2$}
    {
    Provided that $e_1\Or e_2,\sigma\Consistent_M \tilde{e},\tilde{\sigma},\Phi$ and \userule{E-Or},
    then by application of the induction hypothesis we obtain $\tilde{e}_1,\tilde{\sigma}\tilde{\eval}\tilde{t}_1,\tilde{\sigma}',\phi_1$
    and $t_1,\sigma'\Consistent_M \tilde{t}_1,\tilde{\sigma}',\Phi\land\phi_1$.
    A second application of the induction hypothesis gives us  $\tilde{e}_2,\tilde{\sigma}'\tilde{\eval}\tilde{t}_2,\tilde{\sigma}'',\phi_2$
    and $t_2,\sigma''\Consistent_M \tilde{t}_2,\tilde{\sigma}'',\Phi\land\phi_2$.
    By \refrule{SE-Or}, we have $\tilde{e}_1\Or \tilde{e}_2,\tilde{\sigma}\tilde{\eval}\tilde{t}_1\Or\tilde{t}_2,\tilde{\sigma}'',\phi_1\land\phi_2$ and $t_1\Or t_2 ,\sigma''\Consistent_M \tilde{t}_1\Or\tilde{t}_2 ,\tilde{\sigma}'',\Phi\land\phi_1\land\phi_2$.

    }

  \Case{$e=e_1\Xor e_2$}
    {  One rule applies, namely \userule{E-Xor}\\
    Since $e_1\Xor e_2,\sigma\Consistent_M \tilde{e},\tilde{\sigma},\Phi$, we know that $\tilde{e}=\tilde{e}_1\Xor \tilde{e}_2$.
    By \refrule{SE-Xor}, we have $\tilde{e}_1\Xor \tilde{e}_2,\tilde{\sigma}\tilde{\eval}\tilde{e}_1\Xor \tilde{e}_2,\tilde{\sigma},\True$.
    Since the expressions did not change, this case holds trivially.

    }

  \Case{$e=\Fail$}
    {  One rule applies, namely \userule{E-Fail}\\
    Since $\Fail,\sigma\Consistent_M \tilde{e},\tilde{\sigma},\Phi$, we know that $\tilde{e}=\Fail$.
    By \refrule{SE-Fail}, we have $\Fail,\tilde{\sigma}\tilde{\eval}\Fail,\tilde{\sigma},\True$.
    Since the expressions did not change, this case holds trivially.


    }
\end{proof}
