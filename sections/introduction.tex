% !TEX root=../main.tex

\section{Introduction}
\label{sec:intro}

%There are many workflow systems
Software that supports people working together is used in most workplaces nowadays.
Its aim is to automate business workflows, in order to simplify processes, to improve service, or to contain cost.
In settings like hospitals, first responders and military operations, these systems could even prevent the loss of lives.

%they are a problem because

Automation and digitalisation of workflows and business processes comes at a cost.
Using such applications requires workers to undergo training, and often relies on their experience and expertise.
\todo{Fix problem description.}

%our approach overcomes this because
To overcome these drawbacks, we propose to integrate a next step hint assistive system into the workflow software.
By combining previous research on symbolic execution for Task-Oriented Programming~\cite{Naus2019} and end-user feedback systems for rule based problems~\cite{DBLP:conf/sfp/NausJ16},
we develop a next step hint end-user feedback system for the Task-Oriented Programming language \TOPHAT~\cite{DBLP:conf/ppdp/SteenvoordenNK19}.
Our solution, which we call Assistive \TOPHAT, generates next step hints from existing code, and does not require extra work by the programmer.

\todo{Specify a bit more the scope of the paper. Say something about the uniqueness of this work, as we do in the related work section.}

\subsection{Contributions}

This paper makes the following contributions.

\begin{itemize}
  \item We describe an end user next step feedback system for \TOPHAT.
  \item We prove soundness and completeness of next step hints generated by this system.
  \item We present an implementation of the user feedback system in Haskell.
  \item We prove the symbolic execution semantics of \TOPHAT to be sound and complete.
\end{itemize}


\subsection{Structure}

\cref{sec:tophat} first introduces the Task-Oriented Programming (\TOP) paradigm and the Task-Oriented Programming language \TOPHAT.
\cref{sec:examples} lists three example programs to illustrate how \TOPHAT works and to show what we like to achieve with Assistive \TOPHAT.
In \cref{sec:symbolic} we introduce briefly the symbolic execution semantics for \TOPHAT.
\cref{sec:assistive} follows with a description of assistive \TOPHAT
and describes the implementation of the system in Haskell.
In \cref{sec:properties} soundness and completeness of the assistive system are shown.
% An implementation of the system in Haskell is described in \cref{sec:implementation}.
\cref{sec:relatedwork} gives an overview of related work, and finally \cref{sec:conclusion} concludes.
