% This is samplepaper.tex, a sample chapter demonstrating the
% LLNCS macro package for Springer Computer Science proceedings;
% Version 2.20 of 2017/10/04
%
\documentclass[runningheads]{llncs}
%


\input{macros/imports}
\input{macros/setups}
\input{macros/general}
\input{macros/languages}
\input{macros/abbreviations}
% !TEX root=../main.tex



\let\phi\varphi

\newmacro{Today}{\text{13 Feb 2020}}
\newmacro{OneYear}{\text{365 days}}


%% Host language %%%%%%%%%%%%%%%%%%%%%%%%%%%%%%%%%%%%%%%%%%%%%%%%%%%%%%%%%%%%%%%


\newkeyword[IF]  {if}
\newkeyword[THEN]{then}
\newkeyword[ELSE]{else}

\newkeyword[Let]{let}
\newkeyword[In]{in}

\newkeyword[Ref] {ref}


\newmacro{If}[3]
  {\IF #1 \THEN #2 \ELSE #3}



%% Values %%


\newmathcommand{unit}{\<\>}


\newvalue{True}
\newvalue{False}
\newvalue[Not]{not}


\newmacro{str}[1]
  {\text{``#1''}}

\newvalue[Map]{map}
\newvalue[Fst]{fst}
\newvalue[Snd]{snd}
\newvalue[Head]{head}
\newvalue[Tail]{tail}
\newvalue[Uniq]{uniq}
\newvalue[Len]{len}



%% Types %%


\newtype{Unit}
\newtype{Bool}
\newtype{Nat}
\newtype{Int}
\newtype{String}
\newtype[Reference]{Ref}
\newtype{Task}
\newtype{Maybe}
\newtype{List}

\newtype{Euro}



%% Object language %%%%%%%%%%%%%%%%%%%%%%%%%%%%%%%%%%%%%%%%%%%%%%%%%%%%%%%%%%%%%


\let\And\relax
\newoperator{Then}  {\blacktriangleright}
\newoperator{Next}  {\vartriangleright}
\newoperator{And}   {\Join}
\newoperator{Or}    {\blacklozenge}
\newoperator{Xor}   {\lozenge}
\newoperator{Edit}  {\square}
\newoperator{View}  {\overline{\square}}
\newoperator{Enter} {\boxtimes}
\newoperator{Update}{\blacksquare}
\newoperator{Watch} {\overline{\blacksquare}}
\newoperator{Fail}  {\lightning}
\newoperator{At}    {@}

\newoperator{AndOr} {\DEPRECATED}



%% Events %%


\newvalue[Left]   {L}
\newvalue[Right]  {R}


\newvalue[Empty]   {E}
\newvalue[Continue]{C}
\newvalue[Pick]    {P}


\newvalue[First]  {F}
\newvalue[Second] {S}
\newvalue[Here]   {H}



%% Semantic functions %%%%%%%%%%%%%%%%%%%%%%%%%%%%%%%%%%%%%%%%%%%%%%%%%%%%%%%%%%


\newmathcommand{eval}[rel]
  {\;\downarrow\;}
\newmathcommand{stride}[rel]
  % {\;\rightarrow\!\shortmid\;}
  % {\;\rightsquigarrow\;}
  {\;\mapsto\;}
\newmathcommand{normalise}[rel]
  {\;\Downarrow\;}
\newmacro{handle}[1]
  {\mathrel{\;\xrightarrow{#1}\;}}
\newmacro{interact}[1]
  {\mathrel{\;\xRightarrow{#1}\;}}
\newmacro{execute}[1]
  {\mathrel{\;\xRightarrow{#1}\!\!^*\;}}

\newmathcommand{Leadsto}[rel]
  {\raisebox{-0.25ex}{$\leadsto$}\mathllap{\raisebox{+0.25ex}{$\leadsto$}}}

\newmathcommand{simeval}[rel]
  {\;\rotatebox[origin=c]{-90}{$\leadsto$}\;}
\newmathcommand{simstride}[rel]
  {\;\mapstochar\kern+0.1em\leadsto\;}
\newmathcommand{simnormalise}[rel]
  {\;\rotatebox[origin=c]{-90}{$\Leadsto$}\;}
\newmathcommand{simhandle}[rel]
  {\;\leadsto\;}
\newmathcommand{siminteract}[rel]
  {\;\Leadsto\;}
\newmathcommand{simulate}[rel]
  {\;\Leadsto^*\;}
  % {\;\raisebox{-0.4ex}{$\leadsto$}\mathllap{\raisebox{+0.4ex}{$\leadsto$}}\kern-0.4ex\circ
  % \approx\!\!>\!\!\circ\;}
  % {\;\bullet\!\!\!\Rightarrow\;\wr\!\!\!\Rightarrow\;}
  % {\;\Longmapsto\;}
  % {\;\wr\!\!\!\Rightarrow\;}

\newmathcommand{Consistent}[rel]
  {\leftrightarrows}

\newmathcommand{Simulate}[it]
  {simulate}
\newmathcommand{Evaluate}[it]
  {evaluate}
\newmathcommand{Again}[it]
  {again}

\newmathcommand{Value}[cal]
  {V}
\newmathcommand{Inputs}[cal]
  {I}
\newmathcommand{Interface}[cal]
  {U}
\newmathcommand{Failing}[cal]
  {F}
\newmathcommand{Watching}[cal]
  {W}
\newmathcommand{Dirty}
  {\Delta}
\newmathcommand{UserInterface}[cal]
  {U}
\newmathcommand{Sat}[cal]
  {S}
\newmathcommand{Hints}[cal]
  {H}


%% Proofs %%%%%%%%%%%%%%%%%%%%%%%%%%%%%%%%%%%%%%%%%%%%%%%%%%%%%%%%%%%%%%%%%%%%%%


% \newcommand{\case}[2]{
%     \noindent\textbf{Case} #1\\
%     \vspace{5mm}
%     \indent\begin{minipage}{\dimexpr\textwidth-3cm}
%     #2
%   \end{minipage}\\\\}

\newcommand{\Case}[2]{
  \bigskip
  \noindent\textbf{Case} #1
  \nopagebreak[4]
  \smallskip
  \par
  \begingroup
    \leftskip\parindent
    \noindent
    #2
    \par
  \endgroup
}

% \newcommand{\case}[2]{
%   \noindent
%   \begin{tabular*}{\textwidth}{lp{0.8\textwidth}}
%     \textbf{Case} & #1 \\
%     \addlinespace
%     & #2
%   \end{tabular*}
%   \medskip
% }



%% Depricated %%%%%%%%%%%%%%%%%%%%%%%%%%%%%%%%%%%%%%%%%%%%%%%%%%%%%%%%%%%%%%%%%%

% !TEX root=main.tex


%% Typing %%%%%%%%%%%%%%%%%%%%%%%%%%%%%%%%%%%%%%%%%%%%%%%%%%%%%%%%%%%%%%%%%%%%%%

\newrule{T-Sym}
  {s:\tau \in \Gamma}
  {\Gamma,\Sigma \infers s:\tau}


%% Evaluation %%%%%%%%%%%%%%%%%%%%%%%%%%%%%%%%%%%%%%%%%%%%%%%%%%%%%%%%%%%%%%%%%%


\newmacro{RelationSE}
  {\tilde{e},\tilde{\sigma} \tilde{\eval} \overline{\tilde{v},\tilde{\sigma}',\phi}}


\newrule{SE-Value}
  {}
  {\tilde{v},\tilde{\sigma}\tilde{\eval} \tilde{v},{\tilde{\sigma},\True}}


\newrule{SE-App}
  {\tilde{e}_1,\tilde{\sigma}\tilde{\eval} \overline{\lambda x:\tau.\tilde{e}_1',\tilde{\sigma}',{\phi_1}} \Quad
   \tilde{e}_2,\tilde{\sigma}'\tilde{\eval} \overline{\tilde{v}_2,\tilde{\sigma}'',{\phi_2}} \Quad
   \tilde{e}_1'[x\mapsto \tilde{v}_2],\tilde{\sigma}''\tilde{\eval} \overline{\tilde{v}_1,\tilde{\sigma}''',{\phi_3}}}
  {\tilde{e}_1 \tilde{e}_2,\tilde{\sigma} \tilde{\eval} \overline{\tilde{v}_1,\tilde{\sigma}''',{\phi_1\land\phi_2\land\phi_3}}}

\newrule{SE-If}
  {\tilde{e}_1,\tilde{\sigma}\tilde{\eval} \overline{\tilde{v}_1,\tilde{\sigma}',{\phi_1}} \Quad
   {\tilde{e}_2,\tilde{\sigma}'\tilde{\eval} \overline{\tilde{v}_2,\tilde{\sigma}'',\phi_2}} \Quad
   {\tilde{e}_3,\tilde{\sigma}'\tilde{\eval} \overline{\tilde{v}_3,\tilde{\sigma}''',\phi_3}}}
  {\If{\tilde{e}_1}{\tilde{e}_2}{\tilde{e}_3},\tilde{\sigma}\tilde{\eval} {\overline{\tilde{v}_2,\tilde{\sigma}'',\phi_1 \land \phi_2\land \tilde{v}_1} \cup \overline{\tilde{v}_3,\tilde{\sigma}''',\phi_1 \land \phi_3 \land \lnot \tilde{v}_1}}}

\newrule{SE-Pair}
  {\tilde{e}_1,\tilde{\sigma}\tilde{\eval} \overline{\tilde{v}_1,\tilde{\sigma}',{\phi_1}} \Quad
   \tilde{e}_2,\tilde{\sigma}'\tilde{\eval} \overline{\tilde{v}_2,\tilde{\sigma}'',{\phi_2}}}
  {\tuple{\tilde{e}_1,\tilde{e}_2},\tilde{\sigma}\tilde{\eval}\overline{\tuple{\tilde{v}_1,\tilde{v}_2},\tilde{\sigma}'',{\phi_1\land\phi_2}}}

\newrule{SE-First}
  {\tilde{e}_1,\tilde{\sigma}\tilde{\eval}\overline{\tilde{v}_1,\tilde{\sigma}',{\phi}}}
  {\Fst\tuple{\tilde{e}_1,\tilde{e}_2},\tilde{\sigma}\tilde{\eval}\overline{\tilde{v}_1,\tilde{\sigma}',{\phi}} }

\newrule{SE-Second}
  {\tilde{e}_2,\tilde{\sigma}\tilde{\eval}\overline{\tilde{v}_2,\tilde{\sigma}',{\phi}}}
  {\Snd\tuple{\tilde{e}_1,\tilde{e}_2},\tilde{\sigma}\tilde{\eval}\overline{\tilde{v}_2,\tilde{\sigma}',\phi} }


%%%%%%%

\newrule{SE-Cons}
  {\tilde{e}_1,\tilde{\sigma} \tilde{\eval} \tilde{v}_1,\tilde{\sigma}',\phi_1\Quad
   \tilde{e}_2,\tilde{\sigma}' \tilde{\eval} \tilde{v}_2,\tilde{\sigma}'',\phi_2}
  {\tilde{e}_1 :: \tilde{e}_2,\tilde{\sigma} \tilde{\eval} \tilde{v}_1:: \tilde{v}_2,\tilde{\sigma}'',\phi_1\land\phi_2}

\newrule{SE-Head}
  {\tilde{e},\tilde{\sigma} \tilde{\eval} \tilde{v}_1::\tilde{v}_2,\tilde{\sigma}',{\phi}}
  {\Head \tilde{e},\tilde{\sigma} \tilde{\eval} \tilde{v}_1,\tilde{\sigma}',{\phi}}

\newrule{SE-Tail}
{\tilde{e},\tilde{\sigma} \tilde{\eval} \tilde{v}_1::\tilde{v}_2,\tilde{\sigma}',{\phi}}
{\Tail \tilde{e},\tilde{\sigma} \tilde{\eval} \tilde{v}_2,\tilde{\sigma}',{\phi}}


%%%%%
\newrule{SE-Ref}
  {\tilde{e},\tilde{\sigma}\tilde{\eval} \overline{\tilde{v},\tilde{\sigma}',\phi} \Quad
   l\not\in Dom(\sigma')}
  {\Ref \tilde{e},\tilde{\sigma}\tilde{\eval} \overline{l,\tilde{\sigma}'[l\mapsto \tilde{v}],\phi}}

\newrule{SE-Deref}
  {\tilde{e},\tilde{\sigma}\tilde{\eval} \overline{l,\tilde{\sigma}',{\phi}}}
  {!\tilde{e},\tilde{\sigma}\tilde{\eval} \overline{\tilde{\sigma}'(l),\tilde{\sigma}',{\phi}}}

\newrule{SE-Assign}
  {\tilde{e}_1,\tilde{\sigma}\tilde{\eval} \overline{l,\tilde{\sigma}',\phi_1} \Quad
   \tilde{e}_2,\tilde{\sigma}'\tilde{\eval} \overline{\tilde{v}_2,\tilde{\sigma}'',\phi_2}}
  {\tilde{e}_1:=\tilde{e}_2,\tilde{\sigma}\tilde{\eval} \overline{\unit,\tilde{\sigma}''[l\mapsto \tilde{v}_2],\phi_1\wedge\phi_2}}

\newrule{SE-Edit}
  {\tilde{e},\tilde{\sigma} \tilde{\eval} \overline{\tilde{v},\tilde{\sigma}',\phi}}
  {\Edit \tilde{e} , \tilde{\sigma}\tilde{\eval} \overline{\Edit \tilde{v},\tilde{\sigma}',\phi}}

\newrule{SE-Enter}
  {}
  {\Enter \tau,\tilde{\sigma} \tilde{\eval} \Enter \tau,\tilde{\sigma},\True}

\newrule{SE-Update}
  {\tilde{e},\tilde{\sigma}\tilde{\eval} \overline{l,\tilde{\sigma}',\phi}}
  {\Update \tilde{e} ,\tilde{\sigma}\tilde{\eval} \overline{\Update l,\tilde{\sigma}',\phi}}


\newrule{SE-Fail}
  {}
  {\Fail,\tilde{\sigma} \tilde{\eval} \Fail,\tilde{\sigma},{\True}}


\newrule{SE-Then}
  {\tilde{e}_1 ,\tilde{\sigma}\tilde{\eval} \overline{\tilde{t}_1,\tilde{\sigma}',{\phi}}}
  {\tilde{e}_1 \Then \tilde{e}_2,\tilde{\sigma} \tilde{\eval} \overline{\tilde{t}_1 \Then \tilde{e}_2,\tilde{\sigma}',{\phi}}}

\newrule{SE-Next}
  {\tilde{e}_1 ,\tilde{\sigma}\tilde{\eval} \overline{\tilde{t}_1,\tilde{\sigma}',{\phi}}}
  {\tilde{e}_1 \Next \tilde{e}_2 ,\tilde{\sigma}\tilde{\eval} \overline{\tilde{t}_1 \Next \tilde{e}_2,\tilde{\sigma}',{\phi}}}


\newrule{SE-And}
  {\tilde{e}_1 ,\tilde{\sigma}\tilde{\eval} \overline{\tilde{t}_1 ,\tilde{\sigma}',\phi_1} \Quad
   \tilde{e}_2 ,\tilde{\sigma}'\tilde{\eval} \overline{\tilde{t}_2,\tilde{\sigma}'',\phi_2}}
  {\tilde{e}_1 \And \tilde{e}_2 ,\tilde{\sigma}\tilde{\eval} \overline{\tilde{t}_1 \And \tilde{t}_2,\tilde{\sigma}'',\phi_1\land\phi_2}}


\newrule{SE-Or}
  {\tilde{e}_1 ,\tilde{\sigma}\tilde{\eval} \overline{\tilde{t}_1 ,\tilde{\sigma}',\phi_1} \Quad
   \tilde{e}_2 ,\tilde{\sigma}'\tilde{\eval} \overline{\tilde{t}_2,\tilde{\sigma}'',\phi_2}}
  {\tilde{e}_1 \Or \tilde{e}_2 ,\tilde{\sigma}\tilde{\eval} \overline{\tilde{t}_1 \Or \tilde{t}_2,\tilde{\sigma}'',\phi_1\land\phi_2}}

\newrule{SE-Xor}
  {}
  {\tilde{e}_1 \Xor \tilde{e}_2 ,\tilde{\sigma}\tilde{\eval} \tilde{e}_1 \Xor \tilde{e}_2,\tilde{\sigma},{\True}}

%% Normalisation %%%%%%%%%%%%%%%%%%%%%%%%%%%%%%%%%%%%%%%%%%%%%%%%%%%%%%%%%%%%%%%


\newmacro{RelationSS}
  {\tilde{t},\tilde{\sigma}\tilde{\stride} \overline{\tilde{t}',\tilde{\sigma}',\phi}}


\newrule{SS-Edit}
  { }
  {\Edit \tilde{v},\tilde{\sigma} \tilde{\stride} \Edit \tilde{v},\tilde{\sigma},\True}

\newrule{SS-Fill}
  { }
  {\Enter \tau,\tilde{\sigma} \tilde{\stride} \Enter \tau,\tilde{\sigma},\True}

\newrule{SS-Update}
  { }
  {\Update l,\tilde{\sigma} \tilde{\stride} \Update l,\tilde{\sigma},\True}


\newrule{SS-Fail}
  { }
  {\Fail,\tilde{\sigma} \tilde{\stride} \Fail,\tilde{\sigma},\True}


\newrule{SS-ThenStay}
  {\tilde{t}_1,\tilde{\sigma} \tilde{\stride} \overline{\tilde{t}_1',\tilde{\sigma}',\phi}}
  {\tilde{t}_1 \Then \tilde{e}_2,\tilde{\sigma} \tilde{\stride} \overline{\tilde{t}_1' \Then \tilde{e}_2,\tilde{\sigma}',\phi}}
  [\Value(\tilde{t}_1',\tilde{\sigma}') = \bot]

\newrule{SS-ThenFail}
  {\tilde{t}_1,\tilde{\sigma} \tilde{\stride} \overline{\tilde{t}_1',\tilde{\sigma}',\phi} \Quad
   \tilde{e}_2\ \tilde{v}_1,\tilde{\sigma}' \tilde{\eval} \overline{\tilde{t}_2,\tilde{\sigma}'',\_}}
  {\tilde{t}_1 \Then \tilde{e}_2,\tilde{\sigma} \tilde{\stride} \overline{\tilde{t}_1' \Then \tilde{e}_2,\tilde{\sigma}',\phi}}
  [\Value(\tilde{t}_1',\tilde{\sigma}') = \tilde{v}_1 \land \Failing(\tilde{t}_2,\tilde{\sigma}'')]

\newrule{SS-ThenCont}
  {\tilde{t}_1,\tilde{\sigma} \tilde{\stride} \overline{\tilde{t}_1',\tilde{\sigma}',\phi_1} \Quad
  \tilde{e}_2\ \tilde{v}_1,\tilde{\sigma}' \tilde{\eval} \overline{\tilde{t}_2 ,\tilde{\sigma}'',\phi_2}}
   % t_2,\sigma'' \stride t_2',\sigma'''}
  {\tilde{t}_1 \Then \tilde{e}_2,\tilde{\sigma} \tilde{\stride} \overline{t_2,\sigma'',{\phi_1\land\phi_2}}}
  [\Value(\tilde{t}_1',\tilde{\sigma}') = \tilde{v}_1 \land \lnot\Failing(\tilde{t}_2,\tilde{\sigma}'')]

\newrule{SS-Next}
  {\tilde{t}_1,\tilde{\sigma} \tilde{\stride} \overline{\tilde{t}_1',\tilde{\sigma}',\phi}}
  {\tilde{t}_1 \Next \tilde{e}_2,\tilde{\sigma} \tilde{\stride} \overline{\tilde{t}_1' \Next \tilde{e}_2,\tilde{\sigma}',\phi}}


\newrule{SS-And}
  {\tilde{t}_1,\tilde{\sigma}  \tilde{\stride} \overline{\tilde{t}_1',\tilde{\sigma}',\phi_1 } \Quad
   \tilde{t}_2,\tilde{\sigma}' \tilde{\stride} \overline{\tilde{t}_2',\tilde{\sigma}'',\phi_2}}
  {\tilde{t}_1 \And \tilde{t}_2,\tilde{\sigma} \tilde{\stride} \overline{\tilde{t}_1' \And \tilde{t}_2',\tilde{\sigma}'',\phi_1\land\phi_2}}


\newrule{SS-OrLeft}
 {\tilde{t}_1,\tilde{\sigma}  \tilde{\stride} \overline{\tilde{t}_1',\tilde{\sigma}',\phi}}
 {\tilde{t}_1 \Or \tilde{t}_2,\tilde{\sigma} \tilde{\stride} \overline{\tilde{t}_1',\tilde{\sigma}',\phi}}
  [\Value(\tilde{t}_1',\tilde{\sigma}') = \tilde{v}_1]

\newrule{SS-OrRight}
  {\tilde{t}_1,\tilde{\sigma}  \tilde{\stride} \overline{\tilde{t}_1',\tilde{\sigma}',\phi_1}  \Quad
   \tilde{t}_2,\tilde{\sigma}' \tilde{\stride} \overline{\tilde{t}_2',\tilde{\sigma}'',\phi_2}}
  {\tilde{t}_1 \Or \tilde{t}_2,\tilde{\sigma} \tilde{\stride} \overline{\tilde{t}_2',\tilde{\sigma}'',\phi_1\land\phi_2}}
  [\Value(\tilde{t}_1',\tilde{\sigma}') = \bot \land \Value(\tilde{t}_2',\tilde{\sigma}'') = \tilde{v}_2]

\newrule{SS-OrNone}
  {\tilde{t}_1,\tilde{\sigma}  \tilde{\stride} \overline{\tilde{t}_1',\tilde{\sigma}' ,\phi_1} \Quad
   \tilde{t}_2,\tilde{\sigma}' \tilde{\stride} \overline{\tilde{t}_2',\tilde{\sigma}'',\phi_2}}
  {\tilde{t}_1 \Or \tilde{t}_2,\tilde{\sigma} \tilde{\stride} \overline{\tilde{t}_1' \Or \tilde{t}_2',\tilde{\sigma}'',\phi_1\land\phi_2}}
  [\Value(\tilde{t}_1',\tilde{\sigma}') = \bot \land \Value(\tilde{t}_2',\tilde{\sigma}'') = \bot]


\newrule{SS-Xor}
  {\ }
  {\tilde{e}_1 \Xor \tilde{e}_2,\tilde{\sigma} \tilde{\stride} \tilde{e}_1 \Xor \tilde{e}_2,\tilde{\sigma},\True}

\newrule{SS-Eval}
    {\tilde{e},\tilde{\sigma} \tilde{\eval} \overline{\tilde{e}',\tilde{\sigma}',\phi_1}  \Quad
     \tilde{e}',\tilde{\sigma}' \tilde{\stride} \overline{\tilde{e}'',\tilde{\sigma}'',\phi_2}}
    {\tilde{e},\tilde{\sigma} \tilde{\stride} \overline{\tilde{e}'',\tilde{\sigma}'',\phi_1\land\phi_2}}
    [\tilde{e} \neq \tilde{e}']

%% Normalisation %%


\newmacro{RelationSN}
  {\tilde{e},\tilde{\sigma} \tilde{\normalise} \overline{\tilde{t},\tilde{\sigma}',\phi}}


\newrule{SN-Done}
    {\tilde{e},\tilde{\sigma} \tilde{\eval} \overline{\tilde{t},\tilde{\sigma}',\phi_1}  \Quad
     \tilde{t},\tilde{\sigma}' \tilde{\stride} \overline{\tilde{t}',\tilde{\sigma}'',\phi_2}}
    {\tilde{e},\tilde{\sigma} \tilde{\normalise} \overline{\tilde{t},\tilde{\sigma}',\phi_1}}
    [\tilde{\sigma}'=\tilde{\sigma}'' \land \tilde{t}=\tilde{t}']

\newrule{SN-Repeat}
    {\tilde{e},\tilde{\sigma} \tilde{\eval} \overline{\tilde{t},\tilde{\sigma}',\phi_1}  \Quad
     \tilde{t},\tilde{\sigma}' \tilde{\stride} \overline{\tilde{t}',\tilde{\sigma}'',\phi_2 } \Quad
     \tilde{t}',\tilde{\sigma}'' \tilde{\normalise} \overline{\tilde{t}'',\tilde{\sigma}''',\phi_3}}
    {\tilde{e},\tilde{\sigma} \tilde{\normalise} \overline{\tilde{t}'',\tilde{\sigma}''',\phi_1 \land \phi_2 \land \phi_3}}
    [\tilde{\sigma}'\neq \tilde{\sigma}''\vee \tilde{t}\neq \tilde{t}']



%% Handling %%


\newmacro{RelationSH}
  {\tilde{t},\tilde{\sigma} \handle{} \overline{\tilde{t}',\tilde{\sigma}',\tilde{i},\phi}}


\newrule{SH-Change}
  { \text{fresh }s}
  {\Edit \tilde{v},\tilde{\sigma} \handle{} \Edit s,\tilde{\sigma},s,\True}
  [\tilde{v},s:\tau]

% \newrule{SH-Empty}
%   { }
%   {\Edit v,\sigma \handle{\Empty} \Enter \tau,{\sigma,\True}}
%   [v : \tau]

\newrule{SH-Fill}
  { \text{fresh }\tilde{s}}
  {\Enter \tau,\tilde{\sigma} \handle{} \Edit s,\tilde{\sigma},s,\True}
  [s:\tau]

\newrule{SH-Update}
  { \text{fresh }s}
  {\Update l,\tilde{\sigma} \handle{} \Update l,\tilde{\sigma}[l \mapsto s],s,\True}
  [\sigma(l),s:\tau]

\newrule{SH-PassThen}
  {\tilde{t}_1,\tilde{\sigma} \handle{} \overline{\tilde{t}_1',\tilde{\sigma}',\tilde{i},\phi}}
  {\tilde{t}_1 \Then \tilde{e}_2,\tilde{\sigma} \handle{} \overline{t}_1' \Then \tilde{e}_2,\tilde{\sigma}',\tilde{i},\phi}

\newrule{SH-PassNext}
  {\tilde{t}_1,\tilde{\sigma} \handle{} \overline{\tilde{t}_1',\tilde{\sigma}',\tilde{i},\phi}}
  {\tilde{t}_1 \Next \tilde{e}_2,\tilde{\sigma} \handle{} \overline{\tilde{t}_1' \Next \tilde{e}_2,\tilde{\sigma}',\tilde{i},\phi}}
  [\Value{(\tilde{t}_1,\tilde{\sigma})} = \bot]

\newrule{SH-PassNextFail}
  {\tilde{t}_1,\tilde{\sigma} \handle{} \overline{\tilde{t}_1',\tilde{\sigma}_1',\tilde{i},\phi} \Quad
  \tilde{e}_2\ \tilde{v}_1,\tilde{\sigma} \tilde{\normalise} \overline{\tilde{t}_2,\tilde{\sigma}_2',{\vphantom{i}\_}}}
  {\tilde{t}_1 \Next \tilde{e}_2,\tilde{\sigma} \handle{} \overline{\tilde{t}_1' \Next \tilde{e}_2,\tilde{\sigma}_1',\tilde{i},\phi}}
  [\Value{(\tilde{t}_1,\tilde{\sigma})} = \tilde{v}_1 \land \Failing{(\tilde{t}_2,\tilde{\sigma}_2')}]

\newrule{SH-Next}
  {\tilde{t}_1,\tilde{\sigma} \handle{} \overline{\tilde{t}_1',\tilde{\sigma}_1',\tilde{i},\phi_1} \Quad
  \tilde{e}_2\ \tilde{v}_1,\tilde{\sigma} \tilde{\normalise} \overline{\tilde{t}_2,\tilde{\sigma}_2',\phi_2}}
  {\tilde{t}_1 \Next \tilde{e}_2,\tilde{\sigma} \handle{} {\overline{\tilde{t}_1' \Next \tilde{e}_2,\tilde{\sigma}_1',\tilde{i},\phi_1}\cup\overline{\tilde{t}_2,\tilde{\sigma}_2',\Continue,\phi_2}}}
  [\Value{(\tilde{t}_1,\tilde{\sigma})} = \tilde{v}_1 \land \neg\Failing{(\tilde{t}_2,\tilde{\sigma}')}]


\newrule{SH-And}
  {\tilde{t}_1,\tilde{\sigma} \handle{} \overline{\tilde{t}_1',\tilde{\sigma}_1',\tilde{i}_1,\phi_1} \Quad
   \tilde{t}_2,\tilde{\sigma} \handle{} \overline{\tilde{t}_2',\tilde{\sigma}_2',\tilde{i}_2,\phi_2}}
  {\tilde{t}_1 \And \tilde{t}_2,\tilde{\sigma} \handle{} {\overline{\tilde{t}_1' \And \tilde{t}_2,\tilde{\sigma}_1',\First \tilde{i}_1,\phi_1}\cup \overline{\tilde{t}_1 \And \tilde{t}_2',\tilde{\sigma}_2'',\Second \tilde{i}_2,\phi_2}}}

\newrule{SH-Or}
  {\tilde{t}_1,\tilde{\sigma} \handle{} \overline{\tilde{t}_1',\tilde{\sigma}_1',\tilde{i}_1,\phi_1}\Quad
  \tilde{t}_2,\tilde{\sigma} \handle{} \overline{\tilde{t}_2',\tilde{\sigma}_2',\tilde{i}_2,\phi_2}}
  {\tilde{t}_1 \Or \tilde{t}_2,\tilde{\sigma} \handle{} {\overline{\tilde{t}_1' \Or \tilde{t}_2,\tilde{\sigma}_1',\First \tilde{i}_1,\phi_1}\cup\overline{\tilde{t}_1 \Or \tilde{t}_2',\tilde{\sigma}_2',\Second \tilde{i}_2,\phi_2}}}


\newrule{SH-PickLeft}
  {\tilde{e}_1,\tilde{\sigma}\tilde{\normalise} \overline{\tilde{t}_1,\tilde{\sigma}_1,\phi_1} \Quad
   \tilde{e}_2,\tilde{\sigma} \tilde{\normalise} \overline{\tilde{t}_2,\tilde{\sigma}_2,\phi_2}}
  {\tilde{e}_1 \Xor \tilde{e}_2,\tilde{\sigma} \handle{} \tilde{t}_1,\tilde{\sigma}_1,\Left,\phi_1}
  [\neg\Failing(\tilde{t}_1,\tilde{\sigma}_1) \land \Failing(\tilde{t}_2,\tilde{\sigma}_2)]

\newrule{SH-PickRight}
  {\tilde{e}_1,\tilde{\sigma} \tilde{\normalise} \overline{\tilde{t}_1,\tilde{\sigma}_1,\phi_1} \Quad
   \tilde{e}_2,\tilde{\sigma} \tilde{\normalise} \overline{\tilde{t}_2,\tilde{\sigma}_2,\phi_2}}
  {\tilde{e}_1 \Xor \tilde{e}_2,\tilde{\sigma} \handle{} \tilde{t}_2,\tilde{\sigma}_2,\Right,\phi_2}
  [\Failing(\tilde{t}_1,\tilde{\sigma}_1) \land \neg\Failing(\tilde{t}_2,\tilde{\sigma}_2)]

\newrule{SH-Pick}
  {\tilde{e}_1,\tilde{\sigma} \normalise \overline{\tilde{t}_1,\tilde{\sigma}_1,\phi_1} \Quad
   \tilde{e}_2,\tilde{\sigma} \normalise \overline{\tilde{t}_2,\tilde{\sigma}_2,\phi_2}}
  {\tilde{e}_1 \Xor \tilde{e}_2,\tilde{\sigma} \handle{} {\overline{\tilde{t}_1,\tilde{\sigma}_1,\Left,\phi_1}\cup\overline{\tilde{t}_2,\tilde{\sigma}_2,\Right,\phi_2}}}
  [\neg\Failing(\tilde{t}_1,\tilde{\sigma}_1) \land \neg\Failing(\tilde{t}_2,\tilde{\sigma}_2)]


%% Driving %%


\newmacro{RelationSI}
  {\tilde{t},\tilde{\sigma} \drive{} \overline{\tilde{t}',\tilde{\sigma}',\tilde{i},\phi}}


\newrule{SI-Handle}
  {\tilde{t},\tilde{\sigma} \handle{} \overline{\tilde{t}',\tilde{\sigma}',{\tilde{i},\phi_1}} \Quad
   \tilde{t}',\tilde{\sigma}' \tilde{\normalise} \overline{\tilde{t}'',\tilde{\sigma}'',\phi_2}}
  {\tilde{t},\tilde{\sigma} \drive{} \overline{\tilde{t}'',\tilde{\sigma}'',\tilde{i},\phi_1 \land \phi_2}}

\input{concrete-rules}
\input{grammars}
\input{maps}


\usepackage{graphicx}
% Used for displaying a sample figure. If possible, figure files should
% be included in EPS format.
%
% If you use the hyperref package, please uncomment the following line
% to display URLs in blue roman font according to Springer's eBook style:
% \renewcommand\UrlFont{\color{blue}\rmfamily}

\begin{document}
%
\title{End user feedback using symbolic execution}
%
%\titlerunning{Abbreviated paper title}
% If the paper title is too long for the running head, you can set
% an abbreviated paper title here
%
\author{Nico Naus\inst{1}\and
Tim Steenvoorden\inst{2}}
%
\authorrunning{Naus and Steenvoorden}
% First names are abbreviated in the running head.
% If there are more than two authors, 'et al.' is used.
%
\institute{Utrecht University, Princetonplein 5, Utrecht, The Netherlands \email{n.naus@uu.nl}\and
Radboud University, Nijmegen, The Netherlands
\email{????????}}
%
\maketitle              % typeset the header of the contribution
%
\begin{abstract}
  % !TEX root=../main.tex

% Context
Software that models business workflows is omnipresent in today's society.
These systems coordinate collaboration in hospitals, companies, and military institutions.
% Inquiry
Unfortunately, workflow systems may obfuscate the influence of current user actions on the desired end result.
In order to make the right decision, users need to oversee the full process and all information available,
both of which are usually buried in the system.
%spend one sentence introducing TOP
% Approach
We have developed a way to automatically generate next step hints for task oriented programs.
Task oriented programming provides programmers with an abstraction over workflow software, while still being expressive enough to describe real world collaboration.
By leveraging symbolic execution, we can calculate these hints without modification of the original program.
% Knowledge
To our knowledge, this is the first time that symbolic execution is used to automatically generate next step hints for end users.
% Grounding
We prove the generated hints to be sound and complete,
and also demonstrate that the symbolic execution semantics we employ is correct for sequential input.
In addition, we have developed a Haskell implementation of our automatic next step hint generation system.
% Importance
By providing next step hints, the chance of human error is reduced, while still allowing end users to intervene if required.
The overall performance is raised, since the quality of decisions will improve.

% Context: What is the broad context of the work? What is the importance of the general research area?
% Inquiry: What problem or question does the paper address? How has this problem or question been addressed by others (if at all)?
% Approach: What was done that unveiled new knowledge?
% Knowledge: What new facts were uncovered? If the research was not results oriented, what new capabilities are enabled by the work?
% Grounding: What argument, feasibility proof, artifacts, or results and evaluation support this work?
% Importance: Why does this work matter?

\keywords{Task-oriented programming \and Next step hint generation \and Symbolic execution.}

\end{abstract}
%
%
%
% !TEX root=../main.tex

\section{Introduction}
\label{sec:intro}

%There are many workflow systems
Software that supports people working together is present in most workplaces nowadays.
Its aim is to automate business and workflow processes, in order to simplify processes, to improve service or to contain cost.
In settings like hospitals, first responders and military operations, these systems could even prevent the loss of lives.

%they are a problem because

Automation and digitalisation of workflows and business processes comes at a cost.
Using them requires workers to undergo training, and often relies on their experience and expertise.

%our approach overcomes this because
To overcome these drawbacks, we propose to integrate a next step hint assistive system into the workflow software.
By combining previous research on symbolic execution for Task Oriented Programming~\cite{Naus2019} and end-user feedback systems for rule based problems~\cite{DBLP:conf/sfp/NausJ16},
we were able to develop a next step hint end-user feedback system for the Task Oriented Programming langugage \TOPHAT~\cite{Steenvoorden2019}.


\todo{Specify a bit more the scope of the paper.}
\todo{Introduce Assistive name.}


\subsection{Contributions}

This paper makes the following contributions.

\begin{itemize}
  \item We present an end user next step feedback system for \TOPHAT.
  \item We prove soundness and completeness of next step hints generated by this system.
  \item We present an implementation of the user feedback system in Haskell.
\end{itemize}


\subsection{Structure}

\cref{sec:tophat} first introduces the Task-Oriented Programming (\TOP) paradigm and the Task-Oriented Programming language \TOPHAT.
In \cref{sec:symbolic}, symbolic execution for \TOPHAT is briefly introduced.
\cref{sec:examples} then goes on to illustrate how assistive \TOPHAT should work, by giving an example.
\cref{sec:assistive} follows with a description of assistive \TOPHAT.
In \cref{sec:properties} soundness and completeness of the assistive system are shown.
An implementation of the system in Haskell is described in \cref{sec:implementation}.
\cref{sec:relatedwork} gives an overview of related work, and finally \cref{sec:conclusion} concludes.

% !TEX root=../main.tex

\section{TopHat}
\label{sec:tophat}

\usemacro{O-Value}

% !TEX root=../main.tex

\section{Examples}
\label{sec:examples}

\subsection{Dining Computer Scientists Problem}

The dining philosophers problem is a classic concurrency problem in computer science.
A number of philosophers sit at a round table with a meal in front of them.
In between the plates lies one fork.
In order to eat their meal, each philosopher has to acquire two forks.
This, of course, means that all philosophers cannot eat at the same time, since there are not enough forks.
Deadlock can occur when all philosophers pick up the fork to their right (or left).
Then, everybody has one fork.
Therefore, each philosopher cannot start his meal, and is also not allowed to put his fork back on the table.

\lstset{emph={this, that, name, left, right}}
\begin{TASK}[
    float=ht,
    numbers=right,
    caption={Dining philosophers problem with three computer scientists.},
    label=lst:dining]
  let fork0 = ref True in $\label{lst:phil:fork0}$
  let fork1 = ref True in $\label{lst:phil:fork1}$
  let fork2 = ref True in $\label{lst:phil:fork2}$
  let pickup = \ this. \ that. $\label{lst:phil:this}$
    if !this $\label{lst:phil:deref}$
      then this := False >>? \ <<>>. $\label{lst:phil:mark-used}$
        if !that then this := True else fail $\label{lst:phil:that}$
      else fail in
  let scientist = \ name. \ left. \ right. $\label{lst:phil:scientist}$
    pickup left right <?> pickup right left in $\label{lst:phil:pick}$
  scientist "Alan Turing" fork0 fork1 <&> $\label{lst:phil:scientist0}$
  scientist "Grace Hopper" fork1 fork2 <&> $\label{lst:phil:scientist1}$
  scientist "Ada Lovelace" fork2 fork0 >>= \ _ . $\label{lst:phil:scientist2}$
    edit "Full bellies"
\end{TASK}

We look at dining computer scientists instead.
\cref{lst:dining} lists an implementation in \TOPHAT for this problem, with three computer scientists.
The forks are represented by references containing Booleans (\cref{lst:phil:fork0,lst:phil:fork1,lst:phil:fork2}).
The value $\True$ indicates that the fork is available,
$\False$ indicates that the fork is being used.

Picking up a fork is only possible when the fork is available,
i.e the reading the reference results is $\True$ (\cref{lst:phil:this}).
This fork is then marked as being used (\cref{lst:phil:mark-used}).
The use of references ensures that the neighbouring scientist cannot pick up this fork: this choice will be disabled.
After that, one can press continue if the second fork is also available (\cref{lst:phil:that}).
For the sake of simplicity, one returns the first fork, rather than setting the second fork to $\False$, and then setting both to $\True$ again.

Each computer scientist takes as arguments a name and references to the two forks that he or she can reach (\cref{lst:phil:scientist}).
They have a choice to take either the left or the right fork.
This is represented with an user choice ($\Xor$, \cref{lst:phil:pick}).
The last lines instantiate three computer scientists sitting next to each other (\cref{lst:phil:scientist0,lst:phil:scientist1,lst:phil:scientist2}).
In \TOP terms, this means they collaborate in parallel ($\And$) while eating their dinner while sharing some resources,
in this cease $\lbl{fork0}$, $\lbl{fork1}$, and $\lbl{fork2}$.

Note that the events of picking up a fork are performed sequentially.
That is, when one computer scientist decides to pick up his right fork, we will handle that event first.
After that, we will handle the choices from the other scientists.
So, the order of the events is explicitly determined by the scientists themselves.
% This is regardless of choices interfering or not.

\todo{Write analysis.}
In \cref{sec:todo} we will analyse this example.
Our goal is to hint each scientist which choice to make, in order to reach the common goal of full bellies.
When the scientists follow these hints, no deadlock will occur.


\subsection{Tax subsidy request}

The example program listed in this section is taken from our previous work on symbolic execution for \TOPHAT~\cite{Steenvoorden2019}.
It models a simplified tax subsidy application process for citizens who have installed solar panels.

A subsidy is only given under the following conditions.
\begin{itemize}
\item The roofing company has confirmed that they installed solar panels for the citizen.
\item The tax officer has approved the request.
\item The tax officer can only approve the request if the roofing company has confirmed, and the request is filed within one year of the invoice date.
\item The amount of the granted subsidy is maximal 600 EUR.
\end{itemize}

The listing below gives the \TOPHAT code for this example.

\lstset{emph={invoiceDate,today,confirmed,invoiceAmount,approved}}
\begin{TASK}[
    float=ht,
    numbers=right,
    caption={Subsidy request and approval workflow at the Dutch tax office.},
    label=lst:tax]
  let getCurrentDate = enter Date in
  let provideDocuments = enter Amount <&> enter Date in
  let companyConfirm = edit True <?> edit False in
  let officerApprove = \ invoiceDate. \ today. \ confirmed.
    edit False <?> if (today - invoiceDate < 365 /\ confirmed) $\label{lst:tax:officer-approve-def}$ then edit True else fail in
  getCurrentDate >>= \ today.$\label{lst:tax:citizen-info}$
  provideDocuments <&> companyConfirm >>= \ <<<<invoiceAmount, invoiceDate>>, confirmed>>. $\label{lst:tax:documents-and-company-confirm}$
  officerApprove invoiceDate today confirmed >>= \ approved.$\label{lst:tax:officer-approve}$
  let subsidyAmount = if approved then min 600 (invoiceAmount / 10) else 0 in
    edit <<subsidyAmount, approved, confirmed, invoiceDate, today>>$\label{lst:tax:result}$
\end{TASK}

In previous work, we have shown that this code indeed adheres to the requirements listed above.

Instead of assisting the developer, by proving the program correct, we would now like to support the end user that is requesting a subsidy.

The end user wants the outcome of this program to be a subsidy amount larger than zero.

Running the simulation function will result in the following set:

\begin{table}[ht]
  \centering
  \begin{tabular}{LLL}
    \toprule
    \text{Symbolic value} & \text{Symbolic input} & \text{Path condition} \\
    \midrule
    \tuple{\min\ 600\ (s_{\id{a}}/10),  \True, \True, s_{\id{i}}, s_{\id{t}}} & [s_{\id{t}},\First \First s_{\id{a}}, \First \Second s_{\id{i}}, \Second \First, \Second] & s_{\id{t}}-s_{\id{i}}<365 \\
    \tuple{\min\ 600\ (s_{\id{a}}/10),  \True, \True, s_{\id{i}}, s_{\id{t}}} & [s_{\id{t}},\First \Second s_{\id{i}}, \First \First s_{\id{a}}, \Second \First, \Second] & s_{\id{t}}-s_{\id{i}}<365 \\
    \tuple{\min\ 600\ (s_{\id{a}}/10),  \True, \True, s_{\id{i}}, s_{\id{t}}} & [s_{\id{t}},\Second \First, \First \First s_{\id{a}}, \First \Second s_{\id{i}}, \Second] & s_{\id{t}}-s_{\id{i}}<365 \\
    \tuple{\min\ 600\ (s_{\id{a}}/10),  \True, \True, s_{\id{i}}, s_{\id{t}}} & [s_{\id{t}},\Second \First, \First \Second s_{\id{i}}, \First \First s_{\id{a}}, \Second] & s_{\id{t}}-s_{\id{i}}<365 \\
    \tuple{\min\ 600\ (s_{\id{a}}/10),  \True, \True, s_{\id{i}}, s_{\id{t}}} & [s_{\id{t}},\First \Second s_{\id{i}}, \Second \First, \First \First s_{\id{a}}, \Second] & s_{\id{t}}-s_{\id{i}}<365 \\
    \tuple{\min\ 600\ (s_{\id{a}}/10),  \True, \True, s_{\id{i}}, s_{\id{t}}} & [s_{\id{t}},\First \First s_{\id{a}}, \Second \First, \First \Second s_{\id{i}}, \Second] & s_{\id{t}}-s_{\id{i}}<365 \\
    \tuple{                        0,  \False, \True, s_{\id{i}}, s_{\id{t}}} & [s_{\id{t}},\First \First s_{\id{a}}, \First \Second s_{\id{i}}, \Second \First, \First]  & \True \\
    \tuple{                        0,  \False, \True, s_{\id{i}}, s_{\id{t}}} & [s_{\id{t}},\First \Second s_{\id{i}}, \First \First s_{\id{a}}, \Second \First, \First]  & \True \\
    \tuple{                        0,  \False, \True, s_{\id{i}}, s_{\id{t}}} & [s_{\id{t}},\Second \First, \First \First s_{\id{a}}, \First \Second s_{\id{i}}, \First]  & \True \\
    \tuple{                        0,  \False, \True, s_{\id{i}}, s_{\id{t}}} & [s_{\id{t}},\Second \First, \First \Second s_{\id{i}}, \First \First s_{\id{a}}, \First]  & \True \\
    \tuple{                        0,  \False, \True, s_{\id{i}}, s_{\id{t}}} & [s_{\id{t}},\First \Second s_{\id{i}}, \Second \First, \First \First s_{\id{a}}, \First]  & \True \\
    \tuple{                        0,  \False, \True, s_{\id{i}}, s_{\id{t}}} & [s_{\id{t}},\First \First s_{\id{a}}, \Second \First, \First \Second s_{\id{i}}, \First]  & \True \\
    \tuple{                        0,  \False,\False, s_{\id{i}}, s_{\id{t}}} & [s_{\id{t}},\First \First s_{\id{a}}, \First \Second s_{\id{i}}, \Second, \First]  & \True \\
    \tuple{                        0,  \False,\False, s_{\id{i}}, s_{\id{t}}} & [s_{\id{t}},\First \Second s_{\id{i}}, \First \First s_{\id{a}}, \Second, \First]  & \True \\
    \tuple{                        0,  \False,\False, s_{\id{i}}, s_{\id{t}}} & [s_{\id{t}},\Second \Second,\First \First s_{\id{a}}, \First \Second s_{\id{i}}, \First]  & \True \\
    \tuple{                        0,  \False,\False, s_{\id{i}}, s_{\id{t}}} & [s_{\id{t}},\Second, \First \Second s_{\id{i}}, \First \First s_{\id{a}}, \First]  & \True \\
    \tuple{                        0,  \False,\False, s_{\id{i}}, s_{\id{t}}} & [s_{\id{t}},\First \Second s_{\id{i}}, \Second, \First \First s_{\id{a}}, \First]  & \True \\
    \tuple{                        0,  \False,\False, s_{\id{i}}, s_{\id{t}}} & [s_{\id{t}},\First \First s_{\id{a}}, \Second, \First \Second s_{\id{i}}, \First]  & \True \\
    \bottomrule
  \end{tabular}
  \begin{tabular}{LLL}
    \toprule
    \text{Symbolic value} & \text{Symbolic input} & \text{Path condition} \\
    \midrule
    \tuple{\min\ 600\ (s_{\id{a}}/10),  \True, \True, s_{\id{i}}, s_{\id{t}}} & [s_{\id{t}},\First \First s_{\id{a}}, \First \Second s_{\id{i}}, \Second \First, \Second] & s_{\id{t}}-s_{\id{i}}<365 \\
    \tuple{\min\ 600\ (s_{\id{a}}/10),  \True, \True, s_{\id{i}}, s_{\id{t}}} & [s_{\id{t}},\First \Second s_{\id{i}}, \First \First s_{\id{a}}, \Second \First, \Second] & s_{\id{t}}-s_{\id{i}}<365 \\
    \tuple{\min\ 600\ (s_{\id{a}}/10),  \True, \True, s_{\id{i}}, s_{\id{t}}} & [s_{\id{t}},\Second \First, \First \First s_{\id{a}}, \First \Second s_{\id{i}}, \Second] & s_{\id{t}}-s_{\id{i}}<365 \\
    \tuple{\min\ 600\ (s_{\id{a}}/10),  \True, \True, s_{\id{i}}, s_{\id{t}}} & [s_{\id{t}},\Second \First, \First \Second s_{\id{i}}, \First \First s_{\id{a}}, \Second] & s_{\id{t}}-s_{\id{i}}<365 \\
    \tuple{\min\ 600\ (s_{\id{a}}/10),  \True, \True, s_{\id{i}}, s_{\id{t}}} & [s_{\id{t}},\First \Second s_{\id{i}}, \Second \First, \First \First s_{\id{a}}, \Second] & s_{\id{t}}-s_{\id{i}}<365 \\
    \tuple{\min\ 600\ (s_{\id{a}}/10),  \True, \True, s_{\id{i}}, s_{\id{t}}} & [s_{\id{t}},\First \First s_{\id{a}}, \Second \First, \First \Second s_{\id{i}}, \Second] & s_{\id{t}}-s_{\id{i}}<365 \\
    \bottomrule
  \end{tabular}
  \caption{}
  \label{}
\end{table}

As a goal we set that the subsidy amount should be larger than zero. This leaves us with the following cases.

From the end states we can conclude that invoiceAmount should be larger than zero in order for the result to be larger than zero. This results in an additional constraint.

In the end, we can use the above to give hints to the user.

First, a date must be entered. Then the user can choose to ender an amount larger than zero, a date that is within 365 from now, or a "RL".
etc.

% !TEX root=../main.tex

\section{Symbolic TopHat}
\label{sec:symbolic}

\begin{figure*}[t]
  \begin{function}
    \signature{\Simulate : \mathrm{Tasks} \times \mathrm{States} \times [\mathrm{Inputs}]  \times \mathrm{Predicates}
      \rightarrow \powerset{\mathrm{Values} \times [\mathrm{Inputs}] \times \mathrm{Predicates}}} \\
    \Simulate(t, \sigma, I, \phi) &=&
      \bigcup \set{ \Simulate'(\True, t, I, \phi, t', \sigma', i', \phi') \mid t, \sigma \drive{} t', \sigma', i', \phi' } \\
    \addlinespace
    \signature{\Simulate' : \mathrm{Booleans} \times \mathrm{Tasks} \times [\mathrm{Inputs}] \times \mathrm{Predicates} \times \mathrm{Tasks} \times \mathrm{States} \times \mathrm{Inputs} \times \mathrm{Predicates}
      \rightarrow \powerset{\mathrm{Values} \times [\mathrm{Inputs}] \times \mathrm{Predicates}}} \\
    \Simulate'(\Again, t, I, \phi, t', i', \sigma', \phi') &=& \\
      \multicolumn{3}{L}{ \left\{
        \begin{array}{lr@{\ }c@{\ }l@{\ }c@{\ }l@{\ }c@{\ }r}
          \nothing                                                                                              & \neg\Sat(\phi'\land\phi) &&&&&&\\
          \set{(v, I\oplus[i'], \phi\land\phi')}                                                                & \Sat(\phi'\land\phi)     &\land& \Value(t',\sigma') = v &&&& \\
          \Simulate(t', \sigma', I\oplus[i'], \phi\land\phi')                                           & \Sat(\phi'\land\phi)     &\land& \Value(t',\sigma') = \bot &\land& t' \neq t &&\\
          \bigcup \set{\Simulate'(\False, t', I\oplus[i'], \phi\land\phi', t'', \sigma'', i'', \phi'')
            \mid  t',\sigma' \drive{} t'', \sigma'', i'', \phi''}                                               & \Sat(\phi'\land\phi)     &\land& \Value(t',\sigma') = \bot &\land& t' = t    &\land& \Again\\
          \nothing                                                                                              & \Sat(\phi'\land\phi)     &\land& \Value(t',\sigma') = \bot &\land& t' = t    &\land& \neg\Again
        \end{array}
        \right.}
  \end{function}
  \caption{Simulation function definition.}
  \label{fig:simulate}
\end{figure*}

% !TEX root=../main.tex

\section{Assistive TopHat}
\label{sec:assistive}

Our goal is to provide next step hints, that lead users closer to their goal.
Goals are formulated in therms of the result value of a task.

A task $t$ is considered done, as soon as it has an observable value $v$.
To observe this value, the function $\Value$ as defined in section~\ref{sec:tophat} is used.

After obtaining the set of all symbolic executions, we need to filter out executions that do not fulfil the goal condition, or those where the goal is in conflict with the path condition.

Then, we take the first input so that we can return this to the user as a hint.

\begin{figure*}[t]
  \begin{function}
    \signature{firsts : \mathrm{Tasks} \times \mathrm{States} \times \mathrm{Goal}
      \rightarrow \powerset{\mathrm{Inputs} \times \mathrm{Predicates}}} \\
    firsts\ (t,\sigma, g) &=& [(i,\Phi')\mid (\tilde{v},\tilde{i}:\tilde{is},\Phi)\leftarrow t,\sigma\drive{}^*\ ,\ \Phi'=\Phi\land g \tilde{v}\ ,\  \Sat(\Phi')]
  \end{function}
  \caption{Firsts function definition.}
  \label{fig:firsts}
\end{figure*}

% !TEX root=../main.tex

\section{Properties}
\label{sec:properties}

In this section, we want to validate our approach by proving correctness.
For the hints function, which forms the heart of \ASTOPHAT, we want to prove that its results are both sound and complete.
Since the hints function relies on \STOPHAT,
and more specifically, the updated definition of the simulate relation,
we first prove correctness of simulate.

\subsection{Correctness of simulate}

The symbolic execution semantics is correct when all symbolic runs relate to a concrete run,
and the other way around, when all concrete runs are contained in the set of all symbolic executions.
These properties are, respectively, soundness and completeness.

The simulation applies symbolic interaction multiple times.
In order to prove certain properties with respect to the concrete semantics,
we need a concrete analog of simulation.
Therefore, we define \emph{execution}, which applies concrete interaction multiple times.

\begin{definition}[Execution ($\execute{}$)]
  \label{def:execution}
  Let $t$ be a concrete task, $\sigma$ a concrete state, and $I = i_1 \cdots i_n$ a list of $n$ concrete inputs.
  We define the execution relation
  \begin{equation*}
    t, \sigma \execute{I} v
  \end{equation*}
  to be the value of task $t$ after performing concrete interaction for each input $i$ in $I$:
  \begin{equation*}
    t, \sigma
      \interact{i_1} t_1, \sigma_1
      \interact{i_2} \cdots
      \interact{i_n} t_n, \sigma_n
  \end{equation*}
  where
  \begin{itemize}
    \item $v$ is the value of $t_n$: $\Value(t_n, \sigma_n) = v$; and
    \item all tasks before $t_n$ do not have a value: $\Value(t_{i<n},\sigma_{i<n}) = \bot$.
  \end{itemize}
\end{definition}

Using execution, we can state soundness and completeness for simulation as follows.

\begin{lemma}[Soundness of simulate]
  \label{lem:soundsimulate}
  For all tasks $t$ and states $\sigma$
  such that $t,\sigma\simulate\overline{\tilde{v},\tilde{I},\Phi}$
  where $\tilde{I} = \simi_0 \cdots \simi_n$,
  for each triple of results $\<\tilde{v},\tilde{I},\Phi\>$
  there exists a concrete input $I$ with the same length as the symbolic input $\tilde{I}$
  such that $t,\sigma\execute{I}v$
  with $[s_i\mapsto c_i]\tilde{v}=v$ and $[s_i\mapsto c_i]\Phi$
  where $\SymOf(\simi_i)=s_i$ and $\ValOf(i_i)=c_i$.
\end{lemma}

\begin{lemma}[Completeness of simulate]
  \label{lem:completesimulate}
  For all tasks $t$, states $\sigma$, and lists of input $I$
  such that $t,\sigma\execute{I}v$,
  there exists a symbolic value $\tilde{v}$ and a symbolic input $\tilde{I}$ with the same length as $I$,
  such that $(\tilde{v},\tilde{I},\Phi)\in t,\sigma\simulate$,
  with $\tilde{i_i}\sim i_i$, $[s_i\mapsto c_i]\tilde{v}=v$ and $[s_i\mapsto c_i]\Phi$,
  where $\SymOf(\simi_i)=s_i$ and $\ValOf(i_i)=c_i$.
\end{lemma}

Where $\simi\sim i$ is defined as follows.

\begin{definition}[Input simulation]
  A symbolic input $\simi$ simulates a concrete input $i$ denoted as $\simi\sim i$ in the following cases.\\
  $s\sim a$, where $s$ is a symbol and $a$ a concrete action.\\
  $\simi\sim i\implies \First \simi \sim \First i$\\
  $\simi\sim i\implies \Second \simi \sim \Second i$
\end{definition}

And $\SymOf(\simi)=s$ and $\ValOf(i)=c$ are defined as follows.\\
\\
\noindent
\begin{minipage}[c]{0.45\textwidth}
  \begin{definition}[Value from input]\\
    \begin{function}
      \signature{\ValOf : \mathrm{Inputs} \to \mathrm{Values}} \\
      \ValOf(\First i)    &=& \ValOf(i) \\
      \ValOf(\Second i)   &=& \ValOf(i) \\
      \ValOf(c)           &=& c \\
      \ValOf(\_)          &=& \bot
    \end{function}
  \end{definition}
\end{minipage}
\begin{minipage}[c]{0.55\textwidth}
  \begin{definition}[Symbol from input]\\
    \begin{function}
      \signature{\SymOf : \mathrm{Symbolic\ Inputs} \to \mathrm{Symbolic\ Values}} \\
      \SymOf(\First i)    &=& \SymOf(i) \\
      \SymOf(\Second i)   &=& \SymOf(i) \\
      \SymOf(s)           &=& s \\
      \SymOf(\_)          &=& \bot
    \end{function}
  \end{definition}
\end{minipage}\\

\begin{figure}[t]
  \tikzstyle{drive} = [decoration={markings,mark=at position
     1 with {\arrow[semithick]{angle 60}}},
     double distance=1.4pt, shorten >= 2.3pt,
     preaction = {decorate},
     postaction = {draw,line width=1pt, white,shorten >= 2.3pt}]
  \tikzstyle{sdrive} = [->,decorate, decoration={coil,aspect=0,amplitude=.5mm},
     double distance=1.4pt, shorten >= 2.3pt,]

\begin{tikzpicture}[
            > = stealth, % arrow head style
            shorten > = 1pt, % don't touch arrow head to node
            auto,
            node distance = 3cm, % distance between nodes
            semithick, % line style
        ]


        % j = 0
        \node (0)  {$t,\sigma$};
        \node (l0) [right of=0,xshift=1.4cm] {$t,\sigma\Consistent_{[\ ]} t,\sigma,\True$};

        % j = 1
        \node (c1) [below left of=0] {$t_1,\sigma_1$};
        \node (s1) [below right of=0] {$\tilde{t}_1,\tilde{\sigma}_1,\tilde{i}_1,\phi_1$};
        \node (l1) [right of=s1,text width=4cm] {\begin{tabular}{l}
        $t_1,\sigma_1\Consistent_{[s_1\mapsto c_1]}\tilde{t_1},\tilde{\sigma_1},\phi_1$\\
                                  $\Sat(\phi_1)$\\
                                  $\Value(t_1,\sigma_1)=\bot$\end{tabular}};

        %
        \node (cc) [below of=c1,yshift=1.5cm] {\vdots};
        \node (ss) [below of=s1,yshift=1.5cm] {\vdots};

        % j = k
        \node (ck) [below of=cc,yshift=1.5cm] {$t_k,\sigma_k$};
        \node (sk) [below of=ss,yshift=1.5cm] {$\tilde{t}_k,\tilde{\sigma}_k,\tilde{i}_k\phi_k$};
        \node (lk) [right of=sk,text width=4cm] {\begin{tabular}{l}
        $t_k,\sigma_k\Consistent_{[s_1\mapsto c_1,\cdots,s_k\mapsto c_k]}\tilde{t_k},\tilde{\sigma_k},\phi_1\land\cdots\land\phi_k$\\
        $\Sat(\phi_1\land\cdots\land\phi_k)$\\
        $\Value(t_k,\sigma_k)=\bot$\end{tabular}};

        %
        \node (ccc) [below of=ck,yshift=1.5cm] {\vdots};
        \node (sss) [below of=sk,yshift=1.5cm] {\vdots};

        \node (cn) [below of=ccc,yshift=1.5cm] {$t_n,\sigma_n$};
        \node (sn) [below of=sss,yshift=1.5cm] {$\tilde{t}_n,\tilde{\sigma}_n,\tilde{i}_n,\phi_n$};
        \node (l1) [right of=sn,text width=4cm] {\begin{tabular}{l}
        $t_n,\sigma_n\Consistent_{[s_1\mapsto c_1,\cdots,s_n\mapsto c_n]}\tilde{t_n},\tilde{\sigma_n},\phi_1\land\cdots\land\phi_n$\\
        $\Sat(\phi_1\land\cdots\land\phi_n)$\\
        $\Value(t_n,\sigma_n)=v$\Quad $\Value(\tilde{t}_n,\tilde{\sigma}_n)=\tilde{v}$\\
        $I=[i_1,\cdots,i_n]$\Quad $\tilde{I}=[\tilde{i}_1,\cdots,\tilde{i}_n]$\end{tabular}};

        \draw[drive] (0) to (c1);
        \node (0label) [below left of=0,yshift=1.3cm,xshift=0.8cm] {$i_1$};
        \draw[sdrive] (0) -- (s1);

        \draw[drive] (c1) to (cc);
        \draw[sdrive] (s1) -- (ss);

        \draw[drive] (cc) to (ck);
        \node (cclabel) [below left of=cc,yshift=1.5cm,xshift=1.8cm] {$i_k$};
        \draw[sdrive] (ss) -- (sk);

        \draw[drive] (ck) to (ccc);
        \draw[sdrive] (sk) -- (sss);

        \draw[drive] (ccc) to (cn);
        \node (cnlabel) [below left of=ccc,yshift=1.5cm,xshift=1.8cm] {$i_n$};
        \draw[sdrive] (sss) -- (sn);


        \path[dashed,->] ([yshift=.25em]c1.east) edge node {\cref{lem:completeinteracting}} ([yshift=.25em]s1.west);
        \path[dashed,->] ([yshift=-.25em]s1.west) edge node {\cref{lem:soundinteracting}} ([yshift=-.25em]c1.east);

        \path[dashed,->] ([yshift=.25em]ck.east) edge node {\cref{lem:completeinteracting}} ([yshift=.25em]sk.west);
        \path[dashed,->] ([yshift=-.25em]sk.west) edge node {\cref{lem:soundinteracting}} ([yshift=-.25em]ck.east);

        \path[dashed,->] ([yshift=.25em]cn.east) edge node {\cref{lem:completesimulate}} ([yshift=.25em]sn.west);
        \path[dashed,->] ([yshift=-.25em]sn.west) edge node {\cref{lem:soundsimulate}} ([yshift=-.25em]cn.east);


    \end{tikzpicture}
    \caption{Proof structure}
      \label{fig:proofstructure}
\end{figure}

Our strategy to prove these two lemma's is outlined in \cref{fig:proofstructure}.
At the top, we start out with any task $t$ and state $\sigma$.
The left side of the diagram is an overview of the evaluate function.
Inputs $i_1$ until $i_n$ are sequentially applied, until the task has an observable value.

On the right side, symbolic execution is performed.
One step of the symbolic interaction semantics is taken, which results in a symbolic task, state, input and a path condition.
Provided that the path condition holds, interaction is executed sequentially until the symbolic task has an observable symbolic value.

Proving soundness and completeness of simulation now comes down to relating the left and right side of the diagram.
From symbolic to concrete (right to left) is soundness, as stated in \cref{lem:soundsimulate}.
From concrete to symbolic (left to right) is completeness, as stated in \cref{lem:completesimulate}.

Since simulation and execution rely on the (symbolic) handling semantics,
we prove soundness and completeness of those semantics first.
Looking at \cref{fig:proofstructure}, there are two different settings in which the (symbolic) handling semantics are applied.
At the top, both symbolic and concrete execution start out with the same task and state.
But further down, the task and state differ for both semantics.
The task and state are related to each other however.
The symbolic semantics introduces symbols, the concrete semantics handles concrete values.
This relation is expressed by the consistence relation listed in \cref{def:consistence}.

\begin{definition}[Consistence relation $\Consistent$]
  \label{def:consistence}
  A concrete task $t$ and concrete state $\sigma$
  are considered to be consistent with a symbolic task $\tilde{t}$, symbolic state $\tilde{\sigma}$ and path condition $\Phi$
  under a certain mapping $M=[s_1\mapsto c_1,\cdots,s_n,\mapsto c_n]$, denoted as $t,\sigma \Consistent_M \tilde{t},\tilde{\sigma},\Phi$
  if and only if $M\tilde{t}=t$, $M\tilde{\sigma}=\sigma$ and $M\Phi$
\end{definition}

Now \cref{lem:soundinteracting} and \cref{lem:completeinteracting} express soundness and completeness of interacting respectively.

\begin{lemma}[Soundness of interacting]
  \label{lem:soundinteracting}
  For all concrete tasks $t$, concrete states $\sigma$, symbolic tasks $\tilde{t}$, symbolic states $\tilde{\sigma}$ path conditions $\Phi$ and mappings $M$,
  we have that $t,\sigma\Consistent_M\tilde{t},\tilde{\sigma},\Phi$ implies
  that for all pairs $(\tilde{t}',\tilde{\sigma}',\simi,\phi)$ in $\tilde{t},\tilde{\sigma}\siminteract\overline{\tilde{t}',\tilde{\sigma}',\simi,\phi}$,
  $\Sat(\Phi\land\phi)$ implies that there exists an input $i$ such that $\simi\sim i$,  $t,\sigma\interact{i}t',\sigma'$ and $t',\sigma' \Consistent_{M.[s\mapsto c]} \tilde{t}',\tilde{\sigma}',\Phi\land\phi$ where where $\SymOf(\simi)=s$ and $\ValOf(i)=c$.
\end{lemma}

\begin{lemma}[Completeness of interacting]
  \label{lem:completeinteracting}
  For all concrete tasks $t$, concrete states $\sigma$, symbolic tasks $\tilde{t}$, symbolic states $\tilde{\sigma}$ path conditions $\Phi$ and mappings $M$,
  we have that $t,\sigma\Consistent_{M}\tilde{t},\tilde{\sigma},\Phi$ implies
  that for all inputs $i$ such that $t,\sigma\interact{i}t',\sigma'$,
  there exists a symbolic input $\simi$, $\simi\sim i$ such that
  $\tilde{t},\tilde{\sigma}\siminteract\overline{\tilde{t}',\tilde{\sigma}',\simi,\phi}$, $\Sat(\Phi\land\phi)$ and $t',\sigma'\Consistent_{M.[s\mapsto c]}\tilde{t}',\tilde{\sigma}',\Phi\land\phi$ where where $\SymOf(\simi)=s$ and $\ValOf(i)=c$.
\end{lemma}

In other words, if a symbolic and concrete task and state are related, they will still be related after (symbolic) handling.
The top case, where both the symbolic and concrete semantics start out with the same task and state,
can be seen as a special case of the consistence relation.
Obviously a task and state are consistent with themselves, using the empty mapping and the path condition $\True$.

The full proof of all four lemma's is listed in the appendix online\footnote{https://github.com/timjs/assistive-tophat/raw/master/appendix.pdf}.


\subsection{Correctness of hints}

Now that soundness and completeness of simulate have been proven, we can prove that our hints function produces correct hints.
Intuitively, for a next step hint to be correct, it should adhere to the following requirements:
\begin{itemize}
  \item it leads to concrete input users can actually insert; and
  \item when users follow the hint, the end goal is still reachable.
\end{itemize}
Moreover, a set of next step hints is correct when:
\begin{itemize}
  \item each hint it contains is correct; and
  \item it covers all possible inputs that lead to the end goal.
\end{itemize}

We separate these requirements into two lemma's, namely soundness and completeness.

\begin{theorem}[Soundness of hints]
  \label{thm:soundhint}
  For all tasks $t$, states $\sigma$, and goals $g$,
  for every next step hint $\<\simi,\Phi\>$ in $\Hints(t,\sigma,g)$,
  there exists a sequence of concrete inputs $I$ and a concrete input $i$ such that $\simi\sim i$,
  $\Sat([s\mapsto c]\Phi)$, $t,\sigma\interact{i} t',\sigma'\execute{I} v$ and $g(v)$.
\end{theorem}

\begin{theorem}[Completeness of hints]
  \label{thm:completehint}
  For all tasks $t$, states $\sigma$, lists of input $i \cdot I$, and goals $g$,
  if $t,\sigma,\execute{i \cdot I} v$ and $g(v)$, then there exists a symbolic input $\simi$ and path condition $\Phi$
  such that $\<\simi,\Phi\> \in \Hints(t,\sigma,g)$ with $\simi\sim i$ and $\Sat\big([s\mapsto c]\Phi\big)$ with $\ValOf(i) = c$ and $\SymOf(\simi) = s$.
\end{theorem}

The proofs of these two threorems are quite straight forward.

\begin{proof}[\cref{thm:soundhint}]
  \cref{thm:soundhint} follows from the definition of $\Hints$ and \cref{lem:soundsimulate} as follows.

  The definition of $\Hints$ gives us that for every pair $\<\simi,\Phi\>$ produced by $\Hints$,
  there exists a triple $\<\tilde{v},\simi:\tilde{is},\Phi\>$ with $\Sat\big(\Phi \land g(\tilde{v})\big)$.
  Then by \cref{lem:soundsimulate} we have that there exists a sequence of concrete inputs $I$ such that
  $t,\sigma\execute{I}v$ and $g(v)$.
\end{proof}


\begin{proof}[\cref{thm:completehint}]
  In order to prove that $i$ is contained in $\Hints(t,\sigma,g)$, we need to show that $t,\sigma \simulate \<\tilde{v},\simi \cdot \tilde{I},\Phi\>$, with $\simi\sim i$ and $\Sat\big([s_0\mapsto c_0,\cdots,s_n\mapsto c_n] \land g(\tilde{v})\big)$, where $\ValOf(i_0)= c_0, \cdots, \ValOf(i_n) = c_n$ and $[c_0,\cdots,c_n]\in i \cdot I$ and $\SymOf(\simi_0) = s_0, \cdots, \SymOf(\simi_n) = s_n$.

  By \cref{lem:completesimulate}, we directly obtain that this indeed exists. Therefore we know that $\simi$ and $\Phi$ exist.
\end{proof}

% !TEX root=../main.tex

\section{Implementation}
\label{sec:implementation}


\begin{verbatim}
  firsts ::
  MonadTrack Text m => MonadSupply Nat m => MonadState Heap m => MonadZero m =>
  Val ('TyTask ('TyPrim t)) -> Goal t -> m ( Input, Pred 'TyPrimBool )
firsts t g = [ ( i, p' ) | ( v, i:_, p ) <- simulate t [] Yes, let p' = p :/\: g v, satisfiable p' ]

\end{verbatim}

% !TEX root=../main.tex

\section{Related work}
\label{sec:relatedwork}

In previous work, we have attempted to provide end-users with next step hints by viewing workflows as rule based problems~\cite{DBLP:conf/sfp/NausJ16}.
By abstracting over workflows, reasoning about them becomes simpler.
A standard search algorithm can be run to find a path to the desired goal state.
Two drawbacks of this approach however are that only very general hints can be given, that range over multiple steps, and that a programmer needs to augment existing workflows with extra information in order to convert it to a rule-based problem.

Stutterheim et al.~\cite{DBLP:conf/sfp/StutterheimPA14} have developed Tonic, a task visualiser for iTasks with limited path prediction capabilities.
The main goal is not to provide hints to end-users, but the system is able to handle the complete task language, and visualise the effects of user input on the progression of tasks.

In order to overcome the problems of our own previous research and the limited use of Tonic for end-user hints, we have combined symbolic execution, together with workflow modelling and next-step hint generation.
To our knowledge, this is the first work describing the combination of these techniques in this way.
The different components coming together in this paper have been studied extensively.
The following sections give an overview of the work done in those areas.

\subsection{Symbolic execution}


Symbolic execution \cite{King1975,Boyer1975} is typically being applied to imperative programming languages,
but in recent years it has been used for functional programming languages as well.
Ongoing work by Hallahan et al.~\cite{HallahanXP2017,DBLP:conf/pldi/HallahanXBJP19} aims to implement a symbolic execution engine for Haskell.
Giantios et al.~\cite{GiantsiosPS2017} use symbolic execution for a mix of concrete and symbolic testing of programs written in a subset of Core Erlang.
Their goal is to find executions that lead to a runtime error, either due to an assertion violation or an unhandled exception.
Chang et al.~\cite{ChangKT2018} present a symbolic execution engine for a typed lambda calculus with mutable state where only some language constructs recognise symbolic values.
They claim that their approach is easier to implement than full symbolic execution and simplifies the burden on the solver, while still considering all execution paths.

\subsection{Workflow modelling}

Workflow modelling have been studied extensively from different viewpoints.
Since many software exists that automates workflows, it is a research topic that potentially has a huge impact on society.

\emph{Workflow patterns} are regarded as special design patterns in software engineering.
Similar to the combinators in \TOP, they describe recurring patterns in workflow systems.
Van der Aalst et al.~\cite{journals/dpd/AalstHKB03} identify common patterns, and examine their availability in industry workflow frameworks.

\emph{Workflow Nets} allow for the modelling an analysis of business processes~\cite{DBLP:journals/jcsc/Aalst98}.
Worflow Nets are a subclass of Petri nets, and are therefore graphical in nature.
Research on Workflow Nets includes verification of models~\cite{DBLP:conf/apn/Aalst97} and complexity analysis~\cite{DBLP:journals/infsof/LassenA09}, just to name a few.


\emph{iTasks}~\cite{DBLP:conf/ppdp/PlasmeijerLMAK12} is an implementation of \TOP in the programming language Clean.
It differs from the above mentioned modelling techniques, since it is not graphical in nature.
iTasks supports higher order workflows, and leverages techniques from functional and generic programming.


\subsection{Automatic hint generation in intelligent tutoring systems}

The intelligent tutoring systems (ITS) research community is very large.
Work that is most relevant to our own is the research into automatic hint generation.
More traditional ITS rely heavily on experts to write dedicated hints for every specific case of an exercise.
Automatic hint generation attempts to overcome this burden by calculating a hint rather than having every case specified.

Heeren et al.~\cite{DBLP:journals/scp/HeerenJ14} develop a framework for so called domain reasoners that allow for automatic hint generation.
Feedback is calculated automatically from a high-level description of an exercise class.
Their approach is applicable to domains like logic, mathematics and linear algebra.
Paquette et al.~\cite{DBLP:conf/its/PaquetteLBM12} present a different automatic next-step hint ITS, that is used to provide hints to students in a programming exercise.

Based on the work mentioned above by Heeren et al., an ITS for Haskell exercises has been developed by Gerdes et al.~\cite{DBLP:journals/aiedu/GerdesHJB17}.
It tuns out that programming exercises is a popular area for Automatic hint generation.
Keuning et al.~\cite{DBLP:journals/jeric/KeuningJH19} have written an excellent literature study of this research area.

% !TEX root=../main.tex

\section{Conclusion}
\label{sec:conclusion}


\subsection{Future work}


- supporting loops?
- developing a full application that allows for user testing
- iTasks support?


%% Acknowledgments
  \input{sections/acknowledgements}

%
% ---- Bibliography ----
%
% BibTeX users should specify bibliography style 'splncs04'.
% References will then be sorted and formatted in the correct style.
%
\bibliographystyle{splncs04}
% \bibliography{mybibliography}
%
\bibliography{bibliography}
%% Appendix
\pagebreak \appendix

% !TEX root=../main.tex

\section{Soundness proofs}
\label{sec:soundnessproofs}

Restatement of Lemma:\\

\begin{lemma}[Soundness of simulate]

  For all tasks $t$, states $\sigma$,
  such that for all $(v,I,\phi)\in \Simulate (t,\sigma,[],\True)$,
  there exists a mapping $M = [s_0\mapsto c_0,\cdots,s_n\mapsto c_n]$
  such that $M\phi$ implies $\Evaluate (M t,M \sigma,M I) = M v$.
\end{lemma}

\begin{proof}
  \case{
    $\Value(t,\sigma)=v$\\
    In this case, we have $\Simulate(t,\sigma,[],\True)=\set{(v,I,\phi)}$.
    When we take $M$ to be the identity mapping, we obtain that
    $\Evaluate (M t,M\sigma,M []) = \Evaluate (t,\sigma,[]) = v$, by definition of $\Evaluate$.
    }

  \case{
    $\Value(t,\sigma)=\bot$\\
    In this case, the symbolic driving semantics is applied. $t,\sigma\drive{}\overline{t',\sigma',i',\phi'}$.
    To every element of the result $(t',\sigma',i',\phi')$, $\Simulate'$ is applied as follows.
    $\Simulate'(\True,t,t',\sigma',[]\oplus [i'],\True\land\phi')$
    \case{
      $\Sat(\True\land\phi')$\\
        \case{
          $\Value(t',\sigma')=v$\\
          In this case, $\Simulate'$ results in $\set{(v,[i'],\phi')}$.
          We need to show that there exists a mapping $M$ such that $\Evaluate (M t,M \sigma, M [i']) = M v$. This by definition of $\Evaluate$
          means that $t,\sigma\drive{M i'}M t',M\sigma'$
          with $\Value(M t',M\sigma')= M v$ must hold.
          By Lemma~\ref{lem:sounddrive}, we obtain that:\\
          For all tasks $t$, states $\sigma$ and mappings $M=[s_0\mapsto c_0,\cdots,s_n\mapsto c_n]$,
          such that $t,\sigma\drive{} \overline{t',\sigma',i,\phi}$
          we have $M\phi$ implies
          $t,M\sigma \xRightarrow[]{M i} t'',\sigma''$, $Mt' \equiv t''$ and $M\sigma' \equiv \sigma''$.\\
          Which gives us exactly what we need to prove.
        }
        \case{
          $\Value(t',\sigma')=\bot$ and $t\neq t'$.\\

        }
    }
  }
\end{proof}




\end{document}
