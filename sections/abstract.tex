% !TEX root=../main.tex

% Context
Software modelling business workflows are omnipresent in today's society.
They coordinate collaboration in hospitals, companies, and military institutions.
% Inquiry
These systems obfuscate the influence of current user actions on the desired end result.
In order to make the right decision, users need to oversee the full process and all information available,
both of which are usually buried deep down in the system.

% Approach
We have developed a way to automatically generate next step hints for task oriented programs.
By leveraging symbolic execution, we can calculate these hints without modification of the original program.
% Knowledge
To our knowledge this is the first time that symbolic execution is used to automatically generate next step hints for end users.
% Grounding
We prove the generated hints to be sound and complete,
and also demonstrate that the symbolic execution semantics we employ is correct for sequential input.
Additionally, we implemented next step hint generation for TopHat programs, a formal system for task oriented programming, in Haskell.

% Importance
Next step hints reduce the chance of human error, while still allowing end users to intervene if required.
The overall performance is raised, since the quality of decisions will improve.

% Context: What is the broad context of the work? What is the importance of the general research area?
% Inquiry: What problem or question does the paper address? How has this problem or question been addressed by others (if at all)?
% Approach: What was done that unveiled new knowledge?
% Knowledge: What new facts were uncovered? If the research was not results oriented, what new capabilities are enabled by the work?
% Grounding: What argument, feasibility proof, artifacts, or results and evaluation support this work?
% Importance: Why does this work matter?

\keywords{Task-oriented programming \and Next step hint generation \and Symbolic execution.}
