% !TEX root=../main.tex

\section{Generating hints for TopHat}
\label{sec:assistive}

Our goal is to provide next step hints, that lead users closer to their goal.
Goals are formulated in therms of the result value of a task.

A task $t$ is considered done, as soon as it has an observable value $v$.
To observe this value, the function $\Value$ as defined in section~\ref{sec:tophat} is used.

After obtaining the set of all symbolic executions, we need to filter out executions that do not fulfil the goal condition, or those where the goal is in conflict with the path condition.

Then, we take the first input so that we can return this to the user as a hint.

\begin{figure}
  \usemacro{O-Hints}
  \caption{Definition of next step hint function.}
  \label{}
\end{figure}


\subsection{Tax subsidy request}

Running the simulation function will result in the following set:

\begin{table}[ht]
  \centering
  \begin{tabular}{LLL}
    \toprule
    \text{Symbolic value} & \text{Symbolic input} & \text{Path condition} \\
    \midrule
    \tuple{\min\ 600\ (s_{\id{a}}/10),  \True, \True, s_{\id{i}}, s_{\id{t}}} & [s_{\id{t}},\First \First s_{\id{a}}, \First \Second s_{\id{i}}, \Second \First, \Second] & s_{\id{t}}-s_{\id{i}}<365 \\
    \tuple{\min\ 600\ (s_{\id{a}}/10),  \True, \True, s_{\id{i}}, s_{\id{t}}} & [s_{\id{t}},\First \Second s_{\id{i}}, \First \First s_{\id{a}}, \Second \First, \Second] & s_{\id{t}}-s_{\id{i}}<365 \\
    \tuple{\min\ 600\ (s_{\id{a}}/10),  \True, \True, s_{\id{i}}, s_{\id{t}}} & [s_{\id{t}},\Second \First, \First \First s_{\id{a}}, \First \Second s_{\id{i}}, \Second] & s_{\id{t}}-s_{\id{i}}<365 \\
    \tuple{\min\ 600\ (s_{\id{a}}/10),  \True, \True, s_{\id{i}}, s_{\id{t}}} & [s_{\id{t}},\Second \First, \First \Second s_{\id{i}}, \First \First s_{\id{a}}, \Second] & s_{\id{t}}-s_{\id{i}}<365 \\
    \tuple{\min\ 600\ (s_{\id{a}}/10),  \True, \True, s_{\id{i}}, s_{\id{t}}} & [s_{\id{t}},\First \Second s_{\id{i}}, \Second \First, \First \First s_{\id{a}}, \Second] & s_{\id{t}}-s_{\id{i}}<365 \\
    \tuple{\min\ 600\ (s_{\id{a}}/10),  \True, \True, s_{\id{i}}, s_{\id{t}}} & [s_{\id{t}},\First \First s_{\id{a}}, \Second \First, \First \Second s_{\id{i}}, \Second] & s_{\id{t}}-s_{\id{i}}<365 \\
    \tuple{                        0,  \False, \True, s_{\id{i}}, s_{\id{t}}} & [s_{\id{t}},\First \First s_{\id{a}}, \First \Second s_{\id{i}}, \Second \First, \First]  & \True \\
    \tuple{                        0,  \False, \True, s_{\id{i}}, s_{\id{t}}} & [s_{\id{t}},\First \Second s_{\id{i}}, \First \First s_{\id{a}}, \Second \First, \First]  & \True \\
    \tuple{                        0,  \False, \True, s_{\id{i}}, s_{\id{t}}} & [s_{\id{t}},\Second \First, \First \First s_{\id{a}}, \First \Second s_{\id{i}}, \First]  & \True \\
    \tuple{                        0,  \False, \True, s_{\id{i}}, s_{\id{t}}} & [s_{\id{t}},\Second \First, \First \Second s_{\id{i}}, \First \First s_{\id{a}}, \First]  & \True \\
    \tuple{                        0,  \False, \True, s_{\id{i}}, s_{\id{t}}} & [s_{\id{t}},\First \Second s_{\id{i}}, \Second \First, \First \First s_{\id{a}}, \First]  & \True \\
    \tuple{                        0,  \False, \True, s_{\id{i}}, s_{\id{t}}} & [s_{\id{t}},\First \First s_{\id{a}}, \Second \First, \First \Second s_{\id{i}}, \First]  & \True \\
    \tuple{                        0,  \False,\False, s_{\id{i}}, s_{\id{t}}} & [s_{\id{t}},\First \First s_{\id{a}}, \First \Second s_{\id{i}}, \Second, \First]  & \True \\
    \tuple{                        0,  \False,\False, s_{\id{i}}, s_{\id{t}}} & [s_{\id{t}},\First \Second s_{\id{i}}, \First \First s_{\id{a}}, \Second, \First]  & \True \\
    \tuple{                        0,  \False,\False, s_{\id{i}}, s_{\id{t}}} & [s_{\id{t}},\Second \Second,\First \First s_{\id{a}}, \First \Second s_{\id{i}}, \First]  & \True \\
    \tuple{                        0,  \False,\False, s_{\id{i}}, s_{\id{t}}} & [s_{\id{t}},\Second, \First \Second s_{\id{i}}, \First \First s_{\id{a}}, \First]  & \True \\
    \tuple{                        0,  \False,\False, s_{\id{i}}, s_{\id{t}}} & [s_{\id{t}},\First \Second s_{\id{i}}, \Second, \First \First s_{\id{a}}, \First]  & \True \\
    \tuple{                        0,  \False,\False, s_{\id{i}}, s_{\id{t}}} & [s_{\id{t}},\First \First s_{\id{a}}, \Second, \First \Second s_{\id{i}}, \First]  & \True \\
    \bottomrule
  \end{tabular}
  \begin{tabular}{LLL}
    \toprule
    \text{Symbolic value} & \text{Symbolic input} & \text{Path condition} \\
    \midrule
    \tuple{\min\ 600\ (s_{\id{a}}/10),  \True, \True, s_{\id{i}}, s_{\id{t}}} & [s_{\id{t}},\First \First s_{\id{a}}, \First \Second s_{\id{i}}, \Second \First, \Second] & s_{\id{t}}-s_{\id{i}}<365 \\
    \tuple{\min\ 600\ (s_{\id{a}}/10),  \True, \True, s_{\id{i}}, s_{\id{t}}} & [s_{\id{t}},\First \Second s_{\id{i}}, \First \First s_{\id{a}}, \Second \First, \Second] & s_{\id{t}}-s_{\id{i}}<365 \\
    \tuple{\min\ 600\ (s_{\id{a}}/10),  \True, \True, s_{\id{i}}, s_{\id{t}}} & [s_{\id{t}},\Second \First, \First \First s_{\id{a}}, \First \Second s_{\id{i}}, \Second] & s_{\id{t}}-s_{\id{i}}<365 \\
    \tuple{\min\ 600\ (s_{\id{a}}/10),  \True, \True, s_{\id{i}}, s_{\id{t}}} & [s_{\id{t}},\Second \First, \First \Second s_{\id{i}}, \First \First s_{\id{a}}, \Second] & s_{\id{t}}-s_{\id{i}}<365 \\
    \tuple{\min\ 600\ (s_{\id{a}}/10),  \True, \True, s_{\id{i}}, s_{\id{t}}} & [s_{\id{t}},\First \Second s_{\id{i}}, \Second \First, \First \First s_{\id{a}}, \Second] & s_{\id{t}}-s_{\id{i}}<365 \\
    \tuple{\min\ 600\ (s_{\id{a}}/10),  \True, \True, s_{\id{i}}, s_{\id{t}}} & [s_{\id{t}},\First \First s_{\id{a}}, \Second \First, \First \Second s_{\id{i}}, \Second] & s_{\id{t}}-s_{\id{i}}<365 \\
    \bottomrule
  \end{tabular}
  \caption{}
  \label{}
\end{table}

As a goal we set that the subsidy amount should be larger than zero. This leaves us with the following cases.

From the end states we can conclude that invoiceAmount should be larger than zero in order for the result to be larger than zero. This results in an additional constraint.

In the end, we can use the above to give hints to the user.

First, a date must be entered. Then the user can choose to ender an amount larger than zero, a date that is within 365 from now, or a "RL".
etc.


\subsection{Implementation}
\label{sec:implementation}

\lstset{language=Haskell}
\footnotesize\noindent%
\texttt{firsts :: MonadTrack Text m => MonadSupply Nat m => MonadState Heap m => MonadZero m =>}\\
\texttt{Val ('TyTask ('TyPrim t)) -> Goal t -> m ( Input, Pred 'TyPrimBool )}\\
%\texttt{firsts t g = [ ( i, p' ) | ( v, i:_, p ) <- simulate t [] Yes, let p' = p :/\: g v, satisfiable p' ]}
\normalsize
